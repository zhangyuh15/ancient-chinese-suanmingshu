\newpage
\appendix
\chapter*{附录一 聊备一格的称骨算命法}

% 在旧时算命术中,有一种托名为唐代命相学家袁天罡先
% 师的秤骨法。这种方法,只要对照一个人农历出生年、月、日、
% 时,分别查得年、月 、日 、时的秤骨份量,然后再把这些份量汇
% 总起来,便可在秤骨歌中找到有关自己一生荣枯的所谓 “断
% 语”了。因为这种方法简便易行,所以能“无师自通二 虽然其
% 法无学术意义,可是却也聊备一格,足供遣兴。
% 出生年、月 、日、时的秤骨份量
% 1. 出生年份六十花甲秤骨份量
% 〔甲子(鼠)〕
% 〔乙丑(牛)〕
% 〔丙寅(虎)〕
% 〔丁卯(兔)〕
% 〔戊辰(龙然
% 〔己巳(蛇)〕
% (:庚午(马)〕
% 〔辛未(羊)〕
% 一两二钱
% 九钱
% 六钱
% 七钱
% 一两二钱
% 五钱
% 九钱
% 八钱
% • 343»
% (:壬申(猴)J 七钱
% 〔癸酉(鸡)〕 八钱
% 〔甲戌(狗)〕 一两五钱
% 〔乙亥(猪 九钱
% 〔丙子(鼠)〕 一两六钱
% 〔丁丑(牛 八钱
% 〔戊寅(虎)〕 八钱
% 〔己卯(兔)〕 一两九钱
% 〔庚辰(龙内 一两二钱
% 〔辛巳(蛇)〕 六钱
% 〔壬午(马内 八钱
% 〔癸未(羊 七钱
% 〔甲申(猴 五钱
% 〔乙酉(鸡)〕一两五钱
% 〔丙戌(狗内 六钱
% 〔丁亥(猪)] 一两六钱
% 〔戊子(鼠)〕 1两五钱
% 〔己丑(牛 七钱
% 〔庚寅(虎)〕 九钱
% 〔辛卯(兔)〕 一两二钱
% 〔壬辰(龙)〕 一两
% 〔癸巳(蛇)〕七钱
% 〔甲午(马 一两五钱
% @〔乙未(羊)〕 六钱
% @〔丙申(猴)〕 五钱
% @ C丁酉(鸡内 一两四钱
% • 344 •
% 颈〔戊戌(狗内
% 一两四钱
% 〔己
% 亥(猪)〕 九钱
% ©〔庚子(鼠 七钱
% 〔辛
% 丑(牛 七钱
% 〔壬寅(虎)〕 九钱 @〔癸卯(兔)〕 一两二钱 @〔甲辰(龙内 八钱
% .
% ©〔乙
% 巳(蛇)〕 七钱
% 〔丙午(马
% 一两三钱
% 〔丁未(羊)〕 五钱
% ©〔戊申(猴力
% 一两四钱
% (:己酉(鸡)〕 五钱 @〔庚戌(狗口 九钱 ©〔辛亥(猪A 一两七钱
% (:壬子(鼠
% 内 五钱
% 〔癸丑(牛刀 七钱
% ®〔甲寅(虎内
% 一两二钱
% ®
% t乙卯(兔)〕 八钱
% 侈〔丙辰(龙)〕 八钱
% CT
% 巳(蛇)>
% 六
% ® (:应午(马)
% J
% 一两九钱
% @
% C己未(羊)
% J 六钱
% 一
% -
% 例工痍申(猴内 八钱 T. ®〔辛酉(鸡)〕 一两六钱 飞
% 二
% 与
% C壬戌(狗)〕 一两 C癸亥(猪)】 六钱
% •345
% 〔十一月〕 九钱
% 〔十二月 五钱
% 3. 出生日期秤骨份量
% 2. 出生月份秤骨份量
% 〔正月〕 六钱
% 〔二月〕 七钱
% t三月〕 一两八钱
% 〔四月〕 九钱
% 〔五月〕 五钱
% • 〔六月〕 —*两六钱
% [:七月〕 九钱
% [八月〕 一两五钱
% (1九月〕 一两八钱
% 〔十月〕八钱
% 〔初一〕 五钱
% 〔初二〕 —两
% 〔初三J 八钱
% 〔初四〕 一两五钱
% 〔初五〕 一两六钱
% 〔初六〕 一两五钱
% 〔初七〕 八钱
% 〔初八〕 一两六钱
% 〔初九〕 八钱
% 〔初十〕 一两六钱
% 〔十一〕 九饯
% 〔十二〕一两七钱
% ・346・
% 一两六钱
% 六钱
% 七钱
% 一两
% 九钱
% 一两六钱
% 一两
% 〔二十一〕
% 〔二十二〕
% (:二十三〕
% 〔二十四〕
% 〔二十五〕
% 〔二十六]
% 〔二十七〕
% (:二十八〕
% (:二十九〕
% (:三十] 六钱
% 4. 出生时辰秤骨份量
% 〔子时(23 1时):]
% 〔丑时( 1 3时
% 〔寅时( 3 5时
% 〔卯时( 5 7时)]
% 〔辰时( 7 9时〉]
% 〔巳时( 9 11时)]
% 〔午时 时)
% 八钱
% 一两七钱
% —两
% 八钱
% 九钱
% 一两八钱
% 五钱
% 一两五钱
% 一两
% 九钱
% 八钱
% 九钱
% 一两五钱
% 一两八钱
% 七钱
% 八钱
% 二两六钱
% 〔十三〕
% [十四〕
% (:十五〕
% 〔十六〕
% '〔十七〕
% T十八]
% 〔十九〕
% [二十〕
% •347
% 〔未时(13 15时)] 八钱
% 〔申时(15 17时)〕 八钱
% 〔酉时(17 19时 九钱
% 1戌时(19 21时)〕 六钱
% 〔亥时(21 23时)〕 六钱
% 这里要注意的是,时辰如果正 1点的,就算是丑时开始,
% 而不属于子时;正 3点的,就算是寅时开始,而不属于丑时;
% 正 5 点的,就算是卯时开始,而不屈于寅时。其他类推。若逢
% 夏时制出生的,就提前一小时算。
% 通过自己出生年、月 、日 、时,把从上表中分别查出的秤骨
% 份量总数加起来,然后再把加起来份量的总数,对照下面列举
% 的弧骨欧. 就可轻而易举地杳出每个人一生命运穷通的大致
% 情况了.
% 毛 骨 歌 .
% 在对照一个人农族定生的年、月、日、时的秤骨份量,并把
% 它们一一汇总之后,我们就可按照下列份员轻重,按图索骥地
% 找出每个人相关的秤骨歌诀。
% (:二两二钱〕
% 身寒骨玲苦仃伶,此命推来行乞人。
% 碌碌巴巴无度日,终年打拱过平生。
% 〔二两三钱〕
% 此命推来骨自轻,求谋作事事难成。
% 妻儿兄弟应难许,别处他乡作散人。
% (二两四钱〕
% • 348-
% 此命推来福禄无,门庭困苦总难荣。
% 六亲 骨肉皆无靠,流到他乡作老翁。
% 〔二两五钱〕
% 此命推详祖业微,门庭营度似稀奇。
% 六亲骨内如冰炭,一世勤劳自把持。
% (:二两六钱〕
% 平生衣禄苦中求,独自营谋事不休。
% 离祖出门宜早计,晚年衣裸庶无忧。
% 〔二两七钱〕
% 一生作事少商量,难靠祖宗作主张。
% 匹马单枪空做去,早年晚岁总无长。
% 〔二两八钱〕
% 一生行事似飘蓬,祖宗产业在梦中。
% 若不过房改名姓,也当移徙二三通。
% 〔二两九钱〕
% 初年运限未曾亨,纵有功名在后成。
% 筑到四句方可立,移居改姓始为良。
% t三两J
% 劳劳充碌苦中求,东奔西走何日休?
% 若系终身勘与他,老来殖 可免忧愁。.
% 〔三两一线〕 二
% 忙忙碌碌做何求,儿日云开见日美?
% 难得祖墓家可立,中年衣食浙能周。
% (:三两二线〕
% 初年运奏事难谋,需南财源如水流。
% 到得• • 中年衣食狂,-^w*~
% 那时名利一齐收。 - ~^r _r -.1i
% ..»>
% •349 •
% 〔三两三钱〕
% 早年作事事难成,百计 勤劳枉用心。
% 半世自如流水去,后来运至得黄金。
% 〔三两四钱〕
% 此命福气果如何?僧道门中衣禄多。
% 离祖出家方为妙,终年拜佛念弥陀。
% C三两五钱〕
% 生平福量不周全,祖业根基觉少传。
% 营事 生涯宜守旧,时来衣食胜从前。
% 〔三两六钱〕
% 不须 劳碌过平生,独自成家福不轻。
% 早有福星常 照命,任君行去百般成。
% (1三两七钱〕
% 此命般般事不成,弟兄少力自孤行。
% 虽然祖业须微有,来得明时去不明。
% 〔三两八钱]
% 一身 骨肉最 清高,早入鬟门姓氏标。
% 待到 年将三十六,蓝衫脱去换红袍。
% 〔三两九钱〕
% 此命终身运不通,劳劳作事尽皆空。
% 苦心竭力成家计,到得那时在梦中。
% C四两〕
% 乎生衣禄是绵长,件件心中自主张。
% 前面风霜多受过,后来必定享安康。
% 〔四两一钱〕
% 此命推来自不同,为人能干异凡庸。
% • 350 •
% 中年还有逍遥福,
% 〔四两二钱〕
% 得宽怀处且宽怀,
% 若使中年命运济,
% 二四两三钱〕
% 为人心性最聪明,
% 衣禄一生天数定,
% 〔四两四钱〕
% 万事由天莫苦求,
% 当年财帛难如意,
% 〔四两五钱〕
% 名利推求竟若何?
% 命中难养男和女,
% 〔四两六钱〕
% 东西南北尽皆施,
% 衣禄无穷天数定,
% 〔四两七钱〕
% 此命推求旺末年,
% 平生源有滔滔福,
% C四两八钱〕
% 初年运道未曾通,
% 兄弟六亲无依靠,
% 〔四两九钱〕
% 此命推来福不轻,
% 从来富贵人钦敬,
% 不比前时运未通。
% 何用双眉皱不开。
% 那时名利一齐来。
% 作事轩昂近贵人。
% 不须劳碌是丰亨。
% 须知福禄赖人修。
% 晚景欣然便不忧。
% 前番辛苦后奔波。
% 骨肉扶持也不多。
% 出姓移居更 觉隆。
% 中年晚景一般同。
% 妻荣子贵自怡然。
% 可卜财源若水泉。
% 几许蹉蛇命亦穷。
% 一生事业晚来隆。
% 自成自立显门庭。
% 使婢差奴过一生。
% 〔五两〕
% • 351 •
% 为利为名终日劳,
% 老来自有财星照,
% 〔五两一钱〕
% 一世荣华事事通,
% 弟兄叔侄皆如意,
% C五两二钱〕
% 一世亨通事事能,
% 宗族有光欣喜甚,
% (:五两三钱]
% 此格推来福泽宏,
% 一生衣食安排定,
% 〔五两四钱〕
% 此格洋推贵气深,
% 丰衣足食多安稳,
% t五两五钱〕
% 走马扬鞭争利名,
% 一朝福禄源源至,
% 〔五两六线〕
% 中年福禄也多遭。
% 不比前番目下高。
% 不须劳碌自亨通。
% 家业成时福禄宏。
% 不须劳苦自然宁。
% 家产丰盈自称心。
% 兴家立业在其中。
% 却是人间一富翁。
% 诗书满腹看功成。
% 正是人间有福人。
% 少年作本费评论。
% 贵荣华显六亲。
% 此命推来礼义通,一生福春用无穷。
% 帝酸苦辣都 娶过,液果财源急而君
% Q五耐七钱J
% 福禄本取万事全,一身荣11乐夭年.
% 多场威黑人争羡,处世逍遥显似仙
% (:五网八线J
% 平生衣食自然来,名利双全富贵僭
% 金榜题名暨甲第,紫袍玉带走金阶
% •362*
% 〔五两九钱〕
% 细推此格秀 而清,必定才高学业成。
% 甲第之中应有分,扬鞭走马显威荣。
% 〔六两〕
% 一朝金榜快题名,显祖荣宗大器成。
% 衣禄定然无欠缺,田园财帛更 丰盈。
% 〔六两一钱〕
% 不作朝中金榜客,定为世上大财翁。
% 聪明天付经书熟,名显商科自是荣。
% 〔六两二钱〕
% 此命生来福不穷,读书必定显亲宗。
% 紫衣金带为卿相,富贵荣华孰与同?
% 〔六两三钱〕
% 命主为官福禄长,得来富贵实非常。
% 名题雁塔传金榜,大显门庭天下扬。
% 〔六两四钱〕 .
% 此格威权不可当,紫袍金带坐高更
% 荣华富贵德能及?万古留名姓氏扬。
% 〔六两五钱〕
% 细推此命福不轻,富贵 荣华孰与争?
% 安国定邦人极品,威声显赫震寰瀛。
% 〔六两六钱〕
% 此格人间一福人,堆金积玉满堂春。
% 从来富贵由天定,金榜题名更显亲。
% 〔六两七钱〕
% 此命生来福自宏,田园家业最 高隆。
% •353
% . 乎生衣禄丰盈足,一路荣华万事通。
% 〔六两八钱〕
% 富贵由天莫苦求,万金家计不须谋。
% 十年不比前番事,祖业根基千古留。
% 〔六两九钱〕
% 君是人间-衣禄星,一生富贵众人钦。
% 总然 福禄由天定,安享荣华过一生。
% 〔七两〕
% 此命推来福不轻,何须愁虑苦劳心。
% 荣华富贵已天定,正笏垂绅拜紫宸。
% 〔七两一钱〕
% 此命生成大不同,公侯卿相在其中。
% 一生自有逍遥福,富贵荣华极品隆。
% 从以上秤骨算命法看,难免简单粗糙,并且形而上学气味
% 浓重。对照八字推命,按 60 甲子年,60 甲子月,60 甲子日,
% 60甲子时计数,这样 60X60 x 60X60 的结果,可以推出 1千
% 2 百 95 万个不同的命造,而此秤骨法则仅归纳为 50 种类型,
% 相去何啻霄壤?而四柱八字推命,如此浩繁,且无法准确为人
% 预测,而此则缺乏理论基础,随意计数的秤骨份量,又何能为
% 人预测前程呢?
% 秤骨法作为一种游戏,或茶余饭后的消遣,则姑妄言之,
% 姑妄听之,自无不可。假若斤斤拘泥于此,则就堕入魔道,非
% 智者所为了。
% •354
