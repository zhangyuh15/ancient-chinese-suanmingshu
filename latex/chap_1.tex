\chapter{算命术的起源}
\section{天命观的产生和先秦诸子的天命观}
上古时期,在人类生产力极不发达,认识水平极度低下的情况下,由于不能解释神秘的自然现象和完全把握自己的穷通寿天,于是就萌发了一种似乎上天有着一种不可抗拒的神秘的力,在支配着世上一切的一切,安排着人间一切的一切的思想。这种思想观点,就是进入阶级社会后古文献上记载的“天”能致命于人,决定人类命运的天命观,或者说是命运观。这用《易经》的话来说,叫做“乾道变化,各正性命”。对于这里的命,后人注释道,“命者,人所禀受,若贵贱夭寿之属也。”


从出土的甲骨卜辞、彝器铭文看,“受命于天”刻辞的不只一次出现,说明早在殷周时期,天命观就已经在一些人的头脑里扎根了。后来的儒家祖师爷孔子,就是位极度信命的老夫子。按理说,孔子是个知识渊博的大儒,对于人类社会有着深刻的认识,怎么就会信起命来呢?原来,他早年风尘仆仆,奔走列国,到处推销自己的政治主张,很想干一番轰轰烈烈的事业。可是到了后来,当他碰了一鼻子灰以后,才深深地省悟到,命运之神竟是如此这般的厉害,然而这时他已是个五十左右的人了。“五十而知天命”,就是他从不知命到知命这一思想转化过程的最好说明。与此同时,他不仅知命”,并且还是个怕命的人,不然,他就不会说出“畏天命”这样的话来。有意味的是,除了自己“知命”、“畏天命”,他和他弟子还不遗余力大肆宜扬“死生有命,富贵在天”,“不知命,无以为君子”,“君子居易以俟命,小人行险以侥幸”的思想。这里他的说教是,一个人的生死贫富,都是命里早就注定了的,作为一个君子来说,非得“知命”不可,否则就够不上做“君子”的资格。正因为君子是“知命”的,所以他能安分守已,服从老天爷的安排,但是小人却不这样,他们不肯听从天命,往往冒险强求,希望有幸能得个好结果。


当然,看问题也不能攻其一点,不及其余。《孔子集语》记录孔子的话说:“古圣人君子博学深谋不遇时者众矣,岂独丘(我孔丘)哉!贤不肖者才也,为不为者人也,遇不遇者时也,死生者命也。”这里,他认为贤和不肖是根据才华来划分的,干和不干是人们自己可以把握的,至于机遇好和不好,是死还是活,那就只得看时运和老天的旨意了。可见他在主张服从天命的同时,又是主张发挥人的主观能动性的。因为人的才华和努力是一回事,境遇和死生的命运安排毕竟又是一回事。

作为儒家学派的创始人,孔子的这种天命思想,又在后来
大儒孟子身上得到了新的反映。《孟子 •万章》上篇说:“莫之
为而为者,天也;莫之致而至者,命也。”意思就是,没有人叫他
干,而他竟干了,这就是天意;没有人叫他来,而他竟来了,就
是命运。同时他还举例说明,尧、舜的儿子都不肖,是因为舜,
禹为相的时间太长,所以尧、舜的儿子不有天下I 禹的儿子启

贤能,而益为相的时间又短,所以启能得到天下。以上这些,
都不是人力所为而自为,不是人力所致而自至。从理来说,这
属于天意,对人来说,这属于命运。天和命,实在是一致的。在
《孟子 •尽心上》中,孟子还说J夭寿不贰,修身以俟之,所以
立命也。”又说「莫非命也,顺受其正。是故知命者不立乎岩
墙之下。尽其道而死者,正命也;桎梏死者,非正命也J 前者
是说,不管命短命长,我都不三心两意,只是培养身心,等待天
命,这就是安身立命的方法。后者是说,天底下人的吉凶祸
下福,无一不是命运,只要顺理而行,接着的就是正命。所以懂
得命运的人不站立在有倾倒危险的墙壁下面。因此,尽力行
道而死的人所受的是天的正命,犯罪而死的人所受的不是天
的正命。这里,孟子虽然认为天命的力量无可抗拒,但是不管
怎样,我还是应该按照我的仁义而行,不能无缘无故地白白送
死。无疑,这对孔子的天命观来说,有着补充的一面。
那个御风而行,洒脱自在得很的列子,也是个坚信命运的
人。在《列子 •力命篇》里,他巧妙地通过“力”和“命”的对话,
宣传了自己的这一思想。力对命说「你的功劳怎么比得上我
呢?”命回答说「你对世间事物有什么了不起的功劳,竟想和
我老子比起高低来?”力说,“一个人的寿夭穷通,贵贱贫富,正
是我力所能及的6” 命听了随即反驳道,“彭祖的智慧不及尧
舜,而活了八百岁。颜回的才华超出众人,可寿只三十二岁。
孔老夫子的德不比诸侯来得差,却在陈、蔡等国遭了难。商纣
王的行够不上仁人,但坐上了帝王的宝座。吴国贤公子季札
没能在吴国做上官,坏蛋田恒却篡了齐国的政。商朝忠臣叔
齐和伯夷饿死在首阳山,鲁国权臣季氏的钱又远远超过了贤
士展禽。假如说你力能发挥作用,又为什么要让彭祖长寿颜

回夭折,孔子困厄纣王登基,季札低贱田恒高升,伯夷、叔齐、
展禽贫困而季氏富贵呢?”力被命一驳,楞了好一会儿,才开口
道「按照你的说法,我虽然对世间事物没有功劳,可社会上搞
得这般模样,你又为什么不出来制止呢?”命从容回答说,“世
界上的一切事物都得听凭他们自己去变化,长寿的长寿,夭折
的夭折,困厄的困厄,通达的通达,贵的贵,贱的贱,富的富,穷
的穷,又难道是我所能够改变的吗?”
通过这段风趣的对话,列子表达了这样一种思想,就是世
界上种种不合情理的事,都不是人力所能够解决的,因为在命
运安排面前,人力原是很渺小有限的。
此外,先秦诸子信命的还很多,而以儒家的势力为最大。
这种思想发展到汉代,随着儒家学说的风行天下,儒家天命观
的思想,就更深入人心了。《河图纪命符》说:“天地有司过之
神,随人所犯轻重,以夺其算纪。恶事大者夺纪(一年),过小
者夺算(一日)。随所犯轻重,所夺有多少也。人受命,得寿,
自本有数。数本多者,纪算难尽,故死迟,若所禀本数以少而
所犯多者,则纪算数尽而死早也 淮南王刘安认为,“仁鄙在
时不在行,利害在命不在智”,扬雄《法言》也说:“遇不遇,命
也「有人向他问命,他说「命是上天决定的,不是人为的。人
为的称不上命,上天决定的命是逃避不了的。”
王充是东汉著名学者,他不信鬼神,一生反对迷信,可是
对于命运,却坚决主张是客观存在着的。他说:“凡人遇偶(碰
上好运〉及遭累害(遭受灾祸),皆由命也。有死生寿夭之命,
亦有贵贱贫富之命。”在他所著《论衡》一书的《命禄》、《气寿》、
《幸偶》、《命义》、《无形》、《偶会》、《初禀》等篇章里,随时都可
碰上对于命运的论述。可见,人们对于命运的信仰,不只是后

来算命家们故弄玄虚所捣的鬼,这里面还有一批大学问家的
参与、坚信和鼓吹,可谓盘根错节,阵容坚强。
在这种社会历史背景下,加上人们对种种不合理的社会
现象如不该飞黄腾达的反而飞黄腾达,不该久处人下的反而
久处人下,不该富贵荣华的反而富贵荣华,不该穷困潦倒的反
而穷困潦倒,以及自己难以把握自己的未来、难以把握自己的
生活、难以把握自己的生命等种种困惑,得不到使人信服的合
理解释,因此便就很自然地信奉起那些大知识分子的说教来。
而统治阶级为了坐稳他们的交椅,也乐于接受这一套,并且不
择手段地加以推广利用。从此以后,那些时运不济的达人,常
常会用“比上不足,比下有余”,“乐天知命”的话来安慰自己,
安于现状;而那些想不通的,除了怨天尤人之外,只好自讨不
服命的苦吃了。

\section{算命术的起源、发展和成熟}
天命观经过先秦学者的一阵鼓吹,其时从上到下,从统治
者到平民百姓,信命的风气一时很盛。早在殷商时期,迷信的
统治者们,就已习惯于在每做一件事之前,总重先占卜一下天
意如何,是凶是吉?后来,又由于人与天地相应观念的影响,
更使得人们普遍认为,整个天下的命运和每个个人的命运,都
和天时星象有关。《周礼 •春官》记载:“冯相氏掌十有二岁,
十有二月,十有二辰,十日,二十有八星之位,辨其叙事,以会
天位。”“保章氏掌天星,以志星辰日月之变动,以观天下之迁,

辨其吉凶。以星土辨九州之地所封,封域皆有分星,以观妖
祥。”这是说冯相氏和保章氏,是专管岁时星象,并从而推测人
间吉凶祸福的一种职官。
用占卜、占星来测候人事的吉凶祸福,这种方法,只能根
据已经发生了的现象进行分析,并且测候的事情,也只能局限
在近阶段所要做的事里。随着人们对预测术越来越高的要求
和越来越大的胃口,这样原始占卜和占星的方法,已经远远不
能满足人们要求预测一生吉凶祸福,贵贱寿夭的难填欲壑。于
是,标榜着能预测人们一生征途,包括过去未来的算命术就应
运而生了。算命术的产生,是人类社会发展到一定阶段的必
然产物,中国如此,外国也如此,只不过是戏法人人会变,各有
巧妙木同罢了。
算命的产生既然有着这种历史的必然性,那末,中国算命
术产生的理论依据又是什么呢?自先秦两汉以来,伴随着哲
学上阴阳五行学说的确立盛行,哲学家们认为,天地间一切事
物的发生、发展和变化,都是阴阳对立平衡,金、木 、水、火、土
五行相互生发制约的结果。既然天地间事物的发生、发展和
阴阳五行有关,那末,作为一个小天地的人身,假如有可能推
知他与生俱来的禀赋,不就可以测知他一生发展的前途了吗?
在这种思想支配下,阴阳五行便就自然成了算命家算命的重
要理论依据了。

在时间上说确切点,我国算命术的起源,大概始于两汉。
反映在文字上,主要见载在《白虎通义》和王充《论衡》等著作
里,而尤以《论衡》为最具代表性。王充《论衡》谈及命理的,主
要有《命禄第三》、《气寿第四》、《幸偶第五》、《命义第六》、《无
形第七》、《吉验第九》、《偶会第十》、《初禀第十二》、《物势第十

四》等篇。这些篇章,不仅空前地宣扬了天命的必然存在,并
且在对命运奥秘的探知上,也作了某些方法上的启发性论证。
在《命禄篇》中,王充说道「凡人遇偶(逢吉)及遭累害,皆
由命也。有死生寿夭之命,亦有贵贱贫富之命。”“命当富贵,
虽贫贱之,犹逢福善矣。故命贵,从贱地自达;命贱,从富位自
危」正因为这样,所以 “才高行厚,未必保其必富贵;智寡德
薄,未可信其必贫贱”。比如才智如孔子这样的睿哲,“犹无成
立之功”,原因自然在于“富贵有命禄,不在贤哲与辩慧”。不仅
如此,王充还断然提出「自王公逮庶人,圣贤及下愚,凡有首
目之类,含血之属,莫不有命」这就进一步把命定论的范围,
从万物之灵的人进一步推向到动物王国去了。篇末,王充还
不惜篇幅这样论述道J信命者,则可幽居俟时,不须劳精苦形
求索之也。"富贵之福,不可求致I 贫贱之祸,不可苟除也「
“越王翳逃山中,至诚不愿,自冀得代。越人熏其穴,遂不得
免,强立为君。而天命当然,虽逃避之,终不得离」耐人寻味
的是,把天性和命混为一谈,也是王充论命的一个侧面,如说,
“人情有不教而自善者,有教而终不善者矣。天性,犹命也J
这无疑是说,和天性连在一起的命,不是人力所能改变得了
的。
《气寿篇》是王充论命有关长寿短寿的重要篇章。文中认
为J凡人禀命有二品,一曰所当触值之命,二曰强弱寿夭之
命。所当触值,谓兵烧压溺也。强寿弱夭,谓禀气渥(厚)薄
也。兵烧压溺遭以所禀为命,未必有审期也。若夫强弱夭寿
以百为数,不至百者,气自不足也。夫禀气渥则其体强,体强
则其命长;气薄则其体弱,体弱则其命短。命短则多病,寿短。
始生而死,未产而伤,禀之薄弱也。渥强之人,不卒其寿。若

夫无所遭遇,虚居困劣,短气而死,此禀之薄,用之竭也。此与
始生而死,未产而伤,一命也。皆由禀气不足,不自致于百也。”
接着又说「人之禀气,或充实而坚强,或虚劣而软弱。充实坚
强,其年寿,虚劣软弱,失弃其身。天地生物,物有不遂;父母
生子,子有不就。物有为实,枯死而堕;人有为儿,夭命而伤。
使实不枯,亦至满岁;使儿不伤,亦至百年。然为实、儿而死枯
者,禀气薄,则虽形体完,其虚劣气少,不能充也。儿生,号啼
之声鸿朗高畅者寿,嘶喝湿下者夭。何则?禀寿夭之命,以气
多少为主性也。妇人疏字者子活,数乳者子死。何则?疏而
气渥,子坚强,数而气薄,子软弱也。怀子而前已产子死,则谓
所怀不活。名之曰怀,其意以为已产之子死,故感伤之子失其
性矣。所产子死,所怀子凶者,字乳亟数,气薄不能成也。虽成
人形体,则易感伤,独先疾病,病独不治。”这里,王充的思想非
常复杂,既有天命的,又有科学的,甚至还包涵着我们今天所
说的优生学的思想成分在内。
从《命义篇》看,王充在这里充分论述了自己的命理思想,
并且初步形成了一套完整的体系。文章开头说:‘墨家之论,
以为人死无命。儒家之议,以为人死有命。”接着经过论证,作
者以“国命胜人命,寿命胜禄命”,“遭逢之祸大,命善禄盛不能
却也”的观点,.否定了墨家提出的 “闻历阳之都,一宿而沉为
湖;秦将白起坑赵降卒于长平之下,四十万众同时皆死; 春秋
之时,败绩之军,死者蔽草,尸且万数;饥傕之岁,饿者满道,温
气疫病,千户灭门。如必有命,何其秦、齐同也”的反驳。接着,
他又阐述传曰「说命有三,一曰正命,二曰随命,三曰遭命”的
正.随、遭‘三命”道「正命,谓本票之自得吉也。性然骨善,故
不假操行以求福而吉自至,故曰正命。”“随命者,戮力操行而

吉福至,纵情施欲而凶祸到,故曰随命J“遭命者,行善得恶,
非所冀望,逢遭于外,而得凶祸,故曰遭命」文中尤为体现其
命学思想的是J凡人受命,在父母施气之时,已得吉凶矣。”认
为人们一生吉凶的命,早在父母交合,得孕之初就已定下来
了。这一学说,为推算怀孕日期论命吉凶,提供了原始的理论
依据,并为后世发展起来的推算出生年、月 、日、时论命,作出
了一定的启发。接着,王充还在篇内论述正、随、遭“三性”道:
“正者,禀五常之性也。随者,随父母之性。遭者,遭得恶物象
之故也。故妊妇食兔,子生缺唇。《月令》曰「是月也,雷将发
声。'有不戒其容者,生子不备,必有大凶,喑聋跛盲。气遭胎
伤,故受性狂悖。羊舌似我初生之时,声似豺狼,长大性恶,被
祸而死。在母身时,遭受此性,丹朱、商均之类是也。” 这些都
是因为,“性命在本,故《礼》有胎教之法, 子在身时,席不正不
坐,割不正不食,非正色目不视,非正声耳不听。及长,置以贤
师良傅,教君臣父子之道,贤不肖在此时矣。受气时,母不谨
慎,心妄虚邪,则子长大,狂悖不善,形体丑恶。”可见这里,王
充在论命的同时,又同时提出了胎教和早期教育对命所产生
的重大影响,真可谓是集唯心和唯物,迷信和科学于一炉了。
“初禀”是王充论命的核心思想之一,在《初禀篇》中,他
说「人生性命当富贵者,初禀自然之气,养育长大,富贵之命
效矣」“命,谓初所禀得而生也。人生受性,则受命矣。性命
俱票,同时并得,非先察性,后乃受命也「因此,周二正作为圣
人,早在“母身之中已受命”了。此后他还才想联翩地说:“夫
王者,天下之资也,禀命定于身中,犹鸟之别雄雌户泊壳之中
也。卵壳孕而雌雄生,日月至而骨甘强,强则雄自号 “ 雄
非生长之后,或教使为雄,然后乃敢符雌,此气性刚如门为之


矣。夫王者,天下之雄也,其命当王。王命定于怀妊,犹富贵
骨生,鸟雄卵成也。非唯人鸟也,万物皆然。”初禀既然如此重
要,那末如果放弃初禀而官论命,就全都落空了。
如果说,王充论命后方法还只处在初级阶段,那末这种
初级阶段的理论,还集中地体现在他的《物势篇》中。文中重
点论述了五行和十二生肖之间的生克关系,从而为后世命理
学家在这方面的发挥奠定了初步基础。他说I “且一人之身,
含五行之气,故一人之行,有五常之操”。“寅,木也,其禽虎
也。戌,土也,其禽犬也。丑未亦土也,丑禽牛,未禽羊也。木
胜土,故犬与牛羊为虎所服也。亥,水也,其禽豕也。巳,火也,
其禽蛇也。子亦水也,其禽鼠也。午亦火也,其禽马也。水胜
火,故豕食蛇。火为水所害,故马食鼠屎而腹胀J
综观王充所论,不但提出了五行论命的依据,并且还进一
步触及到了生肖生克和夫妻贼害都是命的理论实践, 从而成
了我国算命发展史中实际上的先驱。
但是,从后代人的著述看,一般都把战国时的鬼谷子、珞
球子等人推上算命术祖师爷的宝座。因为没有确切的证据可
以证明这种说法,加上先人们有着好古的习惯,一切都喜欢假
托古人,并且越古越好,所以这种说法多半出于附会,大致可
以肯定下来。
东汉以后,魏晋南北朝时,知识阶层对天命的信仰愈益漫
延。《魏书 •孙绍传》云「绍曾与百寮赴朝,东掖未开,守门候
旦。绍于众中,引吏部郎中辛雄于众外,窃谓之曰:‘此中诸
人,寻当死尽,唯吾与卿,犹享富贵。'雄甚骇愕,不测所以。未
几,有河阴之难。绍善推禄命,事验甚多,知者异之「
晋葛洪《抱朴子内篇 • 辨问》并引《玉铃经》的说法,认为

人们一生吉凶,早在父母交合,得妊之初就已定下来的原因,
在于和天上的值日星宿有关。他说「《玉铃》云:《主命原》曰:
人之吉凶,制在结胎受气之日,皆上得列宿之精。其值圣宿则
圣,值贤宿则贤,值文宿则文,值武宿则武,值贵宿则贵,值富
宿则富,值贱宿则贱,值贫宿则贫,值寿宿则寿,值仙宿则仙。
又有神仙圣人之宿,有治世圣人之宿,有兼二圣之宿,有贵而
不富之宿,有富而不贵之宿,有兼富贵之宿,有先富后贫之宿,
有先贵后贱之宿,有兼贫贱之宿,有富贵不终之宿,有忠孝之
宿,有凶恶之宿。如此不可具载,其较略如此。为人生本有定
命,张车子之说是也」
那末何谓“张车子之说”呢? 张车子的典故出在晋代干宝
所著《搜神记》卷第十中。书云:
周章啧者,贫而 好道。夫妇夜耕,困息卧,梦天公过
而哀之,敷 外有以给与。司命按录籍,云「此人相贫,限
不过此。唯有张车子,应赐钱千万。车子未生,请以借
之。”天公曰,“善。”曙 觉言之,于是夫妇戮力,昼夜治生9
所为辄得,资至 千万。先时有张妪者,尝 往周家佣赁,野
合有身,月 满当孕,便 遣出外,驻 车屋下 ,产 得儿。主人往
视,哀其孤寒,作粥麋食之。问:“当名汝儿作何?”妪曰:
“今在车屋下而生,梦天告之,名 为车子。”周 乃悟曰「吾
昔梦从天换钱,外白以张车子钱贷我,必是子也。财当归
之矣。”自 是居日衰减。车子长大,富于周家。
此后,张车子的故事,就常被用来作为人生贫富天定的典
故使用了。
梁朝之时,刘勰《新论》并论命相说「命者,生之本也;相
者,助命而成者也。命则有命,不形于形, 相则有相,而形于

形。有命必有相,有相必有命,同禀于天,相须而成也。人之
命相,贤愚贵贱,修短吉凶,制气结胎,受生之时,其真妙者,或
感五帝三光,或应龙迹气梦,降及凡庶,亦禀天命,皆属星辰,
其值吉宿则吉,值凶宿则凶。受气之始,相命既定,则鬼神不
能改移,而圣智不能回也J
话虽如此,但从方法论看,自从后魏孙绍始推禄命,陶弘
景有《三命抄略》以来的很长一段时间里,算命的方法很是粗
糙简单,有的仅仅只就生日那天所碰上的星象来作推测,所以
还形不成一个完整的体系。然而,也正由于经过了三国魏晋
南北朝一段时期算命家们共同的摸索探讨,到了唐代,才有了
一次大的发展,质的 飞跃。原因是阴阳五行和一个人出生年、
月、日进一步紧密结合,以推断一生休咎的学说,在李唐年间
获得了方法上的确认。同时,随着中外文化的空前交流,印
度、西域的占星术也相继传进中土,促进了中土算命术的发
展。
从宋元人文献记载得知,按照星象历法推算人命,始于唐
代贞元年间,也就是公元 785 年到 805 年的一段时间里,当时
有西域康居国来的一个名叫李弼乾的术士,传来了印度婆罗
门术《聿斯经》。 有了原来的气候土壤,再加上外来术数的一
伍进,中国原有算命术的发展,就振翅起飞了。
在唐代算命术飞速发展,并且正式确立体系的过程中,起
关键核心作用的有李虚中、僧一行、桑道茂等人。其中魏郡
(沿所在今河北大名)人李虚中,字常容。唐德宗贞元年间,李
京中科考顺利,中了进士,后来一直做到殿中侍御史的官。平
:小,他精究阴阳五行,能够根据一个人出生年、月 .日的天干地
工,来推定他一生中的贵贱寿夭,吉凶祸福,并且竟然“百不失

一”。以上这些,都是唐朝大儒韩愈白纸黑字,在《殿中侍御史
李君墓志铭》里明白告诉我们的。由于李虚中本人的本事,再
加上韩愈一吹,后人就把他尊为命理学家开山祖师了。因为
东汉王充那一套,和李虚中“汪洋奥美,关节开解,万端千绪,
参错重出"的学说比较起来,毕竟粗糙得很,还形不成一套完
整的体系。李虚中留下的著述,有署名为鬼谷子撰的《命书》
三卷中的注释部分。但是后人考证结果,《命书》既不是鬼谷
子所撰,就是连注释也不一定出于李虚中之手。
李虚中这种以出生年、月 、日天干地支对一个人一生的吉
凶祸福进行推测的方法,经过五代宋初人徐子平的进一步发
展完善,中国算命术才正式进入成熟完备阶段,并为后来的命
理学家所广泛取法。
史书记载,徐子平名居易,曾和当时看相大师麻衣道人陈
图南一起隐居华山,精研命理之学。他在算命术上的最大贡
献,就是把李虚中推算年、月 、日干支的办法,进一步演汇 ‘j
年、月、日、时同时测算的“四柱”法。所谓“四柱”,就是以出生
年份的天干地支为第一柱,月份的天干地支为第二柱,日期的
天干地支为第三柱,时辰的天干地支为第四柱。这样,每一柱
天干一字,地支一字,共两个字,四柱天干地支加起来的总字
数就是八个字。然后再按照这八个字中所蕴含着的阴阳五行
进行演算,就可推知一个人一生命运的大致情况了。他的著
述,据说有《徐子珞球子赋注》二卷,见《说郛 • 已疟篇》。 后
来,人们为了纪念他,又有把算命术称作“子平术”的。
徐子平以后,宋代文人信命的记载较之前代,有了更加增
多的趋向,苏轼《东坡志林》说到命的有三条,其中一条说帏
愈以磨蝎为身宫,而他本人也以“磨蝎为命,平生多得谤誉,殆

是同病”。王辟之《淹水燕谈录》记载湄州三灵山人程惟象,年
轻时碰上异人传授命相要诀,后来精益求精,为人推算贵贱寿
夭准确率很高。那时有个名叫张宣徽的,问他一个丁酉人命,
他说「天宾星行初度,不当作内臣,寿数五十四岁。'结果果然
被他说中。有棒的是,释文莹在《玉壶清话》中,还提到了一则
瞎子算命的,。那时有个刘童子,从小瞎了眼睛,擅长于“声
骨及命术”。 荆南人夏侯嘉正向他求教,刘童子说「你将来一
定及第,并且有清职。收入除了薪俸,还会有百金的横财。可
惜有了横财,寿数就终了。”后来也果然应了他的话。魏泰《东
轩笔录》卷十四说「章邮公,庆历中罢相知陈州,般舟蔡河上。
张方平、宋子京俱为学士,同谒公,公曰「人生贵贱、莫不有
命,但生年、月 、日 、时、胎有三处合者,不为宰相,亦为枢密副
使。”张、宋退召术者,泛以朝士命推之,惟得梁适、吕公弼二
命,各有三处合,张、宋叹息而已。是时梁、吕皆为小朝官,既
而皇祐中,梁为相,熙宁中,吕为枢密使,皆如郁公之言。这些
记载,都从另一角度,反映了当时学者如苏轼、王 辟之、释文莹
等人都是信命的,并且在他们的笔记中,似乎还带着明显拔高
命术的倾向。
在方法上,徐子平所创立的“四柱”法从宋代开始,已渐渐
地大行于天下了。那时,算命不仅是命理学家的事,就是通儒
学者,也大多精通命理。如南宋著名理学家朱熹的老朋友徐
端叔,就是一个读书也精通命理的人。对于徐端叔命理的大
致情况,朱熹在《赠徐端叔命序》中说「世以人生年月日时所
值枝(支)干纳音(五行的一种),推知其人吉凶寿夭穷达者,其
术虽若浅近,然学之者亦往往不能造其精微。盖天地所以生
物之机,不越乎阴阳五行而已,其伸屈消息,错综变化,固已不

可深穷,而物之所赋,贤愚贵贱之不同,特昏明厚薄毫厘之差
耳,而可易知其说哉。徐君尝为儒,则尝知是说矣。其用志之
密微而言之多中也固宜。世之君子倘一过而问焉,岂惟足以
信徐君之术而振业之,亦足以知夫得于有生之初者,其赋予分
量固已如是,富贵荣显,固非贪慕所得致,而贫贱祸患,固非巧
力所可辞也J
从其时著作流传看,现在仍在港台流行着的《渊海子平》,
就是宋代徐子升根据徐子平命理研究成果,纂辑而成的一部
重要著述。
元朝时期,上层统治者虽然由汉贵族换成了蒙古贵族,可
是汉族社会中算命的风气却依然盛行不衰。元末明初陶宗仪
《辍耕录》记载「术士俞竹心者,居庆元,嗜酒落魄,与人寡合,
顺其意者,即与推算,醉笔如飞,略不构思,顷刻千余言,道已
往之事极验,时皆以为异人。至元(元惠宗年号)己卯(公元
1339 年)间,娄敬之为本路治中,尝以休咎叩之。答曰「公他
日直至一品便休; 娄深信其说,弃职别进,适值壬午(公元
1342 年),更化俯就省掾,升除益都府判,改换押字,宛然真书
'一品'二字,未几卒于官所。此偶然耶?抑数使然耶?”书中
并另有故事一则道,元代有个旷达不羁的富家子弟,好几个算
命先生为他推算命程的结臬,都说他的寿元只有三十岁。富
家子弟听后,知道自己将不久人世,就把家里的资财都慷慨地
接济了穷人。后来在一个风浪险恶的渡头,这富家子弟又救
了个伫立江边,正要纵身跳下翻滚波涛自寻短见的丫鬟。事
情过去一年以后,富家子一行又来到这风恶浪险的渡头摆渡,
正巧上次被他救起的那个丫鬟从旁走来,坚决要邀他回家和
丈夫一起拜谢他的救命之恩。原来这丫鬟后来被主家辞退,已

【16页图片】

嫁了人。没奈何,富家子只得让其他同行的二十八人先摆渡
过去,自己则跟着丫鬟来到她家。待到吃过茶点,辞别主人出
来,只听得街上沸沸扬扬,才知道刚才摆渡的船只,已被风浪
卷没,而同行的二十八人,全都无一生还。这里,陶宗仪在宣
扬信命的同时,又掺进了佛家积德可以扭转命运的因果报应
思想,色彩极为瑰奇。由于元朝享国不长,命理学著述较少,
只李钦夫所撰《子平三命渊源注》少数几种而已。
到了明代,中国算命术在社会流传上,达到了一个前所未
有的高峰。那时的开国功臣宋濂,曾写《禄命辨》一文,第一次
系统地总结了我国命理学的历史渊源。而一时间有关的命理
著述,也如雨后春笋大量涌现。其中《滴天髓原注》出于明初
重臣刘基之手,它的参考价值,至今有目共睹。此外沈孝瞻的
《子平真钱》、万育吾的《三命通会》、张神峰的《神峰通考命理
真宗》等,也都是一时的佼佼者。
至清一代,算命术依然盛行不歇,那时社会上人们不管富
贵贫贱,男女老少,也不管碰上婚姻、赴考、经营等什么事,都
存在着一种请人算算的心理上的强烈欲望。虽然有时他们中
的一些人嘴上总会挂起“君子问凶不问吉”的堂皇话,可心里
想的,则最好是吉星高照,鸿运亨通。由于知识分子如纪啊、
俞栅等人的推波助澜和介入,社会上研究命理的风气十分浓
厚,主要著述有陈素庵的《命理约言》、《滴天髓辑要》,任铁樵
的《滴天髓阐微》,以及无名氏所撰先名为《拦江网》,后来又改
名为《穷通宝鉴》等多种a
综括以上所说,算命术在东汉末年确立概念,形成雏形以
后,经过唐朝李虚中的发展,基本形成了体系,后来又经由徐
子平创“四柱法”,从而标志了中国算命术的瓜熟蒂落,正式成

熟。徐子平以后,宋元明清,人们寄希望于命运,相信算命术
的犹如堤岸决口,洪波汹涌,多得不可收拾。就是连天下至尊
的万岁爷,也还要请人算算,看看往后的结局如何,寿数长短。
这就证明了即使坐着皇帝宝座的,心里也并不那么踏实,更不
要说是深受剥削压迫,处在底层,生活没有保障的黎民百姓
了。
民国以来,大军阀和大官僚们,包括蒋介石本人,也都相
信算命。1927 年蒋介石初次下野,回到奉化溪口,特地请名
僧太虚法师到雪窦寺为他老婆毛氏讲解《心经》。一天,蒋介
苏和他大哥蒋介卿一起去雪窦寺漫游。太虚精于星相 之术,
当蒋介卿告知太虚蒋介石生于光绪十三年丁亥九月十五日未
时后,太虚就为蒋介石聚西会神地推算起来,不久并向蒋介石
祝贺道「今岁流年丁卯东顺,明年交入戊辰,不仅东山再起,
还有大喜临门呢。”当时蒋介石听毕,微笑而已。第二年,蒋介
石从日本归国,非但重掌党国大权,并且和美貌聪慧的宋美龄
共偕连理。这一来,蒋对太虚法师真是佩服得五体投地。此
后由于蒋的推荐,太虚法师更是名噪朝野,红极二时。
当时,社会上相信5运的,更是多得数不过来,包括那些
吃墨水的读书人。当年夏丐尊撰《命相家》一文,文中提到自
己因事到南京去,住在 x x 饭店,二楼楼梯旁就寓着一位命相
家,门口招牌写着「青田刘知机星相谈命。”原来正是自己十
年前的同事刘子岐。一阵寒暄过后,得知刘知机最初在上海
挂牌,后来到了南京。正谈话间,门忽然开了,进来两位顾客:
一个是戴呢帽穿长袍的,一个是着中山装的,年纪都未满三十
岁。刘子岐
——刘知机丢开了作者,满面春风地立起身来迎
上前去,俨然一副十足的江湖派。这时夏丐尊自感不便再坐,
• 18 •
就把房间号数告诉子岐,约他畅谈,回到自己房里。
夏丐尊满腹狐疑:十年前的中学教师,居然会卖卜?顾客
居然不少,而且大都是青年知识阶层中人。后来刘知机到底
来了,夏丐尊禁不住问道「命相学当真可凭吗?”
“当然不能说一定可凭。不过现今这样的社会上,命相之
说,尚不能说全不足信。你想,机关里当科长的,能力是否一
定胜过科员?当次长的,能力是否一定不如部长?举个例说,
我们从前的朋友里,李 X X 已成了主席了。王 XX学力人品,
平心而论远过于他,革命的功绩也不比他差,可是至今还不过
一个XX部的秘书。还有,一班毕业生几十人中,有的成绩并
不出色,倒有出路,有的成绩很好,却没有人去过问。这种情
况除了命相,该用什么去解释呢?宥人说,现今吃饭全靠八行
书,这在我们命相学上叫 '遇贵人'。这简直是穷通由于先天,
证明命的的确确是有的了」刘知机带点玩世不恭地说。
“这样说来,你们的职业实实在在有着社会的基础J
。到了总理的考试制度真正实行性后,命相也许不能再成
为职业。至于现在,有需要,有供4;先是堂堂皇皇的吃饭职
业。命相学家的身份决不比教师来得低下,我预备把这碗江
湖饭吃下去哩J
“你的营业行目有几种?”
“命,相,风水,合婚择日,什么都干。风水与合婚择日近
来已不行了,只有命、相两项,现在仍有生意。因为大家都在急
迫地要求出路,等机会,出路与机会的条件不一定是资格和能
力,实在是全靠碰运气。任凭大家口口声声喊 '打破迷信',到
了无聊之极的时候,也会瞒了人花几块钱来请教我们。在上
海,顾客大多是商人,他们所问的是财气。在南京,顾客大半
• 19 •
是 '同志'和学校毕业生,他们所问的是官运。老实说,都无
非为了要吃饭。唯其大家想要吃饭,我们也就有饭吃了。哈
哈……”刘知机滔滔不绝,酒已半醺,自负之外又带感慨。
刘知机的沦为命相术士,自然是伤心人别有怀抱。可话
也不能全然这样说,比如在旧上海,以命相为生的术士虽说比
比都是,其中有的就深有学问根基,如袁树珊和韦千里,就是
享有盛名的两位。其中韦千里毕业于复旦大学文学系,解放
前寓居上海南京路大庆里 34 号,解放初迁居香港,为人学识
渊博,著述如林。有趣的是,据近代命理学家太乙子朱为纲
告知,1945 年抗战胜利后,上海还曾有过星相公会组织,后来
并亦筹备命理哲学研究会,袁树珊、丁太炎、朱为纲等人,都是
研究会中的理事。其时百姓由于生活在水深火热之中,受着
寄希望于晚年转运的心理支配,请教算命先生的更是多得难
以计数。即使到了现在,在科学经济都已十分昌明繁荣的香
港、台湾,人们相信命运的仍然趋之若鹫,包括社会上层人士
在内。如当前香港流行的,就有子平术、紫微斗数、铁板神数
等术。 图仁
台湾梁心铭著《现代命学》,书中《论吉凶祸福篇》中提到:
*命里有时终须有,命里无时莫强求。中华电视台七月二日开
播 '天机'来讨论人生之命运,其主旨是要让人创造命运,不要
向命运低头。”接着还振振有词地说「当你了解命运之奥秘之
后,则命运就掌握在你手中,行好运时宜把握良机,努力奋斗,
即可多得利益,开创你美好新前程。行坏运时应宜守为安,勿
踏危机,减少灾害。若是吉凶参半运时,得意须防失意时,往
来知节要稳步,进退谨饿要三思。总而言之一个人,得意之时
处以淡,失意之时处以思,一切言行需安详。朋友1 你认为
• 20 •
呢?”
发展到运用现代先进传播媒介电视开播“天机”, 进行大
范围的讨论人生命运,并且形成了一系列新的命运观念,确是
当今港台有关命运观流行的一大特点。
由于习惯势力根深蒂固,积重难返,即使在大陆,经过解
放多年来历史唯物主义和辩证唯物主义的教育,要是进行一
次民意测验,人们相信命运的肯定还大有人在。如笔者收到
的读者来信中,知识界中肯定命运存在的真还为数不少,至于
广大农村,那就更是不在话下了。由此足以说明,在祖国土地
上盛行了一千多年的算命风气,是多么的难以彻底根除。

\section{古今命理典籍概览}

在历史长河的积淀中,千百年来,我国命理著作,可谓林
林总总。为了对我国古代命理典籍的概况有个大致的了解,这
里择要而述。
《李虚中命书》
《李虚中命书》计有三卷,旧本题为鬼谷子撰,李虚中注。
李虚中,字常容。魏侍中李冲的第八世孙。在仕途上,李
虚中先是中了进士,不久在唐宪宗元和年间(公元 806—820
年),做上了殿中侍御史的官。李虚中死后,大文豪韩愈曾为
他作墓志铭推崇说,李虚中先生最擅长五行书,平时凭着人们
出生年、月 、日 干支五行的相胜相克,以及胜衰王相进行斟酌,
可以推算人的寿夭贵贱,利和不利,有着相当高的命中率,真
。21 •
所谓是“汪洋奥美,万端千绪”。
然而问题在于,韩愈并没有说到李虚中有所著述,《唐书
艺文志》也没有提到过《李虚中命书》这部书。此后直到《宋
史 • 艺文志》,才收有《李虚中命书格局》二卷。可是从这一时
期的文人著录看,则又说法互不一致,没有一个统一的口径:
郑樵《通志 • 艺文略》的说法为,《李虚中命书》一卷,《命书补
遗》一卷1 晁公武《群斋读书志》则题作《李虚中命书》三卷;焦
氏《经籍志》则又在《李虚中命书》三卷之外,另外又别出《命书
补遗》一卷。由于这些传本早已无法看到,所以也就只好把问
题悬在那里,无法考证它们的异同了。
好在明初朝廷编辑《永乐大典》,卷帙浩繁,把这书也收了
进去,所以不致失传。书市载有李虚中自序一篇,序中提到汉
司马季主在壶山之阳碰到鬼谷子,鬼谷子授给司马季主 "逸
文”九篇,论述幽微之理。这些“逸文”到了唐朝李虚中手里,
李就掇拾诸家,作了一道注释成集的工序,于是便就有了《李
虚中命书》的出笼。
关于《水乐大典》女降虚中命书》的情况,《四库全书总
目》这样概括道「详勘书中义例,首论六十甲子,不及生人时
刻干支,其法颇与韩愈墓志所言始生年、月 、日者结合,而后半
乃多称四柱,其说实起于宋时,与前文殊相缪戾,且其他职官
称谓,多涉宋代之'事,其不尽出虚中手,尤为明甚。中间文笔
有古奥难解者,似属唐人所为,又有鄙浅可嗤者,似出后来附
益,真伪杂出,莫可究诘。疑唐代本有此书,宋时谈星学者以
己说阑入其间,托名于虚中之注鬼谷,以自神其术耳」
《四库全书总目》的说法,不是没有道理。由于书中议论
精切近理,多得星命之旨,和后世所述窈渺恍惚大为不同,所
。22 •
以《四库全书》在收录此书时按照晁公武《读书志》所载原目,
个为三卷。对于书中明显可以看出是后人依托的地方,则又
的加以按语,随文纠正,藉以提醒读者,不为所惑。
《玉照定真经》
《玉照定真经》一卷,旧文题为晋郭璞撰,张 顺注。
打开《晋书 • 郭璞传》,没有提到郭璞曾经撰有此书。此
后《隋书》、《唐书》、《宋史》等书的《艺文志》,以及其他诸家书
目,都没有提到过这本著述,只有叶氏所撰《竹堂书目》,载有
此书一册,但也闭口不提作者姓名。由于以上种种原因,大致
可以认为,此书出于后人依托之本。
关于为此书作注的张顺,也不知道是何人。从书中不只
一次出现的江南方言看,很可能本书的正文和注文,都出于张
顺一人之手,也未可知。
书中所述,虽然文句不够驯雅,可却简洁明晰,近切中理,
犹有珞琮子、李虚中命书遗意,如论年仪、月 仪、六害、三奇、三
交、四象之类,多所阐发,这是书中的可采之处。可是,当书中
推及到外亲、女婿等处,便就难免曲傀凝凿,牵强附会了。
《星命溯源》
《星命溯源》五卷,不知何人所编。
本书以五星推命的学说,托名为唐代张果老所传。关于
以星位尊卑推测禄命,从东汉以来已经有了苗子。当时王充
著有《论衡》一书,书中曾说「天施气而众星布精,天所施气而
众星之气在其中矣。人禀气而生,含气而长,得贵则贵,得贱
则贱。贵或秩有高下,富或宽有多少,皆星位尊卑大小之所授
也「唐朝之时,以五星宫推度休囚的说法开始渐渐流行。韩
愈心星行》说「我生之辰,月宿南斗,牛奋其角,箕张其口」
• 23 •
唐末杜牧自作墓志铭时也说「余生于角星昴毕,于角为第八
宫,曰 '病厄宫',亦曰 '八杀宫',土星在焉,火星继木星宫。杨
晞曰,'木在张,于角为第十一福德宫。木为福德,大君子救于
其旁,无虞也。'余曰:'自湖守不周岁迁舍人,木还福于角是
矣,土火死还于角,宜哉!
通观《星命溯源》全书,第一卷为《通元遗书》,杂录唐代张
果之说,计有三篇。第二卷为《果老问答》,称明代中都人李慢
在嘉靖二年(1523)碰上张果,口授此书。第三卷为《元妙经
解》,称张果撰,元代郑希诚注。第四卷为《观星要诀》,没有撰
人姓名,可能为郑希诚编。第五卷为《观星心传口诀补遗》,也
不知为谁人所撰,或者出于术士掇拾,以补郑希诚所未备,也
未可知。关于郑希诚其人,号沧洲,瑞安人,官至主簿,其他就
一无所知了。
从后世所传五星推命书看,《星命溯源》堪称鼻祖。在此
书的基础上,又有《果老星宗》传于世,此外另有《天宫五星术》
一书,但比起《果老星宗》,又另有一套说法。然而据理而论,
正气当以《果老星宗》为主,化气当以《天宫五星术》为要,彼此
可以互参。
《徐氏珞球子赋注》
《徐氏珞球子赋注》二卷,宋徐子平撰。
古今用年、月、日、时推度禄命之法,多认《珞球子三命赋》
为鼻祖。所谓“珞球”,有“球球如玉,珞珞如石”的意思。但对
于《珞球子三命赋》究竟出于何人手笔,至今是个疑团。有人
认为是周世子晋所撰,然而赋中却出现秦代河上公在后汉末
年悬壶化杖等事,可见这种说法是站不住脚的。又有人认为,
珞球子为南北朝时陶弘景的自称,但是从唐代李虚中也只以
• 24 •
年、月 、日来推算人禄命看,陶弘景时还没有出现八字学说。
考证结果,《珞 球子三命赋》最早见载于《宋史 • 艺文志》,
而晁公武《读书志》更是明确指出,直到南北宋之际,此赋;才渐
渐流行开来,那末赋的作者为北宋人,则基本可以肯定下来。
自从《珞珠子三命赋》产生以来,最早为赋作注的有徐子
平、王廷光、李仝、释昙莹四家。其中徐子平注本论运气向背,
金木刚柔得失,青赤父子相应,由于所说切近命理内核,所以
为后世术士所推重。
《珞球子三命消息赋注》
《珞球子三命消息赋注》二卷,宋释昙莹注。
昙莹号月萝,嘉兴人,以擅谈《易》理名噪一时,洪迈《容斋
随笔》把他称为“《易》僧”。 书中,昙莹撮合王廷光、李仝等人
注释,加上自己观点,所以其文实际兼涉王、李两家学说。不
足之处是书中所述,往往以命理附会《易》理,故而不及徐子平
撰注本来得明白切实。然而,书中所列王廷光推演命限,昙莹
自论孤虚等条,则又颇为精确,足有可采之处。
通览全书,上卷之中多为三家注案并载,下卷则以昙莹所
注为主,别有特色。
心命指迷赋》
《三命指迷赋》一卷,旧本题为宋岳珂补注。
史籍所载,岳珂著述虽多,但却没有提到过 心命指迷
赋》,打开《宋史 • 艺文志》,也不见这篇赋的著录。可是,从岳
珂所著《程史》看,书中记有岳珂和瞽者杨艮论韩傅胄禄命,以
及论幕官袁韶禄命一条,颇为详尽,可见岳珂原本精于禄命。
也正因为如此,所以其书出于术家附会,不是没有可能。
书中所论,专以徐子平学说为主,于夹马、夹禄、拱岸、拱
• 25 •
贵,辩论详尽,往往为其他书籍所未道及。其中如所论拱库,
尤为精晰。不足之处是专以月建和胎元作为推测之本,则又
未免失之偏颇了。
《渊海子平》
《渊海子平》五卷,宋代徐升编著。
我国古代探讨禄命的著述,到了《渊海子平》,可谓洋洋大
观,形成气候。后世所著,引用本书文字的时有可见,可知本
书在我国命理发展史上影响之大。
书中第一卷所载,主要为《论五行所生之始》、《论天干地
支所出》、《天干相合》、《十干所属方位及十二支所属论》、《论
六十花甲子纳音并注解》、《论天干生旺死绝》,以及各种神煞,
起大运、小运等有关命理的基本知识和要领,全都囊括其中。
若把它作为全书的基础篇来读,也未尝不可。
第二卷所载,主要为《继善篇》、《看命入式》,以及各种格
局等等。在《看命入式》条下,笔者详论正官、偏官、七杀、印
绶、正财、偏财、食神州官、伤官、劫财、羊刃等等,为后世星命
学家所取法。再如所言落局,有正官格、杂气财官格、月上偏
官格、时上偏财格、时上一位贵格、飞天禄马格、倒冲格、乙巳
鼠贵格、六乙鼠贵格、合禄格、子遥巳格、丑遥巳格、壬骑龙背
格、井栏叉格等,也可谓是洋洋大观,为前所未有。
第三卷则所涉较杂,在六亲方面,有《六亲总论》、《六亲捷
要歌》、《论父》、《论母》、《论妻室》、《论兄弟姐妹》、《论子息》、
《论妇人总诀》、《阴命赋》、《女命富贵贫贱篇》、《论小儿》等篇。
此外并旁及性情、疾病、大运、太岁吉凶、杂论口诀等等。卷末
并另有《心镜歌》、《妖祥赋》、《络绎赋》、《相心赋》、《玄机赋》、
《幽微赋》、《五行原理消息赋》等韵文,可供甄录。
• 26 •
第四卷所收主要为一些歌赋之类的韵文。主要有:《金玉
赋》、《碧渊赋》、《爱憎赋》、《万金赋》、《子下百章歌》、《四言独
步》、《五言独步》等等。此外,有些篇章虽然名为《造微论》、
《人鉴论》、《渊海集说》等,可是从形式上看,却也属于赋的一
类。
第五卷为全书的最后一卷,卷中所收,除《五行相克赋》、
《珞珠子消息赋》外,纯为七言诗诀如《正官诗诀》、《偏官诗
诀》、《印绶诗诀》、《正财诗诀》、《偏财诗诀》、《食神诗诀》、《伤
官诗诀》等等。
纵观全书,既有作者自己所撰,又多收有古来所传歌赋,
所以内容庞杂。然而也正因为作者抱着多多益善的观点编著
此书,所以许多有关命理的古诗古赋,赖此书而得以保存下
来。
《星命总括》
《星命总括》三卷,旧本题为辽耶律纯撰。
耶律氏为辽代贵仕,然而对于耶律纯其人,后世却一无所
知,或许因为这个缘故,所以许多人都认为这也是部后人托名
之作。
书中所述,剖析义理,往往造微,颇有参考价值。然而所
称宫有偏、正,则又未免故弄玄虚,出奇立异,流于乖谬,反至
窒塞了。
《星学大成》
《星学大成》十卷,明万民英撰。
万民英,字育吾,大宁都司人,嘉靖二十九年(公元 1550
年)进士,历官河南道监察御史,出为布政司右参议。
全书依次编排星学家言,中间插进注释论断,对于星家古
• 27 •
法,巨细不遗,可谓大备。
书中第一卷为《星曜图例》,第二卷为《观星节要宫度主用
十二位论》,第三卷为《诸家限例琴堂虚实》,第四卷为《耶律秘
诀》,第五卷到第七卷为《仙城望斗三辰通载》,第八卷为《总龟
紫府珍藏星经杂著》,第九卷为《碧玉真经邓史乔庙》?第十卷
为《光福渊微显曜格局》。
然而,术家为了事事皆验,就各出心裁,希望能够提高吉
凶的预测率,可是天下哪有这等好事,所以其法愈多而愈加不
能测中,结果其术往往流于荒谬难信,离五行生克之理越来越
远了。但是,万民英对于这些却全然不顾,一概收入书中,致
使全书有芳有秽,披阅为难。
《三命通会》
《三命通会》十二卷,不署撰人姓名,卷首只题育吾山人。
据查《明史 • 艺文志》所列书目载有万民育《三命会通》十
二卷,和《三命通会》的卷数一样,只是“通会”和“会通”有所不
同。按照《星学大成》作者万民英,字育吾推测,很可能和育吾
山人就是同一个人了。假若果真如此,则此书的作者,便应题
为万民英了。
由于书中阐述子平遗法,对于官、印、财、禄、食、伤的名
义,用神的轻重,各种神煞所牵涉到的吉凶,都能采撮群言,得
其精要,所以当时几乎家家备有此书,并且从此以后,研究星
命的,并多把此书看成为是总汇之作。
不过,书中也自有其不足之处,正如《四库全书总目》所说
的那样「至其立论多取正官、正印、正财,而不知偏官、偏印、
偏财亦能得力;知食神之能吐秀,而不知伤官之亦可出奇。是
则其偏执之见,未为圆彻。且胎元等论,施之今日,亦多有不
• 28 •
验。言命学者当得其大意,而变通之可矣」
《星平会海》
《星平会海》十卷,不署撰人姓名。
黄虞稷《千顷堂书目》载有《星平会海》书名,前有自题,称
“武当山玉虚宫三逢甲子日金山人”编集,要是果真甲子三逢
的话,那末其人撰集此书当时已经 180 多岁,可见不可凭信。
书为明代人所撰,书中既论五星命理,又论子平命理,内
容有《五星起例》,《周天七政四余行给,《七政四余变曜贵贱
格分野图》,《七政四余入庙乘旺好乐宫歌诀》,《五星忌宫歌
诀》,《五星解神歌诀》,《命宫踱度几岁出童限行大限过宫量天
总尺》,《先看三星》,《最 紧四事》,《步天经诀》,《星曜疆度歌》,
《星曜入宫歌》,《五星变局》,《入门四十四看法》,《二十四要
法》,《步天惊句》,《琴堂总诀》,《琴堂指金歌》,《玉衡经凡《张
果老先天口诀》,《后天口诀》,《看五星捷诀》,《命理何知经》,
《琴堂五星经》,《琴堂指金赋》等好几十篇。
书中子平命理部分,除《起八字例》,《六亲论》,《子平举要
歌》外,尚有《天元秀气巫咸经》,《爱格赋》,《万金赋》,《金玉
赋》,《一行禅师天元赋》,《玉照神应真经》,《兰台妙选》,《取格
指诀》,以及《命理正格》、《命理杂格》等等。
全书所采不只一家,虽内容丰富,但也多有谬误之处,比
如加盘、乔庙等法,推衍家如果按照书中所述去推,往往大多
推算错误,闹出笑话。
《命理约言》
《命理约言》四卷,清陈素庵著。
书中第一卷为法,第二卷为赋,第三卷为论,第四卷为杂
论。民国年间,全书经过当时命理学家浙江嘉兴人韦千里的
29 •
校刊删注,名为《精选命理约言》,于是就渐渐通行开来了。
《精选命理约言》第一卷计有《看命总法》、《看格局法》,
《看用神法》、《看生年法》、《看月令法》、《看日主法》、《看生时
法》、《看运法》、《看 流年法》、《看六亲法》、《看贵贱法》、《看贫
富法》、《看吉凶法》、《看寿夭法》、《看科第法》、《看性情法》、
《看疾病法》、《看女命法》、《看小儿命法》等 48 篇。
第二卷为《总纲赋》、《格局赋》、《行运赋》、《流年赋》、《正
官赋》、《偏官赋》、《正印赋》、《偏 印赋》、《正财赋》、《偏财赋》、
《食神赋》、《伤官赋》、《比劫赋》、《禄刃赋》、《从局赋》、《化局
赋》、《一行得气赋》、《两神成象赋》、《暗冲暗合赋》、《女命赋》
等 20 篇赋文。韦千里并说「赋二十篇,乃论命之精华,余略
加诠注J
第三卷为论,计有《天干论》、《地支论》、《干合论》、《干冲
论》、《支三合论》、《支六合论》、《支方论》、《支冲论》、《支刑
论》、《支害论》、《五行旺相休囚论》、《十二支作用论》、《干支覆
载论》、《诸神煞论》、《太岁论》等 48 篇,可谓洋洋洒洒。
第四卷主要为杂论 24 则,并附张神峰《辟五行诸谬论》11
贝h总计 35 则,以作为全书的补充。
书中韦千里的识文引清渠先生的话说:“是书湮没人世,
垂三百年,今竟赖君(指韦千里)毅力,得以公诸天下,使命学
日进昌明,则人人知命,人人守份,上无战争之害,下无攘夺之
虞,其功不亦大哉
《滴天髓阐微》
《滴天髓阐微》四卷,清任铁樵撰注。
我国古代命学著作,京图《滴天髓》是一篇纯以五行盛衰,
生克理气推断命运顺逆,带有较多思辨哲学的名著。但不足
30 •
之处是通篇采用赋的形式,言简意赅,理解为难。为此,历来
为此书作注的不乏其人。最早为此篇作注的是明代开国重臣
刘基所撰《滴天髓原注》,而任铁樵所撰注的《滴天髓阐微》则
又在《滴天髓原注》的基础上,逐句逐段加进自己的心得体会。
可贵的是,任氏还在必要之处,加进大量平时推命实例,予以
阐微抉幽,所以一编在手,对于研究此道者当大有裨益。真如
后来袁树珊为本书影印抄本作序时所说那样:“翌日,亦君偶
以精抄本任铁樵先生增注之《滴天髓阐微》见示,余披阅之再,
知其以古本《滴天髓》正文为纲,古注为目,古往外复增新注,
阐发要旨,并于逐条排列命造,以资佐证。学宗陈沈,笔有炉
锤,理必求精,语无泛设,诚命学中罕见之孤本也
为什么袁树珊要称此书为“命学中罕见之孤本”呢? 原来
本书原刻早就希如星凤,当时海宁陈家曾经有所收藏。后来
有个自号观复居士的学者从陈氏家里把书借出,并且花了好
长时间,把书抄了下来。不久,原书归还到陈家后,因陈家不
幸发生火灾而把此书焚为灰烬。从此以后,这一抄本便就放
了海内孤本,弥可宝贵。要是当时陈家秘藏不肯示人,或者虽
然示人而没有观复居士勤为抄录,那末这本撰著的命文使
可想而知了。
关于此书的价值,袁树珊认为:“其(指任铁樵)论五行
克衰旺颠倒之理,固极玄妙,而尤以旺者宜克,旺极宜冷 ,F 之
宜生,弱极宜克二条最为精湛。至云人有厚薄,山川不同,命
有贵贱,世德悬殊,此又以天命而合地理人事言也。故其为人
论命,尝曰某造纯粹中和,太平宰相;某造仕路清高,才华卓
越I某造经营获利,勤俭成功;某造背井离乡,润身富屋:某造
贪婪无厌,性情乖张I 某造挥金如土,破家亡身; 某造不宁生
• 31 •
产,必有后灾;某造出身贫寒,为人贤淑;某造青年守节,教子
成名;某造爱富嫌贫,背夫弃子;某造若不急流勇退,能无意外
风波?某造蒲柳望秋而即,松柏经霜弥茂。衮褒斧贬,莫不各
具苦心,大义微言,要皆有关世道。古之君子所谓既没而言立
者,其在斯人乎
《穷通宝鉴》
《穷通宝鉴》又名《拦江网》,原书二卷,不知出于何人手
笔。
书中卷首先列《五行总论》、《四书子评》、《月 谈赋》作为总
纲。然后分别在第一卷中详论甲木、乙木、丙火、丁火在一年
四季,十二个月中的盛衰喜忌;第二卷中详论戊土、己土 、庚
金、辛金、壬水、癸水在一年四季,十二个月中的衰旺喜忌。III
于本书专论五行在四时十二个月的旺相休囚,所以阐发深细,
别具一格。
后来,命理学家论定四时五行的衰旺喜忌,多以此书为
宗。可见泛泛命理,伤其十指,倒反而不及本书断其一指来得
深入有得。 - 鹏
- 《命理探原》
《命理探原》共八卷,民国年间袁树珊撰。
全书前五卷为四时五行,星宿神煞,以及六亲、大运、流年
等命学基本理诒。
第六卷为《先贤名论》,卷中所采依次为明《子平源流考》
(明万育吾),《明通赋》(唐徐子平〉,《元理赋》(宋徐大升),《气
象篇》(明醉醒子3《五行生克赋》(明《星平会海》),《六神篇》
(明张神峰),《碧渊赋》,《玄机赋》(《渊海子平》),《形象篇》、
《八格论》、《体用论》、《清浊论》、《真假论》、《寒暖燥湿论》(京
• 32 •
图撰,明刘基注),《盖头论》(明张神峰),《论阴阳生克》、《论十
干有得时不旺失时不弱》、《论外格用舍》、《论时说拘泥格局》、
《论杂格》(清沈孝瞻)等。
第七卷为《泗德堂存稿》,卷中排列笔者平时所批的 30 几
个命例,目的是为了“便于初学”,有个借鉴。
第八卷分《星家十要》、《星命事实丛谈》两个部分。在《星
家十要》中,作者仿效清代名医张璐玉《医家十戒》之例,开列
“学问”、"常变”J言语”、“教品”、“廉 洁”、“劝勉”、“警励”、“治
生”、“济贫”、“节义”等十要,目的不外崇尚当年司马季主对臣
说忠,对子说孝的宗旨。其中对于“学问”、“廉洁”两条,更为
重耍。《星命事实丛谈》则采集古今名论,旁及诗文,作者的用
意在于使学者不仅仅局限于星命的一隅之见,更重要的是要
“明道救世”,这才是最根本的。
读大著
第八
通观
《杂
卷之后
全书,
说门》(指
,尚有
丹徒
第五
《补
刘恒瑞
卷),
遗事实
在跋语
多发
丛谈》
中曾提
前人所
,作为
出这样
未发。
全书的
如子
的评介
尾声。
时分别
「今
~
— 前后二日取用,合婚须培补男女用神等论,发明新理,信而有
征,与鄙见实不谋而合,足坚余志。又起例论行运扣足年月日
时,及六亲总论与评判大运,须运岁宫限合断吉凶诸说,皆为
古人不传之秘。”比如子时分别前后二日取用,推命者如果不
知这一原理,往往可使时间差讹一天。如若正好碰上立春,那
么还将差讹年份,闹出笑话。书中,袁树珊这样举例说明道:
“假如甲寅年正月初十日辛酉夜子时立春,其人是年正月初十
日下午九点钟后,十一点前亥时生,即作癸丑年、乙丑月、辛酉
日、己亥时推。如在初十日下午十一点钟后,十二点钟前夜子
时生,即作甲寅年、丙寅月、辛酉日、庚子时推(用壬日起庚子
• 33 •
时),所谓‘今日之夜,非今日之早也'。如在初十日下午十二
点钟后,一点钟前子时正生,即作甲寅年、丙寅月、壬戌日、庚
子时推,所谓‘今日之早,非昨日之晚也 若夫推行运之零
借,命宫之过气,尤当如此。”
除以上各书之外,如元朝李钦夫《子平三命渊源注》,明张
神峰《神峰通考命理真宗》,清沈孝瞻《子平真诠》,近代韦千里
《八字提要》、《新命学讲义》,徐乐吾《命理入门》、《命理寻源》,
以及港台多种命理典籍等,也都各擅胜场,这里就无法逐一细
述了。