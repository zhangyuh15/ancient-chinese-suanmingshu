\chapter{算命术的起源}
\section{天命观的产生和先秦诸子的天命观}
上古时期,在人类生产力极不发达,认识水平极度低下的情况下,由于不能解释神秘的自然现象和完全把握自己的穷通寿天,于是就萌发了一种似乎上天有着一种不可抗拒的神秘的力,在支配着世上一切的一切,安排着人间一切的一切的思想。这种思想观点,就是进入阶级社会后古文献上记载的“天”能致命于人,决定人类命运的天命观,或者说是命运观。这用《易经》的话来说,叫做“乾道变化,各正性命”。对于这里的命,后人注释道,“命者,人所禀受,若贵贱夭寿之属也。”


从出土的甲骨卜辞、彝器铭文看,“受命于天”刻辞的不只一次出现,说明早在殷周时期,天命观就已经在一些人的头脑里扎根了。后来的儒家祖师爷孔子,就是位极度信命的老夫子。按理说,孔子是个知识渊博的大儒,对于人类社会有着深刻的认识,怎么就会信起命来呢?原来,他早年风尘仆仆,奔走列国,到处推销自己的政治主张,很想干一番轰轰烈烈的事业。可是到了后来,当他碰了一鼻子灰以后,才深深地省悟到,命运之神竟是如此这般的厉害,然而这时他已是个五十左右的人了。“五十而知天命”,就是他从不知命到知命这一思想转化过程的最好说明。与此同时,他不仅知命”,并且还是个怕命的人,不然,他就不会说出“畏天命”这样的话来。有意味的是,除了自己“知命”、“畏天命”,他和他弟子还不遗余力大肆宜扬“死生有命,富贵在天”,“不知命,无以为君子”,“君子居易以俟命,小人行险以侥幸”的思想。这里他的说教是,一个人的生死贫富,都是命里早就注定了的,作为一个君子来说,非得“知命”不可,否则就够不上做“君子”的资格。正因为君子是“知命”的,所以他能安分守已,服从老天爷的安排,但是小人却不这样,他们不肯听从天命,往往冒险强求,希望有幸能得个好结果。


当然,看问题也不能攻其一点,不及其余。《孔子集语》记录孔子的话说:“古圣人君子博学深谋不遇时者众矣,岂独丘(我孔丘)哉!贤不肖者才也,为不为者人也,遇不遇者时也,死生者命也。”这里,他认为贤和不肖是根据才华来划分的,干和不干是人们自己可以把握的,至于机遇好和不好,是死还是活,那就只得看时运和老天的旨意了。可见他在主张服从天命的同时,又是主张发挥人的主观能动性的。因为人的才华和努力是一回事,境遇和死生的命运安排毕竟又是一回事。

作为儒家学派的创始人,孔子的这种天命思想,又在后来
大儒孟子身上得到了新的反映。《孟子 •万章》上篇说:“莫之
为而为者,天也;莫之致而至者,命也。”意思就是,没有人叫他
干,而他竟干了,这就是天意;没有人叫他来,而他竟来了,就
是命运。同时他还举例说明,尧、舜的儿子都不肖,是因为舜,
禹为相的时间太长,所以尧、舜的儿子不有天下I 禹的儿子启

贤能,而益为相的时间又短,所以启能得到天下。以上这些,
都不是人力所为而自为,不是人力所致而自至。从理来说,这
属于天意,对人来说,这属于命运。天和命,实在是一致的。在
《孟子 •尽心上》中,孟子还说J夭寿不贰,修身以俟之,所以
立命也。”又说「莫非命也,顺受其正。是故知命者不立乎岩
墙之下。尽其道而死者,正命也;桎梏死者,非正命也J 前者
是说,不管命短命长,我都不三心两意,只是培养身心,等待天
命,这就是安身立命的方法。后者是说,天底下人的吉凶祸
下福,无一不是命运,只要顺理而行,接着的就是正命。所以懂
得命运的人不站立在有倾倒危险的墙壁下面。因此,尽力行
道而死的人所受的是天的正命,犯罪而死的人所受的不是天
的正命。这里,孟子虽然认为天命的力量无可抗拒,但是不管
怎样,我还是应该按照我的仁义而行,不能无缘无故地白白送
死。无疑,这对孔子的天命观来说,有着补充的一面。
那个御风而行,洒脱自在得很的列子,也是个坚信命运的
人。在《列子 •力命篇》里,他巧妙地通过“力”和“命”的对话,
宣传了自己的这一思想。力对命说「你的功劳怎么比得上我
呢?”命回答说「你对世间事物有什么了不起的功劳,竟想和
我老子比起高低来?”力说,“一个人的寿夭穷通,贵贱贫富,正
是我力所能及的6” 命听了随即反驳道,“彭祖的智慧不及尧
舜,而活了八百岁。颜回的才华超出众人,可寿只三十二岁。
孔老夫子的德不比诸侯来得差,却在陈、蔡等国遭了难。商纣
王的行够不上仁人,但坐上了帝王的宝座。吴国贤公子季札
没能在吴国做上官,坏蛋田恒却篡了齐国的政。商朝忠臣叔
齐和伯夷饿死在首阳山,鲁国权臣季氏的钱又远远超过了贤
士展禽。假如说你力能发挥作用,又为什么要让彭祖长寿颜

回夭折,孔子困厄纣王登基,季札低贱田恒高升,伯夷、叔齐、
展禽贫困而季氏富贵呢?”力被命一驳,楞了好一会儿,才开口
道「按照你的说法,我虽然对世间事物没有功劳,可社会上搞
得这般模样,你又为什么不出来制止呢?”命从容回答说,“世
界上的一切事物都得听凭他们自己去变化,长寿的长寿,夭折
的夭折,困厄的困厄,通达的通达,贵的贵,贱的贱,富的富,穷
的穷,又难道是我所能够改变的吗?”
通过这段风趣的对话,列子表达了这样一种思想,就是世
界上种种不合情理的事,都不是人力所能够解决的,因为在命
运安排面前,人力原是很渺小有限的。
此外,先秦诸子信命的还很多,而以儒家的势力为最大。
这种思想发展到汉代,随着儒家学说的风行天下,儒家天命观
的思想,就更深入人心了。《河图纪命符》说:“天地有司过之
神,随人所犯轻重,以夺其算纪。恶事大者夺纪(一年),过小
者夺算(一日)。随所犯轻重,所夺有多少也。人受命,得寿,
自本有数。数本多者,纪算难尽,故死迟,若所禀本数以少而
所犯多者,则纪算数尽而死早也 淮南王刘安认为,“仁鄙在
时不在行,利害在命不在智”,扬雄《法言》也说:“遇不遇,命
也「有人向他问命,他说「命是上天决定的,不是人为的。人
为的称不上命,上天决定的命是逃避不了的。”
王充是东汉著名学者,他不信鬼神,一生反对迷信,可是
对于命运,却坚决主张是客观存在着的。他说:“凡人遇偶(碰
上好运〉及遭累害(遭受灾祸),皆由命也。有死生寿夭之命,
亦有贵贱贫富之命。”在他所著《论衡》一书的《命禄》、《气寿》、
《幸偶》、《命义》、《无形》、《偶会》、《初禀》等篇章里,随时都可
碰上对于命运的论述。可见,人们对于命运的信仰,不只是后

来算命家们故弄玄虚所捣的鬼,这里面还有一批大学问家的
参与、坚信和鼓吹,可谓盘根错节,阵容坚强。
在这种社会历史背景下,加上人们对种种不合理的社会
现象如不该飞黄腾达的反而飞黄腾达,不该久处人下的反而
久处人下,不该富贵荣华的反而富贵荣华,不该穷困潦倒的反
而穷困潦倒,以及自己难以把握自己的未来、难以把握自己的
生活、难以把握自己的生命等种种困惑,得不到使人信服的合
理解释,因此便就很自然地信奉起那些大知识分子的说教来。
而统治阶级为了坐稳他们的交椅,也乐于接受这一套,并且不
择手段地加以推广利用。从此以后,那些时运不济的达人,常
常会用“比上不足,比下有余”,“乐天知命”的话来安慰自己,
安于现状;而那些想不通的,除了怨天尤人之外,只好自讨不
服命的苦吃了。

\section{算命术的起源、发展和成熟}
天命观经过先秦学者的一阵鼓吹,其时从上到下,从统治
者到平民百姓,信命的风气一时很盛。早在殷商时期,迷信的
统治者们,就已习惯于在每做一件事之前,总重先占卜一下天
意如何,是凶是吉?后来,又由于人与天地相应观念的影响,
更使得人们普遍认为,整个天下的命运和每个个人的命运,都
和天时星象有关。《周礼 •春官》记载:“冯相氏掌十有二岁,
十有二月,十有二辰,十日,二十有八星之位,辨其叙事,以会
天位。”“保章氏掌天星,以志星辰日月之变动,以观天下之迁,

