\chapter{算命术的起源}
\section{天命观的产生和先秦诸子的天命观}
上古时期,在人类生产力极不发达,认识水平极度低下的情况下,由于不能解释神秘的自然现象和完全把握自己的穷通寿天,于是就萌发了一种似乎上天有着一种不可抗拒的神秘的力,在支配着世上一切的一切,安排着人间一切的一切的思想。这种思想观点,就是进入阶级社会后古文献上记载的“天”能致命于人,决定人类命运的天命观,或者说是命运观。这用《易经》的话来说,叫做“乾道变化,各正性命”。对于这里的命,后人注释道,“命者,人所禀受,若贵贱夭寿之属也。”


从出土的甲骨卜辞、彝器铭文看,“受命于天”刻辞的不只一次出现,说明早在殷周时期,天命观就已经在一些人的头脑里扎根了。后来的儒家祖师爷孔子,就是位极度信命的老夫子。按理说,孔子是个知识渊博的大儒,对于人类社会有着深刻的认识,怎么就会信起命来呢?原来,他早年风尘仆仆,奔走列国,到处推销自己的政治主张,很想干一番轰轰烈烈的事业。可是到了后来,当他碰了一鼻子灰以后,才深深地省悟到,命运之神竟是如此这般的厉害,然而这时他已是个五十左右的人了。“五十而知天命”,就是他从不知命到知命这一思想转化过程的最好说明。与此同时,他不仅知命”,并且还是个怕命的人,不然,他就不会说出“畏天命”这样的话来。有意味的是,除了自己“知命”、“畏天命”,他和他弟子还不遗余力大肆宜扬“死生有命,富贵在天”,“不知命,无以为君子”,“君子居易以俟命,小人行险以侥幸”的思想。这里他的说教是,一个人的生死贫富,都是命里早就注定了的,作为一个君子来说,非得“知命”不可,否则就够不上做“君子”的资格。正因为君子是“知命”的,所以他能安分守已,服从老天爷的安排,但是小人却不这样,他们不肯听从天命,往往冒险强求,希望有幸能得个好结果。


当然,看问题也不能攻其一点,不及其余。《孔子集语》记录孔子的话说:“古圣人君子博学深谋不遇时者众矣,岂独丘(我孔丘)哉!贤不肖者才也,为不为者人也,遇不遇者时也,死生者命也。”这里,他认为贤和不肖是根据才华来划分的,干和不干是人们自己可以把握的,至于机遇好和不好,是死还是活,那就只得看时运和老天的旨意了。可见他在主张服从天命的同时,又是主张发挥人的主观能动性的。因为人的才华和努力是一回事,境遇和死生的命运安排毕竟又是一回事。