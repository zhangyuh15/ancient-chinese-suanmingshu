\chapter{四柱算命的具体方法}
\section{怎 样 排 八 字}
命理学家看命,先要排出一个人出生年、月、日、时的干
支。年、月 、日、时加起来共四项,徐子平称为“四柱”,每柱
个天干一个地支,共八个字,所以叫做“八字”。算命排出八字,
是最事关重要的。八字排出以后,再根据八字之间五行生克
等千变万化的关系,从而推论一个人一生的吉凶祸福。如果排
不出八字,那就一切都落空了。
那末,怎样排出年、月、日、时四柱的八字呢?
1. 推年法 算命根据的是农历,农历哪一年出生的,那
么这一年的干支就是本人年柱上的干支。比如庚辰龙年出生
的,他年柱的干支就是庚辰,辛巳蛇年出生的,他年柱的干支
就是辛巳,其他类推。
推算年柱的方法大致有三种: 查看万年历,比如你只
知道是公元 1940 年生,而不知道这一年农历叫什么年,那么
打开《新编万年历》一看,就知道是庚辰年了。 《万年
历》的,可以自己排一张六十花甲表,然后根据当年的干支和
• 76 •
自己虚臂反推上去,也可得出结论。 是用手指推算,顺
推反推,这在瞎子中用得最为纯熟。
这里我们且看 1941 年到 2000 年的六十花甲表,便就了
解推年法的大概了。
1941 年,辛巳;
1942 年,壬午;
1943 年,癸未;
1944 年,甲申;
1945 年,乙酉;
1946 年,丙戌;
1947 年,丁亥;
1948 年,戊子;
1949 年,己丑;
1950 年,庚寅,
1951 年,辛卯;
1952 年,壬辰;
1953 年,癸巳;
1954 年,甲午;
1955 年,乙未;
1956 年,丙申;
1957 年,丁酉;
1958 年,戊戌;
1959 年,己亥;
1960 年,庚子;
1961 年,辛丑;
1962 年,壬寅;
. 77 •
1963 年,癸卯;
1964 年,甲辰;
1965 年,乙巳;
1966 年,丙午;
1967 年,丁未;
1968 年,戊申;
1969 年,己酉;
1970 年,庚戌;
1971 年,辛亥;
© 1972 年,壬子;
@ 1973 年,癸丑;
1974 年,甲寅;
© 1975 年,乙卯;
@ 1976 年,丙辰;
© 1977 年,丁巳;
® 1978 年,戊午;
@ 1979 年,己未;
@ 1980 年,庚申;
@ 1981 年,辛酉;
1982 年,壬戌;
© 1983 年,癸亥;
@ 1984 年,甲子;
© 1985 年,乙丑;
© 1986 年,丙寅;
© 1987 年,丁卯;
© 1988 年,戊辰;
• 78 •
1989 年,己巳;
1990 年,庚午;
. 1991 年,辛未;
1992 年,壬申;
@ 1993 年,癸酉,
额 1994 年,甲戌;
1995 年,乙亥;
1996 年,丙子;
励 1997 年,丁丑;
1998 年,戊寅;
1999 年,己卯;
2000 年,庚辰。
推年不管用什么方法,都必须严格划定以农历的立春作
为一年的界限。比如正月立春后生的,用本年干支,虽然生在
正月,可时间却在立春之前(还没到立春),那末就得算到上年
出生,而用上一年的干支作为年柱。同样道理,虽然同是出生
在农历十二月的,可是立春前生的用本年干支,立春后生的就
要划到下一年去了。
2. 推月法 推月的办法虽然每月的地支是固定的,如前
文《天干地支》所说正月寅月,二月卯月,三月辰月,四月E月
便是,可是月份的天干却不固定,要经过一定的推算才能排
出。推算的歌诀是:
甲己之年丙作首,乙庚之岁成为头,
丙辛必定寻庚起,丁壬壬位顺行流,
更有戊癸何方觅,甲寅之上好追求。
具体办法是,如农历甲午年四月生的,那末先根据歌诀“甲己
• 79 •
之年丙作首"推出作首的正月丙寅月,然后再顺次推出二月丁
卯,三月戊辰,四月己巳,可知这一年四月的干支就是己巳了。
现把歌诀化成简表,以便查阅(见 80 页上表)。
从表中,我们可以看出月份的干支也和年份一样,从丙寅
月起六十甲子周转下来,重新又回到丙寅月,这时已经是五年
过去了。因为按照五年十二个月算,六十甲子周转下来,不就
正好是五年吗?

【80 页表】

这里必须注意的是,一定要注意结合节气来推算月份。在
一年二十四个节营里;立春、惊蛰、清明、立夏、芒种、小暑、立
秋、白露、寒露、立冬、大雪、小寒是节,雨水、春分、蓉雨、小满、
夏至、大暑、处暑、秋分、霜降、小雪、冬至、大寒是气。而推月
则严格以节作为界限。如在本月节前生的,就用上个月的干
支,本月下一个节后生的,也就是下一个月的节提前来到本
月,就得用下个月的干支。因为在通常情况下,一个月只有一
个节和一个气,然而有时又有打乱了的。比如公元 1986 年丙
寅年正月廿六出生,查《新编万年历》,这一天正好是下个月的
惊蛰节提前来到,那就算不得是庚寅正月出生,而要算到下月
辛卯二月出生了。现将二十四节气和月份的分配情况,列表
• 80 ・
如下:

这就是说,命书中十二个月的划定应该是这样的:
〔正月寅月〕立春经雨水到交惊蛰为止。
〔二月卯月:)惊蛰经春分到交清明为止。
〔三月辰月〕清明经谷雨到交立夏为止。
〔四月巳月〕立夏经小满到交芒种为止。
〔五月午月〕芒种经夏至到交小暑为止。
〔六月未月1小暑经大暑到交立秋为止。
〔七月申月〕立秋经抽暑到交白露为止。
〔八月酉月〕白露经秋分到交寒露为止。
〔九月戌月〕寒露经霜降到交立冬为止。
〔十月亥月〕立冬经小雪到交大雪为止。
〔十一月子月〕大雪经冬至到交小寒为止。
〔十二月丑月〕小寒经大寒到交立春为止。
为了便于记忆,有二十四节气歌一首道:
正月立春雨水节,二月 惊蛰及春分,
三月 清明并谷雨,四月立夏小满方,
五月芒种与夏至,六月小暑大暑当,
七月立秋兼处暑,八月白露秋分忙,
九月寒露还霜降,十月 立冬小雪张,
• 81 •
子月大雪共冬至,腊月小寒大寒昌。
3. 推日法 如果手头有本万年历,推日的办法比较筒
单,只要一查一推,就可知道每天的具体干支了。比如科学普
及出版社出版的《新编万年历》,就把公元 1840 年庚子到公元
2000 年庚辰一百六十年间农历每月初一、十一、二十一的干
支写得明明白白,使用起来只要根据天干地支的顺序一推便
知。如以民国二十九年庚辰年九月初十为例,现在查知这一年
的九月十一日是丁亥日,那末倒推一天便可得知初十那天就
是丙戌日了。
4. 推时法 推时法和推月法一样,要有个周折。因为时
柱下面的一个地支是已知的,即由要求算命的人自报,即使他
报的是现代的钟点,也无关紧要。但时柱上面的天干,就要费
一番踌躇了。好在推得的办法,也就是遁得的办法,并不复
杂,只要知道出生一天的干支,就可根据口诀求得了。歌曰:
甲己还生甲,乙庚丙作初,
丙辛从戊起,丁壬庚子居,
成癸何方发?壬子是真途。
这就是说,天干是甲、己天出生的,那末他出生时间便从甲子
(半夜 23 点 1点钟)开始推算。如仍以刚才九月初十生人为
例,现知道此人是辰时生,而九月初十是丙戌,那末根据歌诀
“丙辛从戊起”,也就是从戊子起开始推算,依次是己丑、庚寅、
辛卯、壬辰。这样推算下来,可以得知那人生时的干支,就是
壬辰了。
为了便于查阅,现将根据出生日天干推时的口诀,化成表
格(见第 84 页)。
根据表格,如乙、庚等日辰时(7—9时以前)生的,我们便
• 82 •
可查得时柱的干支是庚辰;丁、壬等日亥时生的,同样可以查
得时柱的干支是辛亥。
关键还是对于子时的推算。前人曾说「子正者,今日之
早,非昨日之晚也;夜子者,今日之夜,非今日之早也。”这是
说,“子正”为凌晨 0 点到 1点结束,属于“今日之早,非昨日之
晚”「夜子”为晚上 11 点到 12 点结束,属于今日之夜,非今日
之早”。由此可见,“今日之早”和“今日之夜”是有分别的。
此话怎讲?回答是在推排子时时,如果为“子正”,也就是
“今日之早”的,那末日柱、时柱都以今天作为标准推排;如果
为“夜子”,也就是“今日之夜”的,那末日柱则仍以今天作为标
准推排,而时柱则要以明天的日柱作为基准进行推排了。
举个例说,有两个人,一人出生在 1928 年农历二月二十
日凌晨 0 点 40 分,另一人出生在 1928 年农历二月二十日晚
11点 25 分,那末两人虽属同天子时出生,可是一个出生在“今
日之早”的“子正”,一个却出生在“今日之夜”的“夜子”,因此
有关时柱的演算,自然便就有所不同。
推演的结果,前者的四柱八字当为戊辰年,乙卯月,庚戌
日,丙子时;后者的四柱八字则当为戊辰年,乙卯月,由戌日,
戊子时了。
有关子时的这种推法,近代命理学家袁树珊《命理探源》
第五卷有《论时刻及夜子时与子时正不同》篇加以专门的论
述。文中袁氏还从另一角度举例说:“假如甲寅年正月初十日
辛酉夜子时立春,其人是年正月初十日下午九点钟后,十一点
斜前亥时生,即作癸丑年、乙丑月、辛酉日、己亥时推;如在初
十日下午十一点钟后,十二点钟前夜子时生,即作甲寅年、丙
寅月、辛酉日、庚子时推(用壬日起庚子时),所谓 '今日之夜,
• 83 *
非今日之早也'。如在初十日下午十二点钟后,一点钟前子时
正生,即作甲寅年、丙寅月、壬戌日、庚子时推,所谓'今日之
早,非昨日之晚也 可见如果不懂这一原理,弄得不好,还会
导致一年的差讹。
这里还有一个问题,就是有些人不知道自己出生的时辰,
或虽知也模模糊糊,记不真切,该怎么办?无可奈何之中,可
用这样一个办法进行约略的推测:
寅、申、巳、亥时出生,男三、六、九胎,女一、四、七胎,
头螺偏右,侧睡。
子、午、卯、酉时出生,男一、四、七胎,女二、五、八胎,
头螺正中或偏左,仰睡。
辰、戌、丑、未时出生,男二、五、八胎,女三、六、九胎,
有两个头螺,侧睡。
以上也只是仅供参考而已,尤其是第几胎生,更为不准。
眼下大陆除了那些超生游击队外,多半只生一胎,最多二胎,
用来匡算推测,往往豁边。
现在根据八字中年柱、月柱、日柱、时柱的推排方法,试综

【84 页表】

合推算一下民国二十九年九月初十辰时出生的“四柱”。首先
我们可以按照万年历查得,民国二十九年就是夏历的庚辰年,
可见这人的年柱是“庚辰”两字。接着根据推月法的口诀或表
格,查得九月戌月的干支属于“丙戌”。因为九月初十的这一天
处在寒露和霜降之间,所以属于的的确确的九月“丙戌”无疑。
再接下来是排日柱,翻开《新编万年历》公元 1940 年(民国二
十九年)庚辰年九月十一日的干支是丁亥,现在初十倒推一
位,便可查知属于丙戌。末了再根据出生“丙戌”日的天干,推
出辰时的干支属于“壬辰”。这样,庚辰年、丙戌月、丙戌日、壬
辰时的四柱八字就被推排出来了。过去,命理学家写“四柱”
的八字,总是从1覆雅右到左竖着写过来的,现在我们不
妨写成这样:
(年柱) 庚辰
(月柱) 丙戌
(日柱) 丙戌
(时柱) 壬辰
八字排出来后,我们便可走下一步棋了。

\section{推算大运、小运、流年和命宫}
了解八字的推算方法以后,接下来还得交待一下大运、小
运、流年和命宫推法。
所谓大运,就是一生中哪一阶段走运,哪一阶段不走运的
• 85 •
意思。因此按照命理学家的说法,“命运”两个字除了经常合
在一起解释外,还可拆开来解释。其中“命”管人的一生,主要
体现在八字里,“运”管人一生中的各个阶段,主要体现在由八
字基础上推算出来的大运里。按照前人说法,对于一个人来
说,运是很重要的:有的人八字虽生得好,可就是一直不走
运,在衣食无愁中碌碌无为地过一辈子;有的人八字虽然生得
一般,甚至还有破缺,经常处于逆境之中,可就是碰上那么一
二次大运,从而干出一番出人头地的事业。当然此外还有命
好运好,命坏运坏,或先好后坏,先坏后好等种种不同,情况千
变万化,难以悉举。
那末,怎样排算大运的起运岁数呢?唯一的依据就是,假
如是天干逢甲、丙、戊、庚、王等阳年生的男性,或者天干逢乙、
丁、己、辛、癸等阴年生的女性,从本人生日的那天起顺数,到
下一个节止,以三天为一岁;反过来,假如是阳年生的女性或
阴年生的男性,就要反其道而行之,从本人生日的那天起,逆
数到上一个节止,也是三天为一岁。多下来的一天抵四个月,
一个时辰抵十天。这里要注意的是,在农历二十四个节气里,
只有立春、惊蛰、清明、立夏、芒种、小暑、立秋、白露、寒露、立
冬、大雪、小寒等十二个才能算作节,其余的只能称为气。这
就是说,每个月中只有一个节和一个气,一年十二个月合起
来,就成了二十四个节气。
举个例说,假如是公元 1943 年农历癸未年八月二十日生
的女命,按照规定起运的岁数应该从生日那天起顺数到下一
个节。查《新编万年历》,癸未年的八月初九是白露,九月十一
日是寒露,现在八月二十日处在白露和寒露二个节之间,倒数
是白露节,顺数是寒露节,可知顺数的下一个节应该是寒露
• 86 ・
节。因为那一年的八月是小月,从八月二十日顺数到九月十
一寒露节的天数是二十天。这时,起运的岁数便可用三去除,
得出的结果是六岁又八个月。也就是说,如果要给这位女性
看命,她的起运岁数就得从六岁零八个月看起。
除了算出起运的岁数外,我们还必须排一排大运的天干
地支。大运的干支是根据生月的干支推排出来的。起运岁数如
果是顺数的,就由生月干支的下一个干支依次顺排下去I逆数
的,就由生月干支的上一个干支依次倒排上去。比如生月是丁
卯,那末顺数的大运干支就依次是戊辰、己巳、庚午、辛未……,
逆数的大运干支就依次是丙寅、乙丑、甲子、癸亥……。命书
规定,大运的每个干和支各管五年吉凶,看夭干时可结合地支
一起看,看地支时因为天干的五年已经过去,所以就撇开不管
了。这里假设一个人三岁起运,顺数的大运干支是戊辰、己
巳、庚午、辛未等等,那末这就是说,他三到十二岁的大运是戊
辰,十三到二十二岁的大运是己巳•…“。看他三到八岁的大
运好坏,要结合戊辰二个字的五行一起分析,接下来看八到
十二岁,就要撇去戊字的五行,单看一个辰字所含的五行就够
了。
大运之外,还有一种小运。因为孩子如果还没交上大运,
小运可以补大运的不足。比如有人八岁起运,那八岁以前的
吉凶,除了配合流年太岁,还可参看小运。
关于小运的起法,有的认为男起丙寅顺行,女起壬申逆
行,有的认为“甲子旬,男起丙寅,女起壬申, 甲申旬,男起丙
戌,女起壬辰,甲午旬,男起丙申,女起壬寅;甲辰旬,男起丙
午,女起壬子,甲寅旬,男起丙辰,女起壬戌”。然而这些办法不
是刻板,就是有背圣人原立起运的深义,所以都不及醉醒子的
• 87 •
小运推法那样容易被人接受。醉醒子的方法是,以时辰的干
支作为出发基点,男命阳年生的顺推,男命阴年生的倒推,女
命阴年生的顺推,阳年生的倒推。庚辰年甲子时生的男命,按
照上面阳年生男命从时柱出发顺推的原则,依次为一岁小运
乙丑,二岁小运丙寅,三岁小运丁卯,一直接到大运。也就是
说,这命从降生到世上的第一天起,就行乙丑的小运了。
在叫法上,小运又叫行年。好多命理学家认为,当孩子行
运进入大运后,也要结合小运一起观察,如果大运虽吉,小运
不通,不可一下就认为是吉;反过来说,如果大运虽凶,小运吉
利,也不可一下就认定是凶。然而从习惯上来说,古人对小运
是并不十分重视的。
所谓流年,比较简单,就是根据农历甲子、乙丑、丙寅、丁
卯……的年份,哪一年去算命哪一年就算是流年。比如有一 命,去算那年的干支是丙寅,那末流年就是丙寅,去算那年的
干支是丁卯,那末流年就是丁卯。有时人们为了问问当年的
吉凶,这时看命的人就常会先看他出生那天的天干,然后结合
流年,在命书上注明“流年丙寅,X X 主事”等字样。还是举例
能说明问题,有人丙寅年去算命,而他出生那天的天干逢庚,
那末看命人就会批上“流年丙寅,偏官主事”。因为丙火克庚
金,命理学家称之为偏官,这在下面命理学家关于五行生克术
语的章节里,我们还要细谈。
末了再说命宫推法。推命宫一般可以利用自己的 指掌。
如以左手为例,无名指末节近掌侧横纹是子位,中指末节近掌
侧横纹是丑位,食指末节近掌侧横纹是寅位,食指末节上端横
续是卯位等等,这样顺时针一周下来,正好是地支的十二位
数。关于十二地支位数和月份的位置,命书上另有如下图的
• 88 •
逆向分配法。
巳八月 午七月 未六月 申五月
辰九月 酉四月
卯十月 戌三月
寅十一月 丑十二月 子正月 亥二月
有了这基础后,我们就可以分两步走来推算命宫的干支
了。比如庚辰年十月辰时生,第一步先算命宫地支,算法是先
由图上子位作正月逆推上去,或对照附图,可知十月的地支处
在卯位;接着再将卯位作为出生的辰时,开始作顺时针计数,
数到卯时,正好停在十二地支的寅位上面。这样推出来的“寅”
字,就算是命宫的地支了。又如甲子年三月酉时生,第一步是
先由图上子位作正月逆推上去,或对照附图,可知三月的地支
处在戌位I接着再将戌位作为出生的酉时,顺时针计数,一直
数到卯时为止,这样酉、戌、亥、子、丑、寅、卯依次数来,卯正好
停在了十二地支的辰位上面,于是这“辰”字就算是命宫的地
支了。这里所要注意的关键问题是,当逆推得到月支位置后,
即将此支改作出生的时辰,并顺时针序依次数到“卯”字为止。
命宫地支推出以后,第二步是再推命宫的天干。推法是
根据前面所说“甲己之年丙作首"的歌诀或图表(参见《怎样排
八字》篇〉,这样我们就可分别找出庚辰年十月辰时生的命宫
天干,是和地支“寅”字相配的“戊”字,以及甲子年三月酉时生
的命宫天干,是和地支“辰”字相配的“戊”字了。
命宫找出来后,算命时便可在命书后面批上“安命戊寅”、
“安命戊辰”,或“安命寅宫”、“安命辰宫”等字样了。
在大多数情况下,为了简便起见,命理学家有时常将命宫
略去不推,这又存乎其人了。因为有好多命理学家认为,推算
• 89 •
命宫本属毫无意义,画蛇添足之举。但也有认为,命宫地支可
作参考,而天干则没有注重的必要,这又存乎其人了。
此外,有的命书上还有什么推胎元等法的。所谓胎元,就
是受胎月份的意思。推胎元的目的主要是推算命主怀孕月份
的干支五行禀赋,从而作为算命时的一种参考依据。关于胎元
的推法,从出生月份起,天干向前顺推一位,地支向前顺推三
位,这样推出来的干支就是受胎月份的干支了。例如甲子月
出生的人,天干甲向前顺推一位是乙,地支向前顺推三位是
卯,那末这人受胎的月份就是乙卯月了。从六十花甲周转一
周来看,从乙卯月到甲子月出生,正好是十个月。古人说十月
怀胎,可见这种推法是以十月怀胎为基数的。可是在事实上,
除了十月怀胎外,既有怀胎七月、八月、九月不足月的,又有怀
胎十一月、十二月等过月而产的,因为有着这种因素在内,所
以在一般情况下,算命先生也是大多略去不推的。

\section{关于五行生克的术语和用神}
命理学家论命,很有一套理论,其中最为重要的,就是五
行论后。由于五行论命有着它一整套完整的体系,所以在封
建社会中,较易为士大夫阶层或知识分子所信仰。
五行论命过去有以年柱为主,结合其他三柱进行推论的;
也有以日柱为主,结合其他三柱进行推论的I 但以日柱为主,
结合其他三柱五行进行推论的算法最具权威。
所谓以日柱为主,结合其他三柱五行进行论命,就是在算
• 90 •
命时,先把一个人出生的年、月、日、时四柱八字排出,并以日
柱天干作为我自身论命的出发点,把四柱的八字都化成五行,
然后再根据日柱天干和周围其他千支五行之间错综复杂的关
系,进行具体的分析推论。
具体说来,比如一个人出生在公元 1940 年农历九月十一
日辰时,我们可以先按照《怎样排八字》篇里所说的方法,依次
排出他年、月 、日、时的生辰八字:
(年) 庚辰
(月) 丙戌
丙戌
(时) 壬辰
然后再根据日柱中自身天干和周围干支所含的五行的关系用
笔注出:
f乙木 正印
偏财庚辰戊土 食神
I癸水 正官
f丁火 劫财
比肩丙戌戊土 食神
I辛金 正财
f丁火 劫财
丙戌{辛金 食神
I戊土 正财
f乙木 正印
偏官壬辰戊土 食神
I癸水 正官
这里,不管是年柱、时柱天干上所注的偏财、偏官,月柱天干上
所注的比肩,还是年、月 、日 、时地支下所注的正财、劫财、正
印、正官、食神,都是算命术中最常见的术语。这些术语,有的
书中叫做六神。不懂这些术语,就较难与之论命了。
• 91 •
从前文所论五行关系来看,都有一个与我同类,还是生
我,我生,克我,我克的问题。上面我们看到的一些如正财、偏
财、伤官、食神之类的有关术语,就是以日柱天干作为自身出
发点,与周围其他有关干支发生生克关系的结果。现把八字
中有关五行生克的术语列举如下:
〔生我者为正印、偏印〕其中以阳母生阴我,阴母生
阳我为正印,如戊土生辛金,辛金生壬水,戊土就是辛金的正
印,辛金便是壬水的正印I阳母生阳我,阴母生阴我为偏印,如
戊土生庚金,辛金生癸水,戊土就是庚金的偏印,辛金便是癸
水的偏印。
〔我生者为伤官、食神〕 其中阳我生阴子,阴我生阳
子为伤官,如甲木生了火,丁火生戊土,丁火就是甲木的伤官,
戊土便是丁火的伤官;阳我生阳子,阴我生阴子为食神,如戊
土生庚金,庚金生壬水,庚金就是戊土的食神,壬水便是庚金
的食神。
〔克我者为正官、偏宜〕其中阳干克阴我,阴干克阳
我为正官,如壬水克丁火,癸水克丙火,壬水就是丁火的正官,
癸水便是丙火的正官;阳干克阳我,阴干克阴我为偏官,又称
。七杀”或七煞”,如壬水克丙火,癸水克丁火,壬水就是丙火
的偏官,癸水就是丁火的偏官。
〔我克者为正财、偏财〕 其中阳我克阴干,阴我克阳
干为正财,如庚金克乙木,辛金克甲木,乙木就是庚金的正财,
甲木便是辛金的正财;阳我克阳干,阴我克阴干为偏财,如庚
金克甲木,辛金克乙木,甲木就是庚金的偏财,乙木便是辛金
的偏财。
〔与我同类者为劫财、比肩〕 其中阳与阴,阴与阳同
• 92 ・
类为劫财,如甲木逢乙木,丁火遇丙火,乙木就是甲木的劫财,
丙火便是丁火的劫财;阳与阳,阴与阴同类为比肩,如庚金逢
庚金,癸水逢癸水,庚金就是庚金的比肩,癸水便是癸水的比
肩。
从以上列举五项可以看出,一切术语都是从日干的自我
和周围干支的关系生发出来的,其中阳与阴,阴与阳发生关系
为正,也就是异性的为正;阳与阳,阴与阴发生关系为偏,也就
是同性的为偏。这些术语,因为在取用神中常常提到,所以也
有直接称之为用神的。.
为什么会有这些印绶、食神等古怪的名字出现呢?原来
命理学家认为造化流行在天地间,不过阴阳五行而已,而阴阳
五行的交相为用,又不过生克制化而已。所以对于这些古怪
名字来说,也就是阴阳五行交相为用、生克制化的直接产物
了。
先说生我的印绶。因为生我的好比父母,所以便取了个
印绶的名称。所谓 就是荫庇的意思,所谓绶,就是授受
的意思。好比父母有恩德荫庇子孙,学孙就借了光一样。
关于命局中印绶宜忌的情况,比如日主身强,这时如果再
有印绶叠见,就必须财星破印,才能避免“满招损”的祸患。《玄
机赋》说:“印多者行财而发J 反之日主身弱,命局中官杀太
重,比劫力薄,这时如得印绶生扶,就有救了。“用之印绶不可
破”,《子平撮要》的这句话,同样是包含了偏印在内的。
清朝陈素庵著《命理约言》,内有《看正偏印法》一节,说得
较为详尽。他说「大抵印不论正偏,但当月令而取之为格,必
不可伤;即不当月令而倚之为用,尤不可伤,在局在运皆然。术
家往往重财官而轻印,不知印被伤,与官被克,财被劫相同,其
• 93 •
有时而轻者,局偶不用印也。若局用印,而无显印,则暗印亦
可取,或木日取申中之壬、辰、丑中之癸,或火日取亥中之甲,
辰、未中之乙,此处二三处有之,方可取用。行运透出为吉,克
坏为凶,仅止一点,亦不济事。总之,局印太轻,须以官、煞运
生之I局印太多,须以财运制之。若太多而强不可制,竟为下
命。盖印乃生我之神,既无弃命从印之法,又无比劫泄印之法
也。至于枭印夺食,惟枭、食两透于干,或并见于支,而无制无
• 化则忌。苟制化得宜,或干支异处,则亦不忌。又旧忌印行死
地,亦不尽然,盖所贵乎印者,以扶其身耳。印之病、死,即身
之禄、旺,何害之有?若但取印旺,则印之禄、旺,即官之病,何
利之有乎?”
对于陈素庵文中所说“既无弃命从印之法”, 近代命理学
家韦千里认为「以理衡之,局中印绶太多,亦可从印。盖七杀
为克我之神,尚且可从,则印为生我之神,如子投母,岂不可
从?任铁樵所著之《滴天髓阐微》一书,载有从强之说,即此意
也。印之病、死,即身之禄、旺,此指阴阳同生同死而言,若以
阳生阴死,阴死阳生而论,则又穿凿不附矣。”
次说我生的食神。因为我生的是孩子,孩子长大后报答
恩典,致养父母,所以说是食神。至于我生的伤官,因为能制
约官星,所以一般认为并不吉利,有“伤官见官,为祸百端”的
说法。再说女命见伤官无财,多主克夫,那就更怕人了。
然而,食神吉而伤官凶,也要根据具体情况,才能定其宜
忌。如有关食神的宜忌,命局日主强而比劫林立,又没有官煞
制身,财运劳身的,这时如逢食神泄身,这食神就成了致中和
的用神了。假如同时又遇印绶夺食,则又不吉。《子平撮要》
说「用之食神不可夺。”说的就是这一情况。相反,命局日主
°94 •
弱而食神重重泄身,而又只见财官,不见生身夺食的印绶,这
时如遇比劫助身,就有救了。所以古歌有“食神最喜劫财乡”
的说法。
伤官虽说是个忌神,但也必须根据具体情况,才能论定。
如命中日主太强,财星稀少,这时就又全靠伤官泄秀生财,以
尽其妙了。古歌有云,“伤官伤尽最为宜。”这时命局中如有官
星一点,没有财星作为中介的,则就怕见伤官戕害。至如日元
衰弱,八字中又多伤官盗泄其气,这时就全赖印绶生扶,制伤
为吉了。因为这时如逢财星,虽然也能盗泄伤官之气,可是又
因财能破印,印绶一坏,失却扶持,日主就不堪忍受。正如《玄
机赋》所说那样「伤官用印宜去财 此外,如遇比劫助身也
好,古歌说「伤官不怕比劫逢。”就是这个道理。
有关食神、伤官,《命理约言》有《看伤官法》、《看食神法》、
《看食伤法》等篇段加以论述。他在《看食伤法》中说:'食伤格
中有尤秀者,曰木火通明,曰金白水清,曰水木清奇,曰土金毓
秀。今略举取用之法,木火通明格,以春三月木日遇火为妙,
妙在木旺能任火相,方进也,四月亦取,盖火当令而未燥,但木
须得势通根耳。金白水清格,以七、八月金日遇水为合,亦妙
在金旺水相g 水木清奇格,以二月癸日遇乙,及卯木为上;土
金毓秀格,以八月己日遇辛,及酉金为上。盖卯、酉气专而清,
但癸与己,亦须得气通根耳。凡合此四格者,皆清贵上命,其
喜忌之理,随格详审之,然不特此也。凡日主强旺,喜泄甚于
喜克。局中官杀与食伤并见,势均力敌,照常取断;若官煞轻
浅,其情恒向食伤,不必当时得令,但透干成象,即可取用,反
以官杀为病神矣。术家于此等局面,只泥官杀为用,所以往往
不验。是亦所谓六神通变之端,不可不知也
• 95 。
接说克我的官煞。所谓“官者棺也,煞者害也”。 朝廷一
旦封人做官,此身就自多属于公家,政绩是好是坏,直到最后
盖棺才能定论,可谓被曾拖害苦了. 平时人家梦棺得官,就是
这个道理。
官为正官,煞为七煞。就正官言,命局中日主身强,比劫
叠见,可是财星寥寥,这时若有官星制约比劫,就有利于保持
全局的平稳。如果一旦伤官损伤官星,比劫就可因无制而猖
獗为害。为此《子平撮要》指出「用之正官不可伤「又如日主
身强,比肩林立,而官星无力,难以制服比劫,这时就必须藉财
生官,以化伤而增加官星的威慑力量。“官轻见财为福刹”,就
是针对这种情况而说的。反之日主身弱,又无比劫助身,八字
中官星强旺,这时就必须印绶,才能使官星为我所用。“有官
有印,无破作庙廊之材」指的正是这一情况。又如 H主身弱,
命局中非但没有印绶生扶,反而多官星克伐,因为有克无生,
没奈何,这时如遇伤官伤害官星,也能转危为安,不作凶看。
进而言之,陈素庵《看正官法》发挥道「看官之法,先论日
干强弱,日干强则当扶官,日干弱则当扶日。再看官星得时得
势与否,适当月令,又透天手为上,如甲生酉月,天干透辛,乙
生申月,天干透庚是也,次则或当月令而不透干,或不当月令
而干官通支,支官通干;又次则干有支无,支有干无。皆须财
以生之,则官之根茂;印以卫之,则伤官之害远。必须正财配
偏印,偏财配正印,则财印不相战。或财在干,印在支,或印在
干,财在支,虽皆正皆偏,各有理会,亦不相战也。若官星太
多,亦须食伤制之,然不作杀论。其切忌有二,一曰冲破,一曰
伤官;须忌有三:一曰食众暗损,一曰印众泄气,一曰时归死
绝。大抵官之强旺者遇此五忌,但减贵气,官之衰弱者遇此五
• 96 ・
忌,则坏矣。至于逢官看财,虽一定之理,然官衰倚财,以多为
贵,官旺亦不甚倚财,略见已足J又说「至于日主无气,满局
皆官,当弃命从之,与从杀同法。”与此同时,他还指出,“旧书
有官不见官之说,谓甲日见丙辛,则甲得辛为官,辛又得丙为
官,此乃节外生枝,不足信也J
再如就七煞言,也同样有宜有忌。日主身强,八字中比劫
重逢,财星力薄,这时正赖官星照耀为福,而官星却又隐迹不
显,没奈何,如果命局出现七煞,以补官星不足而制约比劫,使
比劫不敢觊觎财星,则也可以持平格局。《继善篇》说「身强杀
浅,借杀为权。”关键是杀星不能太重,否则就会“危及自身”。
又如日主身强,比劫重逢,而所需的七杀又相对力薄,不足以
制伏比劫,这时就要命中财多生杀,方才有用。“杀轻者善财
生之”,就是说的这一情况。反之 H主身弱,比劫零落,没有夺
财的忧虑,这时如果七煞太重,就戕伐自身了。然而,要是命
局中同时又出现伤官、食神的,因为食伤可以制煞,可不构成
危害。《玄机赋》说「杀重身轻,制乡有益「再如身弱煞旺,得
见印绶生身,也就无需多虑,因为《玄机赋》早就说过「身弱有
印,杀旺无妨。”又如日主衰弱,八字中七煞林立,既没有印绶
护身,比劫助势,又没有食伤制煞,在孤立无援的情况下,就索
性“弃命从煞”,就好比弃命从财一样,反而取七杀作为用神,
让自己作为傀儡。子平有云「日主无根,弃命从杀」
七煞又称“偏官”,陈素庵《看偏官法》阐发「看杀之法,先
论日干强弱。日干强,则一点杀星,亦不可制;日干弱,则不问
杀之多寡,必须制之。再看杀星得时得势与否,当令而又透
干,为杀旺1次则或当令而不透干,或不当令而干杀通支,支杀
通干;又次则干有支无,支有干无。制之用食伤,食较有力;合
• 97 •
之用刃劫,刃较有势;化之用印,偏正同功。杀太旺,则制化两
用,但须食神配正印,伤官配偏印,则不相战也;或食伤配正
印,干支异处,各有理会,亦不相战也。若刃劫合煞,阴日不如
阳日,盖甲用卯中之乙合庚,乃卯之本气,乙用寅中之丙合辛,
视本气有间矣。甲用乙合庚,庚贪合则忘杀,乙用甲止能帮
身,视合煞又有间矣,故阴日以制化为急。若杀星太弱,宜财
神滋之;制神太过,宜偏印破之。至杀星太强而无制,日主太
弱而无根,宜弃命从之。总之,日干能任财杀为要,苟日干衰
绝,又不能从杀,即有制化,岁运财杀旺地,必成灾祸,倘更无
制无化,岁运财杀旺地,无不危亡。若身杀两停,行运宁可扶
身。”此外,对于古书所说「杀不离印,印不离杀。”“印无杀不
显,杀无刃不威。”陈素庵也自有他的理解「盖印所以生日主,
刃所以护日主,虽不言扶身,而扶身在其中矣。又有杀强于
主,行杀运反利者,此必日主本非衰绝,而原局印绶成象有力,
杀生印,印生身也。惟忌行财运,坏印助杀,则必为祸矣
再说我克的妻财。因为妻子是事奉我而终身无违的,财
产是被我自然享用的,两者都被我掌管,所以我克的就是妻财
了。
有关财星的宜忌,如日主太强,八字中财星不多,又少官
煞制身,这时这不多而堪用的财星就不能再遭劫了。《子平撮
要》说「用之财星不可劫J 就是指的这一情况。又如日主强
而比劫多,这时就要财旺,如有官星制劫,则就更上一层楼。为
此古歌说「身强财旺皆为福,若带官星更妙哉 反之,如果
日主衰弱,又没有比劫助身,而八字中偏逢财星叠出,这时就
全靠印绶扶持了。古歌有云「日主无根财太重,全凭印绶扶
身躯。”换个位置,八字中如果日主弱而财星叠见,又无印绶扶
• 98 •
持,同样道理,这时又要靠比劫的力量来制财助身了。这就是
《玄机赋》所说的「财旺者遇比何妨「此外还有一种 “弃命从
财”的情况,这是说日主太弱,命局中财星叠见,要借印绶扶持
而偏偏没有比劫,没奈何,于是索性“弃命从财”,反而可取财
为用神。
在《命理约言》中,陈素庵有《看正偏财法》一段,详为阐解
道「看财之法,不论正偏,只取得时得势。适当月令而有气为
得时,不当月令而成象为得势,然看日干强弱为要。日干强,
则当扶财;日干弱,则当扶日。旧云 '逢财看官'者,不尽然。凡
我克我生,一件入格得气,皆可取贵,但恐止此一件,便是滞
物,故财与食伤,又欲其辗转生化,非必以生出、克我为贵也。
每见用财之命,或财轻而行生财之运,或财重而行制财之运,
一生不行官杀,往往富贵,但局中运中见官杀,亦其所宜耳。苟
财多身弱,又加以官杀,取祸必矣。旧谓正财乃分内之财,遇
之非奇,偏财乃众人之财,得之为美,夫不安己之分,而喜取人
之物,此贪夫之见耳。特正财能伤正印,偏财能制枭神,然不
可因此而贵偏贱正也。旧又有恶露喜藏之说,此亦谬认财为
钱币耳。即以钱币论之,源远流长,挥霍岂忧睥睨'力微势弱,
扃钥何难劫夺乎?至于财神太旺,而用比劫,盖爱其助主,非
取其分财;财神太衰而用食伤,虽藉其生财,亦防其泄主。若
财多而强不可制,当弃命从之,行助财运则吉,行夺财助主运
则凶。他如时上偏财,时上财库,日时专财,夹财拱财等格,皆
多立名目,不若四柱通融取用,较为简当也「
末说与我同类的比劫。正因为彼此都是同类,大家肩比
着肩,高低差不了多少,所以便把比肩称为兄弟。在命书中,
除了把阳干见同类的阳干、阴干见同类的阴干称为比肩外,还
• 99 ・
把阳干同类的阴干,阴干同类的阳干称为劫财。命书认为,命
中见劫财多克妻害子,多破耗,并要提防小人。
至于比劫的具休情况,也有宜与不宜之分。命局中日主
强旺,不宜比劫助身,如遇比劫助身,又没有食神、伤官泄身
的,那就必须官煞制服,才能致于中和。“比肩要逢官煞制”,
《子平撮要》的这句话 是针对这一情况而说的。相反,如果
日主衰弱,又见宣煞制身,伤官泄身,财运劳身的,就喜见比劫
出来助身了。《玄机赋》说「日干无气,遇劫为强。”可见命遇
比劫,也并不全是不吉的。
我们这里且看陈素庵《看比劫禄刃法》对于比劫禄刃的一
番阐述「天干各有比劫,地支惟戊己遇辰戌丑未为比劫。甲
乙遇寅卯,丙丁逢日午,庚辛遇申酉,壬癸遇亥子,皆禄刃也。
盖本气纯粹为禄,本气刚暴为刃。凡阴阳之禄刃,交互取之。
乙丁己辛癸之刃,确在寅申巳亥,向来但知禄前一位为刃,而
不知阳以前为前,阴以后为前,妄谓辰戌丑未为阴刃,试以阴
阳同生同死之法推之,四者皆衰绝,何得有刃?即以阳生阴死
之法推之,四者皆冠带,何以成刃?又有谓阳有刃,阴无刃者,
既非通理,甚有讹阳为羊,谓如以刃到羊者,尤属谬误。至于支
有刃,而干见刃,谓之刃透,往往以支无劫,以干劫当之,然则
支无禄,可以干比当之耶?总之比劫禄刃,异情而同劫,皆助
身之神,特比纯而劫驳,禄和而刃暴耳。比与劫,主衰杀旺则
用之,身弱财多则用之。刃则取以助干,尤妙于合杀,盖刃杀
皆刚暴之物,相合则如猛将悍卒,处置得宜,为我宣威奋武,人
命值之,贵而有权。禄则能扶日主,亦能助诸贵神,旧谓建禄
离祖,专禄伤妻,间亦有验,然财印得时得势,此一端未便为害
也.”文中认为天干与日干同类的为比劫,地支和日干同类的
• 100 •
为禄刃,并且剖解宜忌,足资参考。
在这些彼此错综的关系中,归纳起来,“官系福身之物,财
是养命之源,印乃资生之本,在人最为切要”,然而命中亦要活
看,不能刻板。此外,这些关系彼此之间还有一些避忌,就是:
“官怕伤,被伤则祸I财怕劫,劫则被分,印怕财,贪财则坏;食
怕枭,逢枭则夺。”这些话看来简单,可用意却还不小哩。
对于这些由五行之间的生克而造成的彼此间的 错综关
系,还派生出了好些有关术语。比如:
(:杀重身轻〕 如自身天干是乙木,没有生在当令的春月,
现在周围又布满重重克我的辛金,也就是克我的七杀(偏官)
太重,所以叫做“杀重身轻
〔身强杀浅〕 如自身天干是甲木,又生于春月当令之时,
势必自身强旺,而周围克我的七杀庚金却少得可怜,所以叫做
“身强杀浅
〔财多身弱〕 如自身天干是甲木,没有生在当令的春月,
而周围却是一片我克的戊土、己土,因为我克的是财,所以叫
做*财多身弱”。 反之则叫做“财弱身强”或“身强财弱”。
〔食神生财〕 如自身天干甲木,柱中有我生的食神丙火,
火能生土,土对甲木来说,不是正财就是偏财。所以八字中如
果缺财的,碰上食神也好。
〔比肩重重〕 如自身天干是甲木,周围的干支里又密布
着重重甲木,因为与我同类而又同性的叫做比肩,所以便就有
了这种说法。
(比劫夺财〕 如自身天干是甲木,而周围干支又布满了
与我同类同性的比肩甲木和同类异性的劫财乙木,而我所克
的正财、偏财戊己土却少得可怜,这样自己本已不多的财,就
•101 •
被比肩和劫财分夺掉了。
〔伤官损印〕 如日干自身甲木,逢柱中丁火就是我生的
伤官,而甲木的印绶则是生我的癸水。如果局中伤官丁火太
旺,不利自身,这时癸水虽能前来克制,可是在水火力量相比
悬殊的情况下,作为印绶的癸水就受损了。
〔印绶护身〕 如自身天干为不当令的甲木,而又没有同
类相扶,这时如果周围干支中遇上生我的水,因为水是木的
印,所以有“印绶护身”的叫法。
〔官印双全〕 如自身天干为甲木,遇克我的金为官,生我
的水为印,并且扶抑相当,没有太过不及,这就叫做“官印双
金‘0
〔财官相生〕 如日干自身甲木,这时财为我克的戊己土,
上能生金,其中辛金,就是克我的正官,所以说财官相生。
象以上这样的术语还有好多,但总的精神是好命要五行
生克扶抑得当,如果五行生克太过或不及的,都不是好命。比
如财多身弱,财多原是好事,只是自己身弱掌管克制不住,没
有这个福份享用,因此命理学家如果算上这种命的,反可有时
断定他的一生没有什么大的财产,或有财也不属于他。再比
如“印绶护身”自然属于好事,但是如果自身太强,周围又多与
自身同类的比肩劫财,这时如再碰上生我的印,就会物极必
反,走向反面,反而弄出祸患来了。
在旧时的算命书中,这种术语充斥纸面,随时可见。但是
由于多少有点莫测高深的味儿,使人难以望见项背,所以遭到
民国时命理学家巨擘袁树珊先生的竭力反对。平时,袁氏论
命详于五行,并在他的著述《命理探原》中得到相当的体现,所
以在旧中国知识界中有着一定的影响。
• 102 •
再说用神。顾名思义,所谓用神,就是八字或大运五行中
对于自身的日干来说,具有补弊救偏或促进助成作用,为我所
用的一种五行代称。其中用神出现在八字命局中的,叫做原
局用神,出现在大运中的,叫做行运用神。
这种补弊救偏或促进助成作用,包涵很广,凡是四柱八字
中对于日干能起扶其过弱、抑其过强作用的,都可取作用神。
任铁樵说「命中至理,只存用神,不拘财官、印绶、比劫、食伤,
皆可为用,勿以名之美者为佳,恶者为憎,果能审日主之衰旺,
用神之喜忌,当抑则抑,当扶则扶,所谓去留舒配,取裁确当,
则运途否泰,显然明白,祸福灾祥,无不验矣。”比如日干乙木,
生不逢春,又少比肩:劫财同类的扶持,这时如果碰上八字或
大运中有生我的水,就可'印绶护身*,逢凶化吉了。再如日•'干
不论乙木或甲木,生于春月,而周围又多比劫,不仅自身强旺,
并且扶持太多,有物极必反的忧虑。这时论命就往往取制木
的金,也就是官煞作为用神,从而抑其太过,达到平衡。如果
八字中不见官煞金,在大运或流年中碰上也好。如果八字或
大运、流年中都碰不上官煞,或碰上也力量不够的,那这人的
用神就不得力,一辈子都别想交好运了。
直接扶抑之外,间接的扶抑也可采作用神,为此便就又有
不只专恃一神为用,或用神之外再辅以喜神、闲神等说法。任
铁樵说「有用神必有喜神,喜神者,辅格助用之神也。然有喜
神,亦必有忌神,忌神者,破格损用之神也。自用神、喜神、忌
神之外,皆闲神也,惟闲神居多,故有一二半局之称。闲神不
伤体用,不碍喜神,可不必动他也,任其闲着,至岁运遇破格损
用之时•而喜神不能铺格护用之际,谓要紧之场,得闲神制化
岁运之凶神忌物,匡扶格局,喜用,或得闲神合岁运之神,化为
• 103-
喜用而辅格助用,为我一家人也。”比如自身日干木弱,八字或
大运中逢上适量的官,就可官印相生,生水扶木了。因此这官
对于木来说,也是用神。当然这种作为用神的官也不能太强
旺了,否则强金克木,就能轻而易举地把木置于死地。同样道
理,比如自身日干木强,八字或大运中偏又逢上较多的水来
生木,也就是印生自身,这就使人担忧木太强了反会走向反
面,这时看命的如果看到八字或大运中有制水的土,也就是自
身的财,就可认定它能抑水生木,把它取作用神,大概不致大
错。
在命理分析中,看准用神,被认为是算命准与不准的关键
一着。在大多数情况下,命理学家都把五行中对自身天干起
最重要扶抑作用的五行看作用神,但有时也把一个人的八字
总起来作通盘的考虑,如太寒太热,太湿太燥,隔塞不通之类,
于是便就又有调候,通关等说法。
何谓“调候”? 通常说来,日主以中和为贵,但有时又不免
会出现全局寒、暖、燥、湿等不均衡的状况。举个例说,有这样
一个命造:
(年) 庚辰
(月) 丁亥
庚申
(时) 庚辰
日元庚金得地得势,偏于强旺,如从扶抑角度看,当取月
干丁火正官制约日元,或取月支亥中壬水泄秀,亥中甲木劳身
为用。但如果从调候角度看,庚金生于冬月,未免金寒水冷,
这时就当急取月干丁火作为调候之神,然而冬月之火,本属虚
脱,加之全局火势不足,便就只好让它暂且闲在那里,此后一
• 104 •
旦行入木运火运,丁火得助,那就大大发挥调候威力了。
总之是,命局太寒,当用暖来调候,命局太热,就要用寒来
调候;命局太燥,就要用湿来调候;命局太湿,就要用燥来调
候。所谓“调候”,就是调节命局寒、暖、燥、湿的气候。
何谓“通关”? 日元以中和、均衡为贵,但除了日元,有时
八字命局,又常会出现两神对立,势均力敌,不相上下的弊病。
有病就得有药,这时如能找到和解或消除两神对立,使之彼此
流通生化的五行,就可把它视之为药,名曰“通关”。比如有一
命造半壁为水,半壁为火,形成两神对垒,不相上下之势,这时
医治最好的药,就莫过于以木进行通关。因为有木作为通关
之神,求但可以消弭两神对垒,同时还可以水生木,以木生火,
造成一种气势流通,生化有情的新局面。可见“通关”之药,原
也不是可有可无的。
用前面这种办法能找到用神的,一般以出现在月份干支
中的为最有力,其次是出现在时辰中,最末才是出现在年份
中。比如一个人日干是秋月出生的辛金得地得势,自身较强,
需要适量的火来加以炼制,然后方才能够冶铸成器。这时如
果月份的干支出现丙寅火,那末算命家就可认定丙寅正官作
为这个人的用神,并且这用神还非常的有力。如果月份的干
支不出现丙火,而是壬子、癸亥等一片水地,那末由于金能生
水,水泄金气,也可看作用神。这里需要注意的是,这人的月
份中如果碰上丙子(丙火、癸水)、癸巳(癸水和巳中丙火)等
水、火同时出现的现象,就又要比较一下整个八字中是火得力
还是水得力,是更迫切需要火还是更迫切需耍水,然后把这更
得力更迫切需要的看作用神。当然,在八字中取用神时还不
要顾此失彼,忘了天干地支间彼此的刑、冲、化、合等等因素,
•105
否则用神取错看借,便就通盘都错了。比如这样一个命造:
(年) 壬戌
(月) 己酉
(H) 丁丑
(时) 甲辰
本命丁火,理该夏月生旺,然而却生于八月酉月火囚之
时,所以没能得时,而年干壬水,月干己土,又都克我、泄我,伤
我元气,加之地支戌、酉、丑、辰,一片金土,又属我克我泄之
神,现在亏得丁火通根年支戌库(火库),又得时柱甲木坐辰生
我为印,然而从全局来看,自身仍属偏弱。弱者宜扶宜生,这
时如果取年柱正官壬水作为用神,以期官印相生,有利印绶甲
木,可是却有食神己土损官为病,所以权衡下来,不如直接取
正印甲木作为用神,既可生扶丁火,又可克制己土,祛除壬水
被制之病,这样水来生木,木来生火,岂不美善?但是,甲木在
命局地支一片金土克我我克的情况下,也毕竟能力有限,所以
又要结合大运来看了。大运如果行到甲寅、乙卯木运,用神得
比肩相助,必定富贵优游。反之大运如入金土,用神受损,那
就困苦不堪了。再如:
(年) 财甲子伤
财
(月) 印戊辰印
伤
比
(日) 庚申食
印
(时) 食壬午高
大运 己巳
• 106 •
庚午
辛未
壬申
癸酉
甲戌
这一命造,虽然天干透出甲戊庚“天上三奇”,地支逢午申拱
贵(午申拱未,未为庚金的天乙贵人),且又申子辰会成水局,
不冲时支午中丁火,看去大有官星得用,名利双收之喜,可是
毕竟因为水势太旺,火力不足,所以难以取丁火作为用神。丁
火之外,再看年干甲木,按理说,甲木泄庚金伤官癸水之气而
生官星丁火,似可为用,不知辰月甲木退气,戊土当权,故而即
使勉强以甲木为用,也属假神。结合行运,前半辈子运走西南
甲木休囚之地,故可卜知虽有祖业,亦一败而尽,且不免刑妻
克子,受尽孤苦。照此看来,若丢却用神而以三奇、拱贵等格
论命,自明清以来,就为学术派所不取。又如:
(年) 丙子
(月) 己亥
(日) 乙丑
(时) 壬午
大运 庚子
辛丑
壬寅
癸卯
甲辰
乙巳
这命粗粗看来,一无可取:夫干壬丙交战,地支子午遥冲,况且
•107-
乙木生于亥月,木寒喜火,正遇水势泛滥,火气因被克而处于
绝地。然而仔细推究,又可发现水势虽旺,乏金相生,火势虽
弱,有土制水救母(火为土母),何况时干壬水生木为印,年干
丙火生月干己土增强制水能力,而此己土,又通根禄旺,其势
足以止水卫火,所谓“有病得药”,故取丙火伤官秀气作为用
神。结合行运,中年后运走东南木火之地,一交寅运,因火木
生旺而连登甲榜,入翰苑,此后则更是青云得路,苦尽甘来了。
关于在大运中看用神,主要有三种情况。一是八字中不
乏用神,而在大运中又重新碰上的;二是八字中缺乏对自身夭
干弱扶强抑的用神,而偏偏出现在大运中的;三是八字和大运
都没碰上对自己强有力的用神。对于末一种,一般认为都是
一身偃蹇,不好的命。对于第二种,命书中有大缺大补、大偏
大纠的说法,生了这种命的,倒起霉来倒煞,可是一行到大运,
不是大补就是大纠,来个彻底的翻身。对于第一种,不用说就
更好了。
这里很重要的一点是,既然用神对于一个人一生命运的
荣枯有着这样重要的作用,那末用神不得逢冲,就被提上议事
日程来了,这就是说,在一个人的八字或大运中,用神被冲被
克是不吉利的事,良之,用神得到生扶或同类相助,也就转吉
有望了。
对于用神,算命先生总是爱用比、食、财、官、印等术语来
加以分析,用神正印、用神食神、用神偏财等等便是。古书初
学捷径《用神喜忌歌》一首道:
用之官星不可伤,不用 官星尽可伤,
用之财星不可劫,不用 财星尽可劫,
用之印绶不可坏,不用 印授尽可坏,
•108 •
用之食神不可夺,不用 食神尽可夺,
用之七杀不可制,制之太多反为凶,
身杀两停宜制煞,杀重身轻宜化杀,
身 强杀浅宜生杀,羊刃重重喜食伤,
若逢官煞亦生殃,财多身弱宜劫刃,
劫重财轻 立食神,官旺身衰宜印地,
官衰印旺利财乡,莫道枭神无用处,
杀多食重最为良,勿谓羊刃是凶物,
财多杀党亦为贞。此是子平真要诀,
后之学者仔细吟。
综观有关用神的一些说法,陈素庵《命理约言》卷一有《看
用神法》总结说:
命以用神为紧要,看用神之法,不过扶抑而巳。
凡弱者宜扶,扶之者即用神也,扶之太过,抑其扶者’
为用神,扶之不及,扶其扶者为用神;凡强者宜抑,抑
之者即用神也,抑之太过,抑其抑者为用神,抑之不
及,扶其抑者为 用神。如木弱扶之取水,水扶太过,
制水以土,水 扶不及,生水以金;木强抑之以金,金抑
太过,制 金以火,金抑不及,生金以土。至同类之相
助,财气之相资,亦扶也;生物泄其气,克物杀其势,
亦抑也。是故有日主之用神焉,六神之扶抑日主者
是也;有六神之用神焉,六神之互相扶抑者是也。六
神之用神,即为日主用也。有原局之用神焉,局中本
具之扶抑是也,有行运之用神焉,运中补足之扶抑是
也。行运之用神,即为原局用也。用神无破为吉,有
助则更吉;用神有损为凶,无 救则 更凶。命譬之身,
•109 •
用神警之身之精神,精神厚则身旺,精神薄则身衰,
精神长存则身生,精神坏尽则身死。看命者,看 用神
而 已矣。然取用神之法,虽 当专一而不眩,亦 宜变通
而勿拘,如正、偏官格,有时制化互用,甚或 生制参
用,况行运数十年,无具木具金之理。尝见大富贵之
命,不恃一神为用,其专恃一神者,乃补偏救弊之病
耳,抑更有说焉。有体而后有用,日主六神体也,扶
抑日主六神者,用 也。苟日主六神,或强不可制,或
衰不堪扶,或散漫无伦,或战争不定,是则体先不成,
用于何有?其为下命运矣。
在封建社会中,算命先定大都认为八字中五行俱全,管人
一生衣禄不愁,当然八字中有缺在运中补上也好。由于这种
思想经过日积月累的渗透,早已普遍地蔓延到了广大的平民
百姓中间,所以民间父母为孩子取名,又常会根据算命先生推
算结果,把所缺的五行在名字中补上,以讨吉利。今天我们如
果看到老一辈中名字有叫森、茶、春、鑫、淼的,便可大致推断,
他命里是缺了什么五行的。

\section{八字中有关的星宿神煞}
我国早期的看命法中,有一种流行很久的星宿照命和神
煞入命的观念。把天上星宿神煞和人的命运结合起来,出于
古代人们对于星和神的一种崇拜心理。
在一个人的四柱八字中,看星宿神煞大多以代表自身的
•110 •
日柱干支为出发点,再联系年、月、时或大运、流年等其他干支
进行观察比照。翻开命书,自身干支中的什么字碰到年、月、
时,或大运、流年干支中的什么字便算遇上了什么神煞,命书
都有一定的规定。譬如自身日干庚金,碰上年、月、时中地支
的亥,就被认为是“文昌入命”了。 这种文昌,是个吉星,假如
读书人碰到了它,一定事业出人头地,春风得意。然而对于有
的神煞,也尽有不从日柱天干出发的。
在古往今来的命书里,有关神煞极多,这里择要介绍如
下:
1. 吉星照命或吉神入命
〔天德〕 也称“天德贵人”,这是以出生月份的地支,结合
出生日期、时辰的天干所反映出来的一种吉星。古歌说:
% _ 正丁二坤(申)中,三壬四辛同,
% 五乾(亥)六甲上,七 癸八艮(寅)逢,
% 九丙十居乙,子巽(巳)丑庚中。
% 然而根据近人研究成果,也有列成下表的:

% 【111页表】

% 如表,寅月出生的人在日或时的天干上碰上壬或丁做
% 出生效A至巨型鲤粘赛巳更史等,就被认为是有天
% 德贵 函嬴
% 命里有天德贵人星的人,一生吉利,荣华富贵。现在港台
% 等地的命书中,还曾广泛记载了世界各国名人的命。比如举
% •iii
% 天德贵人星为例,书载:“命运中有天德贵人星的人,尤其以日
% 命为主,在社会上能出人头地的非常多。例如许多出名的歌
% 星、影星美空云雀、吉永小百合等,甚至英国前首相邱吉尔、日
% 本美智子妃殿下,这些人都是日命中有天德贵人星相助」
% 〔月德〕 这是一种以出生月份地支,结合出生日期天干
% 反映出来的吉星,规律是:
% 寅午戌月在丙,
% 申子辰月在壬,
% 亥卯未月在甲,
% 巳酉丑月在庚。
% 这个规律看起来象不好记,其实好记。在前面《天干地支的刑
% 冲害化合》篇中,我们在谈地支的“三合”时,曾有过“申子辰合
% 水,亥卯未合木,寅午戌合火,巳酉丑合金”的说法,这里“寅午
% 戌月在丙”,就是寅午戌月见丙火日,“申子辰月在壬”,就是申
% 子辰月见壬水日,“亥卯未月在甲”,就是亥卯未月见甲木日,
% “巳酉丑月在庚”,就是巳酉丑月见庚金日,并且见的日干都是
% 阳干,还不好记?命中有月德的人,也和天德一样,一生无险 J
% 无基
% t三奇〕 三奇有“天上三奇”、“地下三奇”、“人中三奇”三
% 种情况。但不论哪一奇,都要些见月,或月、日 、时的天干
% 挨次顺排下来才是,如果位置.康不是了。歌曰:
% 天上 三奇甲成庚,
% 地下三奇乙丙丁,
% 人中 三奇壬癸辛。
% 这就是说甲年生的人,月干、日干中同时挨次出现戊、庚,或甲
% 月生的人,日干、时干中挨次出现戊、庚,就算是应了“天上三
% • 112 •
% 奇”。其他类推。
% 按照命书说法,八字中逢“天上三奇”、“地下三奇”、“人中
% 三奇”的都是襟怀卓越,博学多能,大富大贵,不属凡类的人。
% (:天乙贵人] 天乙贵人星的看法以日柱的天干为主,结
% 合其他三柱地支观察。方法是:
% 甲戍庚见丑未,
% 乙己见子中,
% 丙丁见亥酉,
% 壬癸见巳卯,
% 辛见寅午。
% 这就是说,甲日戊日或庚日出生的人,见八字地支中有丑或未
% 的,就可认定是有天乙贵人星了。其他类推。
% 命书认为,天乙贵人星也是一种很有用的吉星。《三车一 览赋》说:“天乙贵人,得之聪明。”此外命里有这种星的人,还
% 可逢凶化吉,因为有贵人相助。
% 这里有个现象,就是十天干配合十地支,把十二地支中的
% /良、唐两支排斥掉了。为什么会有这种现象呢?《渊海子平》
% 编释道「十干临十支,皆贵人所临之方,惟辰、戌两宫,贵人不
% 临,何也?殊不知辰、戌乃魁罡恶弱之地,天乙不临,所以不为
% 贵也」
% 〔天赦〕 天赦也是个吉星,这种吉星出现不多,看法是:
% “春戊寅,夏甲午,秋戊申,冬甲子。”解释是春月逢戊寅日出
% 生,夏月逢甲午日出生,秋月逢戊申日出生,冬月逢甲子日出
% 生的,都是逢上了天赦星。
% “命中若逢天赦,一生处世无忧”,这就是命书对天赦星的
% 说项。
•113 •
〔十干禄】《渊海子平》说「甲禄在寅,乙禄在卯,丙戊禄
在巳,丁己禄在午,庚禄在申,辛禄在酉,壬禄在亥,癸禄在
子。”看法以日干五行为主,结合年、月、日 、时地支。如果从自
身天干出发,禄在年支的叫做岁禄,禄在月支的叫建禄,禄在
日支的叫坐禄,禄在时支的叫归禄。例如庚日出生的人,年支
逢申,就是岁禄,月支逢申,就是建禄,日支逢申,就是坐禄,时
支逢申,就是归禄。其他日干见禄依此类推。
禄为养命之源,命中逢上,一生衣禄不愁,然而最怕犯冲
或入空亡(见下〔六甲空亡〕),如果这样,反而衣禄不足了。
〔文昌〕 命里出现这种吉星的,对于知识分子来说,尤其
有用。《命理探原》说,“文昌者,乃食神之临官长生所也。”看
法以日柱天干为主,结合其他有关地支进行观察。具体情况
是l
甲见巳,
乙见午,
丙成见申,
丁己见酉,
庚见亥,
辛见子,
壬见寅,
癸见卯。
对此,古歌有云:
甲乙巳午 报君知,丙戊申宫丁己鸡,
庚猪辛鼠壬逢虎,癸人见兔入云梯。
命书认为,八字中见文昌星的,非但聪明过人,才华出众,并且
另外有着逢凶化吉的妙处。 <
114 •
〔将星〕 这名字听上去很威严。八字中出现将星的情况
是:
寅午戌见午,
巳酉丑见酉,
申子辰见子,
亥卯未见卯。
意思是寅、午、戌日出生的人,碰上年、月、时地支中有午字的,
便是有了将星。其他类推。《三命通会》说「将星者,如大将
驻扎中军也,故以三合中位为将军「
据认为,命中出现将星的人,有掌权之能,众人皆服。又
云「将星文武两相宜,禄重权高足可知。”
2. 偏于中性,有吉有凶的星煞
〔魁罡〕 魁罡是一种天冲地击之煞,凡是日干碰上戊戌、
庚戌的叫天罡,逢上庚辰、壬辰的叫地罡鼠 命中有魁罡星的,
主人性格聪明,文章振发,临事果断,秉权好杀。如果运行身
旺,必定发福百端,假若一见财官,那就祸患立至了。又如日
柱魁罡,碰上刑冲,非 但不吉,反而是个贫寒的穷人。此外,女
性以阴柔为美,命中如果碰上魁罡,也是犯忌的。
〔华盖〕 鲁迅诗中曾说「运交华盖欲何求?未敢翻身已
碰头」把交华盖看成是交了坏运。其实,华盖在命柱中出现,
按照命书的说法,也不一定全是坏事。看法是:
寅午戌见戌,
巳酉丑见丑,.
申子辰见辰,
亥卯未见未。 1.
比如寅、午、戌日出生的人,在年、月、时的地支中碰上戌
•115
字,便被认为是有了华盖星。
命书说,华盖为艺术文章之星,主人必定读书刻苦,做事
勤恳,但性格却不免孤僻。如果华盖多逢印绶,并且处在旺相
之地,可能在政界有一定地位。又如“华盖逢空,偏宜僧道”,
那就孤而不吉了。
〔驿马〕 在古代,命中出现驿马有两种情况,就是贵人驿
马多升跃,常人驿马多奔波。《身命赋》说:“马奔财乡,发如猛
虎。”《造微论》说:“马头带剑(驿马天干见庚、辛,或纳音见
金),威镇边疆。”这说明驿马可吉可凶。如果四柱五行阴阳配
合得宜,驿马和财、官、禄处在同一地支上,并且居于生旺之
地,不逢克伐,那就不贵也富。反之,如果驿马处在死绝之地,
并且又逢克伐,或竟处在空亡之乡,那就难免奔波流浪了。然
而,又有“逢冲譬之加鞭,遇合等于繁足”的说法。观察命里有
没有驿马,口诀是:
寅午成见申,
巳酉丑见亥,
申子辰见寅,
亥卯未见巳。
意思是寅、午、戌日出生的人,逢年、月 、时地支中逢申的,就可
认为是有了驿马。由于申为金,由此扩展开来,凡四柱地支或
大运流年碰上巳、酉、丑三合金局地支的,也可认为是驿马发
动的迹象。
同样道理,既然寅、午、戌日出生的人,驿马可在巳、酉、
丑、申年、月发应,那末依次类推:申、子、辰日出生的人,驿马
当在亥、卯、未、寅年、月发应;巳、酉、丑日出生的人,驿马当在
申、子、辰、亥年、月发应。
•116 •
3. 凶神恶煞
〔羊刃〕《星平会海》说:“甲禄到寅,卯为羊刃;乙襟到
卯,辰为羊刃;丙戊禄在巳,午为羊刃,丁己禄在午,未为羊刃;
庚禄居申,酉为羊刃;辛禄到酉,戌为羊刃;壬禄到亥,子为羊
刃;癸禄到子,丑为羊刃。”这就是说,禄过则刃生,判定的方法
是,从自身日柱天干出发,凡是阳干甲见卯,丙戊见午,庚见
酉,壬见子,其地支必定处在禄后一位;同样,阴干乙见辰,丁
己见未,辛见戌,癸见丑,其地支处在禄后一位的,也可以羊刃
视之。
命书认为,羊刃是个性子急躁,刚强凶狠的神。命中男多
羊刃,妻宫有损;女带羊刃,刑夫克子。然而有时也耍具体分
析,如果身弱,那就不一定是凶,因为羊刃有卫禄帮身的职能;
反过来,如果身强则就难免招灾惹祸了。关于身弱身强的看
法,一看日干五行和降生月份之间的旺、相、休、囚、死关系,二
看日干得四柱干支生助的多少,三看日干和年、月、日、时地支
在寄生十二宫中处于什么状态。以上这三种看法,结合有关
章节一查就知。
〔桃花煞〕 在命书中,桃花煞也叫“咸池”。所谓煞,就
是凶神恶煞的意思。命中出现桃花煞,多与酒色有关。看法
是:
寅午戌见卯,
巳酉丑见午,
申子辰见酉,
亥卯未见子。
意即寅、午、戌日出生的,碰上卯年、卯月或卯时,就是犯了桃
花煞。论命诗说:
•117 •
风淫淫冶号咸池,并集来临祸应期,
酒 色相刑 三二位,更加 神煞血光随。
又有命书说「酒色猖狂,只为桃花带煞。”认为有这种命的人,
“多为男女淫欲之征”。其中又分墙里桃花和墙外桃花;如煞
的位置处在年支或月支的,叫做墙里桃花,说明夫妻恩爱,可
不为害,但如果冲破就不好了;煞的位置在时支的,叫做墙外
桃花,或许在外有另有艳遇。假如一个人的八字里没有桃花
煞,而在大运或流年中碰上的,也可被认为是交了桃花运。
其实现实生活中,正人君子碰上桃花煞的也很多,说明桃
花煞的可信程度并不太高。
〔孤辰、孤宿〕 这是一种孤寡伶仃的神煞,如果命里逢
上,大多孤栖独宿,伶仃清苦。按照《三命通会》的说法,孤辰
和孤宿的看法以年柱地支为主,结合月、日 、时的地支来作判
断:
年支逢东方一气寅、卯、辰时,月、日 、时支中出现巳的
叫孤辰,出现丑的叫孤宿;
年支逢南方一气巳、午、未时,月、日 、时支中出现申的
叫孤辰,出现辰的叫孤宿;
年支逢西方一气申、酉、戌时,月、日 、时支中出规亥的
叫孤辰,出现未的叫孤宿;
年支逢北方一气亥、子、丑时,月、日 、时支中出现寅的
叫孤辰,出现戌的叫孤宿。
这里我们不难看出这样一个规律,就是在年干地支东、
南、西、北的合局中,顺数下去一位的是孤辰,倒数一位的是孤
宿。如以北方一气亥、子、丑为例,亥、子、丑顺数的下一位是
寅,倒数的上一位是戌,那末寅就是亥、子、丑年生人的孤辰,
• 118 •
戌就是亥、子、丑年生人的孤宿了。
《烛神经》说:“凡人命犯孤宿,主形孤骨露,面无和气,不
利六亲。生旺稍可,死绝尤甚。驿马并,放荡他乡;空亡并,幼
少无倚;丧吊并,父母相继而亡。一生多逢重丧叠祸,骨肉伶
仃,单寒不利。入贵格,赘婿妇家;入贱格,移流未免。”
当然,这种说法在现实生活中,也是经不起推敲的。
〔亡神〕 这也是个凶神恶煞,命书的说法是:
申子辰见亥,
寅午戌见巳,
巳酉丑见申,
亥卯未见'寅。
如日支为申、子或辰的人,在年支、月支或时支中出现亥字,就
算是亡神入命了。当然,由于月支和时支更贴近日柱的自身,
所以情况要比年柱出现亡神更糟。
“亡神入命祸非轻,用尽机关心不宁”。 然而,亡神如果和
吉神处在一个柱上,也主谋略深算。
〔六甲空亡3 《渊海子平》说:“甲子旬中无戌、亥,甲戌旬
中无申、酉,甲申旬中无午、未,甲午旬中无辰、巳,甲辰旬中无
寅、卯,甲寅旬中无子、丑。”这是说十天干和十二地支互相配
合,成就六十甲子。如果以甲为基准,就有甲子、甲戌、甲申、
甲午、甲辰、甲寅六旬。旬是十,从甲子到甲戌,正好要过乙
丑、丙寅、丁卯、戊辰、己巳、庚午、辛未、壬申、癸酉十天,这说
明天干从甲到癸,地支从子到酉一旬终了,天干正好用完,而
地支还多着两个没给排进去。于是,这戌、亥便就成了甲子旬
中的空亡。比如日柱干支在甲子旬中的,不管是乙丑、丙寅、
丁卯、戊辰、己巳,还是庚午、辛未、壬申、癸酉,只要年、月 、时
•119 •
支中出现亥、戌的,就算是逢上空亡了。同样道理,在甲戌旬
中见申、酉,在甲申旬中见午、未,在甲午旬中见辰、巳,在甲辰
旬中见寅、卯,在甲寅旬中见子、丑的,也都叫作空亡。大凡看
命,如果喜神落在空亡位置上的,那就虚而不实,空欢喜了一
场。相反,如果忌神落在空亡位置上的,那就又可转忧为喜
了。

六甲空亡一览表:

【120页表】

以上空亡之中,又有“真空工“半空”「填空”「坐空”等说
法。其中失时为“真空”,得时为“半空”,行运逢原空之神为
•120・
“填空”,年、月、日、时四干在空支之上为“坐空”。 此外又有
“阳日空阳,阴日空阴”的说法。
〔十恶大败〕《三命通会》说「十恶者,犯十恶重罪,在所
不赦;大败者,譬兵法中,与敌交战,大败无一生还,喻极凶
也。”看法是以出生年份,结合生日来定。具体情况是:
庚成年见甲辰日,
辛亥年见乙巳日,
壬寅年见丙申日,
癸巳年见丁亥日,
甲戌年见庚辰日,
甲辰年见戊戌日,
乙亥年见辛巳日,
乙未年见己丑日,
丙寅年见壬申日,
丁巳年见癸亥日。
以上十种,不管月、时怎样,都是不好的命。原因是这十天出
生的人,不但年支和日支彼此相冲,并且禄地碰上空亡。譬如
甲辰、乙巳日出生的人,甲禄在寅,乙禄在卯,而从“六甲空亡”
表中我们可以查知,寅、卯正是甲辰旬的空亡。其他类推。
命书看十恶大败,不结合月支、日支综合考虑,所以可信
度不治
〔四废日〕
四废日是说一年四季,每个季度只有一天,作为废日。情
况为:
春天庚申日生,
夏天壬子日生,
•121 •
秋天甲寅日生,
冬天丙午日生。
但在《三历会同》中,又在四废日的基础上增添内容为:
春天庚申、辛酉日生,
夏天壬子、癸亥日生,
秋天甲寅、乙卯日生,
冬天丙午、丁未日生。
所谓“废”,就是“囚死无用”的意思。春天木神用事,金囚
无用,所以金废;夏天火神用事,水囚无用,所以水废,秋天金
神用事,木囚无用,所以木废;冬天水神用神,火囚无用,所以
火废。
大凡命带四废,主要为作事不成,有始无终。其他也没有
什么大的了不起。
[天罗地网〕
关于天罗地网,《渊 海子平》这样说道:
火命人逢戌、亥,是为天罗,
水、土命火逢辰、巳 ,乃为 地网。
为什么要以戌、亥为天罗,辰、巳为地网呢?《三命通会》
解释为,“盖天倾西北,戌、亥者,六阴之终也;地陷东南,辰、巳
者,六阳之终也。阴阳终极,则暗昧不明,如人之在罗河,此其
义也
关于天罗地网的不吉情况,《渊海子平》这样认为「男子
忌之于天罗,女子忌之于地网。多生执滞,如恶杀必至死亡。”
看来好凶,其实也没有什么道理。至于“金、木生人”,则没有
天罗地网的说法。
除了以上神煞之外,还有学堂、金舆禄、红艳煞、元辰,以
•122
及六厄、勾绞、天地转煞等等一些无稽之谈,如唐朝韩愈就曾
自叹命宫磨蝎,让磨蝎宫的凶星闯到命宫里来,哪还会有好的
结果?
为了明白起见,现将日干和地支,日支和地支之间的神煞
关系,列表如下,其他省略。
日干见地支神煞一览表

【123页表】

日支见地支神煞一览表

【124页表】

从以上所列星宿神煞看,多半有点硬性规定的味道,因为
它不从五行生克宜忌着手分析,所以后来大多数的命理学家,
都不赞成单用神煞来断定一个人一生的吉凶祸福。清代命理
学家陈素庵《命理约言》卷三曾批驳神煞说:“旧书称神煞一百
•124 •
二十位,一一细推起例,毫无义理者,十尝七八,且一字每聚吉
凶神煞十余,祸福何以取断?此皆术家逞臆妄造。每一书出,
则增数种,欲以何说惑人,即立何等名色,往往数煞只是一煞。
尝稽历日所载,尚多相沿之弊,何况通书命书乎?”又说J不知
人命吉凶,皆由格局运气,安可以偶合神煞而信之?即如桃
花、流霞、红艳等煞,为男女淫欲之征,然端人正士,烈妇贞女,
犯之者甚多,况桃花煞亥卯未在子,寅午戌在卯,巳酉丑在午,
申子辰在酉,皆五行生印,流霞煞如乙遇申乃正官,丙遇寅乃
长生,辛遇酉乃禄神,何所见其淫亵乎?且春花无不妖冶,何
独桃为淫花?干支字面相见,有何红色艳态?”桃花之外,书中
又否定华盖说「又太岁三合之墓,谓之华盖,或以为文章,或
以为孤高,亦不足凭也。”为此,陈素庵毫不含糊地指出,所谓
“神煞诞妄,皆此类也”,然而要去一一批驳,又太费笔墨,所以
“达理之士,自当晓然耳二 再如当代命理学家陈宝良也明确
认为「五行为本,神煞为末」

\section{命理中五行和干支的分析}
我们有了前面各篇所说的有关基础,就可从所排四柱的
八字入手看命了。
因为在年、月 、日、时四柱中,日柱的干支是管自己的,所
以命理中一切的一切,都得从日柱干支所含五行出发,进行分
析推论。
从总的方面着眼,有关日干五行荣枯的大致情况,《渊海
•125-
子平》有着这样的诗诀:
〔十天干体象〕
甲木
甲木天干作首排,原无枝叶与 根芟。
欲存天地千年久,直向 沙泥万丈埋。
断就栋梁金得用,化成 灰炭火为灾。
蠢然块物 无机事,一任 春秋自往来。
乙木
乙木根茬种得深,只宜阳地不宜阴。
漂 浮最怕 多逢水,刻断何当若用金。
南去火灾灾不浅,西行 土重祸尤侵。
栋梁不是连根本,辨别 工夫好用心。
丙火 ,
丙火明明一太阳,屈从X大立纲常。
洪光不独窥千里,臣隗尤能遍八荒。
出 世肯为 浮木子,传生不作湿泥娘。
江 湖宛水安能克,惟怕成林木作殃。
丁火
丁火其形一烛灯,太阳相见夺光明。
得时能化千金铁,失令难熔一寸金。
虽少干柴 渚可引,纵多 湿木不能生。
其间衰旺当分别,旺比一炉衰一渠。
戊土
戊土城墙堤岸同,振江 河海要 根重。
柱中带合形弥壮,日下乘虚势必崩。
力薄不胜金漏泄,功成安用木疏通。
•126
平生最要东南健,身旺东南健失中。
己土
己 土田园属四维,坤深万物为之谋。
水金旺处身还弱,火土功成局最奇。
失令岂能埋剑戟,得时方可用磁基。
谩夸即旺兼多合,不遇刑冲总不宜。
庚金
庚金顽钝性偏刚,火制功成怕火乡。
夏产东南过纹炼,秋生西北亦光芒。
水漂反是他相克,木旺能令我自伤。
我 己干支重遇土,不逢冲破即埋藏。
辛金
辛金珠玉性虚灵,最受阳河沙水清。
成就不劳炎火煨,费扶 偏爱湿泥生。
水多火旺宜西北,水冷金寒要丙丁。
坐禄通根身旺地,何愁埋上没其形。
壬水
壬水汪洋并百川,漫流天下总无边。
干支多聚成飘荡,火土重逢木水源。
养性结胎需未午,长生 归禄属乾坤。
身 强原是无财禄,西北行程厄少年。
癸水
癸水应非雨露么,根通亥子即江河。
柱无坤坎还身弱,局有财官不尚多。
申子辰全成上格,寅午戌备要中和。
假饶火土 生深夏,西北行程岂太过?
•127
由于诗的语言简炼概括,并且惯常还带有一定的模糊性,
所以我们这里稍许再行展开,作点分析:
凡是日主属木的,必先看一看木势的盛衰。木重水多的
是盛,这样就宜金砍木,如果金不足的逢土也好,因为一方面
土能生金制水,另一方面木又能克土,使木自身的盛势有了阻
抑的处所。木微金强的是衰,这就宜火制金,火少的逢木也
好,因为木与木为同类,既可壮大自己声势,又可生火制金。至
于水太盛则木漂,取土制水为上,没有土的有火也好,因为火
能生土。土太重则木弱,难以制约,取木帮助克土为上,没有
木的有水也好,因为水能生木制土。火太多则木焚,取水制火
为上,没有水的有金也好,因为金能生水。
凡是日主属火的,必先辨别一下火力的有余与不足。火
炎木多的是有余,这样就宜水济火,如果水不足的逢金也好,
因为金能生水制木,另一方面火又能克金,使火自身的有余有
了发挥的处所。火弱水旺的是不足,这就宜土制水,水不足的
逢火也好,因为火与火是同类,既可壮大自己声势,又可生土
制水。至于木太多则火炽,取水制火为上,没有水的有金也
好,因为金能生水。金太多则火弱难以制约,取火帮助克金为
上,没有火的有木也好,因为木能生火。土太多则火晦,取木
生火制土为上,没有木的有水也好,因为水能生本。
凡是日主属土的,必先辨别一下土质的厚薄。土重水少
的是厚,这就宜木疏土,假如木弱的逢水也好,因为水能生木
制土,另一方面土又能够克水,使土自身塞滞的情况得以改
善。土轻木盛的是薄,这就宜金制木,金不足的逢土也好,因
为土与土是同类,既可壮大自己声势,又可生金制木。至于火
多则土焦,取水制火为上,没有水的有金也好,因为金能生水。
•128 •
水太多则土流而难以保持,取土帮助克水为上,没有土的有火
也好,因为火能生土。金太多则土弱,取火生土制金为上,没
有火的有木也好,因为木能生火。
凡是日主属金的,必先辨别一下金的老嫩。金多土厚的
是会,这就宜火炼金,假如火衰的逢木也好,因为木能生火制
金,另一方面木又能克土,从而使土不能生金太过,同时,金又
能克木,使重金有所发挥。木重金轻的是嫩,这就宜土生金,
假如土不足的逢金也好。至于土多则金埋,取木制土为上,没
有木的有水也好,因为水能生木。水太多则金沉而难以出头,
取土帮助克水为上,土少的有火也好,因为火能生土。火太烈
则金销,取水制火为上,水不足的有金也好,可以壮大自身力
量。
凡是日主属水的,必先辨一下水势的大小。水多金重的
是大,这就宜土御水,假如土弱的逢火也好,因为火能生土以
制水,另一方面火又能克金,从而使金不能生水太多,同时,水
又能克火,使水有所发挥。水少土多的是小,这就宜木克土,
如果木弱的逢水也好。至于金多则水浊,取火克金为上,火少
的有木也好,因为木能生火。火太炎则水灼而易于消耗,取水
制火壮大自身为上,水少的有金也好,因为金能生水。木太多
则水缩,取金生水为上,金不足的有土也好,因为这样可以引
木克土,分散木对水的贪汲。
对于命学原理来说,算命时分析八字中五行盛衰,是断定
一个人一生荣枯,走运不走运的必不可少的重要前提。对此,
《三命通会》引徐大升的话总结道:
金赖土生,土多金埋;土 赖火生,火多 土焦;火赖木
生,木 多火炽,木 赖水生,水 多木漂,水 赖金生,金多水浊。
•129 •
金能生水,水多金沉;水能生木,木盛水缩;木能生
火,火多木焚;火能生土,土多 火晦;土能生金,金多 土变。
金能克木,木坚金缺, 木能克土,土重 木折, 土能克
水,水多土流,水 能克火,火炎水热;火能克金,金多火熄。
金衰遇火,必见销熔;火 弱逢水,必为 熄灭;水弱逢
土,必为淤塞r土衰遇木,必遭倾陷;木 弱逢金,必为 砍折。
强金得水,方 到其锋 强水得木,方泄其势, 强木得
火,方化其顽;强 火得土,方止其焰;强 士得金,方制其害。
以上这种分析,都建立在对自身五行太过不及的扶抑上面,可
见对于一个人来说,最好是能求得五行的动态平衡,否则太过
不及,没有相应可以作为喜神、用神的五行来作恰到好处的补
救,那就糟了。
这种分析,如再结合京图《滴天髓》的那套学说作为参照
补充,那就更加全面,更加深细,更加灵活。
甲木参天,脱胎要火。春不容金,秋不容土。火炽乘
龙,水荡骑虎。地阔天和,植立千古。
这段针对甲木说话的原理为:
“甲木参天,脱胎要火 甲木为体本坚固,参天雄壮的纯
阳之木,若生于春初馀寒未尽的寅月,那就木嫩气寒,需耍得
火温养,才能舒叶展枝,敷荣发生,否则便无法欣欣向荣,成其
气势。
“春不容金”。若甲木不生于初春寅月而生于仲春卯月木
势极旺之时,旺极者宜泄而不宜克,所谓“强木得火,方化其
顽",更何况春令克木的金正处于无能为力的休囚状态,所以
此时欲以衰金而克旺木,其后果就必然是木坚金缺而不为所
容了。
•130
“秋不容土”。任铁樵《滴天髓阐微》说:“(甲木)生于秋,
失时就衰,但枝叶虽凋落渐稀,根气却收敛下达,受克者土。秋
土生金泄气,最为虚薄。以虚气之土,遇下攻之木,不能培木
之根,必反遭其倾陷,故秋不容土也。”
“火炽乘龙”。日干甲木逢四柱干支火势炽盛的,那末其
所乘日支,就以坐在生木泄火的水库辰宫为宜。对此,任铁樵
阐述原理说:“柱中寅、午、戌全,又透丙、丁,不惟泄气太过,而
木且被焚,宜坐辰。辰为水库,其土湿,湿土能生木泄火,所谓
'火炽乘龙'也。”
“水荡骑虎”。 日干甲木逢四柱干支水势浩荡的,一那末其
所骑日支,就以坐在纳水坚木的寅宫为宜。任铁樵对此亦阐
述原理道:“申、子、辰全,又透壬、癸,水泛木浮,宜坐寅。寅为
, 火土生地,木之禄旺,能纳水气,不致浮泛,所谓 '水宕(荡)骑
虎'也。” •
乙木虽柔,到羊解牛。怀丁抱丙,跨凤乘猴。虚湿之
地,骑马亦忧。藤萝系甲,可春可秋。
其所述原理可作这样的分析:
“乙木虽柔,到羊解牛”。羊在地支为未,牛在地支为丑,
所以“到羊解牛”,特指乙木生于未日丑日,或未月丑月。由于
未为木库,乙木得以蟠根,丑为湿土,乙木可以受气,所以不足
为忧,适堪为喜。
“怀丁抱丙,跨凤乘猴”。 日干乙木,若四柱透出丙、丁火
的,因为火能制金,所以虽然出生在典或申月理月金旺之
时,蛇退四 十二地支中,酉为鸡,席凝。这里,京图把鸡
美称为凤,所以才有“跨凤乘猴”的说法。
“虚湿之地,骑马亦忧”。“虚湿之地”指日干乙木,生于子
•131 •
月亥月,四柱天干既无丙、丁透出,而地支之中又乏未、戌燥
土。如若这样的话,那末即使日支坐在属马的午火之上,也无
济于事而难以生发。
“藤萝系甲,可春可秋”。 乙木为藤萝之木,甲木为松柏之
木。日干乙木之人,四柱中如遇天干透甲,地支藏寅的,就可
谓之藤萝系于松柏了。由于若此则乙木有所扶助,所以非但
可春可秋,并且生于四季也不足忧。
丙火猛烈,欺霜侮雪。能烦庚金,逢辛反怯。土众生
慈,水猖显节。虎马犬乡,甲来焚灭。
其所说原理大致是这样的:
“丙火猛烈,欺霜侮雪”。在甲、丙、戊、庚、壬五阳干中,唯
独丙火秉阳之精,而为阳中之阳,所以灼阳之至,不畏秋而欺
霜,不畏冬而侮雪。任铁樵说J丙乃纯阳之火,其势猛烈,欺
霜侮雪,有除寒解冻之功。”
“能烦庚金,逢辛反怯 丙火克制庚金,所以能燃;丙辛
合而化水,所以反怯。对此刘基有云:“庚金虽顽,力能燃之;
辛金本柔,合而反弱」
“土众生慈,水猖显节二 土为丙火之子,所以日主丙火而
四柱中土多的,其人有慈爱之心。水为丙火的克星,为此日元
丙火而四柱中水多的,其人多眼范之风。
“虎马犬乡,甲来焚灭”。焉谓“虎马犬乡”,就是日下丙
火,支合寅、走出局的,由于这时火势已经过于猛烈,所以
如果再见甲木来生,反而连得甲木也一起被焚灭了。
对于以上丙火的情况,任铁樵总括要领道「由此论之,泄
其威,须用己土I遏其炎,必要壬水;顺其性,还须辛金。己土
卑湿之体,能收元阳之气,戊土高燥,见丙火而焦拆矣。壬水
•132 •
刚中之德,能制暴烈之火;癸水阴柔,逢丙火而燥干矣。辛金
柔软之物,明作合而相亲,暗化水而相济I 庚金刚健,刚又逢
刚,势不两立。此虽举五行而论,然世事人情,何莫不然?”
丁火柔中,内性昭融。抱乙而孝,合壬而忠。旺而不
烈,衰而不穷。如有嫡母,可秋可冬。
“丁火柔中,内性昭融”。 丁干属阴,火性属阳,所以有柔
而得中,外柔顺而内文明之象。任铁樵说「丁非灯烛之谓,较
丙火则柔中耳。内性昭融者,文明之象也。”这是就丁火的性
质而说的。
“抱乙而孝,合壬而忠 乙木为丁火的父母,最怕七煞辛
金克制,现在丁火制伐辛金,明使辛金不能损伤乙木,所以其
孝异乎常人。壬水为丁火的正君,最怕偏官戊土制约,可是丁
壬合而化木,暗使戊土不能欺侮壬水,所以其忠也异乎常人。
“旺而不烈,衰而不穷”。由于丁火的性质,有柔中昭融之
美,所以时当乘旺,不至赫炎,时值就衰,不至熄灭,而无太过
不及的弊端。
“如有嫡母,可秋可冬”。丁火的嫡母为甲木,所以刘基注
曰「生于秋冬,得一甲木,则倚之不灭,而焰至于无穷也。”而
任铁樵则把乙木也囊括了进来说:“干透甲、乙,秋生不畏金;
支藏寅、卯,冬产不忌水。”
戊土固重,既中且正。静翕动辟,万物司命。水润物
生,火燥物病。若在艮坤,怕冲宜静。
这种说法的解释为:
“戊土固重,既中且正”。刘基认为,戊土并非城墙堤岸之
土,只是与己土相比之下,戊土属于高原刚燥之土,并为己土
的发源地,所以说是“得乎中气而且正大矣”。
• 133 •
“静翕动辟,万物司命”。土为万物之母,春夏气动而辟则
万物发生,秋冬气静而翕则万物收藏,所以人们把它说成为万
物的司命。
“水润物生,火燥物病”。对于水润火燥,物生物病,要根
据出生月份和其他坐支等具体情况,进行分析。任铁樵认为:
“(戊土)其气高亢,生于春夏火旺,宜水润之,则万物发生,燥
则物枯。生于秋冬水多,宜火暖之,则万物化成,湿则物病」
“若在艮坤,怕冲宜静二 寅在八卦居于艮位,审在八卦居
于坤位,所以“若在艮坤”两句,是说出生在寅月或申月,因为
春则戊土因克致伤,故而气虚宜静,秋则戊土因泄受损,故而
体薄怕冲。此外如日干戊土坐寅的怕申来冲,坐申的怕寅来
冲,由于冲则根动,所以也怕冲宜静。
己土卑湿,中正蓄藏。不愁木盛,不畏水狂。火少火
晦,金多金光。若要物旺,宜助宜帮。
其具体情况为:
“己土卑湿,中正蓄藏”。己土为戊土枝叶之地,其性质虽
属阴湿之土,可是却也中正蓄藏,贯八方而旺四季,对于万物
有滋生不息的妙用。
“不愁木盛,不畏水狂二 己属柔土,能够生木,所以不愁
木盛制克。己属深土,能够纳水,所以不愁水狂冲荡。
“火少火晦,金多金光”。对于这里的文意,刘基这样解释
道「无根之火,不能生湿土,故火少而火反晦。湿土能润金
气,故金多而金光彩,反清莹可观,此其无为而有为之妙用。”
但任铁樵则认为,“火少火晦者,丁火也,阴土能敛火晦火也。
金多金光者,辛金也,湿土能生金润金也。”
“若要物旺,宜助宜帮”。若要万物充盛长旺,必须四柱土
•134
气深固,又得丙火暖和,去其阴湿之气,否则卑薄软湿之土,又
何以能够滋生万物呢?
庚金带煞,刚健为最。得水而清,得火而锐。土润则
生,土干则脆。能赢甲兄,输于乙妹。
究其原委:
“庚金带煞,刚健为最 对于这句话的解释,刘基这样说
道:“庚金乃天上之太白,带杀而刚健。”任铁樵氏也说「庚乃
秋天肃杀之气,刚健为最J
“得水而清,得火而锐 庚金得水,则气流而清,足以引
通刚杀之性,使之淬厉晶莹;庚金得火,则气纯而锐,然后冶炼
熔铸,使成锋棱如霜的剑戟之器。
“土润则生,土干则脆”。庚金逢湿土则生,逢燥土则脆。
在辰、戌、丑、未四库中,辰、丑为湿土,戌、未为燥土,原因是
辰、丑含水,戌、未含火,所以有湿燥之分。
.“能赢甲兄,输于乙妹”。就甲、乙木而言,甲为兄,乙为
妹。甲木虽强,但以庚金之力,足以克伐;乙木虽柔,以阴木而
合阳金,所以非但不克,反而转觉有褚。
辛金软弱,湿润而清。畏土之叠,乐水之盈。能扶社
稷,能救生灵。热则喜母,寒则喜丁。
这些话的原理为:
“辛金软弱,湿润而清”。 刘基认为,辛为阴金,并非象有
些人所说的属于珠玉之金,任铁樵则说:“辛金乃人间五金之
质,故清润可观
“畏土之叠,乐之水盈”。日元辛金,四柱中如若土势重
叠,则涸水埋金,反为不美;若水流充盈,则润土养金而秀。然
而话虽如此,但是也要根据具体情况,不可一概而论。
135 •
“能扶社稷,能救生灵,对于这句话的解释,任铁樵说得
很有意味:“辛为甲之君也,丙火能焚甲木,合而化水,使丙火
不焚甲木,反有相生之象;辛为丙之臣也,丙火能生戊土,合丙
化水,使丙火不生戊土,反有相助之美,岂非扶社稷、救生灵
乎?”
壬水通河,能泄金气。刚中之德,周流不滞。通根透
癸,冲天奔地。化则有情,从则相济。
其精神大致为:
“壬水通河,能泄金气 壬为通于天河的阳水,为癸水的
发源之地,长生于申,申属金,位于天河之口,所以壬水能泄西
方肃杀的金气。
“刚中之德,周流不滞”。 壬水能泄西方肃杀的金气,所以
称其为“刚中之德”。 壬又为从天而降的昆仑之水,百川之源,
所以有“周流不滞”的说法。
“通根透癸,冲天奔地”。 壬水日干,四柱地支会申、子、辰
水局而天干又透出癸水的,那就其势冲天奔地而不可阻遏了。
对此任铁樵说:“如申、子、辰全,又透癸水,其势泛溢,纵有戊、
己之土,亦不能止其流。若强制之,反冲激而成水患;必须用
木泄之,顺其气势,不至于冲奔也。”
“化则有情,从则相济”。“化”为壬丁化木,又生丁火,所
以有情「从"为身弱从旺,如四柱中火旺从火,土旺从土,这时
也未尝没有相济之功。任铁樵说「合丁化木,又能生火,不息
之妙,化则有情也。生于四、五、六月,柱中火、土并旺,别无金
水相助,火旺透干则从火,土旺透干则从土,调和润泽,仍有相
济之功也。”
癸水至弱,达于天津。得龙而运,功化斯神。不愁火
• 136 •
土,不论庚金。合戊见火,化象斯真。
文中含义为: 」/
“癸水至弱,达于天津 癸水为源远流长的纯阴至弱之
水,并不单指为雨露之水,刘基原注说:“癸水乃阴之纯而至
弱,故扶桑有弱水也,达于天津。”
“得龙而运,功化斯神”。癸水随天而运,得龙而成云雨,
变化不测,龙就是“辰,日干癸水,四柱中天干有甲、乙,地支
有寅、卯的,都能运水气以生木制火,润土养金而定为贵格。
“不愁火土,不论庚金。合戊见火,化象斯真。”癸水至弱,
若癸日生人,四柱中见火、土多的,便就从而化之,所以说是
“不愁火土”。癸水弱而难泄金气,所谓“金多反浊”,故而“不
论庚辛”。 十天干中虽说戊癸合火,然而也要月支藏火,柱中
透出丙、丁,如此则引出化神,才能化得真实;否则生于秋冬金
水旺地,即使支遇辰龙,天干透出丙、丁,也难以从化了。

\section{日干、格局和干支合化刑冲的看法}

入手看命,按照命理学家的常规是先看日干,因为这是代
表自身的一个天干,凡是年、月 、日 、时四柱中的干支,都要围
着这个天干来论定吉凶宜忌。日干有得时和失时的不同,如
日干碰上任、相的月支,就是得时;碰上休、囚、死的月支,就是
失时。比如日干是甲木,木生于春,水能生木,所以月支如果
碰上春月,就属于旺,碰上冬月,就算是相,都属得时;如果日
干甲木不生于冬春之月,而偏偏生在木能生火,火生木休的夏
•137 •
月,木能克土,土旺木囚的四时末一个月,也就是三、六、九、
十二月的土月,甚至生在金能克木、金盛木死的秋月,就都属
于失时。得时的自身强旺,不得时的自身衰弱。关于五行和一
年四季旺、相、休、囚、死的关系,前面我们已有专篇谈过,一阅
便知。此外,.观察日干和月干的关系,还有利于我们对一个人
八字格局的认定。看过日干和月支的关系后,再看日干下坐
的是哪一个地支,这地支对日干来说,在寄生十二宫中处于什
么样的状态?是长生、沐浴、冠带、帝旺,还是衰、病、死、墓、
绝、胎、养? 此外,还不要忘了看看与日柱干支紧贴的左邻右
舍月柱和时柱的干支,以至于年柱的干支,这些干支所代表的
阴阳五行,对于自身日干来说,生克扶抑的情况如何?
这种看法说实在点,就是在日干为主的基础上,以年柱为
根,可以知世脉的盛衰;以月柱为苗,可以知父母亲荫庇的有
无,兄弟的得力不得力;以日柱为己身,日支为妻子,可以知妻
子的贤惠不贤惠;以时柱为花实,可以知小辈的兴旺不兴旺。
这里,要紧的是,我们还千万不要忘记根据日干五行所需
要的生克扶抑取 用神,然后再看看这用神喜的是什么,忌
的是什么。这样才能通盘考虑,以下论断。现将命中日干、格
局和干支的合化刑冲的看法具体解析如下:
1. 先看日干强弱
日干的名称很多,分别有日主、命主、身主、日 元、日神等
叫法。在一个人的八字里,日 干的地位是最为举足轻重的,因
为日干代表的是一个人的本身。因此从这点出发,首先判定
一个人自身日干的衰旺强弱,就成了看命的首要条件。
大凡判定一个人日干强弱的方法,主要有三点。第一,看
日干在所生的月份得令还是不得令6 比如日干甲、乙见月支
・138・
寅、卯,丙、丁见月支巳、午,戊、己见月支巳、午或辰、戌、丑、
未,庚、辛见月支申、酉,壬、癸见月支亥、子,都属于最佳的得
令生旺状态,所以这日干就强。反之,日干在出生的月令中,
如果处于一种或休、或囚、或死的状态,那就是弱。第二,日干
在四柱中得到的生助是多还是少。比如日干属甲、乙木,如四
柱中得水木之助多的,就是旺而得势。反之,日干甲、乙木得
不到四柱中水木之助,甚至反遭金制火泄,那就弱而失势了。
第三,把自身日干对照四柱地支,如果碰上长生、沐浴、冠带、
临官(禄)、帝旺或墓库的,就是得地得气,自身自然强旺。反
之,就是失地失气,强旺不起来。以上得令、得势、得地三者全
都集于一身的,日干处于极旺状态;反过来,失令、失势、失地
三者全都集于一身的,日干处于极弱状态。又有旺、强、中、
衰、弱之分。旺为日干处于极旺状态,强为日干处于较强状
态,中为日干处于中和状态,衰为日干处于较弱状态,弱为日
干处于极弱状态。关于日干旺、强、衰、弱的扶抑原则,大致为
旺极宜泄,强者宜克,衰者宜扶,弱者宜抑。且看举例:
C日主极旺的命〕
(年) 比肩甲寅禄
(月) 伤官丁卯乙木帝旺
甲子癸水沐浴
(时) 比肩甲子癸水沐浴
这一命造,日干甲木生于卯月仲春,处于帝旺状态,所以
得令。甲木在四柱中,生它的有日支和时支两个癸水作为印
绶,和它同类的有年干、时干两个甲木作为比肩,以及月支卯
中乙木作为劫财,所以得势。甲禄到寅,年支寅为甲的禄,月
支卯对甲说,使甲处在帝旺的状态,所以得地。这命甲木得
• 339 •
令、得势、得地,三者兼得,所以日主极旺。
〔日主较旺的命〕
(年) 戊戌
(月) 甲子 ,
(日) 己巳
(时) 己巳
这一命造,日干己土生于子月仲冬绝地,为不得令。可是
由于日支时支巳火,为日干己土的帝旺之乡,而年支戌土又为
己土的养地,所以得地。加之四柱干支比劫重重,有印生扶,
所以得势。纵观命造全局,得地得势,由弱转强,所以当取月
干正官甲木作为用神,这就是命书所说的身强堪任财官。
(日主极弱的命〕
(年) 偏财戊申绝
(月) 七杀庚申绝
(日) 甲午死
(时) 七杀庚午死
这一命造,日主甲木,生于木绝的初秋申月,所以不得时
令。甲木在四柱中,月柱庚申和年支、月支申金,都是克它的
七杀,而日支、时支两个丁火又拼命的泄它,加上没有比劫为
助,所以失势。甲木在年、月 、日、时四个地支中,都处于死绝
的状态,所以失地。失令、失势、失地,三者都丧失尽净,所以
是个日主极弱的命。
(日主较弱的命】
(年) 壬戌
(月) 甲辰
(日) 戊寅
• 140 •
(时) 乙卯
这一命造,日主戊土,生于冠带辰月,辰又为土,并且降生
的那天正好处在立夏前 18 天的土旺季节,所以得令。可是戊
土虽然得令,但地支寅卯辰会成东方木局,而天干甲木又透出
月干,加之除了年支戌土之外,缺少印比为助,故而遍览全局,
在强旺的木势制约下,命主自属处于相对的弱势。
(日主中和的命〕
(年) 劫财甲寅帝旺
(月) 偏印癸酉绝
(日) 乙亥死
(时) 伤官丙子病
这一命造,日主乙木,生在木绝的仲秋酉月,所以不得时
令。乙木在四柱中,得月干、日支、时支和年柱水木的生助,所
以得势。乙木在月支和日支中虽处于绝、病之地,可是年支帝
旺得气,所以中和。综合以上失时,得势,地气得失偏于中和的
分析,所以是个日主中和或偏强的命。
纵观日主强弱的情况,陈素庵在《看日主法》中提出白己
的主张说:
旧书论日主,或专主强旺,或 反尚 衰弱,盖以太强则
得抑有力,太 弱则得扶立效,此即“有病方为贵”之说,皆
偏见也。凡日 主最贵中和,自然吉多凶少,日 主太强太弱,
自然吉少凶多,惟可 抑之强,可扶之弱,则存乎作用耳。作
用之法,如 木日 强则用金克之,用火泄之;木日弱则水生
之,用木助之。若得土而杀其势,亦所以抑之;借土而培
其根,亦所以扶之,其要归 诸中和 而已。旧谓男命日主不
嫌于强,然过 强则亦取咎,女命日主不嫌于弱,然过弱亦
•141 •
受亏。至于日主所坐之支,较为 亲切,但坐财官等吉神,
亦须四柱透露扶助,坐 伤劫等凶神,四柱亦能伐而去之,
非坐下一支,遂定休咎也。
2. 次看命中格局
在四柱命理学中,看取格局也是不可忽视的一个重要环
节。虽然对于这个环节各有各的看法,有的命理学家认为丢
掉格局同样可以看命,然而在大多数情况下,能够看取格局,
总比完全丢掉格局要强得多。按照命书说法,格局有正格和
变格的不同,正格有正官、七杀、正财、偏财、正印、偏印、食神、
伤官八种,如果并掉财、印两格的正偏,也有六种,至于变格,
那就千变万化,较难捉摸了。
那末,怎样来具体看格局呢?首先,可采用“月支藏干”的
原则来看取格局。所谓“月支藏干”,就是月柱的地支隐藏着
什么天干的意思。在采用这一原则时,先要看一看月支所藏
的天干本气有没有透到月干或年干、时干上去。如果有的,譬
如寅月天干透甲,卯月天干透乙,辰月天干透戊,巳月天干透
丙,午月天干透丁,未月天干透己,申月天干透庚,酉月天干透
辛,戌月天干透戊,亥月天干透壬,子月天干透癸,丑月天干透
己,都可根据这一透出天干和日柱天干之间的生克关系,取为
格局。如月支透出是正财的,就是正财格;月支透出是偏财的,
就是偏财格;月支透出正官的,就是正官格;月支透出偏官的,
就是偏官格;月支透出印绶的,就是印绶格,月支透出偏印的,
就是偏印格;月支透出伤官的,就是伤官格;月支透出食神的,
就是食神格等。其次,对于子月、卯月、酉月等月支中只含一
个本气天干的,如果这种本气不在年、月 、时等柱上透发出来,
也可根据月支和日干之间的关系取为格局。第三,如果月支
• 142 •
所藏属于本气的天干没有在月、时、年等三柱上透发出来,那
末再看月支所藏的其他天干有没有透出来的,比如月支寅的
本气是甲木,然而甲木如果没有上透天干,而其中所藏的丙火
或戊土倒有透出来的,那就也可根据丙火或戊土与日柱天干
之间的关系,取为格局,至于到底取丙取戊,这就又要看两者
的力量大小了。第四,如果月支本气和所藏的其他五行一个
也没有透出天干,那就根据月支所藏各干,比较它们之间的强
弱盛衰,挑选其中一个较为得力的,然后再根据这一天干和日
干之间的关系,取为格局。此外,如果月支藏干和日柱之间的
关系属于比、劫、禄、刃的,则一般不取为正式格局,而要特别
取为变格了。比如甲日寅月,乙日卯月,丙日巳月,丁日午月,
戊日巳月,己日午月,庚日申月,辛日酉月,壬日亥月,癸日子
月,由于甲禄在寅,乙禄在卯,丙禄在巳,丁禄在午,戊禄在巳,
己禄在午,庚禄在申,辛禄在酉,壬禄在亥,癸禄在子,所以都
可撇开其他正格,取为建禄的变格。对于以上看取格局的办
法,也少不得举例说明,以求刨根究底。
〔命造举例〕
(年) 辛丑
(月) 正官戊戌戊土 辛金 丁火
(日) 癸未
(时) 壬子
这一命造,生于癸日,而月支戌中藏有戊土、辛金、丁火,
其中少 土透出月干 .辛金透出年干,由于戌的本气是戊土,所
以得取戊土来定格局。对于癸水来说,戊土是克它的正官,所
以这命的格局是F官格。
〔命造举例〕
•143 •
(年) 甲辰
(月) 丙子正官
丙申
(时) 己亥
这一命造,生于丙日,而月支子中藏有癸水,因为子、卯、
酉三支所藏只有本气,所以根据上述取格原则的第二条,按照
癸水和丙火之间形成的正官关系,可径取为正官格。
〔命造举例〕
(年) 己巳
(月) 壬申庚金 壬水 戊土
(日) 丙辰
(时) 己丑
这一命造,生于丙日,而月支申中藏有庚金、壬水、戊土,
其中申的本气庚金没能透出年、月、时三柱,而只有壬水透出
月干,所以根据丙火和壬水之间阳彼克阳我者为偏官的关系,
取格局为偏官格。
〔命造举例〕
(年) 甲寅
(月) 壬申庚金 壬水 戊土
(日) 壬申
(时) 乙巳
这一命造,生于壬日,而月支申中藏有庚金、壬水、戊土,
其中壬水虽然透出月干,可是因为月干和日干之间属于比肩
关系,所以不取为格。再看申中庚金、戊土,由于庚金属于申支
的本气,力量显然超过戊土,所以便取庚金和壬水之间的偏印
关系,定格局为偏印格。
•144 •
至于命中尚有其他种种名目繁多的格局,我们另立专篇
再谈。
3. 三看干支的刑冲化合
八字中天干和天干,地支和地支之间的刑冲化合,因为对
于命局的阴阳五行,有着不可忽视的影响,所以也深为命理学
家所重视。其看法大致是:
〔两干相合,贵乎得中〕比如甲己合土,彼此地支都乘生
旺,这就中而不偏。如果甲太庇,己太柔,这样一者太过,一者
不及,就不中和了。
〔阳得阴合,阴得阳合〕 命书说「天干合,阳得阴合,福
慢;阴得阳合,福紧。”比如阳干甲得阴干己合为财,阴干己得
阳干甲合为官,虽都是福,可是又有前者福慢后者福紧的不
同。又有认为,命中合多,性喜淫乐,所以女命最忌合多,然而
对于甲己和乙庚的彼此相合,又为女命所不忌。
〔两干争合,阴阳偏枯〕 如果碰上两个天干和一个天干
相合,这在命书中称为阴阳偏枯。比如两甲合一己,或者两己
合一甲,就好比夫多妇少,妇多夫少一样,难免相争相妒,用情
不专,所以不是好事。
1日干合化,通根乘旺〕 这是说日干与年、月、时等天干
的相合,要生在本干五行生旺的月份,这样就旺而有根了。比
如甲己合而化土,必须生在辰、戌、丑、未土旺的月份;乙庚合
而化金,必须生在巳、酉、丑月或者申月金旺的月份;丙辛合而
化水,必须生在申、子、辰月或亥月水旺的月份, 丁壬合而化
木,必须生在亥、卯、未月或者寅月木旺的月份;戊癸合而化火,.
必须生在寅、午、戌月或者巳月火旺的月份。否则就不可论化。
(间隔遥远,虽合难化〕 天干的化合除了必须结合出生
• 145 •
说份,还要看看地位远近。如果年干属乙,时干属庚,彼此路途
轲隔遥远,合力单薄,就不一定论化了。
3 1天干相合,有吉有凶〕 天干合掉之后,大多自身还有五
六或六七分力量,比如乙庚合金,金虽被合,然而自身的性质
.却还多半存在。对于天干相合后的是吉是凶,要根据具体情
况而定。在一般情况下,合并不是件坏事,可是一旦如果日干
的喜神或用神被合,那就凶神乱意,情况不妙了。
: 〔地支六合,区别对待】 这就是说,命局所喜的地支被六
合合去之后,就要减吉;所忌的地支被合掉后,就会减凶。此外
地支合局还可解除刑冲不吉。具体情况要作具体分析。例如
命局喜子,地支中有丑字合而化土,就减了吉的分数;反之命
局忌子,地支中有丑字合而化土,则又可以减去凶的程度。又
如命局喜子,但逢午冲,这时如果又有未去合午,那就解了子
午之间的彼此相冲。这里需要注意的是,地支六合要彼此紧
贴,如日支和月支紧贴,日支和时支紧贴,否则彼此隔位,就不
合了。此外,地支如是二卯合一戌,或二戌合一卯'二寅合一
亥,或二亥合一寅的,叫做妒合。
〔地支三合,论吉论凶〕 在地支的申子辰合水,亥卯未合
木,寅午戌合火,巳酉丑合金三合局中,如果合局为命中所喜
则吉,所忌则凶。比如命局喜水的,而地支中出现申子辰三合
水局,就以吉论,命局忌水的,地支中如果出现申子辰三合水
局,那就要以凶论了。此外,如果地支出现申子或子辰合水,
亥卯或卯未合木,寅午或午戎合火,巳酉或酉丑合金,通常叫
做半合局。半合局以紧贴为妙。然而, 无论是三合局的还是
半合局,都怕逢冲,造成破局。
〔地支三会,活看吉凶】 在地支的贯卯层会东方木.巳午
•146・
未会南方火,申酉戌会西方金,亥子丑会北方水等三会方向
中,也和地支的三合局一样,如果会局为命中所喜则吉,所忌
则凶。比如命局喜水的,地支中出现亥子丑会成北方水的,就
以吉论;反之命局忌水,地支中却偏偏出现了亥子丑会成北方
水,那就要以凶论了。在力量上,地支三会方向的威力大于三
合局,而三合局的威力又明显大于六合。为此,如果四支中三
合局或三会方向同时并见,一般都弃合论会。
(;地支六冲,本气为重〕 命中地支相冲,以本气为重。比
如寅申相冲,寅的本气是甲木,申的本气是庚金,所以两者相
冲,首先体现在甲木和庚金的冲克上。在通常的情况下,总是
申金胜而寅木败,可是如果时令碰上火旺金衰,或水旺火衰,
则彼此相冲,又可分别造成寅火胜而申金败或申水胜而寅火
败的局面。在吉凶上,如果命局所喜的地支冲败则凶,所忌的
地支冲败则吉。这里要补充的是,相冲的地支必须彼此紧贴,
才能算冲,否则彼此隔开,就只好作稍有波动看了。六冲和三
合局一起出现,由于三合局的力量大于六冲,所以以合局论。
不过,如果是半合的话,有时逢冲,也可把合解掉。例如酉年
酉月亥日巳时,月支酉和时支巳半合,但日支亥和时支巳相
冲,就解掉了月支酉和时支巳的半合。
〔地支刑害,略微动摇〕 地支子刑卯,卯刑子,原是水木- 相生:巳刑申,巳申本合;丑刑戌,戌刑未,都是同类的土;至于
申刑寅,未刑丑,无非彼此相冲而已。同样,地支相害,也和地
支相刑一样,影响不大,只是略微动摇而已。
以上干支的刑冲化合,《滴天髓》另有*阳支动且强,速达
显灾祥,阴支静且专,否泰每经年”的说法。关于十二地支中
的阳支阴支,有的以子、丑、寅、卯辰、巳六支为阳,午、未、申、
•147
酉、戌、亥六支为阴,但大多数的星命家则以子、寅、辰、午、申、
戌为阳,丑、卯 、巳、未、酉、亥为阴。由于阳支性动而强,所以
吉凶之验多较速显,阴支性静而弱,故而祸福之应每至迟晦。
此外,《滴天髓》还说,“生方怕动库宜开,败地逢冲仔细推」
寅、申、巳、亥为“生方”,所谓“生方怕动”,这是因为“生方”若
有冲动,常易引起两败俱伤的结局。如寅、申逢冲,申中庚金
虽克寅中甲木,但寅中丙火,未尝不克申中庚金I 申中壬水虽
克寅中丙火,但寅中戊土,也未尝不克申中壬水。辰、戌、丑、
未为“四库”,一般来说,库中多藏有印绶财官,宜冲则开,然而
也少不了根据具体情况,难以一概而论。子、午、卯、酉为“四
败”之地,由于其气所藏专而不杂,所以如若逢冲,必须仔细推
详宜与不宜,不可执著。
对于“四生”、“四库”、“四败”的逢冲情况,任铁樵《滴天髓
阐微》举例说,
〔生方逢冲例】
(年) 癸巳
(月) 癸亥
甲寅
(时).壬申
大运 壬戌
辛酉
庚申
己未
戊午
丁巳
日干甲木,生于孟冬亥月,寒木喜火,可是四柱壬癸水泛,无土
• 148 •
制约,亥中壬水又冲巳中丙火为灾,看去似乎不佳,然而妙在
寅亥合木,大使巳火绝处逢生,得以兴发。结合行运,早年运
入西方金地,生水制火,所以碌碌风霜,奔驰不遇。四十以后
运临南方火土之地,助起用神,弃印就财,所以财源滚滚,娶妾
生子。由此看来,印绶作用逢财,为祸不小,不用就财,发福最
大。
〔生方逢冲例②〕
(年) 甲寅
(月) 壬申
(日) 癸已
(时) 癸亥
大运 癸酉
甲戌
乙亥
丙子
丁丑
戊寅
己卯
庚辰
秋水通源,申金当令,水重木囚逢冲,不足为用。火虽休而紧
临日支,况秋初余气未熄,用神必在巳火。坏在巳、亥比邻逢
冲,群劫纷争,所以连克三妻,无子。兼之运走北方水地,以致
破耗异常。至戊寅、己卯,运转东方,喜用合宜,得其温饱。庚
运制伤生劫,又逢酉年,喜、用两伤,不禄。
t败地逢神例]
(年) 伤辛卯官
•149・
(月〉 印丁酉伤
(日) 戊子财
(时) 比戊午劫M
大运 丙申
乙未
1 甲午
癸巳
壬辰
辛卯
此“伤官用印”,喜神即是官星,并非俗论所说的'土金伤官忌
官星”。局中月支酉冲年支卯,致使月上印星丁火失却生助之
神,日支子冲时支午,遂使午中丁火难遏伤官之肆。由此可知,
由于地支金旺水生,木火冲克殆尽,所以天干火土虚脱,无根
可扎。观察命主一生,读书未成,碌碌经营,更兼中运天干,金
水一气,未免有志难申。虽然好在水不透干,为人文采风流,
精于书法,但这丝毫也不能使命主由此而摆脱困境。由此可
知,凡伤官佩印,喜神用神在木火的,一般都忌见金水。
〔支全四库例〕
(年) 辛未
(月) 辛丑
(日) 戊辰
(时) 壬戌
大运 庚子
己亥
戊戌
•150 •
丁酉
丙申
乙未
此造之美,不在支全辰戌丑未四库,而在月支丑中辛金元神透
出,伤官吐秀,泄强土之精英,加之四柱木火,伏而不见,所以
命局纯清不混。结合行运,至酉运时辛金得地,高中乡榜,后
因运行南方,木火并旺,用神辛金受到伤害,所以未能进而有
所发迹。
〔支全四库例②】
(年) 戊辰
(月) 壬戌
(0) 辛未 .
(时) 己丑
此造辛金日元,满局印绶,壬水伤尽,未足为用。若以未支辰
支所藏乙木为用,只待运来引出,便可破印,然而丑戌两库双
双冲破未库辰库,砍伐乙木,最后终至克妻无子。由此而论,
四库必须冲开的说法,关键全在天干调剂得宜,更须用神有
力,然后岁运相辅,才能一路平安,青云直上,否则便就不妙
了。 

\section{富贵贫贱和寿夭疾病的推算}
儒家先师,生前累累若丧家之狗的孔老夫子曾无可奈何
地感叹说:“死生有命,富贵在天J把孔子称为“累累若丧家之
751
狗”,语出《史记 •孔子世家》,原话是郑人姑布子卿在暗中观
察了孔子相后,对他弟子子贡所说的。后来子贡把实话告诉
了孔子,孔子非但不以为怒,还欣然笑曰「形状(人的形相3
末也。而谓似丧家之狗(指狼狈的神态》,然哉! 然哉!”
由于奋斗了一辈子也没能在政治上施展抱负,最后在不
得已中才做起了教书先生的孔子,终于不得不在碰得头破血
流后低头认命了。“不知命,无以为君子”,这就是他在追求上
失败后,心情渐次趋向淡泊的自我写照。那末,怎样才能预先
“知命”呢?这在孔子生活的当时,除了一些零星的相术外,显然
是件不可能的事,因为当时只知有命却不知道命的推算方法。
自从算命术发明以后,由于在很大程度上遵循着这位儒
家先师的遗训,所以推究预算一个人一生的富贵贫贱,穷通寿
夭,自然便成了算命术的首要目的。
为什么同样一个人,生出来后的处境竟然会这么不同呢?
按照命理学家的解释,就是当人在受胎之初,阴阳二气交流,
真精妙合之际,如果禀受的是清气,那就为智为能,禀受的是
浊气,那就为愚为不肖。为智为能的在社会上必定多所获益,所
以或富或寿;为愚为不肖的不能自我奋发,所以贫贱而夭。这
反映到命里,就自然会在每个人自己的生辰八字里显现出来。
说到这种对命主本人未来的推法,命理学家也自有他的
一套办法,这就是推富贵贫贱先看命中自身日干得令不得令,
次看用神得力不得力,未看行运演莉不顺利。如果日干得令,
黑粉得力,运遇财官,往往富贵发福,大吉大利,反之则贫困潦
倒,苦不堪言。推生死寿夭要细论岁运和原局用神的喜忌,如
果岁运碰上忌神冒头,喜神无救,那就轻则凶,重则死了。但也
有一种“以生月定之”的说法,《玉门关集》说,“凡寿以生月定
•152-
之,生月居支干纳音旺处,及五行相生不逆,日时并胎,皆得
数,不相刑克者,主上寿。”此外,《滴天髓》有《何知章》推人富
贵贫贱寿夭,影响很大,其说法为:
何知其人富,财气通门户
何谓“财气通门户”? 刘基注解道「财旺身强,官星卫财,
忌印而财能坏印,喜印而财能生官,伤官重而财神流通,财神
重而伤官有限,无财而暗成财局,财露而伤亦露者,此皆财气
通门户,所以富也。”任铁樵则补充说:“财旺身弱无官者,必要
有食伤。身旺财旺无食伤者,必须有官有煞。身旺印旺食伤轻
者,财星得局。身旺官衰印绶重者,财星当令。身旺劫旺,无
财印而有食伤者。身弱财重,无官印而有比劫者,皆财气通门
户也。”由于命局中论财与论妻之法可以通论,所以有“财神清
而身旺者妻美,财神浊而身旺者财富”的说法。如果深一层论,
则任氏之说可供参考。他说「如身旺有印,官星泄气,四柱不
见食伤,皆财星生官,无食伤,则财星亦浅,主妻美而财薄也。
身旺无印,官弱逢伤,得财星化伤生官,则亦通根,官亦得助,
不特妻美,而且富厚。身旺官弱,食伤重见,财星不与官通,家
虽富而妻必陋也。身旺无官,食伤有气,财星不与劫连,无印
而妻财并美,有印则财旺妻伤。此四者宜细究之。”
【命造八字]
(年) 甲申
(月) 丙子
(日) 壬寅
(时) 辛亥
壬水生于仲冬,羊刃当权。表面看去,日支寅中食神甲木
被申金冲破,然而妙在日支与时支寅亥合局,两者为年干甲和
•153 •
月干丙的木火长生之地,加之子申合水,申金不但不冲寅木,
并且合水后食神甲木反而得到生扶,所以说是财气通于门户,
为百万富翁之命。大凡巨富之命,不在财星多少,只要生化有
情,就是“财气通门户”了。如果财星临于旺地的,则不宜见
官,因为官星能够盗泄财星之气。又如日主失令的,则必定要
有比劫相助,方为美善。
〔命造八字〕
(年) 壬申
(月) 丙午
(日) 癸亥
(时) 戊午
癸水生于仲夏,又逢午时,月柱时柱透出丙火戊土,未免
财官太旺。妙在日柱癸水得地,更妙年干壬劫支坐长生,身旺
则堪任财官。加之五行无木,水不泄而火无助,所以当取年干
壬水作为用神。此后一旦运走西北,金水得地,虽说祖上遗产
不丰,但也白手起家,有四五十万之富。
何知其人贵,官星有理会
对于“官星有理会”的解释,刘基认为「官旺身旺,印绶卫
官,忌劫而官能去劫,喜劫而官能生印,财神旺而官星通达,官
星旺而财神有节,无官而暗成官局,官星藏而财神亦藏者,此
皆官星有理会,所以贵也。”任 铁樵氏则说:“身旺官弱,财能生
官。官庭身弱,官能生印。印旺官衰,财能坏印。印衰官旺,
财星不现。劫重财轻,官能去劫。财星坏印,官能生印。用官
官藏庭亦藏,用印印露官亦露者,皆官星有理会,所以贵显
也。”由于命局中论官与论子之法可以相通,所以又有“官星清
而身旺者必贵,官星浊而身旺者必多子”的说法。对此,任铁
•154 •
樵发挥说「如身旺、官旺、印亦旺,格局最清,而四柱食伤,
点不混,财星又不出现,官星之情,依乎印,印之情,依乎日主企
只生得一个本身,所以有官无子也。纵使稍杂食伤,亦被印星
所克,子亦艰难。如身旺、官旺、印弱,食伤暗藏,不伤官星,不
受印星所克,自然贵而有子。必身旺官衰,食伤有气,有印而
财能坏印,无财而暗成财局,不贵而子多必富。如身旺官衰,
食伤旺而无财,有子必贫。如身弱官旺,食伤旺而无印,贫而
无子。或有印逢财,亦同此论
〔命造八字〕 大运
(年) 丁酉 乙巳
(月) 丙午 甲辰
戊寅寅中甲木为杀,用神 癸卯 }东方木地
(时) 丁巳 壬寅 J •
翼卜匕方水地
这一命造,日主戊土,生于仲夏午月,火气炎盛,又遇年、
月、时三柱干支丙丁之火生扶,戊己之土助身,可谓身旺之极。
旺者宜制宜泄,所以取日支寅中制我的甲木七煞,或年支中泄
我的酉中辛金作为用神。再看行运,早中年寅卯辰合木,运行
东方,得木制克;中晚年又转入北方子丑水运,水旺生杀,所以
是个贵过于富的命造。 .
〔命造八字〕 大运
(年) 杀癸卯
(月) 杀癸亥
丁卯印
:
壬戌 X
辛酉 }西方金地
庚申
(时) 辛亥 己未
戊午 ,南方火地
丁巳 J
局中官杀乘权,似乎可畏,好在地支亥卯拱印,流通水气,
所以官星有所理会。初运辛酉、庚申,生杀坏印,功名偃蹇。己
未一运,大运地支未与命局亥卯会成印局,大运天干透出食神
己土,所以云程直上。此后戊午、丁巳,伤比透干,比劫助身,
故亦仕途辉煌。由此可知,有其命必得其运,否则就是一介寒
O
〔命造八字〕 大运
(年) 官癸酉 丙辰 )
(月) 劫丁巳 乙卯 }东方木地
(日) ‘ 丙午 甲庚 J
(时) 杀壬辰 癸丑
壬子 ,北方水地
辛亥 J
丙火生于孟夏,坐禄临旺,又逢月干丁劫助身,所以自身
属强。好在地支巳酉拱金,财能生官,官又制劫。更妙时干透
出壬水,助起年干癸官,与局中月柱日柱成就既济之象。结合
大运,一旦进入北方水地,登科发甲,名利双辉。为此,不必斤
斤于官煞为嫌,因为命中身旺的,必要官煞混杂,才能助发。
印
〔八字命造〕 大运
(年) 财甲午2 丁卯 、印 戊辰
财
(月) 官丙寅官
•156 •
(日) 辛酉比
印
己巳
(时) 印己丑比 庚午
食
辛未
壬申
癸酉
此造日干自身辛金,年干财星,月 干官星,时干印星,都通
根禄旺。庚午一运,前五年庚运帮身,于运中癸酉年登科发
甲;后五年午运则杀旺病晦,亲属刑丧。行至辛运,比来助身,
在己卯年发甲入词林。其后运行金水,帮身制杀,鹏翅可展。
〔命造八字〕 大运
(年) 乙巳 庚辰
(月) 辛巳 己卯
(日) 庚辰 戊寅
(时) 甲申 丁丑
庚金生于立夏五日,土当令而丙火尚未司权,庚金之生坐
实,加之辰支、申时,生扶并旺,月干辛劫、年支时支长生为助,
可知是个身旺杀浅的命造。分析命中虽有年干透出财星,但
这财星由于无根遭劫,故可卜知出身贫寒。结合行运,一旦交
入丁运,因为官星元神发露,所以于戊寅、己卯两年财星得地,
喜用齐来之时,科甲联登,又入词林。书云:“以杀化权,定显
寒门贵客。”
何知其人贫,财神反不真
所谓“财神不真”,任铁樵认为大致有九种情况:一为财
重而食伤多,二为财轻喜食伤而印旺损财,三为财轻劫重而不
• 157 •
见食伤,四为财多喜劫而官星制劫,五为喜印而财星坏印,六
为忌印而财星生官,七为喜财而财合闲神而化,八为忌财而财
合闲神化财,九为官煞旺喜印而财星得局。并说:“凡败业破
家之命,初看似乎佳美。非财官双美,即干支双清;非煞印相
生,即财临旺地,不知财官虽可养命荣身,必先要日主旺相,方
能任其财官,若太过不及,皆为不真,能散能耗则有之,终不能
致富贵也。”
〔命造八字〕 大运
(年) 癸卯 癸丑
(月) 甲寅 壬子
(日) 丁巳 辛亥
(时) 己酉 庚戌
此造酉财藏而癸杀露,且杀印联珠而生,所以祖业二十余
万。然而由于年干之杀无根,加之杀的精华全被印绶所窃,所
以不用癸水作为用神。再看时支酉金之财,上有己土盖头藏
覆,看去似佳,可是整个格局木旺土虚,相火逢生,故而巳酉不
能会金为财,致使财星不真。结合大运,一交壬子,泄金生木,
败尽祖业。此后行至癸运,印遇长生,竟遭饿死。
〔命造八字〕 大运
(年) 辛丑 乙未
〈月〉 丙申 甲午
(日) 癸巳 癸巳
(时) 庚申 壬辰
命中丙财坐禄,丑中一杀独清,看去似乎佳美,可惜局中
印星太重,五杀生印晦财,加之丙辛金合而化水,变财为劫,庚
中在m,财界不真。行运初入乙未、甲午:木火并旺,食神生
•158
财,祖业颇丰0 此后一旦交入癸巳,巳与命中申金,皆合而为
水,漫天比劫,一败如灰,最后竟沦为乞丐。
何知其人贱,官星还不见
任铁樵认为,官星不见之理有上等、中等、下等三种情况。
官轻、印重、身旺,或官重、印轻、身弱,或官印两平,日主休囚,
此为上等官星不见。官轻、劫重、无财,或官杀重无印,或财 、
轻、劫重、官伏,此为中等官星不见。官旺喜印,财星坏印,或
官煞重无印,食伤强制,或官多忌财,财星得局,或喜官星,而
官星合他神而化伤,或忌官星,他神合官星而又化官,此为下
等官星不见。
〔命造八字〕
(年) 丁丑
(月) 壬子
(日) 丁亥
(时) 甲辰
丁火生于仲冬,月干透出壬水,地支亥子丑会成北方水
局,而辰文为湿土,非但不能制水,反而能够晦火,加之日主虚
弱,甲木凋枯,自 顾不暇,且湿木不能生无焰之火,所以官星旺
极不真,反为清枯之象。好在局中无金,气势纯清,为人学问
真淳,处世无苟,以训蒙作为度日生计,苦守清贫,这就是所谓
的“上等官星不见”了。
〔命造八字〕 大运
(年) 丙辰 辛卯
(月) 庚寅 壬辰
(日) 丙午 癸巳
〈时) 壬辰 甲午
c
159・
乙未
此造庚财临于绝地,根攀全无,而时上官星气亦不足.兼
之运走东南木火之地,所以幼年丧父,依母转嫁他姓。几年之
后,母亡牧牛度日。稍长大后帮佣为生,后来不幸双目失明,
不能佣作,求乞自活。 •
何知其人吉,喜神为辅弼
喜神为辅弼用神之神,属于吉神。四柱中如果有喜神的,
则用神有势,一生吉多凶少。反之,四柱中如无喜神出现,虽
有用神,若岁运不逢忌神冲克则罢,若果一遇忌神,大多难逃
凶灾。
〔命造八字〕
(年) 杀甲子财
杀
(月) 印丙寅印
比
(H)戊寅
(时) 己未
春初土虚,杀旺逢财,所以以枭印丙火为用。所喜年支财
星与枭印相隔,且财生杀,杀生印,有生生不背之妙。纵观局
中,又以未时帮身作为喜神,加之四柱纯粹,主从得宜,所以早
登甲第,一生有吉无凶,仕至观察。晚年退归林下,夫妇齐眉,
寿过八十,儿辈都为科第中人。
〔命造八字〕
(年) 丙申
(月) 己亥
(日) 庚辰
(时) 戊寅
•160 •
此寒金喜火,年干丙火得时支寅木相生,则火有焰。但用
财杀的首先要日干身旺,现在日干庚金喜逢年支申金为禄,并
且月干己土、时干戊土,日支辰土三印贴生,加上月支亥水当
权,申金贪生不冲寅木. 遍览全局,无火则土冻金寒,无木则
水旺火虚,故而计长论短,当取火为用神,木为喜神,木火两
字,缺一不可。现因局中喜、用皆备,所以一生无凶无险,登科
发甲,宦海无波,后裔济美,一直活到八十开外。
何知其人凶,忌神辗转攻
所谓“忌神”,就是损害体用之神。八字中忌神为病,喜神
为药。有忌神而有喜神解救的,叫做有病有药;有忌神而没有
喜神可以解救的,叫做有病无药。有病有药的吉,有病无药的
凶。如寅月出生的人,如不用甲木而用戊土,那么这克土的甲
木就成了当令的忌神了。这时日主命局中如果有火化木,有金
制木,这火金就成了喜神,此后行运若更扶喜抑忌,皆可转凶
为吉。反之,命中运中如若无火化木,无金制木,反而有水生
木,有木党木,那就凶祸多端,到老不吉了。此外,岁运如虽未
能扶喜抑忌,但也不与忌神结党,助纣为虐,那末大致终身不
凶不吉,碌碌到老。对此,刘基说道「财官无气,用神无力,不
过无所发达而已,亦无刑凶也。至于忌神太多,或刑或冲,岁
运助之,辗转攻击局内无备之神,又无主从,不免刑丧破败,犯
罪受难,到老不吉」
〔命造八字〕 大运
(年) 乙亥 丁丑
(月) 戊寅 丙子
丙子 乙亥
(时) 甲午 甲戌
•161 •
此造丙火生于寅月,局中寅亥化木,年时甲乙并透,印星
过于旺盛,而日支子水又被时支午火冲破,所以只得取月干戊
土作为用神。再看局中气势,甲乙木旺,反得亥子水生,这就
是所说的“忌神辗转攻”了。结合运程,初运丁丑,助起用神,
出身巨富,其乐盈盈。一交丙子,火不通根,水助忌神,非但父
母双亡,并且连遭火灾。乙亥一运,水木并旺,忌神鹉张,又遭
火厄,克三妻四子,赴水而死。
【命造八字〕 大运
(年) 辛巳 己丑
(月) 庚寅 戊子
(日) 丙辰 丁亥
(时) 己丑 丙戌
乙酉
丙火虽然生于寅月,但纵观全局,土金并旺,所以当取寅
木作为用神。然而寅为初春嫩木,忌见庚金盖头,故以庚为局
中忌神。行运初交己丑、戊子,生金泄火,父母双亡,孤苦不
堪。丁亥、丙戌,因为火临西北之地,不能去尽忌神,所以历尽
风霜,稍成家业。一交乙酉,乙与月干庚金,酉与日支辰土,皆
合而化金,忌神得势,刑妻克子,遭受水厄而亡。
何知其人寿,性定元神厚
所谓“性定”,意即局中四柱得地,五行停匀,所合都为闲
神,所化都为用神,所冲都为忌神,所留都为喜神,无缺无陷,
不偏不枯。“性定”的人,不生贪恋之私,不作苟且之事,为人
宽厚和平,仁德皆备,所以多福多寿。所谓“元神厚”,意即四
柱中官弱逢财,财轻遇食,身弱有印绶生身,身旺有食伤吐秀,
所喜的为提纲之神,所忌的为失令之物,加之提纲与时支有
• 162 •
情,行运与喜用不悖,所以富而且寿。
(命造八字〕
(年) 辛丑
(月) 癸巳
(H) 甲子
(时) 丙寅
这一命造,四柱通根生旺,五行源头流通,很有特色。先看
四柱,自身甲木,归禄时支:印绶癸水,禄在日支;食神丙火,禄
于月支;官星辛金,支坐财地。再看五行,月支巳火生年支丑
土,年支丑土生年干辛金,年干辛金生月干癸水,月干癸水生
日主甲木,日主甲木生时支丙火,丙火又下坐长生之地,可谓
源远流长。为此,命主为人仁德兼备,刚柔相济,位居三品,富
有百万,一直活到百岁高龄,无疾而终。
〔命造八字〕
(年〕 乙未
(月〕 戊寅
(日) 乙卯
(时) 庚辰
此东方曲直仁寿格。由于格中火气衰微,财神衰弱无气,
木势太旺,官星削薄无根,所以一生操劳,仗义疏财而清贫自
守。然而,妙在东方一气,仁寿成格,刘基有云:“大率甲乙寅
卯之气s不遇冲战泄伤,偏旺浮泛. 而安顿得所考必寿。木属
仁,仁者寿,每每有验J为此,虽则生活清贫,但却寿至九十四
岁而终。 •
何知其人夭,气浊神枯了 .
所渭“气浊”,任氏认为「浊字作一弱字论。气浊者,日主
•163
失令,用神浅薄,忌神深重,提纲与时支不照,年支与日支不
和,喜冲而不冲,忌合而反合,行运与喜用无情,反与忌神结
党,虽不寿而有子。” 所谓“神枯”,任氏接着说,“神枯者,身弱
而印绶太重,身旺而克泄全无,然重用印,而财星坏印,身弱无
印,而重叠食伤,或金寒水冷而土湿,或火焰土燥而木枯者,皆
夭而无子也。”而刘基则从总的方面概括道I “气浊神枯之命,
极易看。印绶太旺,日主无着落I财杀太旺,日主无依倚。忌
神与喜神杂而战,四柱与用神反而绝。冲而不和,旺而无制,
湿而滞,燥而郁,精流气泄,月背时脱,此皆无寿之人也
〔命造八字〕 大运
(年) 丁丑己土伤官 壬寅
(月) 癸卯乙木印绶,用神 辛丑
(H) 丙戌戊土食神 庚子 北方水地
(时)食神 戊戌戊土食神 己亥 )
署}西方金地
这一命造,日主丙火,生于仲春卯月,乙木生火,本属好
事,可惜年支丑中己土,日 支戌中戊土,时柱干支两重戊土,食
伤重重,致使自身泄气太过。综览全局,当取乙木印绶作为用
神,既可生我,又可制服太过的土。再看大运,壬寅以后,亥子
丑一片水地。水虽能够制火,可是水又能够生木,这步运纵然
比不上直接行东方木运来得更好,可是却还勉强说得过去。然
而一旦行运进入戌酉金地,虽说金为财运,可是金能克木,财
星破印,用神被制,这就难保活命了。
〔命造八字〕 大运
(年) 印绶乙丑辛金 甲申
• 164 •
(月) 印绶乙酉辛金,死 癸未
(日) 丙辰 壬午
(时) 正财辛卯 辛巳
这一命造,丙火生于酉月死地,根气全无,加之时干透出
正财辛金,年支月支丑酉也伏藏正财辛金,可谓财多身弱。对
于财多身弱而又没有比劫助身的命来说,用神最好取印,因为
印能生身,所以这里的用神就压在生我的乙木上了。然而年
干、月干的乙木,虽然和日支时支的辰、卯通根相连,然而从这
两个乙木自身的坐支来看,却落在无情相克的财星辛金上面,
可谓财星破印,上下无情。在这种情况下,表面看去用神虽
多,可却不顶真用,加之没有命中所喜的比、劫、禄、刃相辅,不
禁举步艰难。好在大运癸未、壬午,火来助身,日干得地,也可
娶妻生子。可是一经交入辛巳运,运中天干辛削用神乙木,财
能坏印,运中地支巳与命中年支月支丑、酉合成金局,又去大
力克削命中日支时支所藏印星,一时用神彻底被伤,夭亡就在
劫难逃了。
再之,对于生死寿夭,《渊海子平》尚有《格局生死引用》七
条,可供参考:
格局者,自有定论,今略而述之。印绶见财行财运,
又兼死结,必入黄泉;如柱有比肩,庶几可解。
正官见杀及伤官,刑冲破害,岁运相并,必死。
正财偏财,见比肩分夺,劫财羊刃,又见岁运冲合,必
死。
伤官之格,财旺身弱,官杀重见,混杂羊刃,岁运又见,
必死;活则伤残。
拱禄、拱贵,填实又见官冲,刃岁运见,即死。
•165 •
日禄归时,刑冲破害,见七杀官星,空亡冲刃,必死;官
杀大忌,岁运相并必死。
其余诸格并忌杀及填实,岁运并临,必死。会诸凶神
恶杀,印绶空亡,吊客、墓、病、死宫诸杀,十死九生。官星太岁,
财多身弱,元犯七杀,身轻有救则吉,无救则凶。金多夭折,木
盛飘流,木枉则夭,土多痴呆,火多顽愚。太过不及,作此论。
一不可拘,二须敢断。必须理会论之,求要生死要矣。
此外,《三命通会》还在卷八《六丁日乙巳时断》篇中说,
“丁亥日,乙巳时,时日并冲,忧伤妻子。巳酉丑、申子辰金水
二局,财官得用,以富贵轮。” 接着还举这样两个八字命造为
例,一是壬辰、甲辰、丁亥、乙巳,说是王行侍郎的命,一是丁
亥、甲辰、乙亥、乙巳,竟是个乞丐的命。
实际上,由于客观存在着的种种不准,所以陈素庵《看富
贵吉寿贫贱凶夭要法》,大力提倡劝人为善的佛家因果报应思
想。由于其说法对后世命学思想具有较大影响,所以引述如
下: ’
.
一. 富 贵吉寿贫贱凶夭诸局,相 准之故既彻,有定之
理既得矣,禄以推人命不尽验,是有己身之善恶焉,
有家世之善恶焉。福善祸淫,於然之理。如为恶之
人,命应一品之贵,而 减至四五品;命应百万之富,而
减 至六七十万;命应百岁之寿,而减至六七十岁;命
应五福全备,而减其一二。又 如为 善之人,命应 极贱,
而得一命之荣;命应极贫,而 得中人之产;命应早世,
而得数十岁之寿;命应诸凶毕集,而免其什三。世俗
之见,将谓为恶者,何尝不福?为善者,何尝不祸?
岂知福之已损,祸 之已灭乎?知祸福者,非知命也,
•166 •
知善恶之为祸福者,则诚知命耳。虽然,徒知之何
益?是有转移之道焉。昔袁了凡先生遇术士 推命,
止于贡士而无子,因详列将来履历,始则一一神验,
后遇高僧,导以造命之学,积若干善求科第,积若干
善求子息。善数既积,果登两榜而举子。术士所推,
毫不险矣。故凡欲求富贵吉寿,而 兔贫贱凶夭者,当
以积善为要。每日自记功过,必期念念皆仁,本本 皆
善,久必如其所愿。若恃命之善而敢于为恶,咎命之
薄而不思挽回,此为天下至愚之人,无 志之士矣。诸
命法曾来耳;是乃要法也。
除了推命主贫富寿夭,有的命书还不忘同时推人疾病。对
于疾病的推法,先要把五行和五脏六股联系起来,然后再根据
五行生克的原理来加以分析。按照中层理论,五行和五脏六
腑的相互搭配是:
(;甲〕 胆 (:乙〕肝
〔丙〕 小肠 心
〔戊1 胃 (:己〕 牌
〔庚〕 大肠 C辛〕 肺
〔壬〕 膀胱、三焦 〔癸〕 肾、心包络
其中胆、胃、大肠、小肠、三焦、膀胱为六腑,性质属阳,所以都
配阳干;肝、心、脾 、肺、肾为五脏,心包络则附于心系,性质属
阴,所以都配阴干。歌曰:
甲胆乙肝丙小肠,丁 心成胃 己 脾乡,
庚是大肠辛属肺,壬系 膀胱癸肾藏,
三 焦亦向壬中寄,包络同归入癸方。
又曰;
• 167 •
甲头乙项丙肩求,丁心戊胁己属腹。
庚是脐轮辛为股,壬胫 癸足一身覆。
与此同时,古人又有把十二地支和身体各部分联系起来
的,因为不及与五脏联系来得重要,所以一般并不受人重视。
现把十二地支和身体各部分联系的歌括照录如下:
子 属膀胱水道耳,丑为 胞肚及脾乡,
寅胆发脉并两手,卯本十指内肝方,
辰土为脾肩胸类,巳面齿咽下尻肛,
午火精神司眼目,未土胃院膈脊梁,
申金夫版经络肺,酉申精血小您赢,
戌土命门腿踝足,亥水为头及肾囊。
若依此法推人命,岐伯 雷公也播扬。
在具体看法上,以日柱干支为主,结合五行生克太过不
及,以作定论。譬如日干是甲、乙木,四柱八字中出现庚、辛、
申、酉等金多的,木就遭克,可能有肝胆惊悻劳瘵,手足顽麻,
筋骨疼痛,头目昏晕,或口眼歪斜,左瘫右痪,以及跌扑损伤等
症。假如日柱天干仍是甲、乙木,园柱八字中出现丙、丁、巳、
午火多韵,并显没有水来相济,这时木气被泄太过,又可出现
内热口干,痰喘咯血,中风不语,以及女人血气不调,怀孕胎
落,小儿急慢惊风,夜啼咳嗽,面色青‘黯等症。至于为什么木
被金制或火泄过头会出现这些症状?因为这牵涉到祖国传统
医学的理论,这里就不作探讨了。
对于以上所述疾病的看法,陈素庵又自有他的独到见解:
“旧分五行,论人疾病,未尝不合于理,但人身脏腑经络五行俱
全,人命柱中运中,五行未必俱全,必以某行断其病,亦不尽
验。须看日主及所用格局,或朗健,或中和,或平顺,皆无疾之
•168 •
命也;或晦弱,或驳杂,或乖戾,皆有疾之命也。又看其神理气
势,或太过,或不及,兼取柱中运中五行参合论之,即无木而就
生木、克木、木生、木克之神,亦可推木之受病与否。至于干支
配头、目、手、足等类,皆当以意消息之。若必尽取诸病而拟议
之,则名医所论,孰非五行?恐须摘取医书数十百种,列于命
书矣。”
为了便于执简驭繁,现采摘部分古赋,聊备一格于下。赋
云:
筋骨疼痛,盖 因木被金伤;眼目昏暗,必是火遭
水克,土虚逢木旺之乡,脾伤定论;金弱遇火炎之地,
血疾无疑。
又云:
木逢金克,定 主腰胁之灾;火被水伤,必主眼目
之疾;心肺喘满,亦 干金火相刑;脾胃损伤,盖因土
水战克。支水干头有火遭,必腹病心朦,支火 干头有
水遇,则内障睛盲。炎上(火)烦焦蒸土曜,头秃眼
昏;润下(水)纯湿无土制,肾虚耳闭I 荧惑(火星)乘
旺临离巽(火风);风中 (中风)失音;太白(金星)坚
利合兑坤(金土),兵前 落算。
结合当今台湾学者著述,《子平八字大突破》认为:“凡人
命,强金伐木,土重木折,水多木漂,火炎木焚,木重无泄,皆主
肝胆有病 凡人命,水多火熄,土多火晦,金多火变,木多火
塞,火多无泄,皆主小肠及心有病 凡人命,木重土陷,水多
上流,金多土虚,火多土焦,土旺无泄,皆主脾胃有病。”“凡人
命,强火熔金,木坚金伤,土多金埋,水多金沉,金旺无泄,皆主
大肠及肺有病 凡人命,土多水塞,金多水浊,火多水沸,木
•169 •
多水缩,水旺无泄,皆主膀胱及肾有病。”
对于命局疾病的看法,台湾学者梁心铭《现代命学》还积
极地从后天调养着眼,提出了“先天后天相辅互补调和”的说
法。书中《论健康居家正诀》篇中,梁先生指出「调候之神为
药神,在命局中甚为重要。正格之命生在炎热夏令,宜水调候
滋润,不然局中火热过燥不佳也。故在饮食方面须凉性之物
对身体较佳,有关燥热之食物少食为妙。正格之命生在严寒
冬令,宜火调候温暖,调候为先,不然过寒之造,不佳也。故在
饮食方面须热性之物对身体较佳。”接着他特为叮咛J人生旅
程生老病死,在所难免。但预防胜于治疗,很多事情是人为因
素造成的。如一个人明知自己肝脏胃肠不好,但不知自爱节
制,喝酒赌博熬夜,暴饮暴食,吃喝玩乐,这样就是加速身体败
亡严重性。若能后天虚心调养自重,这样不但可好转康复,而
且正常的人生会过得更有意义。”这样就客观辩证了。

\section{五行、喜用神和职业选择}
一个人要求立足社会,以谋进展,在社会上找到一个合适
的位置,是件十分至关重要的事。俗话说:“男人就业宜对行,
女人嫁君宜对郎。”“男怕入错行,女怕嫁错郎J 就是这个意
思。那么,如何才能按照命理学的说教,根据各人四柱八字的
五行喜忌,或喜用神来选择自己的职业,以祈求得-份合适的
二作而摆好自己的位置呢?当然就今天来说,自然还理应包
点半月天的女性在内。其大致情况为:
•170 。
按照五行喜忌选择职业
〔命主喜木〕
命主喜木的,以做和木有关的职业为较佳。由木而引发
开来,这一类工作大致有木材、木器、家具、装潢、园艺、苗圃、
茶叶、药材、水果、棉布、造纸、文教、作家、出版、书店,旁及政
治、文书、工商、教育、医务、宗教等职业。
〔命主喜火〕
命主喜火的,以做与火有关的职业为较佳。由火而引发
开来,这一类职业大致有饮食、饭店、宾馆、工厂、百货、美容、
光学、照相、银镜、电力、电器、爆竹,旁及工艺品、印制、评论
家、军界等职业。
〔命主喜土J
命主喜土的,以做与土有关的职业为较合适。由土而引
发开来,这一类行业大致有农业、畜牧、土产、矿产、房地产、建
筑业、古董买卖,旁及中介、经销、管理、买卖、墓葬、风水师等
业。
〔命主喜金〕
命主喜金,以做与金有关的职业为较相宜。由金而引发
开来,这一类行业大致有五金、金融、机械、交通、汽车、家电、
刀模,旁及法律、影星、音乐、会计、科技、开采、厂矿等职业。
匚命主喜水〕
命主喜水的,以做与水有关的职业为较理想。由此而引
发开来,这一类行业大致有水利、航海、水产、水上运动、茶水
业..澡堂、消防,旁及贸易、旅游业、记者、自由职业等等。
以 上按照五行喜忌选择职业,有两个问题要弓【起注意。
是行业五行属性一时定不下来的,比纪对演员,有的认为演员
•171 •
流动性大,理应属水,但也有人认为演员上台,在强光下进行
演出,当以属火为宜,这又牵涉到仁者见仁,智者见智了。二
是五行喜忌,不得不和大运五行的影响结合起来,尤其是命里
带有驿马,或地支遭受刑冲的,更加容易受到大运五行的影
响。
此外,关于五行职业,还有一个混和五行的问题,比如建
筑业,这里虽然划归喜土类下,其实建筑少不了要用钢窗,要
用木门,这又同时兼有木和金的性质了。
以上仅就个人从事的事业而言,如果在国营、集体、公司、
厂矿,以及中外合资企业里工作的,因为不属个人事业,可以
不受五行限制。否则彼此间你调我动,岂不乱了套吗。
按照用神不同选择职业
〔用神正官J
用神为正官的,一般以从政为佳,比如政府机关、行政管
理部门、司法部门等公职人员。
〔用神偏官〕
用神为偏官的,一般以从事严肃职业为佳,比如军界、警
界,乃至开采、爆破、登山探险、外科医生、音乐指挥等等。
〔用神正印〕
用神为正印的,一般以从事文教事业为较理想,比如文
学、教育、作家、秘书、学术研究,旁及宗教事业、慈善事业等
等。
〔用神偏印〕
用神为偏印的,一般以从事专门性职业为宜,比如科学之
研究、设计,技术之发明、创造等等。
〔用神劫财〕
• 172 •
用神为劫财的,一般以从事自由职业为佳,比如教师、律
师、医生,乃至推销等职均可。
〔用神比肩〕
用神为比肩的,一般也较宜从事自由职业,此外并适宜于
合伙经营,当然也要看准对象。
〔用神正财] '
用神为正财的,一般以从事财政事业为宜,如金融、商务、
税务、会计等职。 '
〔用神偏财〕
用神为偏财的,一般以从事商务等职较为有利,如经商、
投资、经济贸易、证券交易等等。
〔用神伤官〕
用神为伤官的,一般以从事独立性的文艺工作为佳,如书
法家、美术家、雕塑家、音乐家、歌唱家、舞蹈家、演员,乃至文
学家、剧作家、编辑等等。
〔用神食神J
用神为食神的,一般也以从事文艺工作为较有发挥,如书
法家、美术家、文学家,乃至外交家、讲演家等。
以上结合用神选择职业的做法,当今台湾命理学家李铁
笔归纳为「用神为官杀者,宜从政为官、行政管理。用神为财
星,宜金融、财政、经商。用神为印星者,宜文教、学术研究、宗
教慈善业。用神为比劫者,宜自由业、武市竞争、流动或合伙
业。用神为食伤者,宜服务业、艺术技术性
对于这一说法,台 湾颜昭博在《子平八字大突破》中还有
这样的阐发或补充:
•173 •
杀印相生,有权,宜 武界。
伤官生财,财源茂盛,宜 从商。
食神吐秀,聪明、秀气,宜艺 术界。
官 印相生,正官清粹,宜 官场。
财官、日 主两强,飞黄 腾达,可自谋发展。
日 主衰绝,或身 旺无依,但求安定。
伤官,工巧,具有创 新力。
食神,和气,客人宾至如归。
偏印,沉思,可为 智谋。
正印,祥和,调解工作。
劫财为喜,变卖祖产而 致富。
劫财为忌,吃喝 嫖赌而破产。
正印文静,正官小心保官,偏 印闭塞,经 商则宜
门市(指开店)。
七 杀干劲,伤 官敏捷,经商则宜武市。
八字官 杀印星重者,就公职,而喜忌则另有着
法。
八字食伤财星重者,可 经商,而 喜忌则 另有看
法。
经商,八字喜木,则行业属木为佳。余同。
综上所述,定居台湾板桥市的梁心铭认为:“人生谋求事
业宜慎为要,假如谋求之宜如为命中喜用之社为最隹,亦成就
愈大。如所做之业为命中忌Z神,则较劳心劳力,没有那么得
心应手」
话且扣此,但人生左世,以农选择原刊人事、环境、机遇等
• 174 •
多种复杂因素的凑合有关,加之,这里还有一个社会需求问
题,所以必须受诸多外界因素的制约,并不是你要干什么,就
可以干什么的。
同一棵树上的花被风一吹,也还有有的落在绣花枕上,佳
人头上,有的落在了泥潭里,有的被人踩在脚下,又何况人?李
白诗不亦云乎:“天生我材必有用。”这样一想,你就不必为一
时还没找到自己合适的位置而感到忧伤。还是踏踏实实,从
你现在所处的位置出发,发出你的光和热,为人类作出自己应
有的贡献,至于将来的调整,这又有待于机遇了。

\section{从八字五行及用神看人的性情相貌}
命学家在算命时,有时口里总常挂着这人的性情脾气怎
么样怎么样,兴致来时,还会发一番有关其人相貌的高论。这
是怎么回事呢?原来五行推人性情,只在日上时上,而以自身
的日柱五行为主,并且不论纳音。对此,命书自有一番有趣的
说法。
〔木〕东方震位,木号青龙,名曰曲直。五常主仁。其色
青,其味酸,其性直,其情和。旺相(参见《五行的旺相休囚死
和寄坐十二宫》),主有博爱恻隐之心,慈祥恺悌之意,济物利
人,恤孤念寡,直朴清高,行藏慷慨,丰姿秀丽,骨骼修长,手足
纤腻,口尖发美,面色青白,语句轩昂,此则木盛多仁之义。休
囚(参见篇目同前)主瘦长发少,拗性偏心,嫉妒不仁,此则木
衰情寡之义。死绝(即死,参见篇目同前)则眉眼不正,悭吝鄙
• 175 •
啬,肌肉干燥,项长喉结,行坐不稳,身多欹侧。遇火色带赤,
见土色带黄,逢金色带白,见水色带黑。其余四行例见6
〔火〕火属南方,名曰炎上。五常主礼。其色赤,其味
苦,其性急,其情恭。旺相,主有辞让端谨之风,恭敬谦和之
义,威仪凛烈,淳朴尊崇。面貌上尖下阔,形体头小脚长,印堂
窄而眉浓,鼻准露而耳小。精神闪烁,语言急速,性躁无毒,聪
明有为。太过,则声焦面赤,摇膝好动。不及,则黄瘦尖楞,诡
诈妒毒,言语妄诞,有始无终。
〔土〕 土属中央,名曰稼嵇。五常主信。其色黄,其味
甘,其性重,其情厚。旺相,主言行相顾,忠孝至诚,好敬神佛,
不爽期信,背圆腰阔,鼻大口方,眉清目秀,面肥色黄,度量宽
厚,处事有方。太过,则执一古朴,愚拙不明。不及,则颜色忧
滞,面偏鼻低,声音重浊,事理不通,狠毒乖戾,不得众情,颠倒
失信,悭啬妄为。
〔金〕 金属西方,名曰从革。五常主义。其色白,其味
辛,其性刚,其情烈。旺相,主英勇豪杰,仗义疏财,知廉耻,识
羞恶,骨肉相应,体健神清,面方白净,眉高眼深,鼻直耳红,声
音清亮,刚毅果决。太过,则好勇无谋,贪欲不仁。不及,则悭
吝贪酷,事多挫志,有三思,少决断,克薄内毒,贪淫好杀,身材
瘦小。
〔水3 水属北方,名曰润平。五常主智。其色黑,其味
咸,其性聪明,其情良善。旺相,则机关深远,足智多谋,学识
过人,诡诈无极,面黑光彩,语言清和。太过,则是非好动,飘
荡贪淫。不及,则人物矮小,行事反覆,性殖不常,胆小无略。
对于这种五行配人性情相貌,大致逢生旺的好,逢死绝的
差,此外如有太过或不及的,也都因为失掉了中和之美而流入
•176-
偏执一路,成不了上品的人格。
由于五行配人性情相貌,内容较杂,为了钩玄提要,便于
记忆,前人也有把这概括起来,编成小赋的形式。《宰公要诀》
说「智高量远,盖因水处深源;笃信守仁,只因土成山岳;仁慈
敏厚,木成甲乙之方;性速辨明,火应丙丁之位;誉高义重,因
金归合庚辛。处于中者,正性不移。或盛或衰,性情变易。水
乘衰败,性昏无赖;土力太微,蔽执寡用;木归蹇地,太柔而治
事无规;火数未兴,小辨而太伤无决;金当浅薄,虽义而有始无
终。”《子平赋》说「美姿貌者,木生于春夏之时;无智识者,水
困于丑未之日。性质聪明,盖为水象之秀;临事果决,皆因金
气之刚。五行气足,体必丰肥;四柱无情,性多顽鄙」《指迷
赋》也说「文章明敏兮,定须火盛;威武刚烈兮,乃是金多。木
盛则怀恻隐之心,水多则抱机巧之智。至土之性,最重为贵」
这些口诀,易读易记,并且角度不同,可以彼此补充,所以很受
欢迎。
然而,在正式算命中,一个人四柱八字所秉受的五行,又
总常常和这里描绘的一些性情相貌对看起号来,有的甚至还
出入很大,来个一百八十度的截然大相反。所以陈素庵《命理
约言》说:“旧分五行论人性情,此不可拘。如木主仁寿慈,然
有成局入格之木而不仁者矣;金主肃杀,然又有得时乘势之金
而不杀者矣。”为此,陈氏认为:“须先看柱中神情气势,或正
大,或光显,或纯厚,或英发,皆贤人也。或偏驳,或晦昧,或刚
戾,或卑琐,皆不贤人也。又看取格取用,或中正显白,无所贪
恋包藏,或奇巧隐曲,多所牵合攘取,则性情大端可睹矣。然
后以五行推之,深则见其肺腑,浅则得其梗概。其有始正而终
邪,始驳而终粹者,则行运使然耳。至于二德多善,贵人多贤,
• 177 •
空亡多虚,劫杀多暴,理之所有,然执一端取断,亦不验也」
大致人们更多的则是根据日常接触,把仁而有博爱恻隐
之心,直朴清高,骨骼修长的人说成是秉有木性气质, 把礼而
有辞让端谨之风,精神闪烁,聪明性急的人说成是秉有火性气
质;把信而言行相顾,忠孝至诚,背圆腰阔,面肥色黄的人说成
是秉有土性气质;把义而英勇豪杰,仗义疏财,体健神清,面方
白净的人说成是秉有金性气质;把智而机关深远,诡诈无极.
面黑光彩,语言清和的人说成是秉有水性气质。这样,就把八
字五行硬性相配的说法,给反因为果,反果为因地颠倒过来
了。
有趣的是,早在我国第一部医学典籍《黄帝内经》中,也有
与这相类似的通过阴阳五行原理,把整个人群划为二十五种
人的做法,并不厌其繁地详细记述了每一种人性情形貌的大
致情况。不过由于这是医学上用以研讨各种不同类型的人的
性情脾气,从而为治疗服务的一种学术体系,所以不可和这里
的五行划分划上等号。然而无论如何,如果从根有上作进一
步考察,则又由于我国古代阴阳五行哲学原理广泛深入于每
一学术领域,所以两者在貌似不同之中,又有着本质上惊人的
相同一面。
除此之外,还有一种结合用神推断个性的方法。其大要
为,
〔正印〕
正印为用神的为人仁慈端方,聪颖多智,温文善良,有志
向,有内涵。
反之正印为病,那就未免志向过高,脱离实际,流入自愎
的泥潭。
•178
〔偏印〕
偏印为用神的为人精明干练,思想纯熟,领悟力高,有时
往往在特殊领域有所建树。
反之偏印为忌神的,那就未免心思不定,多疑多虑,愁肠
百结而自寻烦恼了。
〔正官〕
正官为用神的人为人光明正直,注重理智,做事有所收敛
约束,并因此而获得人们的尊敬。
反之正官作为忌神,那就未免优柔寡断,办事缺少魄力和
积极性,陷入无能一路。
〔偏官〕
偏官为用神的人侠义好胜,志向远大,有进取性,因而往
往成为权威显赫的人。
反之偏官作为忌神,那就未免性情急躁,叛逆好胜,且又
刚愎自用了。
(正财〕Y
正财作为用神的人性情温和,思想纯正,勤俭持家,肯负
责任。
反之正财作为忌神,那就未免做事刻板,不知变通,并且
比较看重经济,显得吝啬。
〔偏财〕
偏财作为用神的人聪敏奇巧,办事干练,豪爽而又能够遇
事变通,所以比较适宜于经商,并且人缘极好。
反之偏财作为忌神,那就未免风流好色,为感情事而不惜
钱财,并且有时往往性子偏急。
〔食神〕
• 179 •
食神作为用神的人温和恭良,忠厚正直,思想脱俗,气质
高雅,多为有才华的读书人。
反之食神作为忌神,那就未免思想过高,自命不凡,喜欢
胡思乱想而内心往往感到空虚。
(伤官〕
伤官作为用神的人志向高远,英明锐利,聪明绝顶而富于
智谋,并且多才多艺。
反之伤官作为忌神,那就未免性情刚愎,傲骨嶙峋,并且
还常常带有那么点刻薄。
(:比劫〕
比劫作为用神的人率直稳重,充满自信,意志坚强,富有
自我意识。
反之比劫作为忌神,那就未免坚持己见,刻板固执,招惹
是非,操劳一生了。

\section{看八字,论六亲}
在算命中,算命先生除了给本人论命之外,还常少不了从
一个人的八字,来推论他的六亲。对于算命先生这一套看八
字,论六亲的办法,命书中也自有它的说法6 按照惯例,他们
的看法是: •
C祖上〕看祖上位在年宫(柱),一般以偏印为祖父,伤官
为祖母。 .
£父母「 看父母位在月宫,一般以偏财为父,正印为正
•180 •
母,偏印为庶母。但也有不分正偏的。
〔兄弟〕 兄弟位置附于月宫,命书以比肩为兄弟。至于
姐妹,有的命书不论,有的认为和兄弟的看法一样。
〔妻妾〕 看妻妾位在日支,命书以正、偏财为妻妾。
〔子息〕 看子息位在时宫,又以偏官(七杀)为男,正官为
女。
为什么要以偏财和印为父母,比肩和劫财为兄弟,正财偏
财为妻妾,偏官正官为子女呢?这里先从夫妻说起。比如说
东方甲乙木,假设甲是阳木为兄,乙是阴木为妹,现在甲木把
乙木配给庚金为妻,因为古人认为,女子理应柔顺而听命于丈
夫,好跟着丈夫过日子,所以克乙木的庚金便自然成了她丈
夫。同样道理,庚是阳金为兄,辛是阴金为妹,庚把妹妹辛金
配给克她的丙火为妻;丙是阳火为兄,丁是阴火为妹,丙把妹
妹丁火嫁给克她的壬水为妻;壬是阳水为兄,癸是阴水为妹,
壬把妹妹癸水嫁给克她的戊土为妻;戊是阳土为兄,己是阴土
为妹,戊把妹妹己土嫁给克她的甲本为妻。这样一个阳干,娶
一个被克的阴干为妻。在命理学术'语中,被克的称为正财和
偏财,所以正、偏财在八字中就成了算命先生看妻子的一个重
要手段。然而,阳干娶阴干为妻,在实践中却也并不这么绝
对,假如一个人八字中日柱的天干恰是阴干的话,那末他也理
所当然地可娶阳干或阴干为妻,如乙木娶戊、己土为妻妾便
是。
再说何以偏财和印为父母?在通常情况下,庚金是由己
土生出来的,其中己土属于阴性,所以为母。命书以生我者为
正印、偏印,所以印就成了母亲。这里,生庚金的己土在天干
中与甲相合,这在前面的篇章中已经有所提及。阳木甲克阴
•181 •
土己,自然甲就成了己的丈夫。可是对于庚金来说,我克者为
正财、偏财,现在庚金克甲木,阳与阳相克,岂不就是偏财?前
面我们说过,我克的正财、偏财都是妻妾,现在忽然又把偏财
说成了父亲,岂不荒谬?为此,命书又有结合年宫看父母的说
法,进行解围。
那末,克我的偏官为儿子,正官为女儿又何从说起呢?原
来乙庚结为夫妇后,乙木生出丙、丁的火来。丙火克庚金阳克
阳,所以丙火便成了庚金的儿子;丁火克庚金阴克阳,所以丁
火就成了庚金的女儿。
末了再说兄弟。因为兄弟是同类,所以庚金的比肩庚金,
便就成了兄弟。
以上这些用神术语名称来看六亲的理由,《子平真诠》曾
简要总括道:
正印为母,身所自出,取其生 我也。若偏财受我克
制 ,何反 为父?偏肘者母之正夫也,正印为母,则 偏财为
父矣。正财为妻,受 我克制,夫为 妻纲,妻则从夫。若官
杀则克制乎我,何以反为子女者?官 杀者财所生也,财为
妻妾,则官 杀为子女矣。至于比肩为兄弟之类,又 理之显
然者。
以上所述有关六亲的看法,清人任铁樵氏并不完全表示
赞同,他提出理由为,“子平之法,以财为妻。财是我克,人以
财来侍我,此理出于正论,又以财为父者,乃后人之谬也。若
据此为确论,则翁妇同宗,岂不失伦常乎?虽分偏正之说,究
竟勉强。财之偏正,无非阴阳之别,并不换他气,且无犯上之
理,宜辨而辟之。如果财为父,官为子,则人伦灭矣,不特翁妇
同宗,而显然祖去生孙,有是理乎?是以六亲之法,今当更定:
•182 •
“生我者为父母,偏、正印绶是也;我生者为子女,食神 、伤官是
也;我克者为妻妾,偏、正财星是也;克我者为官鬼,祖父是
也I 同我者为兄弟,比肩劫财是也。此理正名顺,乃不易之
法。”
现结合古书所举命造,分析如下:
出身官家例〕
(年) 官癸卯
伤
(月) 印乙丑财
官
(日) 丙子官
(时) 伤己丑
日柱丙子,年月官印透而得禄,财星藏而归库,所以出身
官家,但美中不足的是丑时伤官肆呈,官星退气,日主衰弱,全
凭印绶乙木生火卫官。结合行运,亥运印绶逢生,入泮优游;壬
戌水不通根,破耗异常:酉运财星坏印,竟伏法而死。
(:母寿父夭例〕
(年) 丁酉偏财
(月) 壬子
(H)丁卯
(时) 印绶甲辰
命造日主丁卯,生于冬月,月干壬水为正官,月支癸水为
七煞,官煞气旺,自身衰弱,所以当取生我的印绶甲木为用神。
现在地支卯辰东方会木,印旺有气,而我克的偏财酉中辛金却
偏处一隅,没有根攀。正印为母,偏财为父,所以命主母长寿
而父早丧。
•183
匚娶妻贤淑例〕
(年) 伤癸卯财
印
(月) 财乙丑劫
伤
印
庚申比
食
(时) 官丁丑
这一命造,寒金坐禄,印绶当权,足以用火敌寒,所忌年干
癸水克丁为病,全赖月干乙木通根,泄水生火。由于喜神即是
财星,更喜财星逢合,谓之“财来就我”,所以其妻勤而多能,生
有三子,皆就书香。
〔娶妻得财例〕
枭
(年) 比丁未比
食
劫
(月) 枭乙巳伤 '、
财
(日) 丁酉财
(时) 杀癸卯枭
丁火生于孟夏,四柱中枭劫当权,一点癸水,制枭无力,妙
在坐下酉金,冲卯木而生癸杀,所以出身虽然贫寒,可是一交
癸运,时来运转,入学复得妻财。此后壬运登科,辛丑出选知
县,仕至郡守。为此,日下若无酉金冲卯生癸,不但妻财无着,
并且名也不成了。
匚 娶妻妒悍例〕
•184 •
茶
(年) 印乙亥1
官
(月) 财庚辰印
食
食
(日) 丙申财 .
杀
(时) 杀壬辰
丙火生于季春,印绶通根生旺,时干又透壬水,所以当取
印绶乙木作为用神,然而坏在庚财合乙化金,生杀破印,故卜
其妻妒悍不贤,无子而绝。所有这些,都是财星坏印所造成的
危害。
1妻遭刑克例〕
(年) 偏印癸卯乙木比肩
(月) 比肩乙卯乙木比肩
(日) 乙未丁火食神 乙木比肩 己土偏财
(时) 印经壬午丁火食神 己土偏财
这一命造,日主乙木,生于春月,得令身旺。日支未和时
支午合,其中己土为妻为财。可惜四柱中乙木重叠,比肩太
过,所以不必运行比劫,妻子也会遭克。
〔食伤为子例〕
(年) 印丁酉伤
官
《月) 印丁未劫
印.
印
(日) 戊戌比
伤
•185*
印
(时) 印丁巳比
食
日元戊土,生于季夏,四柱印绶重叠,水气全无,燥土不能
泄火生金,所以连克三妻五子。此后直到交入丑运,湿土晦火
生金,又以运与柱中时支年支巳酉会成食伤金局,所以得一子
而养育成人。
〔转运得子例]
(年)辛卯
(月)辛卯
(日)甲辰
(时)丁卯
这一命造,春木雄壮,喜在时干丁火透露,伤其年干月干
无根辛金。然而当运行己丑、戊子时,因为丁火泄气不能为
用,所以非但得子不育,并且财多破耗。此后一旦运入丁亥、
丙戌,因其或为支拱木而干透火,或为火通根而发库,故而连
得五子,家业增新。为此,任铁樵总结「凡八字之用神即是子
星,如用神是火,其子必在木火运得,或木火流年得,如不是木
火运年得,必子息命中多木火,或木火日主,否则难招或不肖,
试之屡验。然命内用神不特妻财子禄,而穷通寿夭,皆在用神
一字定之,其可勿诸?”
1兄爱弟敬例】
(年) 丁亥
(月) 壬寅
丙子
(时) 丁酉
•186
丙火生于初春,谓之“相火有焰”,不作旺论。再看四柱五
行,虽说月干壬水通根,亥子煞旺无制,然而好在年月干支,
丁、壬、亥、寅皆能合而化印,若说时支酉金财星坏印,却有时
干丁劫盖头制伏。由于丁劫为局中所喜所用之神,所以同胞
七人,兄爱弟敬,并且都是知识分子。
〔兄弟遭累例〕
(年) 癸巳
(月) 戊午
(日) 丙午
(时) 庚寅
这一命造,不但羊刃当权,又逢生旺,并且年干月干戊癸
合而化火,柱中庚金之财,劫夺殆尽,所以兄弟六人,都不成
器,遭累不堪。对照这一命造,任铁樵颇有感慨地自述说「余
造年、月、日皆同,换一壬辰时,弱杀不能相制,亦有六弟,得力
者早亡,其余皆不肖,以致拖累破家」接着,任总结道:“总之,
劫刃太旺,财官无气,兄弟反少,纵有不如无也。然官杀太旺
亦伤残。必须身财并旺,官印通根,可敦友爱之情。”
对于看八字,论六亲,台湾颜昭博《子平八字大突破》另有
研究成果。他认为,看祖上的要义为:以年干支对日干,若年
干支为日干之喜神,则我受其荫;反之,则无荫可受。在干为
明荫,在支为暗荫。明为物质之荫,暗为地理风水之荫。以偏
印为祖父,若偏印为日干之喜神,则祖父和我感情佳;反之,感
情平淡。并看偏印生存力高低。同时指出:以年干对其它三
干、四支,查年干之生存力,若生存力高,则祖父成就高,若生
存力低,则祖父成就低。以年支对年干,若年支为年干的喜
神,则祖父母恩爱。
•187-
看父母的要义,《子平八字大突破》这祥认为:以月干支对
日干,若月干支为日干之喜神,则我受其荫1 反之,则无荫可
受。以正印为母,偏财为父,若为日干之喜神,则感情深;反
之,感情平淡。并看正印、偏财生存刀之高低。以月支对月
干,若月支为月干之喜神,则父母恩爱。
至于夫妻,《子平八字大突破》分析为:凡人论命,不离夫
妻,而夫妻之看法,命书虽杂,万法不离其宗,列下四柱,首观
日支之为喜、为忌,则夫妻助力之大小已知三分;次看财、官之
为善、为恶,则知夫妻之缘深、缘浅;若苟日支被刑冲,则知夫
妻不稳;再之劫、伤之是否太重,可断配偶之是否刑伤。而人
命有伤无官,则如脱缰野马;有官无伤,虽色欲而能自制。岁
运引色,则家有风波,其中内容仔细详推。女命官杀混杂,婚
姻易有第三者介入。
还有子息,《子平八字大突破》的看法是,以时干支对日
干,若时干支为日干之喜神,则我受子荫;反之,则无荫可受。
以食伤为子息,若为日干之喜神,则感情深;反之,感情平淡。
并看干日生存力之高低。以时干对其他三干、四支,查时干之
生存力,若生存力高,则子息成就高;若生存力低,则子息成就
低。
《子平八字大突破》的说法,附有例子,
〔看祖上例〕
(年) 甲寅
(月) 乙亥
壬戌
(时) 丙午
年干甲,年支寅非日主壬水之喜,明荫、暗荫皆无。偏印
•388-
庚金为日干之喜,然命中不见,和祖父缘浅。年干甲木自坐寅
禄,壬水支藏干透相生,丙火透干,祖父成就高。年支为年干
之禄地,祖父母恩爱。
【看父母例〕
(年) 辛卯
(月) 癸巳
(日) 乙丑
(时) 己卯
日干乙木生巳月,伤官生财弱身,癸水生乙并调候,癸为
乙之喜神,巳反为忌,故明荫有,暗荫无。偏财弱身,和父亲感
情平淡;正印为喜,但命中缺如,疏疏离离。癸水生巳月,余气
于丑,又逢辛金相生,唯己土制克,巳火熬沸,癸水生存力非
高,故父亲成就平凡。月支巳非癸之喜,父母感情非美。
〔看父母例〕
(年) 甲寅
(月) 乙亥
壬戌
(时) 丙午
月干乙非日干壬之喜神,不受其荫;亥为壬之禄地,唯寅
亥化木,始受其荫,终虚空。正印为喜,藏于库,母亲有情,体
弱多招。偏财为忌,父亲相处终无蜀。月干乙木自坐长生,年
支羊刃,壬水相生,丙火调候,父亲成就高。月支亥为月干乙
之长生,父母恩爱。
(看婚缘例〕
(年〉 丁卯
(月) 甲辰
•189 •
(日) 丙戌
(时) 甲午 3
乙酉年和日柱丙戌成一级相顺,婚缘。所谓一级相顺,如
日柱丙戌,按天干相顺乙之后为丙,地支相顺酉之后为戌,就
算是一级相顺了。其他日柱干支类推。
(:看婚缘例〕
(年) 甲子
(月) 癸酉
(日) 庚子
(时) 己卯
乙亥大运,乙未流年和时支卯三合财地,婚缘。
〔看婚缘例〕
(年) 庚寅
(月) 丁亥
丙子
(时) 乙未
原命杀星为寅所合,逢巳、申之流年冲开,婚缘。
1看晚婚例〕
(年) 丁丑
(月) 丙午
(日) 辛巳
(时) 壬辰
八字唯一偏财星入库,宜晚婚。
〔看晚婚例〕
(年) 己丑
(月) 癸丑
• 190
戊午
(时) 癸丑
四柱不见官杀,宜晚婚。
〔看夫妻例〕
(年) 乙未
(月) 庚辰
(日) 庚子
(时) 甲申
正财、偏财双露,婚姻易有第三者介入。
〔看夫妻例〕
(年) 戊戌
(月) 癸丑
丙午
(时) 辛卯
夫星被合,夫情不向我,日支喜神,可得夫荫,唯美中不
足。
〔看夫妻例〕
(年) 戊子
(月) 戊午
(日) 壬申
(时) 壬寅
子午冲,官星受损,寅申冲,夫官冲离,婚姻败象。唯夫宫
喜神,又有别论。
〔看子息例〕
(年) 甲寅
(月) 乙亥
壬戌
(时) 丙午
丙午非日干壬水之喜,难受子荫。食伤为忌,子息虽多,
无一旁侍。丙火亥月,令气为死,壬水紧临相克,寅午戌三合
火局,印绶相生,丙火弱中转旺,生存力高低随运岁而迁异,子
息成就高低亦随之起伏。
关于看六亲祸福吉凶,穷通寿夭等办法,近人林惠祥《算
命的研究和批判》还曾概括道:“亲属自己身强者隹,遇克他
的,即遭克死,但如能逢生他的,略有希望。或逢有能克克他
的,也有救。偏财旺者父长寿,比劫多者父早死,正印有力者
母寿,财多破印即主克母。例如己身日主为甲木,其财(父)为
戊己,印(母)为壬癸,戊己土克壬癸水,水(母)被克死。比肩
财多者兄弟多。见比肩、劫财、败财都会克妻妾及父,例如己
身为甲木,比肩、败财即甲乙木,妻妾为正财、偏财,即戌己土。
甲乙木能克戊己土,故妻妾被克死。又父也是偏财,故也被克。
坐下妻宫主妻好,妻即用神也主有贤妻。妻星多主克妻,妻星
两透,偏正杂出,则有多妻。日干坐下的地支遇刑冲会克妻。己
身强妻弱者,应配能补救这种弱势的女人,这称为 '硬配 官
杀多,伤兄弟姐妹,例如正官偏官的庚辛金太多,则伤及比肩、
败财的甲乙木,即伤兄弟姐妹。伤官食神多会伤子息,因为丙
丁火克庚辛金。枭印多克祖父母(壬癸水克丙丁火)。看子息
的方法,应先寻子星,再对照时的地支,照生旺死绝(参看本书
《五行的旺相休囚死和寄生十二宫》)来推,推法依下面的歌:
长 生四子中旬半,沐浴一双保吉祥。
冠带临官 三子位,旺中五子自成行。
衰中二子病中一,死中至老没儿郎。
•192
除非养取他人子,入墓之时命夭亡。
受 气为绝一个子,胎中 头产有姑娘。
养中三子只留一,男女宫中仔细详。
笔者
歌中意思说
注)。 庚如逢
,假如己身
时支巳
是甲,
,便是
子星
在长生
便是庚
的状态
(庚是甲
,可
的偏官
以在中年时—— 得四子;如逢午便是在沐浴的状态,可以有二子;在冠带、临官
都有三子;在帝旺有五子;在衰中有二子;病中有一子;在死的
状态,无子;逢墓的状态子会夭亡;在绝中有一子;在胎中会有
长女;在养中生三子留一子
现综括全文,似乎可作这样的理解,就是四柱中如果年柱
是吉神或用神的,说明命主祖基较丰;月柱是喜神或用神的,
说明命主有父母荫庇,并且兄弟和睦;自支是喜神用神的,说
明命主夫妻协力,爱情甜美;时柱是喜神用神的,说明命主子
女得力。反过来说,如果年柱是忌神的,说明命主祖上破败寒
微;月柱是忌神的;说明命主父母刑伤,兄弟不和;日支是忌神
的,说明命主夫妻爱情生活不谐;时柱是忌神的,说明命主子
女难育或子女不肖。当然,这些出现在年、月 、日、时上的忌神
如果有制,则又可以逢凶化吉,另作别论。
再如从用神结合十二宫等来看,八字中如果以印绶作为
自身的喜神,或印绶遇上长生之地,说明主人有着很厚的福
荫,并且父母双双长寿;月柱上的印绶遭逢死绝之地,或者作
为用神的印绶被破,说明父母不全,或难以得到父母的荫庇。八
字中如果比肩、劫财为自身的喜神用神,或比肩坐在禄地的,
说明兄弟得力;反之,比肩如为忌神的,说明不是兄弟不睦,就
是兄弟有凋难。八字中如果以财为喜神用神,并且生化有力
的,说明妻子贤惠而又得力;反过来如果财为忌神,或冲合争
• 193 •
分,说明妻子不服丈夫,夫妻感情不好。女命以官、煞为丈夫,
八字中如果官星得用,丈夫高贵而自身也贵;以食、伤为子女,
八字中如果食、伤为喜神用神的,说明子女贤孝,有享子女之
福的可能,如果食、伤逢冲,或坐孤辰孤宿,说明子女稀少,或
者命中克子。
此外,在看八字论六亲时,还可结合行运来看。其中父母
结合幼运,夫妻兄弟结合中运,子女结合老运。比如命中幼年
有鸿运的,说明有着双亲的福荫;中年有鸿运的,说明夫妻协
力同心,或者得兄弟的力;晚年有鸿运的,说明得子女之力。
末了再说几句,就是有关六亲相克的问题。对于六亲相
克,我们有必要在这里加以澄清,否则一味迷信,往往造成夫
妻、父子、兄妹等不和,有时还常造成更加严重的后果。比如鲁
迅笔下的祥林嫂,由于夫死子殄,人家都说她的命硬,克夫克
子,结果在社会舆论和习惯势力的强大压力下,当年充满活力
的祥林嫂,就这样活生生地被封建和迷信给吃掉了。想到这
里,我们为祥林嫂感到极大义愤和不平。其实,古人所谓相
克,并不是说真的就要把谁给克死了。为此,清代陈素庵的那
段话,说得很是中肯。他说「世俗相传,父命凶,则能克子,子
命凶,则能克父,夫命凶,则能克妻,妻命凶,则能克夫,遂至有
骨肉相怨憎者,此说殊误。凡父命中子星破坏,可以推其子之
不肖,非因父命而克子也;子命中父星衰绝,可以推其父之早
逝,非因子命而克父也;夫命中有妻星损坏,可以推其妻之频
丧,非因夫命而克妻也;妻命中夫星死绝,可以推其夫之不禄,
非因妻命而克夫也。”
当然,陈素庵的用心虽好,可他的立足点,却仍站在天命
必有的基点上。然而无论如何,我们是不该,也没有必要用我
•194 •
们今天的立场去要求古人的。

\section{怎样看大运和流年的吉凶荣枯}
从前面《八字中大运、流年和命宫的推算》中,我们得知了
大运的推算方法和起运岁数。譬如公元 1940 年庚辰农历十
月十四日出生的男命,可以对照《新编万年历》查出,那一年的
农历十月十四日上一个节是十月初八的立冬,下一个节是十
一月初九的大雪。由于庚辰是阳年,按照规定,阳年生的男命
起运岁数是顺数到下一个节止,然后以三天为一岁去除。那
年庚辰年的十月是小月,所以从十月十四顺数到十一月初九
大雪是二十四天,再用二十四去除三,正好是整数八。这就是
说,这位先生的起运岁数该是八岁。起运岁数算出以后,接下
来是排大运的干支。同样我们知道,大运的干支是根据生月
的干支推排出来的,起运岁数如果是顺数的,就由生月干支的
下一个干支顺排下去,逆数的,就由生月干支的上一个干支倒
排上去。现在已知生月是丁亥,起运的岁数是顺数,所以这命
的大运干支就应该从丁亥起依次顺数为戊子、己丑、庚寅、辛
卯、壬辰、癸巳、甲午、乙未。
由于命书规定,大运的天干地支每字都管五年,所以一个
天干和一个地支加起来是十年。看前五年的虽以天干 为主,
但要结合地支一起推看,看后五年的一般丢掉天干只看地支,
这就是命书所说的大运中地支重于天干的原则。
现在仍以上述八岁起运的男命为例,可以得知他的大运
• 195・
• 196 •
八到十七岁是戊子,十八到二十七岁是己丑,二十八到三十七
岁是庚寅,三十八到四十七岁是辛卯,四十八到五十七岁是壬
辰,五十八到六十七岁是癸巳,六十八岁到七十七岁是甲午,
七十八岁到八十八岁是乙未。
推论大运的吉凶荣枯,先要从本命日柱天干出发,分析本 .
命五行的宜和忌,再结合大运干支所代表的五行对本命日柱
天干的生克扶抑,是宜是忌,以及有没有刑冲化合等等,才能
作出最后的判断。对此,《命理探原》曾引陈素庵的话:
宜与不宜,全凭格局;利与不利,但问天干。破
格者(破坏自身格局的)值之为戚(忌), 助格者遇之
为欢(宜1 日弱者扶之而气盛,日强者抑之而全美。
旺日复到旺乡 (大运的五行对于自身日干来说显得
太旺),必 罹悔吝(凶"衰日再临衰地(大运的五行对
于自身的日干来说显得太衰),定 主摧残(凶 吉若
财官印食,喜于相见;凶如刑冲枭劫,多主不安。
比如日干是金,命中金强,最理想的就是能行水木火的食伤财
官运,因为火能制金,不致使金太旺而走向反面,而金又能够
生水克木,使强金有了疏泄的余地,如果遇上生金的土运和比
肩、劫财的金运,对于本人来说,无疑造成了一种“旺日复到旺
乡”的架势,因此极为不吉利。反之如果日干是金,命中金弱,
那就又要来个一百八十度的大相反,宜行生我扶我的印绶、比
「劫运,否则身弱再遇财官,岂不等于“衰日再临衰地”,又哪有
不“摧残”的?
以上推大运吉凶荣枯的办法,如果结合用神来作判断,就
是八字四柱配合较好,原局中有用神的,那末一生行运一般多
为水流花开,得意得很。然而对于一些八字原局四柱配合得
并不理想,原局中没有用神,或者用神较弱的,就要看他行运
时碰上碰不上用神了。一个人行运一辈子,不可能在方向上都
遇水遇木遇金,如果原局中所缺的用神,在行运中一旦补上,
纠正命中五行的偏差或不足,也可发福或有所作为。对于这两
种原局和行运中的用神,行家们分别有“原局用神”和“行运用
神”的叫法。从总体上说,凡是日主旺的,宜行财、官运。日主旺
而财、官弱的,行到财、官运时必定一下大发。如果日主旺过头
的,则宜行食、伤运,以泄自身过旺之气。反过来,凡是日主弱
的,宜行比、劫或印绶运。日主弱而财、官旺的,行比、劫运比印
绶运更好。假如日干不强不弱,叫做中和,中和的人也适宜于
行财、官运。
还是举例说明来得容易理解,我方这里且看这样一个乾
造:
(年) 庚辰
(月) 丁亥
庚申
(时) 庚辰
八岁起运,大运如下:
M
雷地支会水
二十八 庚寅 ) 三十八 辛卯 地支会木
四十八 壬辰 j
五十八 癸巳 ) 六十八 甲午 地支会火
七十八 乙未 j
• 197 •
八十八 丙申
庚申称专禄,所谓禄,就是寄生十二宫里的临官。男子占
禄,杖地造屋。命里四金,二土,一水,一火,缺木。五行缺木,
亥里藏甲木,辰里藏乙木,生日木日(在纳音五行中,庚申属石
榴木)。
此命生于冬天,金寒而重。年份比肩,月份正官、食神,日
支比肩,时干比肩。
八字中比肩多命硬,爱人年龄相差耍多,否则要重婚。同
岁、兔、狗不配,与猴、鸡、鼠相合。兔属卯,龙属辰,卯辰相害;
狗属戌,龙属辰,辰戌相冲,所以都不配。猴属申,鼠属子,龙
属辰,申子辰合水,所以相合。此外辰酉合金,鸡属酉,所以和
属酉的鸡也相合,然而这也不是绝对的。
八字中用神取食、伤、财、官,水、木、火三神一气。走印
绶、比劫运不好。
八岁起运。八到十二岁偏印,身体多病。十三到十七岁
伤官,也不顺利。十八到二十二岁正印,读书刻苦。二十三到
二十七岁墓库,多损失。二十八到三十二岁比肩,因为命中已
有四金,比肩多,故而不好,自己真心待人,人家却要暗算。三
十三到三十七岁偏财,寅申相冲驿马运,妻星动荡而有财运。
三十八到四十二辛金劫财运,八字缺木,群劫分财,有损失。四
十三到四十八正财,因卯中乙木与庚合金,所以说不尽甜酸苦
辣。四十八到五十二食神生财,且壬丁合木,一生运势,至此
大转。五十三到五十七岁辰土当心身体,做事收心,破财,有
灾厄。五十八到六十二岁伤官癸丁交战,不妙。六十三到六
十八岁巳运长生,亦忧亦喜。六十八到七十八岁偏财、正官,
这步运好。总之六十三岁后大运干支一片木火,用神得力,老
•198
来喜乐无忧。
注意,五十三到五十七防财、防身体,三十二以内吃亏,三
十三以后偏财。四十八岁后成名。
此命幼年吉凶参半,最好与父亲分开些。青年时有较大
的挫折,中年开始转好运,直至晚年。一生有偏财,但也经常
破费。此命东西南北尽皆通,有名望,从文更好。此命要多注
意身体,因为命中金太多。又因为官星为日主庚金喜神,所以
子媳好,晚年幸福。
当然,正如世界上任何事情,都会有不同的看法一样。所
以对于同一命造,由于各人的理解不同,也往往会产生不仅
相同的批语,这是不言而喻的。
为了明白起见,我们这里再以家住天津河北区民权门靖
江南里的命理学家王希金,为出生于 1943 年的台湾女作家三
毛女士所批命书为例。王希金在给笔者的一封来信中说:“去
年(1990 年)仲冬,台湾女作家三毛女士忽殒自身,各界为之
惊叹。余因好奇,据诸刊、报章之载其各方面状况,断其生时,
演成命局(我已写成评述其死因文章),简为评批。”出生时辰
是按照三毛生平各方面的情况,斟酌推断出来的。批为:
(年) 癸未
(月) 乙卯
癸未
(时) 甲寅
四岁起运,大运如下:
四 丙辰
十四 丁巳
二十四 戊午
•199 •
三十四 己未
四十四 庚申
癸未日元,生提乙卯,时落甲寅,木旺于春令,大有木盛水
缩之患。但《喜忌篇》有云广六癸日得寅,岁月怕戊、己二方。”
此乃刑合格已为成立,佳造也。局中食神、伤官秀气满盈。或
云「弃命从儿格。”似较勉强,因为支下未能会成木局。按其
喜忌,尤喜水、木 、火之乡。水助比,木泄秀,火生财。四到十
三岁丙辰,火土交出,忧喜参半。
十四到二十三岁丁巳,大运干支皆火,优佳可卜。又十九
岁之巳火乃驿马,又与时支寅木相刑,故忽走他乡。《气象篇》
说「驿马带剑(无缰之马),山斗文章,潇洒出尘」故著述甚
丰,一时洛阳纸贵。 ’
'
二十四岁到四十三岁,大运戊午,己未,虽然文思泉涌,财
运不断,可是因为平生最忌戊、己之方,所以夫星不禄,说不尽
甜酸苦辣,尤如天边孤雁,日暮觅途。
四十四岁大运庚申,含气彻地透天,凶神大临,金来伐木,
用神溃败。去岁(1990 庠)庚午,月透戊干,双庚克木,岂不殆
乎?又《玄机赋》说:“运贵在于支取,岁重向乎干求。”故戊、
己、庚之盖头土,金优为患耳。未运之华盖,见其三毛女士,文
学著作已达高峰。
总之,观其造化之背顺,命局清秀无比,英华外发,然运途
未能媲美命局,憾也;正谓源清而流浊耳。
又如袁树珊《命理探源》为某比丘所推一命:
(年) 甲申
(月) 辛未
(日) 己未
•200 •
(时) 甲子
安命甲戌。十岁起运,大运如下:
十 壬申
二十 癸酉
三十 甲戌
四十 乙亥
五十 丙子
六十 丁丑
七十 戊寅
八十 己卯
为了保持原批风格,以窥命书文理的一斑,现把袁氏所批
引述如下:“己与甲合,不特正五行属土,即化气五行亦属土。
此化土于小暑后一日,赤帝正司权,土王未用事,格局虽佳,精
神不足。再逢申支藏庚,暗地化金,以泄土气,更难以名公巨
卿言也。所幸时带空亡而会天乙,秉性聪明,致使寄生净土,
机缘凑合,尤应得志沙门。若再具有克治之功夫,非惟免尘俗
纷争之扰累,且可获如来上乘之真诠,岂不妙哉!二十岁前,
运途多舛,困难屡遭。自二十一岁交癸运,春风乍暖,柳眼初
舒矣。二十六岁交酉运,外圆内缺,人岂知之。三十岁与太岁
冲克,芝兰化为荆棘,惜哉1 三十一岁七月十六日交甲运,拂
开天上云千里,捧出波心月一轮。三十六岁交成运,除旨十七
岁之灾惊外,其余四年,登极乐国。四十一交乙运,去彼贪嗔,
保我定慧,慎之。四十六岁交亥运,接丙运、子运,十五年幸福
无量。六十一岁交丁运,与化土之格极形反对,此倦飞州还时
也,请留意焉。法子七二枝,寿元六旬外
对于批命书大运,当今台湾命理学家还有综合分析,批得
•201
更为详尽的。作为一种学术讨论,我们这里且引《子平八字大
突破》所载一例:
乾造,1931 年辛未农历十月初二丑时生
(年) 辛未
(月) 己亥
(日) 庚午
(时) 丁丑
一岁起运,大运如下:
一 戊戌 .
十一 丁酉
二十一 丙申
三十一 乙未
四十一 甲午
五十一 癸巳
以下展开分析:
分析日干生存力之高低
庚金生亥月,令气为体。
十月庚金气寒。
年干辛,月干己,年支未,时支丑生助。
时干丁火,日支午火去寒。
总而论之,日主庚金精神舒畅,气力不弱,可以任财官。
正官生存,力之分析
月干丁火生亥月,名气为死。
月干丁火仅日支午火生助,余皆克泄。
总而论之,正官力量非强,然亦不弱。
偏财之生存力分析
•202 •
月支亥中藏甲,甲生亥月,令气为相。
仅月支亥中壬水生助,余皆克泄。
总而论之,偏财力量非强,偏于衰弱。
四柱结构之分析
年柱和时柱天克地冲,年柱和日柱一级相承。
出身之环境分析
查看初运之日干生存力之高低。
初运戊戌,戊生辛、助己、生庚、泄丁,戌助未,克亥、泄午、
助丑。
总而论之,日干庚金气强,而丁火稍弱,甲木亦衰,故知出
身环境可以,然非大富大贵之家。
学业之分析
原命印绶多,且初运戊戌、二运丁酉皆无伤克原命印绶,
故知幼时读书努力。
原命丁、午支藏干透调候,命局寒暖燥湿中和,且初运戊
戌、二运丁酉皆无克伤丁、午,故知幼时文才必佳。
原命食神受制,然二运丁酉有助食神,故知十一到二十岁
智慧启开。
十六岁流年丙戌,大运丁酉,丙、丁助丁火,戌、酉助日•干,
流年不恶,故考运佳。
十九岁流年己丑,大运丁酉,己丑助庚、泄丁,丁酉弱庚、
助丁,流年一般,故考运平平。
总而论之,此人学业必佳。
婚缘之分析
二十一到三十岁丙申大运,申生亥中壬水,壬生甲,食神
生财,心性开始浮动。
• 203 
二十三岁流年癸巳,巳午未三会,未中藏乙,乙乃妻星,星
宫同合,该年女友入门。
二十五岁流年乙未,乙庚合,乙乃妻星,午未合,星宫同
合,亦主异姓入门。
三十岁流年庚子,丙辛合,丙乃子星,子丑合,主生子。
总而论之,此造二十五岁流年乙未结婚可能性最大。
父母之分析
正卯己土生亥月,丁午丑未相生,正印不弱;偏财生亥月,
满盘克泄,偏财衰弱,母寿高于父寿。
己来生庚,正印生日主而不为忌,母亲疼爱,尤以丁酉、丙
申两运,母子感情深重。
原命偏财甲木较弱,对日干之影响力不大,可以忽略不
论,故父子感情平淡如水,难言亲疏。
丙申大运,申亥相害,二十三岁流年癸巳,巳亥冲,亥中藏
甲,甲乃父亲,父子因故分离。
丙申大运为创业时期,丙助丁、申泄丑,总而言之,此时官
星丁火力量增强,月干己土生助日干正是时机,故决非白手成
家。
正印在月干,偏财在月支,母位母填,父位父填,父母品德
节操,必属贤良方正。
夫妻之分析
命中正财弱小,藏于财库,年干透劫,正财对正干之影响
力可以忽略不论,夫妻感情平淡如水。.
原命日干庚金逢己、辛、未、丑相生,身强可以任官,日支
逢官星,妻有助力。
乙未大运,乙庚合,午未合,外情多有。
• 204 •
甲午大运,午午自刑,四十三岁流年癸丑,丑未冲,未中藏
乙,乙乃妻星,夫妻因故分离。
富贵之分析
原命身强,官星不弱,财生官星,贵而不富之命。
丙申大运,丙助丁,申泄丑,身强官也强,贵气临身。
乙未大运,乙克己、生丁,未助未、泄午、克亥,仍然身强官
也强,官运一帆风顺。
甲午大运,甲克己、生丁,午生未、助午、生丑,仍然身强官
也强,官场顺利。
癸巳大运,癸泄庚、破丁,巳生未、助午、生丑,仍然身强官
也强,官场仍然有利。唯已渐趋下风。
壬辰大运,壬泄庚辛、克丁,辰泄午、助未、助丑,身强官星
弱矣,官场不利,多有是非。
是非祸险之分析
癸巳大运,大运和提纲天克地冲,身体开始走下坡。
五十三岁流年癸亥,癸伤丁,亥伤午,官星用神而被伤,是
非之灾。
原命甲乙衰弱,甲乙属肝胆,故知肝胆弱。十一到三十
岁,申酉连克甲乙,肝胆更是衰弱。
其他
原命土气重,土属黄色,心性自然而然亲近黄色系列色
彩,如黄色衣服、黄色家具、黄色大地。’
原命正官管制适当,为人自省、自制、不越礼犯上。
原命印绶不弱,个性内向。
除此之外,更多的是一些浓缩的简批推算之法。这里我
们且看命理学家太 N子朱为纲的.几个例子:
•205*
朱左,1939 年己卯七月初四未时
(年) 己卯
(月) 壬申
(H) 丁亥
(时) 丁未
三岁起运,大运如下,
三 辛未
十三 庚午
二十三 己巳
三十三 戊辰
四十三 丁卯
五十三 丙寅
六十三 乙丑
日元丁火,地支亥、卯、未合成木局,天干丁壬又合而化
木,病在月支申金作梗,所以作假从强论。丁卯、丙寅、乙运皆
美,丑运逊。
洪左,1942 年壬午六月十一日午时
(年) 壬午
(月) 丁未
(H) 丁丑
(时) 丙午
五岁起运,大运如下:
五 戌申
十五 己酉
二十五 庚戌
三十五 辛亥
•206
四十五 壬子
五十五
丁火司令,
癸丑
八字明暗之丁
,
有八重--之多,可作从强推算,大
利南方。四十五岁壬运,丁壬可作化火论,吐气扬眉。四十九
岁子运亦无妨,因子丑相合为土,可以敌水。惟五十五岁癸丑
之运则逊之,身居南方,亦可减轻阻蹇。
胡左,1955 年乙未
(年) 乙未
(月) 戊寅
(H) 丁酉
(W).癸卯
一岁起运,大运如下:
-
丁丑
十一 丙子 .
二十一 乙亥
三十一 甲戌
四十一 癸酉
五十一 壬申
甲乙成林,官衰印旺,喜值戊土司令,更有丁日酉支,药财
破印,八字上乘。然北方水地,乏善可陈,今后西方金乡,必是
大有可为,择善而从,前程不可限量。
金右,1962 年壬寅农历四月十二亥时
(年) 壬寅
(月) 乙巳
(H) 癸丑
(时) 癸亥
• 207
三岁起运,大运如下,
三 甲辰
十三 癸卯
二十三
•
壬寅
三十三 辛丑
四十三 庚子
五十三 己亥
癸水诞于巳月,庚金司令,水众逢金,天行健旺,作强论。
最喜食伤通根,逢木必发,在寅、卯之途,可以奋发图强,能自
立更生,青出于蓝。辛丑、庚子运途,因已巩固基础,不足为
虞。晚运亨通,当在知命之后,庶境更美。
凌右,1962 年壬寅十月初七申时
(年) 壬寅
(月) 庚戌 •
(日) 乙巳
(时) 甲申
八岁起运,大运如下:
八 己酉
十八 戊申
二十八 丁未
三十八 丙午
四十八 乙巳
五十八 甲辰
戊土司令,甲木进气。九月乙木,根枯叶落,须赖癸水滋
养。时逢甲申,藤萝聚甲。四柱乙庚作合,巳申又合,坤命不
宜。事业尚可,婚姻艰巨,佳运要在知命之后,还可人定胜天。
• 208・
再之,看大运除了结合五行宜忌,还有一种“年管少年,
月、日管中年,时管晚年”的说法,这种说法在《三命通会》卷二
中,还具体有“以生月为初限,管二十五年,生日为中限,管二
十五年,生时为末限,管五十年”的三限划定。看法大致以日
干为出发点,其中年柱干支为喜神用神的则少年发达,为忌神
的则少年困苦I日月干支为喜神的则中年亨通,为忌神的则中
年蹇滞, 时辰干支为喜神的则晚年安荣,为忌神的则晚年零
落。但是一般认为,这种看法比起大运的推算来,未免显得简
单了点。
大运之外,流年和命宫的好坏,也都由日主的天干出发进
行五行宜忌的详细推断,宜的为吉为荣,忌的为凶为枯。不要
忘记的是,看流年时,还必须把流年放到大运里去进行观察分
析。大运吉而流年吉的,其年大吉;大运吉而流年凶的,不致
大凶;大运凶而流年凶的,难逃其凶;大运凶而流年吉的,难保
大吉。大运的力量,足以左右流年。打个比方,大运好比大 ,
河,流年好比小河,大河水满,小河也满,大河水浅,小河也
浅,大河的水势足以影响小河,而小河的水势却难以影响大
河。
再有一种流年和命宫结合的看法。流年法以轮流值年的
‘太岁”为首,“命宫如值流年吉神,其年则福,值凶煞,其年则
祸”。由于那些神煞分布在子、丑、寅、卯等十二年中,每年各不
相同,所以对照命宫来看,每年的吉凶也就各各不同了。然而
又由于这些神煞凶多吉少,并且方法粗糙简单,所以袁树珊
《命理探原》指责这种看法说「凶煞有十之九,吉神尽十之一,
其不适用可知。舍干枝五行生克之至理,而惟务此虚文,宜其
毫无效验,贻讥大雅。”连命理学家本人都不相信,那荒谬的程
•209
度,也就由此可知了。
说到“太岁”,大致有两种情况,一种是四柱中的年柱,叫
做当生太岁;另一种是一年年轮流过来的,叫做游行太岁。当
生太岁管终身,游行太岁则每年游行十二宫,以定一年四时的
吉凶祸福。对于后面的游行太岁,《三命通会》卷二《论太岁》
说道,“经云,'岁伤日干,有祸必轻I 日犯岁君,灾殃必重。
“岁君伤日者,如庚年克甲日为偏官,譬君治臣,父治子,虽有
灾晦,不为大害。何则?上治其下,顺也,其情尚未尽绝。如
甲日克戊年为偏财,譬如臣犯其君,子忤其父,深为不利。何
则?下凌其上,逆也,其凶决不能免。若五行有救,四柱有情,
如甲日克戊年,四柱原有庚申金,或大运中,亦有将甲木制伏
纯粹,不能克戊土为有救。经云,'戊己愁逢甲乙,干头须要庚
辛'是也J

\section{入格八字举例}
在命理学家眼里,虽然人们出生时辰的八字千变万化,错
综复杂,可是总得有个格局统帅全局,否则不就乱了套?这就
是入格八字的由来。
关于八字的格,从来就为命书所重视。如《三命通会》卷
六的《杂取各格》,以及《星平会海》卷十的取格析例,都不惜以
整卷的篇幅,对八字的各种格局,作出详尽的分析。
关于入格八字的看取办法,以代表自身的日干为主,然后
配合月令、年时,而以月令为重,其中逢官看财(财能生官),逢
•210 •
财看煞(财能生煞),逢煞看印(印能化煞),逢印看官(官印相
生九歌云:
一官二印 三财位,四煞 五食六伤官。
立法先详生与死,次分贵贱吉凶看。
在命书中,对于命的格局,有正格和变格的不同。凡是以官、
煞、印、财、食、伤等入局的,叫做正格。正格以外,叫做变格。
现把命书所载有关格局,择要举例如下。
1. 正格
正格命局的判定,通常以月柱干支为主,进行观察。如月
支所藏五行本气透出天干,可再结合司令以定真假,然后取为
格局。比如寅月天干透甲,卯月天干透乙,辰月天干透戊,巳
月天干透丙,午月天干透丁,未月天干透己,申月天干透庚,酉
月天干透辛,戌月天干透戊,亥月天干透壬,子月天干透癸,丑
月天干透己,而这些透出的天干又在命主出生那天司令用事
的,都可从这透出天干和日主天干之间的生克关系,取为格
局。但也有认为只要本气天干透出便可,不必考虑司令用事。
此外又有三种情况,一为如果月支本气没有透出天干,而支中
所含其他五行有所透出的,那末也可以此透出的天干,结合司
令,取为格局。如亥月天干不透壬水而透甲木,并且这甲木又
在命主出生那天司令用事的,那末也可根据整个八字的具体
情况,从甲木和日主天干之间的关系,认定格局。二为月支本
气没有透出天干,而支中所藏其他五行也都没有能够透出天
干,那就只好在月支所藏的各支中比较强弱,从而选一强有力
的,认作格局。如寅月生人,在月干、时干、年干中都不透甲
木、丙火、戊土的,因为寅为春月,甲木得令,所以多数可以甲
木作为代表。但是如果这甲木在四柱中不见生气的话,那末
•211 •
不妨另取丙火或戊土作为代表,以定格局。三为卯月、酉月、
子月三个月中,因为其月支只藏一种天干五行,所以不论月干
本气透与不透,如果通盘考虑下来可以为用的,便可直接定为
格局。
正官格 在六神中,正官是天地的正气,忠信的尊名,
虽然治国齐家,劳苦功高,可是八字中出现正官,只要一位就
足够了,并且出现的部位,以月柱为正,又怕刑冲。如果官星
太多,或官煞(偏官)相混,或部位偏离月柱,或官星逢冲,就难
以入格了。这就是命书所说的「正气官星,切忌刑冲,多则
论煞,一位名真。”如果身旺而时柱上兼有财星,更是贵不可
言。
〔入格八字〕
(年) 癸未
(月) 正官乙卯
(日) 戊寅
(时) 壬子
月柱乙卯,卯中 乙木透出天干,所以取乙木来判定格局。因为
乙木对于日主戊土来说,属 于克我的正官,而时柱壬子又为戌
土的财星,这样财官相生,故而取为正官格。美中不足的是,
局中财官太旺,自身偏弱,好在初运中运一路印绶比劫,助身
为旺,所以堪任财官。
诗曰:
正官须在月中求,无破无伤贵不休。
玉 勒金鞍真赋态,两行旌节上 星州。
偏官格 所谓偏官,就是七煞有制的称谓。如果八字
中同时出现偏印、偏财,身煞平衡,就是大富大贵的命。如果七
•212 •
煞被制过头,或者八字中官煞混杂,那就退职离官,多致凶死。
又如行运进入煞乡,也主不死而穷。此外,日柱天干无根而遇
煞制,或煞重藏根,主人都有被煞制死的可能性。所说煞重藏
根,就是七煞直接藏在自身日柱的地支中,比如乙酉日生的
人,酉是辛金,克乙木为煞,这时如果年柱、时柱中又不见制煞
或化煞的干支,就是很不吉利的命。
〔入格八字〕
(年) 丙寅木制
(月) 戊戌
(日) 壬戌
(时) 辛丑
月柱戊戌,戌中戊土透出月干,所以取戊土来定格局。因为戊
土对于日主壬水来说,属于克我的七杀,而年支寅中甲木又制
七杀戊土,'七杀有制为偏官”,所以属于偏官格。妙在时干透
辛,为生壬水的正印。这就使得日主制中有生,因而是个致中
和的贵命。如果柱中七杀无制,就又属于七杀格局了。诗曰:
偏 官有制化为权,唾手登云发少年。
岁 运若行身旺地,功名 大用福双全。
七煞格
在命局中,七煞是克我的神,需要有制为福,好比恶宿小
人,须要制伏,也可为我所用。在命书中,虽然有。七煞有制,
谓之偏官”的说法,可在举格局中,却也并不分得那么清楚。我
们且看下面一个七煞格局。
〔入格八字〕 李寺丞
(年) 己巳
(月) 丁卯
•213 •
(日) 丙午身旺
(时) 煞壬辰
时上壬水克自身丙火为煞,然而周围又不乏制水的土,可见
“七煞”、“偏官”等格,原也并不区分限制得那么明显,所以有
的命书,索性把“偏官”、“七煞”统称为偏官格或七煞格,倒也
来得干净。
按照“时上一位为贵”的原则,凡是上好的七煞格局,七煞
的位置一定要出现在时柱上,并且只要一位,不可多见。假如
时柱出现七煞,而年、月、日柱上又重复出现的,那就非但不
贵,反而成了辛苦劳碌的命。对于“时上一位为贵”的七煞格,
只要本身自旺而有制伏,行运进入七煞旺乡,必定发迹。反之,
如果命中七煞没有制伏,成了的的确确的七煞格,那末只要行
到对七煞有所制伏的运里,也可发迹;只怕命里七煞没有制
伏,而运又行到煞旺无制的境地,那就难免生出祸患了。诗
说:
时上七煞是偏官,有制 身强好命看。
制 伏喜逢煞旺运,三方 得地发何难?
元无制伏运须看,不怕刑冲多煞攒。
若是身衰官煞旺,定知此命是贫寒。
印绶格
在用神的名称中,印绶是生我的。符合这种格局的人,身
旺为福,四柱中最喜透出官星七煞,以及行官煞的运,因为官
煞能够生印。大忌柱中出现太多的财,这也因为财能伤克印
绶。至于四柱纯都是印,由于印绶太过,反而走向事物的反
面,所以注定主人是孤独的命。
• 214 •
〔入格八字〕 陈都宪
(年) 官癸未
(月) 正印乙卯正印
(日) 丙子胞胎逢印
(时) 官癸巳
八字月柱中乙卯两个乙木,都是自身日干丙火的印绶,而日支
子对于日干丙来说,在寄生十二宫中又正好处在天地气交,氤
意造物的胎的状态,这就更加需要印绶来促成了。妙在年柱、
时柱中透出的两颗官星癸水,也为这正印的格局增添了分数。
诗曰:
月逢印绶喜官星,运入官乡福必清。
死绝运临身不利,后行财运百无成。
正财格
在格局中,正财最喜身旺印绶,忌官星、忌倒印工 偏印),忌
身弱比肩、劫财。忌见官星的原因是怕盗财气,然而正财格中
带有官星,又行上财旺生官的大运,则反而可以更加发迹。反
过来说,如果柱中财多身弱,则怕行财旺生官的运,否则反而
祸患临头。又如财神宜藏,藏则丰厚,露则浮荡,行运如果碰
上比肩、劫财,非但分去财产,弄得不好,恐怕连命都要丢掉
了。此外,也有些情况,比如身强财旺的逢财看煞,见官更好,
所以命书又有“财藏露官者,当作贵推”的说法。
〔入格八字〕 李罗丞相
(年) 壬申
(月) 丙午午中己土为财
甲午
(时) 壬申
•215
这八字月支午中己土,为自身甲木的正财,而自身日支又坐财
地,所以在取格时把它看作是正财格。加之年柱、时柱的壬水
申金,不是生我甲木的印绶就是制我甲木的七煞,所谓“逢财
看煞”,对于印旺生财来说,可以说是致中和的最佳方案了。诗
曰:
财星忌透只宜藏,身旺 逢官大吉昌。
怕逢比劫来相会,一生名利被分张。
偏财格
如果偏财出现在时上的,与时上七煞格局一样,只要一
位,其他三柱不要重复出现。而这位时上的偏财,又怕逢冲,
如果一旦行运进入财旺之乡,那就发福百端了。
〔入格八字〕 李参政
(年) 庚寅 .
(月) 乙酉正官
(日) 甲子
(时) 戊辰戊土偏财
这一命造,月支正官不透,时柱戊土下坐辰支透气通根,所以
考虑取戊土偏财为格局。入偏财格的,除了喜行财运,最怕逢
冲,还大忌行到羊刃败财和劫财的运,因为这样偏财被分被
劫,就全完了。诗说:
时上偏财一位佳,不逢冲破享荣华。
败财劫刃还无遇,富贵双全比石家。
食神格
食神如在月令提纲中出现的,只要一位,并且要身旺,因
为食神能够生财,如逢身弱,那就难以克财了。对于入食神格
的人来说,四柱忌印绶、官煞,以及比肩、羊刃(劫财)为祸。如
•216
果大运一旦进入食神财旺运,便可发福。
〔入格八字〕 蜀王
(年) 己未
(月) 戊辰身旺
(0) 戊辰
(时) 庚申
蜀王八字的食神虽然出现在时干上,可是因为得力,所以便把
它取作食神格局。由于自身戊土,生在春天末一个月的三月
辰月,土令得时,所以身旺。诗曰:
食神身旺喜生财,日主 刚强福禄来。
身 弱食多反为害,或逢枭食主凶灾。
伤官格
“伤官见官,为祸百端”,因为在用神中伤官是正官的克
星,如果官来乘旺,那就祸不可言了。所以入伤官格的,伤官
一定要彻底伤尽才好。所谓伤尽,就是四柱中一点也不出现
官星。八字中如伤官多,有财星,或行身旺运,或行财旺运,都
是富贵发福的命。命理学家认为,“伤官乃小人之情,喜财而
妒官,又行财运,反生富贵”。此外伤官身旺无财的凶,这种人
如果一旦碰上官运,就会大祸临头,理应尽快退身避职。大凡
伤官只喜财旺身旺,如果行运进入财衰和死绝等地,那就脱财
无禄,不是官司打败,就是死期临头了。
〔入格八字〕 通参政
(年) 甲寅
(月) 庚午己土伤官
丙午
(时) 甲午
• 217 •
八字月支午中己土对丙火来说,是我生的伤官。由于格中一点
也没有丙火的官星癸水,所以伤官伤尽;加之伤官多,月干透
出庚金财星,自身丙午,午又是丙的帝旺之乡,所以是个发福
富贵的命。诗曰:
火土伤官伤宜尽,金水伤官要见官,
木火见官 官有旺,土金官去返成官,
惟有水木伤官格,财官 两见始为欢。
以上正官、偏官、七煞、财、印、食伤等正格,每一格局又可
各各化出一些另外格局,如正官格兼煞的叫官煞格、兼印的叫
官印格,正官格兼财的叫财官格,偏官或七煞格兼印的叫煞印
格,兼财的叫财煞格,印绶格兼官的叫官印格,兼煞的叫煞印
格,正、偏财格兼官的叫财官格,兼煞的叫财煞格,食神格用杀
的叫食神制煞格,用财的叫食神生财格,伤官格取印的叫伤官
用印格,取财的叫伤官生财格,取劫的叫伤官用劫格,取伤的
叫伤官用伤格,取官的叫伤官用官格,又有假伤官格等等。
官煞格
命局官煞当令混杂,如果得坐支印绶,引通官煞之气,使
之生化有情,或气贯生时,足以扶身敌煞的,如此则富贵可卜。
反之,如果坐下没有印绶引通官煞旺气,而气又不贯生时,那
末即使不贫也贱。但如果官煞不当令的,则不作此论。
〔入格八字〕
(年) 戊午
(月) 己未
(0) 壬申
(时) 辛亥
此造官煞当令而旺,好在日坐长生,又有印绶,引通财煞
•218
之气,且时逢禄旺,所以足堪敌官搅煞,加之运生西北金水之
乡,故卜少年科甲,文章炳焕,非泛泛之辈。
财煞格
格局中如果财煞得用,或须用财滋煞的,称财煞格,或财
滋弱煞格。入这一格的,大多自身较强,否则难当财煞之用。
〔入格八字〕
(年) 丙申
(月) 庚寅
(日) 庚申
(时) 辛巳
局中庚辛三透,地支两坐禄旺,所以自身强旺,堪任财煞。再看
年干虽透丙煞,挂角而又得禄,然而无奈庚辛元神透露,所以
最须用财滋煞,方为美善。结合行运,辰运木之馀气,采芹生
色;巳运火之禄旺,科甲联登;甲午、乙未,木火并旺,财煞得
势,仕至藩臬。
煞印格
命局中七煞太重的,常须印绶引通,化煞生身,称煞印格
或煞重用印格。
〔入格八字〕
(年) 戊子
(月) 煞甲寅印
(日) 戊午印
(时) 煞甲寅煞
戊土生于寅月寅时,煞旺身衰之象,好在坐下印绶午火,生身
化煞,年支子水之财,又生寅木而不冲午火,所以行运一旦进
入南方火土之乡,化煞庇身,早登黄甲,出仕驰名。
•219 •
食神制煞格
命局中七煞太旺,又无印绶化煞,此时若以食神为用,制
伏七煞,不使克制太过,称之为食神制煞格。
(:入格八字〕
(年) 煞戊辰煞
(月) 煞戊午
(日) 壬辰煞
(时) 食神甲辰
八字中四柱皆煞,好在自身壬水通根辰库,时干透出食
神,而辰又为木之馀气,所以群凶自伏。此后运至癸亥,亥为
食神甲木的长生,和日主壬水的禄地,所以科甲联登。甲子一
运,甲运食神得助,出仕县令,子运衰神冲旺,不禄。
制煞太过格
煞要制化,方才吉而可用,然而制煞太过,煞神受伤,如果
这煞神又为自己命局用神的话,那就并不太妙了。
〔入格八字〕
(年) 财辛卯印
劫
(月) 食戊戌食
财
印
(日) 丙辰食
官
(时) 己或印
时逢独煞,四食相制,年支虽有印绶卯木制食,但却辛金
盖头,况且秋木力薄,难以疏土,好在亥中甲木,制食卫煞,煞
•220 •
可得用。运至乙未,亥卯未会成木局印绶,制食卫煞有功,捷
报南宫,名高翰苑。甲午一运,木死于午,运中甲和时干己合
而化土,故丁外艰I 己巳年,巳又冲去亥水,所以不禄。
伤官用印格
伤官太重,日主之气被泄过度,故而用印补偏救弊,制伤
生身,名伤官用印格。
〔入格八字
(年) 己丑
(月) 辛未
(日) 丙寅
(时) 己丑
日元丙火,四柱干支伤官重叠,致使日主泄气太过,所幸丙火
坐寅长生,寅中甲木偏印,生身制伤为用,然而毕竟独印难敌
众伤,况且又逢月干辛金财星破印,故而早年行运,水覆山重。
此后直至运行丁卯,丁火劫去辛金,卯木破其丑土,所谓“有病
得药”,腾身而登月殿,庆集瑞林。接下来丙寅一运,体用皆
宜,仕至黄堂。
伤官用财格
劫印重重,命主自身偏强,故以泄身破印的伤官财星为
用,使八字命局趋于平稳。
〔入格八字〕
(年) 丙申
(月) 戊戌
(日) 丁卯
(时) 乙巳
此火土伤官,劫印重叠,日主身强可知,所以取年支申金
•221 •
财星作为为用神。其人早年得祖上遗业,当运入辛丑伤官生
财,财星得助之时,经营获利,发财十余万。此后壬寅一运,寅
为申金绝地,且为丙劫长生,又因寅冲申破,所谓“旺者冲衰衰
者拔”,所以不禄。
伤官用劫格
命中伤官生财,财星太重,破印累身,所以必要用劫制财,
以趋中和,为伤官用劫之格。
〔入格八字〕
(年) 癸亥
(月)-辛酉
(0) 戊申
(时) 己未
日元戊土,月干伤官通根,年干年支财星本旺,又逢伤生,
所以财星太重,早岁未免芸窗拂意。妙在时干己未劫财通根
为用,制抑财星,此后一旦运行丁巳、丙辰,印旺劫生,仕至州
牧,官资丰厚。可惜乙卯一运,官星冲克不静,罢职归田。
伤官用伤格
格局中印或自身强旺,必以命中运中伤官为用为助,才能
使格局取得平衡的,称伤官用伤格。
〔入格八字〕
(年) 乙酉 .
(月) 戊寅
癸酉
(时) 癸丑
癸水生于寅月,地支酉丑拱金,印旺生身,必以寅中甲木
伤官为用。乙亥一运,木逢生旺,高中乡榜,此后甲戌,出仕县
・222・
令。转至癸酉,葵运尚佳,酉运支逢三酉,木嫩金多,削职归
田。综观此运,病在火少无药,如若有火制金,虽然运入印金
之地,也无大患。
伤官用官格
书云「伤官见官,为祸百端。”但局中如果有财作为调停,
或伤官受到制约而足以用官的,则不仅无害,亦且有喜。
〔入格八字〕
(年) 庚午
(月) 己卯
(H) 壬申
(时) 己酉
壬水生于卯月,水木伤官之象。所喜同时官星通根年支,
午中丁火为财,足以化伤生官,而卯木伤官又为金印制服,加
之日元生旺,所以足以用官。巳运官星临于旺地,采芹泮水,
折桂月宫。壬午、癸未,仍行南方火运,出宰名区,莺迁州牧。
甲申、乙酉,金得地而木临绝,且官星泄气受制,所以退归田
里,以琴书自乐而终。
假伤官格
格中伤官得用,但不当月支司令用事之时,所以有假伤官
之格。
(:入格八字〕
(年) 戊午 - (月) 丙辰
(B) 戊辰
(时) 辛酉
戊土通根,年月干支火土重重,全赖时柱伤官通根 透 干,
•223
-
泄其秀气。三十岁前运走火土,蹭蹬芸窗。一交庚申,云程直
上。此后辛酉、壬戌、癸亥,体用合宜,伤官生财,从藩臬而转
封疆,宦海无波。
2. 变格
所谓“变格”,是一种特别的命格。通常情况下,四柱八字
如果符合这种特别命局条件的,当作变格论,不作正格。
杂气印绶格
在月份中,辰、未、戌、丑等月,也就是三、六、九、十二月,
月支辰中有乙木、癸水、戊土,未中有丁火、乙木、己土,戌中有
辛金、丁火、戊土,丑中有癸水、辛金、己土,这里面包涵了天地
驳杂不纯之气。比如以东方的甲乙木举例,甲则坐镇寅位阳
木,乙则坐镇卯位阴木,两者司管春令,而夺东方之气,可辰虽
属于暮春三月,然而这时已处于春夏交接之际,方位已经偏向
东南,所以受气不纯,禀命不一,有杂气之称。其他未、戌、丑
三月,也照此原理类比。
在杂气印绶格中,如果自身日干是甲的,要出生在十二月
丑月的才称得上贵,因为丑中辛金是甲木的正官,丑中癸水是
甲木的正印,丑中己土为甲木的正财。如果这时把握不住财、
官、印中取哪一样来定格,可以观察月干中透出的是什么用
神,然后再决定取舍。然而辰、戌、丑、未都是库藏,要有钥匙
打开,才能发福,才能为我所用,而这种打开库藏的钥匙,就是
刑冲破害。但这种刑冲破害,也要恰到好处,否则冲破过头,
反而伤了福份。大抵杂气需要财多,便可为贵。假如在年、时
等柱中有符合其他格局的,则当以其他格局来论。
〔入格八字〕 葛待诏
(年) 庚寅
• 224 •
(月) 丙戌丁火为印
戊子
(时) 癸丑
这种格局,忌行财运官运。八字的主人葛待诏,早先原是卖玳
瑁梳子的,只因杂气中月令透出丙火,月支藏有丁火为印,所
以行运一旦戌库冲破,就发迹了。可是毕竟由于日支子为癸
水,属于戊土的财,而时支丑里又含有一定量的癸水为财,这
样财能破印,水去克火,平时尚可维持过去,然而一旦行入子
运,运中癸水和命中癸水一起呼应起来,那就泛滥成灾,火光
灭没了。后来果然当行到子运时,这位葛待诏就寿终正寝了。
这用算命的术语来说,就是“贪财坏印二诗曰:
辰戌丑未为四季,印绶财官居杂气。
干头透出 格为真,只问财多为 尊贵。
杂气财官格
命书逢辰、戌、丑、未月出生的,有杂气之称。大抵杂气要
财多透露为贵,逢官也好。因为辰、戌 、丑、未属于墓库,需要
冲开,这样库中的财官印绶才能为我所用,否则官墓不显其
名,则库不用于世,印墓不得为信,不就形同虚设?
〔入格八字〕 王尚书
(年) 正财戊子
(月) 壬戌辛金为官 戊土为财
乙亥
(时) 丁丑
自身乙亥,生于戌月,戌中辛金为官,戊土为财,而其中戊土又
透出年干,所以就成了杂气财官的格局。诗曰:
杂气财官 四库中,还须 破害与 刑冲。
• 225 •
天干透出 财源格,财多 身旺禄相同。
羊刃比肩格
所谓比肩,就是同类中阳见阳,阴见阴的称谓,好比兄弟
姐妹的同类一样。同类中阳见阴则不称比肩而称败财,又称
羊刃,阴见阳则不称败财而称劫财。八字中如果“印财身强见
者,能夺伤官七煞,身弱见者,劫财分官见剥。”
〔入格八字〕 高太尉 .
(年) 庚午丁火
(月) 乙酉正官
(日) 甲寅
(时) 乙亥长生
本命天干甲木,生于八月,以酉中辛金为正官。然而年干出现
庚金为七煞,这种官煞相混,就不妙了。好在乙庚合金,甲木
把妹妹乙木嫁给庚金为妻,命书中有“贪合忘煞”的说法,况且
年支中又有丁火制服庚金,不致为灾。再看时支透出乙木,作
为甲木的羊刃,而时支亥中壬水,又使甲木处于长生状态,所
以行运一旦进入丑运,丑中辛金即抑乙木,又使自身甲木官运
亨通,所以官至二品。•诗曰:
春木夏火两相逢,秋金冬水一般同。
不宜羊刃天干透,运至 重逢又反凶。
七煞羊刃格
所谓七煞,就是偏官,喜制伏,喜羊刃。如命局中七煞、羊
刃同时出现的,往往可以把它看作这种格局,但忌财多,否则
便不成格局了。对于七煞羊刃格的人来说,最怕羊刃逢冲,譬
如丙日、戊日生人羊刃在午,因为午中丁火、己土分别属于日
干丙、戊的羊刃,这时如果行运进入正财子地,子午相冲,破了
• 226 •
羊刃,就不妙了。同样,壬日生人羊刃在子,忌行午地正财的
运,庚日生人羊刃在酉,忌行卯地正财的运,甲日生人羊刃在
卯,忌行酉地正财的运。如果格局中,羊刃不被冲破,那末碰
上财运,问题不大。
(:入格八字〕 不花平章
(年) 乙卯
(月) 戊子羊刃
(日) 壬戌七煞
(时) 壬寅
局中命主生于壬日,月柱日柱分坐子、戌两支,子中癸水为壬
水的羊刃,戌中戊土为壬水的七煞。壬水生于仲冬子月,得令
身旺,七煞被时支寅中甲木所制,有制为吉。这样身强煞浅,
七煞羊刃成格,所以是极贵的命。
井栏斜叉格
“井栏叉”解作井口,井中有水,所以济人。入这一格局
的,以庚申、庚子、庚辰三天为主,地支申子辰三合水局,天干
透出三庚。庚金以丁火作为正官,以申子辰冲寅午戌火局,使
庚日得官星为贵。如若柱中天干透出丙丁,则官煞显露,地支
逢上巳午,则井口填实,都减了分数。又时遇丙子,为时上偏
官,时遇甲申,为日禄归时,都不属此格。《三命通会》说「若
天干有壬癸字,则引申子辰为伤官,去寅午戌火力。戊己字克
伤水局,不能冲寅午戌火贵,乃减分数。岁运同。此格须柱无
一点火气,生秋冬为合局,见戊辰、戊子亦不妨。若庚子再见
子时,只作飞天禄马论;在辰月,以印绶论;在子月,以伤官论。
须变通消息,果合此格,主清奇贵显,但不甚富。运喜东方财,
北方伤,忌南方火土,西方平平J
• 227 •
〔入格八字〕 王都统
(年) 庚子
(月) 庚辰
(日) 庚申
(时) 丁丑
天干三庚,地支申子辰合水,全逢润下,虽说生在辰月,然而变
通消息,当可视作井栏斜叉之格。诗曰:
生 遇三庚喜气新,全逢润下井栏真。
金精怕见寅午戌,水秀 偏宜申子辰。
伤贵缘多壬癸见,露官 休共丙丁临。
运行大抵东方美,一世荣华不受贫。
财官双美格
此格以壬午、癸巳两日为主,因坐支地支所含,为日干的
正官正财。其他如甲戌、乙丑、乙巳、丙申、丁丑、戊辰、己亥、
庚寅、辛未、壬戌、癸未等日,支内虽然藏有财官,禄即官,财即
马,堪称禄马同乡,然而因为其禄马财官或偏或正,未能纯一,
所以不入此格。
〔入格八字〕
(年) 己丑
(月) 丁卯
(日) 壬午
(时) 癸卯
壬午日柱,本已财官双美,加之局中年月透出丁、己,而日支午
又为财官丁、己的临官禄地,所以大富大贵。诗曰:
禄马同乡无克夺,财官同处最为荣。
三台八座真奇贵,克夺如强欠利名。
• 228 •
天元暗禄格
此格只取庚寅、乙巳、丙申、己亥四天。庚寅日,庚以丁火
为正官,此时如虽年、月、时干中不见丁火,也有坐支寅火克庚
为官。甲禄在寅,木为火母,母子有相继之义。结合年、月 、时
干,喜有戊己滋助,此时若见乙、丁,更为佳妙,如见丙煞,则宜
以壬、癸、亥、子制服为用。乙巳日,乙坐长生之金为官,戊禄
为财,若年、月、时干中有庚、戊引透巳中财官,更属理想。此外
更要壬癸之印助身,忌辛金七煞为制,因巳中原有丙火,须壬、
癸、亥、子去其火气方美。丙申日,喜庚、辛财,癸水官,甲、乙
印,忌戊、己伤官。己亥日,甲木坐亥中长生为官,忌金伤官P
〔入格八字〕 闻渊尚书
(年) 庚子
(月) 甲申
(日) 庚寅
(时) 丙戌
日元庚寅,寅中长生丙火克庚为官,禄旺甲木被克为财。虽说
时干又透丙煞,好在年支子水制伏,所以成为贵格。
禄元互换格
此格只有四天四时,即戊申日见乙卯时,丁酉日见壬寅
时,丙子日见癸巳时,庚子日见丁亥时。如戊申日见乙卯时,
戊以卯中乙木为官,乙以申中庚金为官,因互换而成贵禄。柱
中若见壬、癸为财,生助乙木官星,再运临官旺之乡,便是贵
命。忌见七煞甲木,辛金伤官,寅支冲申,酉支冲卯。其他丁酉
日见壬寅时,丙子日见癸巳时,庚子日见丁亥时,推法喜忌与
戊申日见丁卯时同。此外古法论禄元互换,如戊午见丁巳之
例,是取临官之禄,又有异于此了。
• 229 •
〔入格八字〕
(年) 癸亥
(月) 壬戌
丙子
(时) 癸巳
此格日时互换禄旺,各临官贵,又无刑冲破害,故贵。
六壬移换格
《三命通会》说「此格柱中有禄、有刃、有官、有印。不就本
身者,遇冲克则变化。有天干地支冲克,或年、月、日冲克,或
日时干克支冲者,当彼此互换为用,以天干常动,施支静,故地
支因冲克动以天干也。”如柱中甲子日见庚午时,因干克支冲,
所以当彼此互换为用,以庚子甲午论定吉凶。此外如壬子日
见丙午时,庚午日见丙子时,癸亥日见丁巳时,也可按此原理
移换,惟有丁酉日遇癸卯时不能移换,因为丁生于酉,癸生于
卯,所以各就天乙贵人长生之位。
〔入格八字)
(年) 己巳
(月) 癸酉
(日) 丁卯
(时) 癸卯
局中二卯一酉,癸丁相克,因地支卯酉冲而撼动天干。表面看
去,一丁夹在二癸之中,似难展步,不知丁为年干太岁己土之
母,此时己见癸水克丁,子来救母,把母接来身边,而癸水为了
逃避己土制克,也乐得和丁火移换位置,移换的结果,月柱变
成丁酉,日 柱变成癸卯,如此则丁火癸水各逢贵地,所以大贵。
子午双包
• 230 •
子为帝座,午为端门,两者都为帝王所居之位,为此命局
如两子两午,或两午包一子,或两子包一午,因为得水火相济
之道,获阳生阴长之机,所以遇者主贵。
〔入格八字〕
(年) 壬午
(月) 壬子
(日) 戊午
(时) 壬子
局中两午两子,所以入于此格。此外如壬子、癸丑、戊午、壬
子,甲子、庚午、丙申、戊子,戊子、戊午、丁未、庚子,戊午、甲
子、甲申、庚午,甲午、壬申、甲子、庚午,也都属于此格的贵命。
阴籍阳生格
古书以寅、申、巳、亥为“四长生”,如果乙、丁、己、辛、癸五
阴日,逢甲 、丙、戊 、庚、壬五阳日的长生,不可便以“阳生阴死”
论定。为此,前人有乙见午为炭柴之木,无亥则不能生;丁见
酉为石精之火,无寅则不能复明;己见酉为粪壤之土,无寅不
能生物;辛见子为流沙之金,无巳则不能生, 癸见卯为脂膏之
水,无申则凝结。
〔入格八字〕
(年) 甲申
(月) 庚午
(日) 乙亥
(时) 丙子
《三命通会》说:“若命入格,更年通月气者大贵。大忌官煞混
杂,贫苦。”诗曰:
五阴日诞喜阳生,若是年支福最亨。
• 231 •
月气得通须大贵,惟嫌官煞主孤贫。
生处聚生格
此格日主遇印绶,又引日长生之地,如此则身强喜见官
星,所以有五马诸侯之贵。
〔入格八字〕
(年) 乙卯
(月) 丁亥
(日) 丙寅
(时) 庚寅
局中木火相生,引身生旺之地,所以为贵。诗曰:
生处聚生福最佳,印绶引旺福无涯。
长生复到长生地,五马诸侯富贵家。
木火交辉
此象如甲戌、甲午、甲寅、丙午、丙寅、丙戌等日,要出生在
春月或夏月,柱中不见金水伤坏,时上透出木火方成。在行运
上,木日火秀,行南方运,火日木秀,行东方运。
〔入格八字〕
(年) 丁巳
(月) 甲辰
(H) 甲寅
(时) 丁卯
局中木火通明,所以为清贵福厚之造。
火土夹杂
火见土则暗,土宿火则晦,所以要火自火,土自土,两不相
掩为妙。局中如若火土夹杂,多半愚浊。
〔入格八字〕
• 232 •
(年) 戊戌
(月) 丁巳
~(日) 己未
(时) 丙寅
局中因火土夹杂,所以平常。经云:“火虚土聚成何用?定是
尘埃碌碌人」
水土败酉格
水土败酉,不利晚景。《三命通会》说:“若更水命土命,
而日主见水土者,尤验。”
(:入格八字〕
(年) 癸亥 辛酉
(月) 乙丑 甲午
(日) 癸酉 戊子
(时) 辛酉 辛酉
入此格的,或为小官而早退闲,或只平常而早弃世,晚景多不
佳妙。
夹库格
此格四柱地支虚夹辰、戌、丑、未等库位,大忌填实,及刑
冲破害空亡,岁运以官印之乡为佳。
〔入格八字〕
(年) 乙亥
(月) 己卯
(0) 己巳
(时) 甲子
卯月巳日,中间虚拱辰中水库为财,四柱又无辰支填实,且不
犯空亡破害刑冲,所以贵为丞相之命。
•233-
地支夹拱格
此格又称“地支连茹”,如四柱地支见子、寅、辰、午之例。
按照十二地支排列顺序,当为子、丑、寅、卯、辰、巳、午、未、申、
酉、戌、亥,现在四柱地支既为子、寅、辰、午,中间虚夹虚拱丑、
卯、巳等支,所以名为“地支夹拱”。
〔入格八字〕 帖干远太师
(年) 甲寅
(月) 戊辰
丙午
(时) 丙申
四柱地支寅、辰夹卯,辰、午夹巳,午、申夹未,故而入于地支美
拱之格。
墓煞格
此格因七煞入于墓库,所以名为“墓煞”。如甲日见庚戌、
庚辰,乙日见辛丑、辛未,丙日见壬辰、壬戌,丁日见癸丑、癸
未,戊日见甲辰、甲戌,己日见乙丑、乙未,庚日见丙辰、丙戌,
辛日见丁丑、丁未,壬日见戊辰、戊戌,癸日见己丑、己未等便
是。《珞球子》说「夹煞持丘,亲姻哭送。”
〔入格八字〕
(年) 己巳
(月) 戊辰
癸丑
(时) 丙辰
《三命通会》说,“癸日见戊为官,己为煞,戊、己并在辰
上,又为癸水库,多主早发早夭。”
金神格
•234 •
金神只有三时,就是癸酉、己巳、乙丑,凡是四柱中时柱上
出现这三个时辰的,就被认为是金神格。但也有认为,要逢六
甲日出生的,才可入这格局,其中甲子、甲辰更好。金神原是
破败之神,凡入这格局的,四柱中要火制伏为贵,或行运进入
火乡也好。如果运入水乡,水泄金气,大祸就临头了。
【入格八字〕 岳武穆
(年) 癸未
(月) 乙卯
甲子
(时) 己巳金神
《星平会海》说:“甲、己为平头煞,生逢春月,身旺财弱,主骨
肉参商,平生做事,弄巧成拙。己巳金神有火制伏,己酉丑合
局,运行南方,名重禄高。柱不见火,残害化气,主凶恶暴亡
“甲子日,己巳时,先贫后富,祖业轻微,妻勤子拗。”诗说,
癸酉己巳并乙丑,时上逢之是福神。
傲物恃才 宜制伏,交逢刃煞贵人真。
性多狠暴才明敏,运入水乡立苦困。
制伏运行逢火局,超迁贵显富无伦。
魁罡格
魁罡有四,就是庚辰、壬辰、戊戌、庚戌,其中辰是水库,属
天罡,戌是火库,属地罡,辰戌相见,所以成了一种天冲地击之
煞。凡是命造中日柱逢庚辰、壬辰、戊戌、庚戌的,就属于这一
魁罡格局。《三命通会》说:“经云:魁罡聚众(四柱中.出现魁
罡的不只日柱一处),发福非常。主为人性格聪明,文章振发,
临事果断,秉权好杀。"如果“运行身旺,发福百端,一见财官,
•235
祸患立至”。《子平总论》说 身值天罡地魁,哀则彻骨贫寒,
强则绝伦贵显「然而对于这种格局,也需活看,比如《三命通
会》举例张时佥事的八字是庚午、丁亥、戊戌、丙辰,刘大受少
卿的八字是丁亥、癸丑、庚戌、戊寅,虽说出生的一天都是魁罡
日,可是却不忌财官印,就是明证。诗曰:
壬辰庚戌与庚辰,戍戌魁罡四座神。
不见财官刑煞并,身行旺地贵天伦。
对于魁罡格的说法,张神峰批判说:“魁罡格取壬辰、庚
戊、庚辰、戊戌临四墓之上,为魁罡,能掌大权。何以临四墓之
上,遂能如此,亦谬说也J
日德格
入这一格局的只有阳干五天,就是甲寅、丙辰、戊辰、庚
辰、壬戌。其中甲坐寅得禄,丙坐辰官库,庚坐辰财印两全,壬
坐戌财官印具备,并且地支寅为三阳之首,辰、成为魁罡之地,
所以这五天的干支,就很有点和其他日子的干支不同了。八字
中出现日德的,不厌其多,如果只有日柱一位日德的,那未取
格时就要按照月柱中财官印食,另作别论了。平时日德除庚
辰自兼魁罡二职外,不管命中还是大运,最忌和魁罡同时出
现,否则便认为是很不好的命运。
(:入格八字〕 张学官
(年) 甲申
(月) 戊辰
(H) 戊辰
(时) 壬戌
命造中有三位日德,由学官而“冠金衣紫”,得五品官诰,很是
不错。又如庚辰、己卯、戊辰、甲寅这样一命,按理有三位日
• 236 •
德,该是好命。然而甲寅忌见身兼魁罡的庚辰,后来运行壬午
财乡之地,午中己土为日干戊土的羊刃,犯了日德的忌讳,到
丁巳年时,寅巳相刑,四月死,寿只三十八岁。这是《三命通
会》所记载的。诗说,
日 德喜煞喜身强,不喜财星官旺乡,
为 性温柔更慈善,一生 福寿乐非常。
日 德不喜见魁罡,化成 煞曜最难当。
局中重见还须疾,运限逢之必定亡。
命理学家张神峰认为,日德格也并不可信,所以他才驳
道「(日德)何以见其为德?不考原委来历,辄以日德名之,岂
非谬说乎?”
日贵格
命造中出生在丁酉、丁亥、癸巳、癸卯四天的人,因为日干
坐在天乙贵人星上,所以便把格局称为“日贵”。其中又有
贵、夜贵之分:丁亥、癸卯日生的,生时要在白天,叫做 “日
贵”,又叫“昼格”;丁酉、癸巳日生的,生时要在黑夜,叫做“夜
贵”,又叫“夜格 《三命通会》说「经云:贵人者,慈祥恺悌
之号,德星尊重之命。遇财官印食则吉,值煞刃冲刑则凶。运
遇魁罡,为害不浅」'对于日贵格的命,八字中如果聚上两三位
贵人的,主人存粹仁义,贵不可言,怕就怕地支贵人逢冲受损,
又怕日上空亡和魁罡加临,这样非但不贵,反而贫贱而夭了。
诗说:
丁遇猪鸡癸兔蛇,刑冲 破害漫 咨嗟。
财临会合方成贵,昼夜分之最为佳。
建禄格
•237 •
八字中自身日柱天干五行,与月建配合起来正好处于临
官禄地,比如甲乙春生,丙丁夏旺,庚辛秋锐,壬癸冬长,以及
戊己生于巳午月等就是。按照五行寄生上二宫的说法,甲禄
(临官)在寅,乙禄在卯,丙禄在巳,丁禄在午,戊禄在巳,丁禄
在午,庚禄在申,辛禄在酉,壬禄在亥,癸禄在子。因此如果日
干甲木,月支在寅,年支如果没有其他大的破坏,就可认为入
了建禄的格。
〔入格八字〕 贺丞相
(年) 辛丑
(月) 庚寅甲禄在寅
(日) 甲辰
(时) 乙亥
八字中日干甲,月建在寅,寅为甲禄。月干和,年干庚、辛金虽
然分别为甲木的七煞和正官,官煞混见,可是当大运进入丙
戌,丁亥制煞之乡时,那就大富大贵了。诗曰:
建禄生提月,财官喜透天。
不宜身再旺,官地是良缘。
归禄格
这种格和建禄格不同的是,建禄格看日干和月支的配合
有否禄地,而归禄格则把日干的禄归到时支中去找,如果时支
中有禄,而四柱又没有官星七煞的,可认为入了归禄的格。经
云「日禄归时没官星,号曰青云得路。”
〔入格八字〕 林枢密
(年) 财戊子
(月) 甲寅
(0) 乙亥
• 238 •
(时) 己卯归禄
此格八字中四柱没有一点官星,所以富贵入格。但如果在行运
中遇到官星,就凶而不吉了。诗曰:
日禄归时格最良,怕官嫌煞喜自强。
若见比肩分劫禄,刑冲 破害更难当。
拱禄格
《三命通会》说:“拱禄有五日五时,癸亥、癸丑,癸丑、癸
亥,拱子禄;丁巳、丁未,己未、己巳,拱午禄,戊辰、戊午,拱巳
禄」
〔入格八字〕
(年) 癸卯
(月) 癸亥
(日) 戊辰
(时) 戊午拱巳
日主戊土,禄在十二支中的巳。而巳在十二支中的次序位置,
处在辰和午之间,现在日支辰和时支午两者之间,中间正好空
出巳的位置,所以叫做“拱禄”。 所谓“拱”,就是“挟”的意思,
辰、午夹巳,而巳又不见,这就叫做虚拱,不可填实。如果四柱
年、月地支中出现巳的,就是填实,那就不属这格局了。《星平
会海》说J忌填实,最怕冲了日时拱位。又怕四柱中有伤,日
干七煞,皆拱不住,则减分数。岁君大运同J 所说'最怕冲了
日时换位”,比如这样一个命造:
(年) 壬子
(月) 丁未 ,
(日) 丁巳拱午
(时) 丁未
• 239 •
日柱丁巳,时柱丁未,己、未中间拱的是午,而年支、月支又不
见午,因此没有填实。坏在年支子,与己、未之间的拱禄午正
好相冲,破了格局,所以一生清贫,不入官场。诗说:
拱夹本身非是我,中间 一位虚中好,
不宜填实见官星,更忌忽来逢冲冒。
壬骑龙背格
这一格局以壬辰日出生为主,四柱又多见壬辰、壬寅,其
中辰字多的贵,寅字多的富,如果纯见寅、辰两支,而没有别的
地支掺杂进来,那就富贵双全了。经云「阳水叠逢辰位,是壬
骑龙背之乡。”这种格大忌官星盛旺,若见戌和辰冲,也不为
福。
I:入格八字〕 王枢密
(年) 壬辰
(月) 甲辰
(H) 壬辰
(时) 壬寅为壬财官
王巨富
(年) 壬寅
(月) 壬寅
(H). 壬辰
(时) 壬寅遍地是财,以致巨富
以上两个八字,前一个辰多,所以贵过于富,后一个寅多,所以
富过于贵,都是很典型的。诗曰:
壬崎龙背怕官居,重叠逢辰贵有余。
假若寅多辰字少,须应 豪富比陶朱。
六乙鼠贵格
• 240 •
此格以六乙日生,时间逢子,而四柱中又没有官星,方才
官高名显。在神煞中,乙见子为贵人,而十二生肖又以子属
鼠,所以有“六乙鼠贵”之名。由于此格子为乙的贵人,所 以
大忌逢冲。若果八字或大运中子午相冲,那就通盘都坏了。
〔入格八字〕 曹尚书
(年) 丁巳
(月) 壬寅 .
•
(H) 乙卯
(时) 丙子贵人
《喜忌篇》说:“阴木独遇子时,为六乙鼠贵之地「诗曰:
乙日生人得子时,名为 鼠贵最为奇。
切嫌午字来冲破,辛酉庚申总不宜。
六甲趋乾格
此格为六甲日生,时间逢亥。亥宫属乾,为甲木的长生之
地,所以有“六甲趋乾”的名称。入此格的,四柱和岁运中不喜
财星,同时又忌寅、巳两支。因为甲财为土,能制亥水,寅与亥
合,巳与亥冲,都不隹妙。
(:入格八字〕 新安伯
(年) 戊辰
(月) 癸亥
(日) 甲子 、
(时) 乙亥亥为乾
大凡逢甲日出生的,亥支不厌其多,又无巳来相冲,这样就自
然富贵了。诗曰:
趋乾六甲最为奇,甲 日生人得亥时6
岁运若逢财旺地,官灾 患难祸相随。
•241 •
“六甲趋乾格”也是一种谬论,所以张神峰这样批判道,
“六甲趋乾格,谓亥乃天之门户,甲日生人临此,谓之 '趋乾
假如别日干生临亥上,何以不谓之 '趋乾'乎?岂天门只好此
六甲来趋乎?夫天体至圆,本无门户,即以乾居西北,类天之
门户,岂可论人之祸福乎?”
六壬趋艮格
入这个格的,要以六壬日生寅时为准,格中寅字多的,又
叫做合禄格。在八字和岁运中,最怕逢申相冲,又忌财官。
〔入格八字〕
(年) 壬寅
(月) 壬寅
(日) 壬寅
(时) 壬寅
寅为艮宫,所以有“趋艮”的叫法。寅宫甲木能合己土,丙火能
合辛金,这样就暗邀己土为壬水的正宫,辛金为壬水的印绶
了。行运不厌身旺之地,如遇申运相冲,那就大打折扣了。诗
曰:
六壬趋艮喜非常,壬日 寅时是贵乡。
大怕刑冲并克制,逢申岁运有灾殃。
在张神峰眼里,“六壬趋艮格”也属谬说「六壬趋艮格,谓
用寅中甲木,能合己土为壬之官,寅中丙火,能合辛金为壬之
印,俱是无中生有。大抵与拱禄、飞禄、禄马之说相为表里,而
此说尤非也。”
勾陈得位格
勾陈在五行中属戊己土。在六戊日和六己日中,遇上财
官的有戊辰、戊子、戊申、己卯、己亥、己未等六日,其中以申子
" 242 •
辰水局为财,亥卯未木局为官。入这种格局的,最怕刑冲煞
旺,反生灾害。
〔入格八字〕 丁都督
(年) 丁亥
(月) 丁未
(日) 己卯
(时) 戊辰
八字中以己卯日为勾陈,遇亥卯未木局为官星得地,所以是个
贵命。诗曰:
日 干戊己 坐财官,号曰 勾陈得位看。
知有大才分瑞气,命中 值此列朝班。
玄武当权格
玄武在五行中属壬癸水。在六壬日和六癸日中,遇上财
官的有壬寅、壬午、壬戌、癸未、癸丑、癸巳六日,其中以寅午戌
火局为财,辰戌丑未土局为官。入这种格局的,在八字和岁运
中最忌身弱冲破。
(:入格八字〕 季都司
(年) 庚戌
(月) 壬午
(日) 壬寅
(时) 辛亥
格中地支寅午戌火局为财,水火既济,所以富贵荣华a 诗曰:
壬癸名为玄武神,财官 两见始成真。
局 无冲破当清贵,辅佐 皇家一老臣。
© 稼福格
稼稿在五行中属中央戊己土。凡是入此格的,不仅日干
•243
要逢戊己土,并且还要地支辰戌丑未全是土局,加上无木克
削,有水为用,自然便就福禄绵绵了。
〔入格八字〕 张真人
(年) 戊戌
(月) 己未
(日) 戊辰水
(时) 癸丑水
此命地支辰戌丑未俱全,得水为财,又无木克,所以有福。诗
曰:
戊 己重逢杂气天,土多只论木居全。
财星得遇堪为福,官一煞 如临有祸缠。
@ 曲直格
曲直在五行中属于东方甲乙木。凡是入此格的,不仅日
干要逢甲乙木,并且地支还要会成亥卯未木局,或寅卯辰会,
加上无金克削,有水为印,命主仁而福寿。
(:入格八字〕 李总兵
(年) 甲寅且''
(月) 丁卯
(日) 乙木
(时) 丙子
此格日干乙木与地支寅卯未会成木局,加上时支癸水生木,又
无官煞相侵,所以盛而为官。诗曰,
甲乙生人寅卯辰,又名 仁寿两堪评。
亥卯未全嫌白帝,若逢坎地必荣身。
诗中“嫌白帝”是说嫌弃庚辛金气,因为金属西方白帝。坎地是
指水地,八卦中坎属于水,所以古人常用坎指代水。
•244 •
@ 炎上格
炎上在五行中属于南方丙丁火。凡是入这格局的,不但
日干要遇丙丁火,并且还要地支会成寅午戌火局,或巳午未
会,加上身旺,运行东南,便就浑成文明,大富大贵了。
〔入格八字〕 张太保
(年) 乙未火
(月) 辛巳火
丙午火
(时) 甲午火
局中自身丙火逢地支巳午未火局,一片炎上之性,故为朝中朱
紫之贵。诗曰:
火 多炎上气冲天,玄武 无侵富贵全。
一路东方行运好,簪缨头顶带腰悬。
@ 润下格
润下在五行中属于北方壬癸水。凡是入这格局的,不但
日干要遇壬癸水,并且还要地支会成申子辰水局,或亥子丑
全。平生忌辰戌丑未官乡,喜西方印地,不宜东南,怕冲克。
〔入格八字J 万宗人
(年) 庚子水
(月) 庚辰水
(日) 壬申水
(时) 辛亥水
这命非但地支申子辰全,子亥水局浑然,并且年、月、时柱
又得庚辛生水,所以一片湛然,福量广阔,为富贵之人。诗
曰:
天干壬癸喜冬生,更值申辰会成局,
• 245 •
或是全归 亥子丑,等闲平步上青云。
领 从革格
从革在五行中属于西方庚辛金。凡是入这格局的,不但
日干要遇庚辛金,并且还要地支会成巳酉丑金局,或申酉戌
全。平生忌南方火运,冲刑克破,喜庚辛旺运。
〔入格八字〕 杨太尉
(年) 辛酉金
(月) 戊戌金
庚申金
(时) 辛巳金
命中自身庚金,地支申酉戌全,月干又透出戊土生金,所以福
高禄深。诗曰:
秋月金居一类看,•名为 从革便相欢,
如无炎帝来临害,定作当朝宰辅官。
@ 弃命从财格
命中日干身弱,四柱中又全无印绶相生,比肩扶助,而天
干地支却透财会财,造成财旺身弱的局势,这时就索性丢弃
自身专以财论,所以叫弃命从财格。此格喜行财旺运,怕入煞
印之乡。比如天干乙木而地支辰戌丑未土全,财神极旺,这时
如果四柱无依,就只有以弃命从财格论。同样道理,日主甲、
乙无根,遇上地支遍地金气,因为金是克制甲乙木的官杀,所
以这时取格,就要从弃命从官格或弃命从煞格来考虑。此外如
弃命从食,弃命从伤等格,也都是根据这一原理派生出来的。
古人认为:“弃命从财,须要会财,从财忌煞。”《星平会海》说:
“此格乃身弱,四柱中全无印绶、比肩扶助,天干透财,地支会
财,得高寿,行财旺之运,忌煞印之乡。”碰上这种格的,主人
• 246 •
平生不是怕老婆,就是做人家的招女婿。因为财者妻也,自身
既无依靠,托的全是妻子的福,所以作这样的分析。但也有大
发特发,非常辉煌的。
〔入格八字〕 王十万
(年) 庚申财
(月) 乙酉财
丙申财
(时) 己丑财
命中财多身弱,自身少有所助,所以只有舍命从之,才能有福。
诗曰:
日 主无根财犯重,全凭时印旺身官。
逢生必主兴家业,破印纷纷总是空。
命 从儿格
刘基说: “此与成象从象伤官不同,只取我生者为儿。如
木遇火,成气象,如戊己日遇申酉戌,成西方气,或巳酉丑全会
金局,不论日主强弱,而又看金能生水气,转成生育之意,此为
流通,必然富贵。”此格食伤为子,行运如遇财乡,为儿又生儿,
因秀气得以流行,所以卜知安享荣华。平生最忌印运、官运,
如果不幸碰上,不伤人丁,就损财产。
〔入格八字〕
(年) 壬寅
(月) 辛亥
辛亥
(时) 壬辰
任铁樵说:“辛金生于孟冬,壬水当权,财逢生旺,金水两
涵,格取从儿。读书一目数行,至甲寅运,登科发甲。乙卯运,
・247・
由署郎出守黄堂。一交丙辰,官、印齐来,又逢丙戌年冲动印
缓,破其伤官,不禄。”
贪合忘煞格
四柱中财官两旺,这时如果柱中透出的煞被合去,就叫贪
合忘煞。入这种格的,虽然衣禄丰厚,可却无官而好酒色。比
如甲日柱中逢庚,庚就是甲的七煞,这时柱中如又透出乙木和
庚相合,便就称作贪合忘煞。又如甲日柱中逢辛,辛是甲的正
官,这时柱中如又透出丙火和辛相合,便又称作贪合忘官。其
中煞要合,官不要合,所以有“合煞不为凶,合官真不美”的说
法。又说「煞无刃不威,刃无煞不显。”这里的刃,就是羊刃,
也就是败财,可见刃煞一起出现,也并不是坏事。这又有点岔
到别处去了。
t入格八字〕 王指挥
(年) 丙申
(月) 合官辛丑
甲辰
(时) 财思辰
这命生于甲日,月建中有官星照临,而戊辰时又财旺生官,本
该大贵,可是偏偏年干丙火与官星辛金相合,命书说「贪合忘
官为颠邪「现在官星既被合掉,只好作戏命看了。加之四十
五岁入丙午运,火势太炎,甲木干燥,木被火焚,则不禄矣6 诗
曰,
贪合忘官合不足,合煞不伤为 己福。
堪叹身弱怕逢败,更历 官乡祸自逐。
@ 天地德合格
所谓“天地德合”,就是局中天干与天干概合,地支与地支
•248 •
相合,因为干为天之清气,支为地之厚载,干合者得贤人之心,
支合者得众人之心,所以有“天地德合”之称。入此格的,时合
为上,日合次之,如果年与月合,日与时合,更加妙不可言。
〔入格八字〕 张舜臣尚书
(年) 乙卯
(月) 丁亥
(日) 戊寅 •
(时) 癸亥
日柱干支与时柱干支两两相合,故而入于此格。此外如林见素
尚书八字:辛未、丁酉、己卯、甲戌,日柱干支与时柱干支亦皆
两两相合,也属此格。
® 化气格
化气格有化土、化金、化水、化木、化火等格。凡是甲日生
人,月干或时干逢己,乙日生人,月干或时干逢甲,因为甲己化
土,这时如果月支偏又属土,可取为化土格;乙日生人,月干或
时干逢庚,庚日生人,月干或心干逢乙,因为乙庚化金,这是如
果月支偏又属金,可取为化金碗;丙日生人;月干或时干逢辛,
辛日生人,月干或时干逢丙,因为丙辛化水',这时如果月支偏
又属水,可取为化水格;丁日生人,月干或时干逢壬,壬日生
人,月干或时干逢丁,因为壬丁化木,这时如果月支偏又属木,
可取为化木格;戊日生人,月干或时干逢癸,癸日生人,月干或
时干逢戊,因为戊癸化火,这时如果月支偏又属火,可取为化
火格。
〔入格八字〕
(年) 甲戌
(月) 丁卯
•249 •
(日) 壬寅
(时) 甲辰
日主壬水,生于卯月木月,地支寅卯辰又会成木局,加之日主
自身天干壬与月村天干丁壬丁合木,年干时干甲木助化,因为
整个格局汇成一片东方秀气,所以自身壬水也就只好跟着化
木,而成化木格了。《饰理推算法》说:“日干与月干或时干化
合,生月地支与化气同一五行,而且无印比、争合、妒合,有者
为假化气之格。”比如:
(年) 癸巳
(月) 丁巳
(日) 癸酉
(时) 戊戌
日主癸水,生于巳月火月,自身癸水与时干戊土合化成
火,又得月干丁火引通,因此似乎化格成象。然而坏在年干癸
水和日主癸水争合时干戊土,因为有了争合,所以这化火格,
便就成了梦幻泡影的假化之格了。
@ 两神成象格
所谓“两神成象”,就是四柱干支中,或水木、或木火、或火
土、或土金、或金水等各占两干两支,又有“相生”和“相成”的
不同。
t入格八字]
(年) 癸亥
(月) 甲寅
(日) 壬子
(时) 乙卯
本造五行,由于在年、月、日、时中水、木各占两干两支,纯清不
•250
杂,五行中水又能够生木,所以就称为水木相生格。同样道理,
如果四柱中木火各占两柱,称木火相生格I火土各占两柱,称
火土相生格;土金各占两柱,称土金相生格;金水各占两柱,称
金水相生格O
反之,如命局四柱中各占两柱的两个五行彼此之间相克,
那就不称“相生”,而称“相成”了。且举这样一造为例:
〔入格八字〕
(年) 己未
.(月) 甲寅
(日) 乙卯
(时) 戊辰
由于命主四柱,土木各占两柱,相克而又相成,所以称为木土
相成格。同样,四柱中如果土水各占两柱,称土水相成格1 水
火各占两柱,称水火相成格,火金各占两柱,称火金相成格;金
木各占两柱,称金木相成格。
@ 两干不杂格
两干不杂,就是四柱只两个天干,并且顺序有次不杂,如
甲年戊月甲日戊时之例。
〔入格八字〕 叶丞相
(年) 庚寅
(月) 戊寅
(日) 庚寅
(时) 戊寅
此为典型的两干不杂格。古书有“两干不杂利名齐”之句,可
见这是一种贵格。然而,如甲子、乙亥、甲戌、乙丑一命,因为
甲人得乙,乙人得甲,谓之偏禄,反而没有科名。
。251 •
天干顺食格
此格年、月 、日、时干支,顺次相生,皆为食神,如甲年丙月
戊日庚时之类便是。
〔入格八字〕 脱脱丞相
(年) 壬辰
(月) 甲辰
丙戌
(时) 戊戌
年干壬水食神为月干甲木,月干甲木食神日干丙火,日 干丙火
食神为时干戊土,所以为天干顺食的格局。加之地支两辰两
戌,也都为日主丙火的食神,先辰后戌不倒,故而格局显贵。
© 干辰一字
入此格的,年、月、日、时四柱夭干,清一色而不杂,所以清
贵。
(:入格八字]
(年) 壬子 戊辰 甲子 庚申 丙寅
(月) 壬子- B
戊午 甲戌 庚辰 丙申
(H) 壬子 戊申 甲寅 庚戌 丙午
(时) 壬寅 戊午 甲子 庚辰 丙申
支辰一字
这种格局,年、月、日、时地支一字不杂,一色纯清,也是一
种贵命。
〔入格八字〕
甲寅 戊辰 乙亥
丙寅 丙辰 丁亥
庚寅 甲辰 己亥
• 252 •
戊寅 戊辰 乙亥
根据宋代吴曾《能改斋漫录》记载,宋代宰相曾布的八字为乙
亥、丁亥、辛亥、己亥,又宰相萧注的八字则为癸丑、乙丑、乙
丑、丁丑,这两人显然都是属于这一格局的贵人。
© 天元一气
这种格局天干一色纯清,地支也一色纯清,所以除了行运
地支被冲,大抵多贵人命。
〔入格八字J 张贵妃
乙酉
乙酉
乙酉
乙酉
此外,庚辰、壬寅、辛卯、甲戌、己巳天元一气,富贵双全。然而
天元如果都是戊午或丁未的话,那就虽贵亦多凶险,克妻不善
终。
对于天元一气的格局,任铁樵多作偏枯论。他说,“是以
人之八字,最宜四柱流通,五行生化,大忌四柱缺陷,五行偏
枯。谬书妄言四戊午者,是圣帝之造,四奏亥者,是张桓侯之
造,究其理,皆后人讹传。试思自汉至今二千余载,周甲循环,
此造不少,谬可知矣。余行道以来,推过四戊午,四丁未,四癸
亥,四乙酉,四辛卯,四庚辰,四甲戌者甚多,皆作偏枯论,无不
应验。同邑史姓有四壬寅者,寅中火土长生,食神禄旺,尚有
生化之情,而妻财子禄,不能全美,只因寅中火土之气无从引
出,以致幼遭孤苦,中受饥寒。至三旬外,运转南方,引出寅中
火气,得际遇,经营发财,后竟无子,家业分夺一空,可知仍作
偏枯论也J
• 253 
以上入格八字举例,命书尚有飞天禄马、子遥巳禄、卯未
遥巳、刑冲带合、刑合得棣、拱禄拱贵、冲合禄马、虎午奔巳、羊
击猪蛇、福德秀气、青龙伏形、白虎持势、朱雀乘风、还魂借气、
金白水清、君臣庆会等格,名目不下百余种之多。
由于八字的格,少说也有百余种之多,所以纵横开合,千
变万化,有的甚至还玄妙百出,使人眼花缭乱,莫测高深。然而
在大多数情况下,命理学家论定八字吉凶,还是以八种正格为
主,从自身日干五行出发,结合八字和岁运中五行盛衰喜忌进
行总体分析,只是在一些较为特殊情况下,对于少数几种明显
入变格的,才以格论。

\section{关于女命的看法}

中国哲学中的阴阳学说,认为女人秉天地的阴柔之气,男
人秉天地的阳刚之气,所以把女人说成属阴,男人说成属阳。
阴和阳是世间一切事物矛盾对立统一中两个截然不同 的 面,
这种思想反映在命理中是,非但在起运的岁数和大运的推排
上,女性和男性有着截然相反的不同,并且在具体的算法上,
两者也有着它明显相异的地方。
在本书前面的有关章节里我们得知,男性八字取我克的
正财或偏财为妻子,可是女性八字中的丈夫,就要来个彻底的
相反,取有我的官(正官)煞(偏官)为丈夫了。同样,在子女的
看法上,男命取克我的偏官(七煞)为儿子,正官为女儿,而女
命则取我生的食神为儿子,伤官为女儿。
•254 •
由于封建社会妇人一切都要依赖于丈夫,"夫利其 妇 必
利,夫困其妇必困”,所以看女命的好坏,先要看夫星官煞的位
置和盛衰,以定贵贱。接着再看子星,因为养儿防老,做妇
人家的本身没有收入,因此晚年的荣辱,就全押在子星的好坏
上了。
在通常情况下,官、煞、财得地,有利于丈夫,食神得地,有
利于孩子。丈夫利则出身富贵,一生享福;孩子利则晚年养
厚,褒宠诰封。由于食神能够生财,财又能够生官。比如有位
夫人八字日干乙木,乙木所生的食神是丁火,然后再由食神丁
火生土,木能克土,所以土是乙木的财,接着又由土生出金来,
金是克乙木的官,为了这个缘故,所以女命多取食神、财、官作
为八字的用神。如果八字中官、煞、财、食生得既不得地,又不
生旺,或者竟告缺如,行运时又没能够补上的,那就一生困苦
潦倒,不用说了。
又如封建礼教崇尚妇人贞洁,从一而终 以八字中如果
见官的就不要见煞,如果见煞的就不要见官,总以一位为好。
如果一旦八字中有两位官星,只要没有煞混在里面,或者四柱
里纯是煞,没有混进官星的,也都是值得称羡的良家妇女。
明代育吾山人所著《三命通会》,曾详细论述了女命的“八
法”和“八格”,现阐释如下:
1.八法
纯 所谓“纯”,就是纯一的意思。比如官星纯一,煞
星纯一,有财(财能生官)有印(印绶护身),又没遇上刑冲,这
就是纯。我们且看下面的一个女命八字,
(年) 癸巳
(月) 戊午
•255・
(日) 辛酉
(时) 丙申
八字中辛酉为自身,而酉对辛来说,由于正处在临官的禄地,
所以自身生旺。古话说「旺不从化」按理天干合局,丙辛应
该化水,现在本身专禄,也就化而不化了。这里,辛金的夫星
是克我的正官丙火,联系该命出生的戊午月,正值农历五月
火旺之时,所以夫星健旺。再联系年干癸水,恰恰与夫星丙火
形成正官关系。在用神中,正官是个吉星,所以也对丈夫十
分有利。若再联系月干戊土,又是夫星丙火的吉神食神,并
且丙火和戊土,又一起归禄(临官)到年柱的地支巳里,可谓
难得。夫星看后再看子星。辛金生壬水为子,而位居时支子
息宫的申里,恰好涵有壬水,而这壬水和申的关系,在十二宫
中又正好处在万物发生向荣的长生之地。加之天干 癸戊合
火,丙辛合水,水火有既济之象;地支巳、午、酉、申,巳里庚金,
申里庚金,酉里辛金,都是夫星丙火和月支午里丁火的财库,
所以也就自然嫁夫为官而食天禄,属于夫荣子贵的命了。
所谓“则”,就是恬静的意思。比如八字中自身柔
弱,只有一位克我营夫星,而四柱又没有攻破冲击的神,这就
是禀其中和之气的“和"了。让我们看这样一个女命八字:
(年) 壬辰
(月) 辛亥
(日) 己卯
(时) 己巳
命中日柱天干己土为自身,月柱亥中甲木为夫星。亥对甲木来
说,处在万物发生向荣的长生之地,既得天时,又得地利。甲木
以辛为正官,现在月逢辛亥,对甲正属有利。己以金为子,时
• 256
支巳中庚金虽为伤官,但是也可活看,况且巳对于庚金来说,
也处在万物发生向荣的长生之地。以上这些,叫做夫得官星,
子得长生,所以旺夫益子。至于日支卯中乙木,虽为自身己土
的七煞,然而有时支巳中庚金为制,所以“去煞留官”,为女命
中的贵象。
清 所谓‘清”,就是洁净的意思。女命中或者只有一
官,或者只有一煞,不相混杂,这就算是清了。总要夫星得时,
柱中有财有官,有印助身,没有一点混浊之气,方才来得清贵。
比如这样一个女命:
(年) 己未 , .
(月) 壬申
乙未
(时) 甲申
自身日柱乙木,以月支时支申中所含庚金为夫星。申对于康
来说,处于临官的禄地,所以夫星得时,又乙木以我生的食神
丁火为子星,而自身日支的未中正好含有丁火,而未对于丁来
说,也正好处在临官的旺地,所以子星得地。乙木以壬水为正
印,而月柱壬水又生于坐下的申金,水源不乏。加之日支未中
己土,又为乙木送来偏财。这样财旺生官,四柱又没有刑冲破
证. 诗云J财官印绶三般物,女命逢之必旺夫。”所以这妇人
有两国之封,夫人之命。
贵 所谓“贵”,这是尊荣之号。命中有官星,并且得
到财气的资生,四柱中又没有刑冲破败,这就贵为女命中的尧
舜了。经云:‘无煞(偏官)女人之命,一贵可作夫人。”又说,
“女命无煞逢二德,可二国之封。”所谓二德,不单只指天德,月
德(见本书《八字中有关的星宿神煞》篇),对于女命来说,财也
• 257-
是德,官也是德,如果再有印绶、食神,那就更加尊贵了。这里
有这样一个女命:
(年) 乙亥
(月) 丙戌
辛卯
(时) 癸巳
自身日柱天干辛金,不仅以我克的年干乙木为偏财,先获一
德,并且又以月干中克我辛金的丙火为官人,而这官人又坐在
以万物成功而藏之的戌的墓库上面,并且还把时支中的巳支
拉来作为临官禄地,所以又得一德。二德之外,时干癸水贵为
夫星丙火的官,自身辛金滋生癸水为子,而这为子的癸水恰
又偏偏坐在临官的巳上,可谓与“夫禄同位 加之时干癸见
日支卯,有天乙贵人之称(见前《八字中有关的星宿神煞》篇)。
这样又是贵人,又是财官双美,所以丈夫和儿子都贵,两遇褒
封。
浊 所谓“浊”,就是混而不清。女命八字如出现五行
失位,水土互伤,自身太旺,代表丈夫的官星显示不出来,而偏
官则一片丛杂,四柱里又没有财官印食,这些大多是娼妓婢
妾,淫巧的妇人。这里且看书中所举这样一个女命:
(年) 己亥
(月) 乙亥
(日) 癸丑
(时) 己未
自身癸水,生于十月亥月,太泛。癸水以戊土为正官,现在正
官不显,而引时干己土为偏夫,可是日支丑和时支未中都有作
为偏夫的己土混杂在里,加之日柱中没有财,乙木本为癸水的
•258
食神,然而乙木生在有力的月干上面,己土受克,这就是五行
失位,难免鬼败临身,先清后浊,不能享福了。
滥 所谓“滥”,就是婪的意思。这是说四柱天干里明
的有多个夫星,地支里暗的又财旺带煞,这就难免酒色猖狂,
私暗得财。碰上此等命的,不是克夫再嫁,就是身为奴婢,因
为太过或是不及,从而走向反面。比如这样一命:
(年) 庚寅
(月) 丙戌
(H) 庚申
(时) 丁亥
自身庚金,生于秋月,日支又逢临官禄地,本身自旺。其中月
柱重于时柱,理应丙火为夫,可是年支和月支寅戌会成火局,
时干之上又透出丁火,难免爱火重重。再如自身庚申金暗克
年支月支的寅亥木为财,而亥中壬水又为庚金的吉神食神,食
神能够生财。因此这个女人虽说长得美貌有福,然而却又少不
了属于滥淫而得财的一路。
娼 所谓“娼”,就是娼妓。八字中如出现身旺夫绝,
官衰食盛,或四柱里不见官煞,或有而被作为凶神的伤官伤
尽,或官煞混杂而食神盛旺,这些如果不是娼妓的命,就是尼
姑婢妾,克夫淫奔的命,两者必居其一。且看这样一命:
(年) 丁亥
(月) 庚戌
(日) 戊辰
(时) 庚申
自身日干戊土,本应以年支亥中甲木为克我的夫星,可是由于
其木正处在九秋的戌月,在旺相休囚死中属于失时无气的死,
•259 •
现在又逢月干庚金监临,所以必定克绝无疑。再看时支申中
庚金,理应属于戊土的食神,而申对庚来说又是临官禄地,所
以食神有力,加之戊辰原属魁罡星辰,有利男子,不利女子,现
在魁罡照临,又能生食,如再结合月干和时,干的庚金,就未免
使得食神旺过了头。虽说辰里乙木也为克我的夫星,但因位
于戊土座下,不能透出,故不能为用。此外年支亥里的壬水,
日支辰里的癸水,时支申里的壬水都是自身戊土的财。戊辰本
属魁罡,自身强旺,现在克我的夫星既已死绝,而四周又充满
一片我生的食神,所以叫做身旺逢生,贪食贪财,是个没有丈
夫的秀丽娼妇。
淫 所谓“淫”,就是淫法。这种人的八字,自身虽然
得地,可是四柱夫星太过,明暗交集,人称日干身旺,四柱中都
是官煞的便是。夫星出现在天干中的叫明,出现在地支里的
叫暗。比如一丁三壬,或丁火同时碰上天干壬水,地支辰中癸
.水,子中癸水的,就都是四柱太过或明暗交集的典型。这样的
女性对于男人,真是无所不纳的了。比如这样一个坤造:
(年) 戊辰
(月) 壬辰
壬戌
(时) 癸亥
命中壬戌、癸亥一个处在临官禄地,一个处在万物成熟的帝旺
状态,可谓自身得地。可是在夫星上,明有年柱戊土为正夫,暗
有二辰一戌所含三个戊土为暗夫,这就属于命书所说的夫星
交集,淫不可言了。
2.八格
安静守分 所谓“安静守分”,就是八字中夫星有气,
• 260 •
日干自旺,财食得所,没有刑冲的女命。且看这样一个坤造,
(年) 癸巳
(月) 官庚申
乙卯禄
(时) 丁亥
自身日干乙木,坐下日支卯为乙木的临官禄地,而时支亥与日
支卯又逢合局,所以日干自旺。再看夫星,乙木以克我的庚金
为正官。妙在夫星坐下的月支申,恰属庚金的临官福地,而年
支的巳,又为庚金的长生之地,加之时支亥中壬水,为庚金的
食神天厨,故而夫食天棣,官星美旺。这样自身、官星,各乘旺
气,两不相侵,四柱又无七煞相混,于此可知是个伉俪和谐,安
静守分的夫人之命。 /
寿两备 这是一种身坐旺乡,通于月气,支干相辅,
组合有情,并且财官印绶,各得其位,不行脱财、坏印、伤官之
局的中和纯粹格局。若果身旺而运行财、食之乡,也属“福寿
两备”之命。如这样一个女命,
(年) 丙午
(月) 庚子
辛酉
(时) 癸巳
日干辛坐酉地,专举自旺。辛金以丙火为官,现在官星归禄于
时支巳,夫星得地;辛金以癸水为子,而时干食神癸水又归裱
于月支子水,可见子星同样得地。加之干支上下相辅,彼此无
损,又生在癸水当令的十一月,成就金白水清之象。这样,命主
美貌端正,夫旺子贵,故可断知是个“福寿两备”的命。
旺夫伤子 看“旺夫伤子”的女命,关键要看时柱,因
•261 •
为时为归宿之地。其法:“夫子二星,引归时上,夫星生旺,子
座衰败是也。”现以这样一个女命为例:
(年) 丙戌
(月) 丙申
丁巳
(时) 辛亥
日主丁火,坐在帝旺的巳上,自身得地。丁火以制我的月支申
中壬水为官星,而时支亥为官星壬水的临官禄地,月支申金,
又为壬水的长生之地。加之七月金旺,申中庚金和时干辛金,
与年、月天干两个丙火成为官星壬水的印绶、财神,所以卜知
夫君聪俊富贵。丁火以食神己土作为自身的子息之垣,现在
柱中不见己土,所以权把支中所藏戊土当作子星看待。然而
把支中所藏戊土引归时支亥宫,亥中甲木,为克伐戊土的七
煞,从而把戊土置于绝地。因为子星被克,所以命里不一定有
小辈,或者有也较难养大。
子伤夫 关于“旺子伤夫”的命,可从月柱、时柱推
知。官星有气得时,丈夫可望清贵发福;如果不得月气,时柱
上又没有旺气,那末丈夫就可因失气失时而人命危浅。再如
子星引归时上,如逢长生、临官、帝旺之乡,又无刑克,说明子
星因为得地而生旺,多半可以有所作为。书中举这样一命为
例:
(年) 己卯
(月) 甲戌
乙卯
(时) 戊寅
自身乙木,以庚金为正官。生于戌月,庚金逢戌,处于无气的
• 262 •
衰地,而引归时上,时支寅又是庚金的克绝之地,所以夫星伤
残。乙木以丙火为子息,而时支寅为丙火的长生之地,并且寅
戌会局,都属于火,故卜子星趋旺。
伤夫克子 这种女命,官星和子星失月失时,又逢克
泄,所以不佳。且看这样一个女命:
(年) 丙子
(月) 官庚子
(日) 乙亥
(时) 丙子
自身乙木,以庚金为官星。现在子月金寒水冷,实为庚金的死
地,加上地支亥子会水,盗泄金气,四柱中又没有土来生金,所
以夫君不永。.乙木以丙火为子息,引至时上子宫,属于水火冲
激,水旺火灭之地,虽然年干时干并有两重丙火,怎奈四柱地
支一片汹涌之水,所以子息也难逃厄运。
少年横夭 命书认为,女命中原有官星受伤,行运再
遇官乡,或无官见伤,行运又临官乡,以及身弱官煞太重,煞重
克身等等,如此则不是凶亡,就是滥淫。此外如带刃无制,行
运又逢合刃之地,也非吉兆。比如:
(年) 官丁卯
(月) 癸丑
(日) 庚辰
(时) 丙子
自身庚金,丁火为官,现在地支子辰与丑中癸水,同为克制官
星的伤官。伤官太重,未免水多金沉,自身有陷落之忧。所以
行运一旦交入丁巳官运,伤官见官,加之时干丙煞又乘机制
身,故而有溺水之害。
• 263。
再如:
(年) 乙酉
(月) 戊子
丙寅
(时) 己亥
日主丙火,坐支寅为丙火的长生之地,可惜生逢子月亥时,官
煞太重,就好比旺火投进盛水一样,所以有临产身亡的忧虑。
又如这样一个八字:
(年) 壬子
(月) 癸卯
(日) 甲戌
(时) 丁卯
自身甲木,月支卯为羊刃,时柱丁卯又为伤官、羊刃。同时,地
支子卯相刑,卯戌相合,柱中夫星财星乏力,因此在癸酉年、乙
丑月、己卯日犯奸而死。
正偏自处 《三命通会》对于“正偏自处”的说法也很
有趣。书云「夫正偏自处者何也,乃夫妇相合,复遇比肩分
争。如一位夫星,有两位妻星相合,谓之 '争合'。若本身自旺,
彼身值衰,四柱不冲,则我正而彼为偏;若彼旺我衰,四柱冲
我,则彼正而我为偏矣。盖我身旺有气,则夫从我为正,我身
衰而别位旺,则夫从别位,我反为偏。谓之彼旺,争走我夫,我
只得为偏,或自旺太过,柱无夫星者,亦为偏,或官煞混杂,或
'伤官太重,亦为偏,更淫滥。”且看:
(年) 壬子
(月) 丙午
(日) 辛酉
•264
(时) 辛卯
自身辛金,以月干丙为官星。可是丙又与时干辛合,可见是个
二女争夫的命。好在自身坐在临官禄地酉支上,身旺有力,而
时干辛金却坐在绝地卯上,衰弱不堪一击。为此争下来的结
果,我做正妻而彼做偏房。
又如:
(年) 癸未
(月) 壬戌
(日) 癸巳
(时) 壬子
自身癸水,戊土为官。可惜贴身壬子水旺,癸巳水弱,由于弱
不胜旺,所以彼胜我衰,只好做个侧室。然而时柱壬水重而泛
滥,壬下坐支子逢年支未,又带桃花,所以彼虽是个正室,可是
却又难以自处。
招嫁不定 何谓“招嫁不定”? 说法为,"月令中有夫
星透干,与己相合,己身从伏,其夫星却无气,时引夫星,或煞
星,却乘旺地,来克己身,又从伏偏夫,故谓之招嫁不定。若夫
星不旺,或受克制,必嫁夫迟,或嫁夫不明,'或夫不济事,或有
外情。”且如:
(年) 癸酉
(月) 甲子
(日) 己未
(时) 乙亥
自身己土,甲木为官。生于子月,失时不旺,可时支逢亥,亥为甲
木的长生之地,官星当旺,然而时干乙木,又制自身为煞,日干
坐未,复为乙木的仓库之地,致使煞星也旺。这样,命主便就处
•265 •
于嫁甲而恋乙,嫁乙而恋甲的犹豫状态,所以叫做“招嫁不定二
以上古书所述八法、八格,封建意识浓重,并且有的纯属
无稽之谈,所以希望读者带着批判的眼光进行分析研究。
《滴天髓》是我国历史上很有影响的一部命学著述,书中
《女命章》说「论夫论子要安详,气静平和妇道章,三奇二德虚
好语,咸池驿马半推详「这是说看女命论夫论子,必须按照命
中五行盛衰仔细推详,不必一定斤斤于以官星为夫,食伤为
子,而对于天上、地下、人中“三奇”,天、月“二德”,以及“咸
池”、“驿马”等等,因为不是虚话好话,就是妄言谬论,所以不
必,所理会。为此,清代任铁樵《滴天髓阐微》举例道S
〔一品之封例〕
财
(年) 财丁巳官
印
官
(月) 官戊申印
劫
杀
(日) 癸丑印
比
(时) 食乙卯食
这一坤造,官星食神坐禄,印绶当令逢生,财生官旺,不伤
印绶。印蒙当令,足以扶身,食神得地,一气相生。纵览局中
五行停匀,安详纯粹,知为夫荣子贵,受两代一品的夫人之命。
t夫贵子秀例J
财
(年) 官己巳杀
印
•266 •
(月) 劫癸酉印
(H) 壬辰
伤
(时) 食甲辰杀
劫 .
秋水通源,印星秉令,官条虽旺,制化合宜,妙在时干透出
甲木,制杀吐秀,一派纯粹之气,所以人品端庄,精于诗书。结
合行运无火,有官不助,印不伤之美,故有夫星显贵,子嗣秀
美,得诰封二品之荣。
t女中才子例〕
(年) 庚辰
(月) 壬午
(H) 乙亥
(时) 癸未
乙木生于午月,火势猛而年干庚官柔脆,好在月干壬水时
干癸水通根制火,年支辰土泄火生金,加之月支午火生年支辰
土,年支辰土生年干庚金,年干庚金生月干壬水,月干壬水生
日干乙木,如此接续相生,使得柱中火不烈,F土不燥,水不涸,
木不枯,所以纯粹中和,为女中才子。
C以印为夫例〕
(年) 丙寅
(月) 辛卯
(H) 癸酉
(时) 戊午
日干癸水,生于卯月泄气之月,兼之柱中财官并旺,日干
柔弱,所以取扶我的印星作为夫星。通览局中,夫星清而得
•267 -
用,故行运行到丑运拱金泄火的几年里,连生二子,而后运行
戊子,因子水冲去时支午中丁火,使酉金不伤,夫婿有登科发
甲之喜。然而一旦交入丁亥,财星肆虐,便就撒手西归了。分
析此造,病在财气太旺,以年干的丙火合去月干的辛金,以时
午的丁火克破日支的酉金,加上寅、卯当权生火,所以当行运
一旦进入丁亥,亥与年支寅合化木,助起旺神,而丁火又紧克
辛金之际,就不禄了。
〔以财为子例】
(年) 丙辰
(月) 癸巳
(日) 丁丑
(时) 甲辰
丁火生于巳月,癸水夫星清透,甲木印绶独秀,所以品格
端庄,持身贞洁。惜在年干月支丙火太旺,生助伤官,以致镜
破钗分。然而不幸中的大幸为月支巳与日支丑拱成金局,财
星得用。古语有'身旺财为子,体衰印作儿”的说法,故而以财
为子,教子成名。后因两子皆贵,得受三品之封。
〔刑夫难守制〕
(年) 丁未
(月) 乙巳
甲午
(时) 丁卯 .
自身甲木,生于巳月,支类南方,千透两丁,似此火势猛
烈,致使甲木泄气太过,局中又无印绶生身,所以只能取时支
卯木作为用神。结合大运,早岁便入火地,故而夫婿早死。由
于其人聪明美貌,且又轻佻异常,此后一旦运至戊申,与木火
・268・
彼此争战,不能守节而一言难尽。
(;夫死自缢例〕
(年) 戊戌
(月) 己未
(H) 丙辰
(时) 戊戌
满局伤官,五行缺木,印星不见,格成顺局,其人聪明美
貌,惜四柱金水太少,伤官之土过于燥厚,而夫星辛金又投墓
于戌,故而淫乱不堪,夫遭凶死。此后随人而走,不两三年,又
克。迨至乙卯木运,犯土之旺,自缢而死。
〔欺夫淫贱例〕
(年) 戊午
(月) 乙丑
(H) 戊戌
(时) 丙辰
日干戊土,生于土旺金藏的丑月,遂使月干官星乙木不能
托根,若以乙木通于时支辰库,则辰中马木又遭戌中辛金克
伐,加之年支、日支、时干印缓生身,所以甘身强旺,足以欺官,
把夫君置于度外。结合中年行入西方金运,知为淫贱之妇。
〔水性杨花例〕
(年) 丁未
(月) 癸丑
(H) 庚子
(时) 丁亥
寒金喜火,可惜地支亥子丑会成水局,月干癸水克去年干
丁火,月支丑中癸水冲灭未中丁火余气,而时干丁火又因虚脱
• 269 •
无根而不足为用,加之四柱五行乏木,所以未能有力地生起丁
火而管伏庚金。由此,日主庚金独自行事,不顾丁火管束,属
于美貌的水性杨花一类女性可知。
@ C合多淫滥例〕
(年) 丁丑
(月) 壬子
(日) 辛巳
(时) 丙申
这一命造,月干壬水合去年干丁杀,时干丙火得禄于巳,
所以出身旧家,美而善媚,人家都称她为“赛杨妃”。 通观她的
一生,四五岁时就长得眉目秀丽异常,迨至十三四岁,更加出
落得象画中人般的娇冶,十八岁那年,和一读书人结为夫妇。
此读书人平时向来醇勤好学,但婚后因与其过度昵爱,非但一
年后学业荒废,并且终于还以纵欲患费瘵病而死。丈夫死后,
此女一发不可收拾,最后因身败名裂,无所依托而走上了自缢
亡身的绝路。分析这一女命,在于命局中天干地支合得太多,
所以任铁樵说「夫十干之合,惟丙辛合以官化伤官,谓贪合忘
官,且巳申合亦化伤官,丁壬合则暗化财星,其意中将丙火置
之度外明矣,其情必向丁壬一边。况乎干支皆合,无往不是意
中人也J
明清以降,近代命理学家袁树珊先生的《命理探原》也颇
享盛名。书中卷七载有他所推的两个女命,亦可作为批判性
研究: 一
〔为某妇推〕
(年) 乙亥
(月) 丙戌
•270
戊午
(时) 壬戌
安命乙酉
初五 丁亥
十五 戊子
廿五 己丑
卅五 庚寅
四五 辛卯
五五 壬辰
六五 癸巳
七五 甲午
戊土日元,以年干乙木为夫星,以戌枝辛金为子星。当此季秋,
木落金藏,似应夫星不旺,子星不多,然得命宫乙酉,为夫子二
星之臂助,仍卜夫兴子盛,况时干壬水,又遥生乙木耶?如夫
星配水木较多之命,则琴瑟调和,子嗣三四,尤在意中,惟傩阳
刃会合,不免人巧多劳。三十岁前,烦恼不一。近来十年,喜气
盈庭,男儿绕膝矣。四十一岁交寅运,三合化火,不无燥土,肺
肝血病,皆宜预防。向后除四十四岁戊午,四十七岁辛酉又见
损伤外,接至六十岁,大都爽健自如,寿逾花甲。
C为某孀妇推】
(年) 丁丑
(月) 己酉
(H) 戊子
(时) 丙辰
安命甲辰
初九 庚戌
•271
十九 辛亥
廿九 壬子 '
卅九 癸丑
四九 甲寅
五九 乙卯
六九 丙辰
七九 丁巳
戊土以辰枝乙木为夫星,以酉之辛金为子星。今木居墓地,而
得命宫甲辰助之,金占提纲,而得丑年合之,似可以夫兴子盛
言也,孰知命宫之甲,与月干之己化土,已失木之作用;丑酉合
金,与木为仇,不能视之为子。因此夫难偕老,子难成立。二十
三岁前犹可,二十四岁大运辛金,流年庚金,同来戕伐甲乙,不
啻拉朽摧枯,此夫丧子夭之痛,不禁而来也。所幸日元土厚,堪
胜水运,为人志坚金石,节凛冰霜,行看德播乡闾,名垂邑乘,
可钦可敬。寿六旬外。
《三命通会》、《滴天髓》、《命理探原》之外,《渊海子平》、
《命理约言》等书,也都有有关女命的论述,可谓名目繁多,应
接不暇。但是归纳起来,不外《渊海子平》所载《女命诗诀》所
说:
财官印绶三般物,女命逢之必旺夫。
不犯杀多无混杂,身强 制伏有称呼。
女命伤官 福不真,无财无印守孤贫。
局中若见伤官透,必作 堂前使唤人。
有夫带合须还正,有合无夫定是偏。
宫 杀重来成下格,伤官 重合不须言。
官 带桃花福寿长,桃花 带杀少祯祥。
• 272 *
合多切忌桃花犯,劫比桃花夫不良。
女命伤官格内嫌,带财 带印福方坚。
伤官旺处伤夫主,破了伤官损寿元。
飞天禄马井栏叉,女命逢之福不佳。
只好为偏并作妓,有财方可享荣华。
再如《命理约言》载述古赋,也有一定影响:
若观女命,则异乎男。富贵者一生官旺,纯粹者
四柱休囚,浊 滥者五行冲旺,娼淫者官煞交差。无官
多合,此为 不良。满 柱煞多,不为 克制。印绶多而老
无子,伤官 旺而幼伤夫。四柱不见夫星,未为 贞洁;
五行多遇子曜(指食神 多),难免荒淫。食神一位逢
生旺,招子须当拜圣明;官煞不杂遇印扶,嫁夫定知
登云路。守寒房 而清洁,金 猪木虎(指辛亥、甲寅日)
相逢(此二日虽克夫而守正);对空帐而孤眠,土猴火
蛇(指戍申、丁 巳日)相遇(此二日克夫不正)。 财旺
生官,辅食无伤,而夫荣子贵;官 食禄旺,一印有助,
而后宠妃褒。伤官查见无财印,败室 刑夫;官 煞重逢
遇三合(见前《天干地支的刑冲害化合》篇),荒淫无
耻。合多官重,贪淫好色之人;官 杂气衰,嗜欲刑夫
之妾。身 旺官凶,非师 尼而为 娼婢;食神变德,先贫
贱而后荣华。
此外,在看女命时还盛行着一种无稽的克夫说法。首先,
“凡女命,生日在官、鬼、死、墓、绝上,主克夫”。比如丙戌、庚
子等日出生的女命,查前《五行的旺相休囚死和寄生十二宫》
篇,丙遇戌正处在人之终而归墓的状态,而庚遇子又处在万物
死的状态,因此都能妨夫。然而也有认为,辛卯日生的女命,
•273 •
虽然逢上绝地,可却“大美小疵”,这就难以一概而论了。再之,
“凡女命,生年生日同一位者,克夫”,“生年生日带六甲者,名
曰带甲,主克夫,月共日俱带者亦然”。举例说,如果甲午年生
的女命,再碰上甲午日的生日,那就少不了要克丈夫。当然,这
都是些站不住脚,没有根据的说法,不必凭信。
有趣的是,命书中还多附有一首推算妇女怀孕,生男还是
生女的歌诀。对此,《三命通会》记载说:
七七四十九,问娘何月有,
除却母生年,单奇双是偶,奇男偶女
奇偶若不常,寿命不长久。
根据歌诀,以 49 为基数,如果母亲岁数是 31 岁(虚岁),怀孕
的月份是农历正月,那末算时 49 加 1(正月)等于 50,50 再减
掉母亲年龄 31 等于 19,19 属于单数,所以生男。如果算出来
单数应生男,双数应生女的,而生下来的结果却单数生女,双
数生男,和这截然相反,那就寿命不长而夭了。然而使人大惑
不解的是,有的命书算法还要在末了再加上十九,这就把《三
命通会》所载的推算怀孕男女法给彻底颠倒过来了。

\section{合 婚宜忌}
公元 1988 年 7 月 17 日,《新民晚报》所载周柏春《法国公
园相亲》一文的文末,作者写道「隔天,有人送来了吴小姐的
生辰八字,在我家灶上搁置三天。三天中,家里碗未打碎一
只,人未跌过一跤。据说这预示着吴小姐将为这个家庭带来
• 274 •
好运」
这里,作者虽然没有进一步谈到男女双方八字的合还是
不合,然而这种婚前看重女方八字的做法,却是我国民间娶妻
合婚的一个内容。合婚在古代也叫合姓,就是合二姓为婚姻
的意思。由于古代结婚娶妻,双方多半没有机会看到对方本
人,更不要说是了解对方的品德操行和性格脾气了。所以,在
合婚过程中,除了周柏春文中提到的一环外,更多的是男方必
先要请人看看女方八字是旺夫益子,还是伤夫克子?如若是
旺夫益子,则男方兴高采烈,合第称庆,若是伤夫克子,则男方
必定改辕易辙,另起炉灶. 在封建社会或旧社会结婚幸与不
幸纯凭运气的情况下,这种社会心理是可以理解的。且按下
这种做法本身的荒诞无稽不说,举个例说,现在男女相亲,女
方送来这样一个八字:
(年) 丁丑
(月) 壬寅
(日) 丁酉
(时) 己酉
命中日干丁火为女方自身,用月干克我的壬水为丁火夫星,而
月支寅中甲木,既为丁火自身的印,又为夫星壬水的吉神食
神。再如儿子寄居的时宫,一是丁火生出己土为子,二是夫星
壬水得己土为官,三是子星己土得寅中甲木为官,四是丁火克
时支酉为财。综合以上分析,必定是个荣夫益子的命,所以男
方高兴还来不及,哪有不满口应承下来的?
再如女方如果送来这样一个八字.
(年) 甲辰
(月) 癸酉
•275
(H) 丙子
(时) 辛卯
其中以日干丙火为自身,既有月干癸水克我为夫,又有地支辰
子会水为暗夫。再如日时干支丙辛相合,子卯相刑,地支刑而
天干合,命书认为是荒淫滚浪,酒色昏迷的命。加上丙火克酉
中辛金财旺,而这财又正处在夫星座下,所以可能卖奸得财。
逢上这种女命,对于合婚的男方来说,就往往难以接受了。
由于我国封建社会,是个以男子为中心的社会,故而表现
在合婚方面,更多的是男方对女方的挑剔。对此,有首古歌
说,
择妇须沉静,细说与 君听。
夫星要 强健,日干当 柔顺。
二德坐 正财,富贵自然来。
四柱带休囚,增命又 增寿。
贵人一位正,两三做宠娉。
金水若相逢,必遭美 丽容。
四贵一位煞,权家富 贵说。
财官若藏库,冲开无不富。
寅申巳亥全,孤淫腹便便。
子午并卯酉,定是随人走。
辰戌兼 丑未,妇道必大忌。
有辰怕见戌,有戌怕见辰。
辰戌若相见,多是淫破人。
有煞不怕合,无煞却怕合。
合神若是多,非妓亦讴歌。
羊刃带伤官,驳杂事多端。
• 276 •
清盘却是印,损子必须定。
天干一字连,孤破祸绵绵。
地支连一字,两度成婚率。
此是妇命诀,千金莫轻视。
话虽这样,然而反过来说,女方挑选丈夫,对于男方送来
八字的研究分析,也是从来不肯马虎的。因为嫁鸡随鸡,嫁犬
随犬,毕竟是件牵涉到女孩儿家终身幸福的大事,又怎能草率
从事呢?在多数情况下,女孩儿家里对于男方八字的要求是
五行中和,不偏不倚,认为这样的男人不但一生丰衣足食,并
且性格中和,寿命绵长。由于古代提倡女子嫁人,从一而终,
结果只考虑到男方的荣华富贵,而不考虑到男方的性情脾气
和生命寿夭,那往后的日子又怎么过下去呢?
所以,出于以上种种对嫁娶问题的看法和忧虑,命书综括
男女合婚的要领是「男家择妇,八字贵看夫子二星,盖夫兴子
益,其福必优也。女家择夫,八字贵得中和之气,盖不偏不倚,
其寿必长也J
然而,世间男女的八字毕竟千变万化,数目繁多,而又哪
,来这么多夫荣子贵,八字中和的命呢?对此,男女间八字如果
有偏的,合婚时互相补偏救弊,转劣为优的学说便就出笼了。
譬如男命自身的日干是木,而八字中比肩、劫财的甲乙木较
重,但女方送来八字的自身偏偏是戊己土,按理木克土,丈夫
制约妻子,在封建伦理中是天经地义的事,可是到底因为男方
木势太重,难免中途克妻,所以这时就得看看女方的食伤庚辛
金如何了。如果食伤重的,由于金能制木,因此招架得住,可
以合婚J如果女方食伤不足,只要戊己土多,能够生金的,也无
伤大雅,同样可以合婚;只是自身衰弱而又无食神庚金可以抵
• 277*
敌的,那就只得彼此说声再见,重新物色对象了。同样道理,反
过来说,如果女命中食伤庚辛金太多,那末找丈夫时最好对方
比肩劫财木多,才能够抵敌得住,因为木多金缺,女方砍伐就
费力了。据说,如果按照这种原则配对起来的夫妻,虽然自身
八字都有偏胜偏衰,这样那样的不足,可是由于彼此取长补
短,取得了动态的平衡,所以还是能够“琴瑟和谐,子嗣蕃衍”
的。把这一男女间彼此救弊补偏的合婚原则归纳起来,就是,
“男命木盛宜金者,得女命之刚金补之,则为尽美,得土生金者
亦佳,得火者较次,得水木则无取矣。如女命刚金喜火者,得
男命之烈火助之,则为尽美,得木生火者亦佳,得水者较次,得
金士则无取矣。”其他五行偏盛偏衰的,可以照此类推。
此外,在合婚中,还有一种用“骨髓破”、“铁扫帚”、“六
害”、“大败”等凶煞来作为避忌的。判定这些凶煞的办法是根
据出生年份的地支,结合农历出生月份的地支来定。例如子
年出生的人生在五月(午月),如果是位女命,就被认为是犯了
“再嫁”的神煞。在合婚时男方家里如果看到这种女命,就会
退避三舍,敬而远之。对于这种没有根据的凶煞说法,陈素庵
《命理约言》卷四《张神峰辟五行诸谬》曾进行了有力的批驳:
“吕才合婚书,俱为理之所无。人之婚姻,由于前定,择婚择
命,不过父母爱子之心。男之择女也,八字贵看夫、子二星,女
之择男也,八字贵得中和之道,何以妄立骨髓破、铁扫帚、六
害、大败、狼籍、飞天、八败、孤虚等谬说,将生年十二支,止以
月家一字为犯,岂有是理耶?”接着进一步说「进财退财,望门
守寡,妻多危,夫多厄,死墓绝,妨夫妻,止以人之生年金、木、
水、火、土,纳音所属月上一字犯之,夫退财进财,系乎自己命
运,安有他人家男女,而能致我之祸耶?”随之,书中还无情地
•278
批驳《合婚书》说「《合婚书》以男女年命宫数,配合天医、福德
为上婚,游魂、归魂为中婚,五鬼、绝命为下婚,若果有是理,则
凡议婚者,俱择上、中者配之,择下者舍之,天下必无怨女旷夫
矣。”比如前文所说“八败”,所谓“八败”,如“猪羊犬吠春三月,
盖以亥、未、戌人,三月生者,遂为八败,不论日、时,不论夫、
子”,其谬可知。又如女命所最忌的桃花煞,“如寅、午、戌兔从
茅(卯)里出,盖取寅、午、戌属火,沐浴于卯,火在卯上沐浴,有
裸体之嫌”,也为不值得一驳的无根游谈。书中行文至此,并
举一女命为例说「吾见夫、子两全富贵老妇,因其幼带八败诸
煞,父母对其八字改造适人,及至临终,始告夫、子真造,以纪
谱志墓。尝取其真造视之,原系夫子明透,理得中和,世俗止
谓其带八败诸凶,而不知其八字甚美也。”

\section{《金瓶梅》和《红楼梦》里的两次算命}
我国算命术自从五代的徐子平奠定基础以后,经过两宋
元朝的氤氢作气,浸涧蔓延,到了明清之时,早已风靡了整个
民间。社会上找人算命的,已经蔚成一种风气。百姓中间,不
管是举士应考,商人经商,还是结婚生子,生老病死,都要找人
问问算算,是吉是凶。到了这时,算命问卜,实际上已成了民
俗中密不可分的一部分了。
对于我国这种算命术的土特产,由于它自始至终打着阴
阳五行哲理的旗号,所以广泛地为知识分子所接受。社会上
除了那些骗饭吃的专业算命先生外,文人学士会算命的也比
•279
比都是。正因为算命术在文人学士中有着这样的基础,所以
又常反映到他们的作品中去。这不仅反映在他们的一些集子
或笔记中,并且还毫不例外地反映到一些优秀的小说中去。
《金瓶梅》是明人小说中熠熠发光的佼佼者。由于作者学
问浩瀚,兼通命相,所以小说里涉及算命看相,占卜问卦的竟
有好几处之多。除了给西门庆算命,书中第六十一回黄先生
为西门庆娇妾、身患重病的李瓶儿算的那个命,就是个很典型
的例子。
连日来,李瓶儿的病愈来愈重,精彩消磨,月水淋漓,六脉
沉细,一灵缥缈。一连请了好几个医生,有的说是重情伤肝,
肺火太旺,以致木旺土虚,血热妄行,犹如山崩而不能节制;有
的说是精冲了血管而起,然后着了气恼,气与血相搏,则血如
崩。这样药石乱投,你治你的,他治他的,早已乱了套儿。一 天晚上,西门庆娘子吴月娘对西门庆道「你也省可与她药吃,
她饮食先阻住了,肚腹中什么儿,只是拿药淘碌他。前者那吴
神仙,算她三九上有血光之灾,今年却不整二十七岁了?你还
使人寻这吴神仙去,却替他打算,算那禄马数上如何?只怕犯
着什么星辰,替他襄保襄保。”西门庆听了,旋差人拿帖儿往周
.守备府里问去。那里回答「吴神仙云游之人,来去不定,但
是,只在城南土地庙下。今岁从四月里,往武当山去了。要打
数算命,真武庙外有个黄先生,打的好数,一数只要三钱银子,
不上人家门。”西门庆随即使陈敬济拿三钱银子,径到北边真
武庙门首黄先生家。门上贴着「妙算先天易数,每命卦金三
钱。”陈敬济向前作揖,奉上卦金,说道,“有一命,烦先生推
算」写与他八字,女命,二十七岁,正月十五日午时。这黄先
生把算子一打,就说:“这个命,辛未年,庚寅月,辛卯日,甲午
• 280 •
时,理取印绶之格。借四岁行运。四岁己未,十四岁戊午,二
十四岁丁巳,三十四岁丙辰。今年流年丁酉,比肩用事。岁伤
日干,计都星照命,又犯丧门五鬼,灾杀作吵。夫计都星者,阴
晦之星也,其象犹如乱丝而无头,变异无常。大运逢之,多主
暗昧之事;引惹疾病,主正二三七九月,病灾有损,小口凶殃,
小人所算,口舌是非,主失财物。或是阴人(女人),大为不
利」抄毕数,敬济拿来家,西门庆正和应伯爵、温秀才坐的,见
抄了数来,拿到后边,解说与月娘听。见命中凶多吉少,不觉:
眉间带上三黄锁,腹内包藏一肚愁。
这里,李瓶儿八字的情况是:
(年) 辛未
(月) 庚寅
(日).辛卯
(时) 甲午
黄先生认为这八字理应取印绶之格,他虽没有说清原因,想来
月支寅中戊土,原是生她自身辛金的印绶,所以就取了这格。
至于行运,则排倒了。
再说流年丁酉,比肩用事。酉属辛金,和自身辛金都是同
类的阴干,所以说是比肩用事。至如岁伤日干,就是流年丁火
太岁,克伤了日干的辛金。按照命书的说法,岁伤日干,未必
就会大祸临头。这里,黄先生把李瓶儿流年说得大不吉利的
主要原因是计都照命,又犯丧门五鬼,灾杀作吵。其中重点发
挥了一通对于计都星照命不利的种种理由。原来命书 认为,
计都是星命家十一星中的一星,和罗喉星相对,十八天行一
度,十八年行一周天。平时经常隐而不见,碰上日月行次即
蚀,所以黄先生才有“夫计都者,阴晦之星也,其象犹如乱丝而
• 281 •
无头,变异无常”等不吉的说法。在阴阳中,女人属阴,阴人再
碰上这倒霉的阴星,也就难怪李瓶儿最终要一命呜呼了。
清朝人算命,不象明朝那样,把八字和神煞连得紧紧的。
因为神煞一般都是硬套的,并且凶多吉少,从而为算命的准确
性和灵活性设下了重重障碍。因此,单从本身八字出发结合
岁运,论定吉凶,就成了清代命理学家的一大特色。当然,论
神煞的也不是说绝对没有,如《红楼梦》曾用薛蝌的话说过:
“既有这个神仙算命的,我想哥哥今年什么恶星照命,遭这么
横祸?快开八字儿,我给他算去,看有妨害么?”然而和明朝人
比较起来,比重要减轻多了。
有趣的是,在《红楼梦》这本封建社会百科全书中,著作者
学识的光华,不仅表现在社会伦理、诗词歌赋、政治经济、琴棋
书画、文物掌故、饮食烹调、儒学佛道等等多方面,并且还深刻
地表现在对医卜星相等三教九流的无所不通上。翻开《红楼
梦》第二回,却说娇杏那丫头,便是当年在窗外掐花儿回顾贾
雨村的,只因为这偶然的一看,便弄出了这样一段奇缘,也是
意想不到的事。谁知她命运两济,不承望自从嫁到贾雨村身
边只一年,就生了一个儿子,又过了半年,雨村嫡配忽然生病
去世,雨村便把她扶为正室夫人。正是,
偶因一回顽,便为人上人。
书中说娇杏丫头 “命运两济”,是说她命好,运气也好。
"命"是指人一生的贵贱祸福,穷通寿夭的总和,'运”是指人一
生中各个阶段的不同机遇和气数。
又如《红楼梦》第六十九回王熙凤借剑杀人,把个尤二姐
折磨得四肢懒动,茶饭不进,渐次黄瘦下去。后来王雅凤叫人
出去算命打卦,算命的说是让属兔的阴人冲了,大家算将起
• 282 •
来,只有秋桐一人属兔,说她冲的。结果气得秋桐大哭大骂,
把个早已恢恨一息的尤二姐,气得当夜五更就吞金自杀了。
然而,书中对算命文化真正花费笔墨的,则是高鹦续作,
第八十六回算命先生给元妃算命的那一段。
书中宝钗说道「不但是外头的讹言舛错,便在家里的,一
听见 '娘娘'两个字,也就都忙了,过后才明白。这两天那府里
头这些丫头婆子来说,他们早知道不是咱们家的娘娘。我说,
‘你们那里拿得定呢?'他说道:'前几年正月,外省荐了一个算
命的,说是很准的'。老太太叫人将元妃八字夹在丫头们八字
里头,送出去叫他推算,他独说: '这正月初一生日的那位姑
娘,只怕时辰错了,不然真是个贵人,也不能在这府中。'老
爷和众人说:’不管他错不错,照八字算去。'那先生便说:'甲
申年,正月丙寅,这四个字内,有伤官、败财,唯申字内有正官、
裸马,这就是家里养不住的,也不见什么好。这日子是乙卯,
初春木旺,虽是比肩,哪里知道愈比愈好,就象那个好木料,愈
经斫削,才成大器。'独喜得时上什么辛金为贵,什么巳中正
官、禄马旺地,这叫作 '飞天禄马格 又说什么日逢专禄,贵
重的很。 '天月二德'坐本命,贵受椒房之宠。这位姑娘,若是
时辰准了,定是一位主子娘娘。这不是算准了么?我们还记
得说「可惜荣华不久,只怕遇着寅年卯月,这就是比而又比,
劫而又劫,譬如好木,太要做玲珑剔透,木质就不坚了。' 他们
把这些话都忘了,只管瞎忙。我才想起来,告诉我们大奶奶,
今年那里是寅年卯月呢?”
可知,高鹘书中给元妃安排的八字是,
(年) 甲申
(月) 丙寅
•283 
(H) 乙卯
(时) 辛巳
日柱乙卯是元妃的自身。在寄生十二宫中,卯是乙木的临官
禄地,所以说“日逢专禄”,是一种很好的命。再如 '辛金为
贵”,命书指出,辛见寅为天乙贵人,贵重得很,现在时干和月
支配合,就应了这命。“巳中正官,禄马独旺”,是说巳中庚金,
为日干乙木的正官,巳支本身又为丙火的临官榇地,加之时支
巳和日支卯相逢,应了驿马启动的命,所以算命的说元妃的命
“真是个贵人,也不能在这府中”。
那末不在常府中,又怎么料定非要受宫中椒房之 宠呢?
这是因为"天月二德坐本命”的缘故。这里,宝钗口中所说“天
月二德坐本命"和命书里排定的天德、月德有所出入,看来当
是指的“归禄逢二德”了。
至于所说的“飞天禄马格”,《喜忌篇》有云,*若逢伤官月
建,如凶处未必为凶,内有倒禄飞冲。”元春生于乙卯日,乙为
阴木,其官星为庚金,而月上丙火能克庚金,这就成了“伤官月
建”。 乙日既得丙火,又生在春初寅木之月,日支上的卯木便
可冲出巳时所含的申金,“倒禄飞冲”,便成了“飞天禄马”格。
且丢开这些不说,无论如何,作者这里用了一定量的篇
幅,借宝钗的口转述了算命先生对命理的一番分析,说明他对
命书有过兴趣,有过研究,则是肯定无疑的。更不要说他在书
中所说“可惜荣华不久,只怕遇着寅年卯月,这就比而又比,劫
而又劫,譬如好木,太要做玲珑剔透,木质就不坚了”的这一段
话,还又十分在行的呢。
那末,高鹑为什么在续作中对命理文化,要借着算命先生
的口作一番如此的发挥呢?这自然和原作者曹雪芹对天命的
• 284 •
看法有关。书中第二回,曹雪芹介绍宝玉来历,以贾雨村之
口
大约
,对
政老前
冷子兴
辈也错
罕然厉
以淫魔
色道:
色鬼看
“可惜你
待了。
们不知
若非
道这人
多读书
的来历
识事,——加
以致知格物之功,悟道参玄之力者,不能知也。”冷子兴见他说
得这样重大,连忙请教缘故。雨村这才从头细述道:
天地生人,除大仁大恶,余者皆无大异。若大仁
者则应运而生,大 恶者则应劫而生。运生世治,劫生
世危。尧、舜、禹 、汤、文 、武 、周 、召 、孔、孟、 董、韩、
周、程 、朱 、张 ,皆 应运而生者;蚩尤、共工、桀、纣、始
皇、王莽、曹操、桓温、安禄、秦桧等,皆应劫而生者。
大仁者修治天下,大恶者扰乱天下。清明灵秀,天地
之正气,仁者所秉也;残忍乖僻,天地之邪气,恶者之
所 秉也。今当 祚永运隆之日,太 平无为之世,清明灵
秀之气所秉者,上自朝廷,下至 草野,比比皆是。所
余之秀气,漫 无所归,遂为 甘露,为和风,洽然 溉及四
海;彼残忍乖邪之气,不能荡溢于光天化日之下,遂
•凝结充塞于深沟大壑之中,偶 因风荡,或 被云催,略
有动摇感发之意,一丝半缕,误而逸出者,值 灵秀之
气适过,正不容邪,邪复 妒正,两不相下,如风水雷
电,地 中既遇,既不能消,又不能让,必致搏击掀发。
既然发泄,那邪气亦必赋之于人,假使或男 或女,偶
秉此气而生者,上 则不能为仁人君子,下亦不能为大
凶大恶。置之千万人之中,其聪俊灵秀之气,则在千
万人之上I 其乖储邪谬不近人情之态,又在千万人之
下。若生于公侯富 贵之家,则 为情痴情种;若 生于诗
书清贫之族,则 为逸士 高人, 纵然生于薄祚寒门,甚
• 285 •
至为 奇优,为名娼,亦断不至为走卒健仆,甘遭庸夫
驱制
——如前之许由、陶潜、阮籍、嵇康、刘 伶、王谢
二族、顾虎头、陈后主、唐明皇、宋徽宗、刘庭芝、温飞
卿、米 南官、石曼卿、柳 耆卿、秦少游,近日 倪云林、唐
伯虎、祝枝山,再如李龟年 、黄播绰、敬新磨、卓 文君、
红拂、薛涛、崔鹭 、朝 云之流,此 皆易地则同之人也。
贾雨村的这番细述,原是曹雪芹对于命理思想淋漓尽致
的发挥。文中,曹雪芹认为,天地生人,除了大仁、大恶,其余
的芸芸众生,很难说得上有什么大的差异。对于大仁、大恶
两者,大仁者应运而生,有修治天下之功,所以大仁者降生就
天下大治;大恶者应劫而生,有扰乱夭下之能,所以大恶者出
世就天下大乱。此外,天地之间又有正气和邪气的不同,正气
清明灵秀,邪气残忍乖僻. 禀受正气出生的人日后成为仁者,
禀受邪气出生的人日后成为恶者。然而,社会上更多的是那些
同时既受正气,又受邪气而出生的人,因为这个原因,所以这
些人上不能成为仁人君子,下不能成为元凶大恶,而只能成为
社会上成千上万的芸芸众生。再如这些芸芸众生,也有层次
上下的不同,禀受聪俊灵秀之气的处在千万人之上,禀受乖僻
邪谬之气,生而不近人情的处在千万人之下。又如虽然芸芸
众生所票的气大致都差不多,但和出身家庭环境也有千丝万
缕的关联。出身在公侯富贵之家的或许成为情种,出身在诗书
清贫之家的或许成为逸士高人,出身在寒门薄祚之家的或许
成为奇优名娼,那跳又要看具体情况,不可一概而论了。
可以看出,曹雪芹这番人禀天地之气出生,并结合出身家
族门第论命的观念,比起王充以来命理学家只论禀气,不问出
身门第的说法,显然推进了一大步,因为他毕竟把后天家庭环
•286
境对每个人所产生的种种影响,提到了议事日程上。可是,从
曹雪芹对命理学的整个认识大轮廓看,他的宿命思想,决定了
他到底不能成为天命论的彻底叛逆者。当然,对于这点,通情
达理的读者是不会用今天的要求,来要求生活在封建社会大
环境、大气候里的曹雪芹的。
《金瓶梅》、《红楼梦》之外,明清小说中有关算命的比比都
是,如极为著名的,就有吴敬梓《儒林外史》第五十四回《病佳
人青楼算命,呆名士妓馆献诗》等有关算命的描述。文中,作
者通过弹三弦瞎子为青楼女聘娘算命,以及陈木南和瞎子之
间的谈话,从一个侧面反映了当时社会算命风气之盛和作者
对算命术的了解。

\section{古代名人八字举要}
这是一个很有趣的课题,可供批判性研究。袁树珊《命
谱》曾为诸葛亮算命道:
诸葛武侯相后汉灵帝光和四年七月二十三日巳时生
〔命造八字 大运
(年) 辛酉辛金偏印 三岁乙未
(庚金正印
(月) 丙申{壬水劫财
1戊±1E官
,
癸水比肩
十三甲午
(B) 癸丑{辛金偏印
1己土偏官
二十三癸巳
•287 •
,庚金正印
(时) 丁巳{丙火正财 三十三壬辰 I戊土正官
f
乙木食神 四十三辛卯
(命宫) 壬辰《戊土正官
1•癸水比肩 五十三庚寅
日元癸水,诞生立秋节后,白帝司权,金正当令,水得金
生,正气充足,再逢年干辛金,年枝酉金,及月枝申藏庚金,又
藏壬水,日枝丑藏辛金,又藏癸水,叠叠生之助之,其为金白水
清,显然易见。仅恃月干单独丙火,不独不能制金,且亦不敷
济水之用,况丙与辛合,同化为水,其火之成分,又复若有若
无,没有生时丁巳之二火,决不能制当令之旺金,济有余之相
水。今既得此为正式之用神,其为雨场时若,天地顺成可知。
接着作者笔锋一转,大致认为诸葛亮中年以后,大运金水
连环,和用神火背道而驰,虽说鞠躬尽力,也只能够事倍功半。
五十四岁大运庚寅,流年甲寅,岁支寅和命中月支申相冲,与
时支巳相刑,所以一旦当生命进入当年八月癸酉,二十八日庚
辰,金水汹涌,助纣为虐之时,也就难逃厄运,卒于军中了。
又如任铁樵《滴天髓阐微》,也举乾隆皇帝御命分析说:
(年) 劫辛卯财
(月) 官丁酉劫
(日) 庚午:
(时) 杀丙子伤
大运 丙申 X
乙未
甲午
• 288 •
癸巳
壬辰
辛卯
庚寅
批曰:天干庚辛丙丁,正配火炼秋金,地支子午卯酉,又配坎离
震兑。支全四正,气贯八方,然五行无土,虽诞秋令,不作旺
论。最喜子午逢冲,水克火,使午火不破酉金,足以辅主,更妙
卯酉逢冲,金克木,则卯木不助午火,制伏得宜。卯酉为震兑,
主仁义之真机,子午为坎离,宰天地之中气,且坎离得日月之
正体,无消无灭,一润一暄。坐下端门,水火既济,所以八方宾
服,四海攸同。金马朱南,并隶版图之内,白狼元兔,咸归覆傅
之中。天下熙宁也。
按下诸葛亮、乾隆帝八字不表,我们这里再罗列古人八字
一束,稍作提示性分析,举要而已。
[孔子〕
(年) 庚戌
(月) 戊子
庚子
(时) 甲申
自身庚金,归禄在时支申中,并且时干甲木为庚金偏财,本属
难得,惜月支子月,寒水当令,虽然金白水清,然而未免寒俭,
况年支戌中官星丁火偏处一隅,旁受月支癸水制约,难以发
挥。纵观孔子一生奔波,劳而无功,政界失意,直到晚年杏坛
设教,弟子三千,说明金水流通,从文却好。孔子生于公元前
551年庚戌,关于他的八字原有多种说法,也有说他出身月份
为乙酉的,录此以备一格。
•289 •
〔关羽〕
(年) 戊午
(月) 戊午
戊午
(时) 戊午
《三命通会》说「戊午日,戊午时,先刑后发,多不善终。”又说:
“纯午,武职威权,名重藩镇。” 基本和关羽的一生吻合。此外
在格局中,这是一种“天元一气”的格局,又名凤凰池。据说,
张飞的八字是“癸亥、癸亥、癸亥、癸亥”四个癸亥,一片铺天盖
地的癸水,和关羽的一片火土正好来个一百八十度的大相反。
火土红黄,癸水纯黑,小说中关羽面如重枣,张飞面如黝漆,大
概和这不无关系。又据袁树珊《命谱》记载,关羽的生辰八字
为,庚子、甲申、戊午、庚申,当属可信,前者恐系出于杜撰,也
未可知。
〔隋炀帝〕
(年) 乙酉
(月) 乙酉
乙酉
(时) 乙酉
隋炀帝杨广为隋文帝的次子,在位期间由于荒淫奢侈,滥
用民力,加之穷兵艘武,统治暴虐,所以当农民大起义的浪潮
推向江都时,为部下宇文化及等发动兵变所缢杀,终年五十
岁。
此外,唐朝那个赫赫有名胖美人杨贵妃的八字,据说也是
乙酉、乙酉、乙酉、乙酉,早年富贵,结局凄惨,和隋炀帝大同小
异。为此古诗有云「乙酉一场空,杨广和太真J
•290
其实,古代命书记载往往不可靠。比如隋炀帝杨广,出生
于公元 569 年己丑,杨贵妃玉环,出生于公元 719 年己未,所
以他俩的四柱,不可能出现乙酉、乙酉、乙酉、乙酉是明摆着
的。
〔唐太宗〕
(年) 庚辰
(月) 庚辰
(日) 庚辰
(时) 庚辰
唐太宗李世民为我国历史上的一代明主,在位期间知人
善用,政绩斐然,死时年仅五十一岁。
又据命书记载,武周神圣皇帝武则天的八字,也为庚辰、
庚辰、庚辰、庚辰,但却享年八十一岁。由于两人都有政声,并
且福禄绵绵,所以古诗说道:“庚辰福禄全,武后与世民J
可是事实则是,唐太宗李世民出生于公元 599 年己未,武
周神圣皇帝武则天出生于公元 624 年甲申,所以两人的八字,
也根本不可能为庚辰、庚辰、庚辰、庚辰。
〔张巡〕
(年) 己酉
(月) 庚午
(0) 癸酉
(时) 乙卯
张巡(公元 709 757 年),邓州南阳(今河南省南阳市)
人。唐玄宗开元末年进士。安禄山造反,张巡和许远苦守唯
阳,浴血奋战,牵制了叛军,保障了江淮,最后终因粮尽援绝,
与城同亡。后来韩愈写了篇《张中丞传后序》的文章,得以使
•291 •
其事迹千古流芳。
近代天虚我生的女儿陈小翠女士精于命理,曾对张巡的
命造作了这样的简析「身弱有印,枭食相争,丑为杀墓,凶死。
乙丑又枭夺食,绝食。”又批为「四十八入丑(运),第二年在己
(死于己土流年)」
〔欧阳修〕
(年) 丁未
(月) 戊申
乙卯
(时) 戊寅
欧阳修为北宋大文学家,纵观乙卯日柱,生于戊申月令,
格属正官,身弱可知。一生行运,以四十一岁起的癸卯、壬寅
两步大运为最辉煌。此后运入辛丑,六十六岁入丑第一年不
禄。陈小翠女士批为「丑为杀墓,宜死。"
(邵雍〕
(年) 辛亥
(月) 辛丑
甲子
(时) 甲戌
邵雍为北宋著名哲学家和道学家,著作有《皇极经世》、
《伊川击壤集》等。命造日元甲木,生于季丑冬月,时支逢戌,
这是移根换叶,甲木逢养的迹象,属 于学界一流的品位。
〔王安石]
(年) 辛酉
(月) 庚子
(日) 癸未
•262 •
(时) 丙辰
王安石为历史上声名显赫的改革家. 日元绕水,建禅于
子,属于气旺有印的建禄格局。五十五岁进入甲午大运;开始
走下坡路。陈小翠女士批为「甲木伤官运,罢相丧子I午运填
实(拱未)冲子,终。”袁树珊《命谱》则分析为,"六十六岁元祐
元年丙寅,四月癸巳,初六日癸巳,卒于金陵者。盖大运逢癸,
月日亦逢癸,乃水归冬旺之命所最忌。虽岁逢火木,亦难胜
之,以多暴寡,莫可如何也J
〔苏轼〕
(年) 丙子
(月) 辛丑
(日) 癸亥
(时) 乙卯
东坡居士苏轼的大名,真是响当当矣。四柱地支亥子丑
全,为北方一气格。陈小翠女士批为「罗网乙巳,辰运试入三
等,丧偶丁忧,余未见凶,子辰化水也。”又说巳运“冲亥,入狱
己未年:未运“冲丑,谪檐耳病”「午运终,戊癸化火」
〔蔡京〕
(年) 丁亥
(月) 壬寅
(日) 壬辰
(时) 辛亥
从这命造的日柱看,既属于壬骑龙背的格局,又可属于魁罡的
格局,所以命主生平看好。然而,《三命通会》却又认为,“壬辰
日,辛亥时,秀贵,恶死”。蔡京本是北宋权臣,后来金兵攻宋,
他带着全家仓皇南逃,被钦宗下令放逐岭南,结果在途中死于
•293 •
潭州(今湖南长沙)。据说那时京城里有个孩子的八字,也和
蔡京生得一模一样,可他却只在十岁就淹死了。
〔秦桧〕
(年〉 庚午
(月) 己丑
(日) 乙卯
(时) 壬午
秦桧奸臣,万民皆知。日元乙木气弱,时干壬水为印,月
柱财星克印,运忌财官。当年陈小翠女士批其大运为:“午未
运皆贪财坏印,故为小人。”‘午运冲子,杀岳飞;未运冲丑合
午。”袁树珊《命谱》批其末运时这样写道「六十六岁一病丧身
者,大运在丙,与年干庚战;太岁乙亥,小限庚辰,会命宫乙酉,
生时壬午,自刑备至,故无可逃避也。”难免生搬硬套,牵强附
会。
〔朱熹〕
(年) 庚戌
(月) 丙戌
(日) 甲寅
(时) 庚午
日干甲木,专棣于寅,地支寅午戌会成火局,木火通明之
象。后入巳运,时干庚煞长生,享年七十一岁而终。袁树珊先
生《命谱》批其末运为「公享寿七旬有 适在巳运,庚申年,
庚辰月,甲子日,庚午时,恬然而逝。具良日元一甲,不胜三庚
之克,寅午戌之火,不胜申子辰之水冲也。然而绍道统,立人
极,为万世宗师,木火齐辉,岂有既极耶J
C贾似道〕
・294・
(年) 癸酉
(月) 庚申
丙子
(时) 丙申
《三命通会》说「丙子日,丙子时,若通火气及寅、卯月,再
行身旺运,吉。年月纯金,弃命就财,亦以吉论。”贾似道为南
宋权相,命书说他“奸臣 南宋末年,元军沿江东下,他率兵
抵抗,兵败革职,在放逐途中,被监送人郑虎臣所杀。
〔元世祖〕
(年) 乙亥
(月) 乙酉
(日) 乙酉
(时) 乙酉
这一格局年、月 、日 、时,天干全是乙木,纯一不杂。按照命书
说法,这是一种*干辰一字”的格局,属于大贵的命。元世祖名
忽必烈,为元代的开国皇帝,一生龙振虎威,功业十分显赫。
〔脱脱臣相〕
(年) 壬辰
(月) 丁未
己丑
(时) 己巳
《三命通会》评此命说,“金神生六月中旬,火旺,未有木库
偏官,年干透壬,丁壬合化真木助官,喜又带刃。运行西方,有
戊己克水,申酉制伏偏官。行戌运,冲开火库,金神有制,贵至
台辅。行亥运水旺之地,三十七岁戊辰,岁君刑开水库,金神
无制,财旺生起官煞为祸,死于鸩毒
•295
〔赵孟殖〕
(年) 甲寅
(月) 甲戌
(日) 己酉
(时) 己巳
这一命造,时逢己巳,属于金神格局。金神原为破败之
神,“要制伐入火乡为胜”,现在月支戌中丁火,年支寅中丙火
一起制伐,加上日干己土遇印生我,比劫助我,所以一旦甲木
% 制我,就格局平衡了。赵孟^本是宋代宗室,入元后元世祖忽
% 必烈搜访遗逸,经程但夫荐举,官刑部主事,后又累官至翰林
% 学士承旨,封魏国公,谥文敏。在艺术上,他的书画几乎一手
% 笼罩了整个元代的天下,后人学他的很多。
% 〔明太祖〕
% (年) 戊辰
% (月) 壬戌
% 丁丑
% (时) 丁未
% 这一命造,如果年、月地支不见辰、戌,单是日、时地支丑、未刑
% 冲,大有不得善终的忧虑,妙在现在年、月、日、时的地支,辰、
% 戌、丑、未四库一应俱全,这就非但无忧,并旦贵为天子了。据
% 说明太祖朱元璋登基以后,听到天下也有一个人和他的八字
% 相同,这就使他大为忧虑,动了杀心。后来召来一看,原来是
% 洛阳地区一个姓李的穷老头儿。朱元璋问他干什么活,他说:
% “老民养蜂十三窠,以之度日。”朱元璋听后宽了口气说「这和
% 我国家享有十三省布政司的税收正好一样。”把十三省税收和
% 十三窠蜂相比,除了表面数字相同,可实质上却有着霄壤之
% •296 •
% 别。
% 〔王守仁〕
% (年) 壬辰
% (月) 辛亥
% (日) 癸亥
% (时) 癸亥
% 王守仁为我国明朝著名哲学家、教育家。他的四柱命造,
% 癸水气旺,支排三亥,遥冲三巳,入于飞天禄马的格局。因为
% 无火,所以清贵。三十五岁行运丙寅,到三十九岁庚午,自抗
% 谏下狱,谪赴贵阳,后来又升为南京刑部主事,忽荣忽辱,备极
% 艰辛。原因之一为寅运伤官见官,故凶;好在寅中藏土之外,又
% 藏木火,所以逢凶化吉,得免危亡。此后行运乙卯,仕途亨通,
% 大吉。此后一路迪吉,终于辰运戊子年。
% 〔严嵩〕
% (年) 庚子
% (月) 己卯
% (日) 癸卯
% (时) 辛酉
% 严嵩奸臣,嘉靖三十一年(公元1542年)任武英殿大学士,
% 入阁,把持朝政二十年,官至太子太师。后被革职,家产籍没,
% 不久病死。陈小翠对他命造的批语为「食格时枭,劫刃伤官
% 终。刑冲太多,必非良善 j金木交争,不仁不义」
% 〔张居正〕
% (年) 乙酉
% (月) 辛巳
% (日) 辛酉
% •297
% (时) 辛卯
% 这一命造,按照《三命通会》说法:“辛酉日、辛酉时,出身孤苦,
% 中年获福,末年封妻荫子,贵。” 张居正是明朝有名的政治家,
% 在他入阁当国的十年间,推行一条鞭法,颇有政绩。公元1582
% 年壬午张居正虚龄五十七岁,这一年,大运丙子,流年壬午,岁
% 和运子午相冲,张居正死。
% 〔戚继光〕
% (年) 戊子
% (月) 癸亥
% (0) 己巳
% (时) 乙亥
% 这一命造,日主己巳,虽然金神位置不在时上,未能作金神格
% 局看,可是月干偏财,时干偏官,自身又得戊土之助,所以扶抑
% 得宜。加之地支子亥年月,所以《三命通会》认为,“以财党煞,
% 作弃看,主大兵权”。
% 〔董其昌〕
% (年) 乙卯
% (月) 戊寅
% 乙卯
% (时) 庚辰
% 董其昌为明末大书画家。日元乙木,专禄于卯,地支寅卯
% 辰,会成东方一气之格。晚年八十三岁,大运莅午,岁值丁丑,
% 丑午相害,火泄木气而终。
% 〔洪承畴〕
% (年) 癸巳
% (月) 壬戌
% • 298
% (H) 癸酉
% (时) 壬戌
% 洪承畴为明末大臣,后来投降清朝,在我国历史上留下了
% 骂名。陈小翠批其命造为「水多必智。正官格、两干不杂格。
% 巳火太旺,为财克印。甲运伤官终,岁甲辰,辰戌冲 又说:
% “戌为正官,不妨;巳则财旺伤印,是其病根。观其后交巳运而
% 失节可信。”
% 〔明毅宗〕
% (年) 辛亥
% (月) 庚寅
% (H) 乙未
% (时) 己卯
% 明毅宗即明朝末年自缢煤山的崇祯皇帝。命中 日干乙
% 木,生于寅月,可谓得令,故月干庚金作为用神。由于其金得
% 柱中己土、未土相生,所以格局清奇,承膺大宝。可惜年干辛
% 金,暗引寅中丙火,化水泄金,地支会成东方木局,而木又能生
% 火制金,所以一旦行运进入丙火,岁值壬午,年干壬和大运丙
% 交战,地支午和月支寅化火,便就难逃厄运了。
% 〔清圣祖〕
% (年) 甲午
% (月) 戊辰
% (H) 戊申
% (时〉 丁巳
% 清圣祖爱新觉罗 •玄坤(公元 1654-1722 年),八岁登皇
% 帝位,年号康熙,在我国历史上是个有所作为的皇帝。柱中日
% 干戊土,月令戊辰,归棣于巳,故取金、水、木作为用神,忌火忌
% •299
% 土。然而康熙晚年,却死于亥运。亥为水,何以不吉?袁树珊
% 《命谱》对此解释为「至六十九岁乃行亥运,岁值壬寅,十一月
% 十三日戌刻,崩于畅春园者,此即刘注所谓 '坐申怕寅',不仅
% 日主戊申,与太岁壬寅,干克支冲也」
% 〔罗聘〕
% (年) 癸丑
% (月) 甲寅
% 己丑
% (时) 甲子
% 罗聘为清朝扬州画派八怪之一。观其四柱,日时天干甲
% 己相合,地支子丑相合,当为“夫妇聚会”的贵格,然而罗聘生
% 前卖画,穷愁潦倒,并不得志。为此,陈小翠分析其命造为:
% “化土悬木,命宫甲寅及月支之寅,皆足为害,非真化也。逢申
% 而伤官见官,其死宜矣。”
% 〔黄景仁〕
% (年) 己巳
% (月) 丙寅
% (日) 癸丑
% (时) 戊午
% 黄景仁为清朝乾隆年间著名诗人,死时仅三十五岁,有
% 《两当轩集》传于世。他的命造,陈小翠女士批为,“化火伤官
% 格,弱不胜财官。化火格,行运逢水,一生不遇。”“亥运马冲巳
% 终。”
% 〔曾国藩:j
% (年) 辛未
% (月) 己亥
% • 300 •
% (日) 丙辰
% (时) 己亥
% 曾国藩(公元 1811 1872 年),字涤生,湖南湘乡人。在
% 仕途上,他权缩四省、位列三公,拜相封侯,谥称“文正”,可谓
% 位极人臣。关于他的命造,近代才女陈小翠批为「火孤立,仗
% 命宫甲午为助;弱有印,为中和。七杀有制,有刃,为伤官用
% 印。癸运伤官见官,终壬申流年,七杀长生,己不能胜矣,故
% 终。”
% 〔李鸿章〕
% (年) 癸未
% (月) 甲寅
% (日) 乙亥
% (时) 己卯
% 李鸿章(公元 1823 1901年),字少荃,安徽合肥人。道
% 光进士,洋务运动的积极倡导者和推行者。日干乙木,归禄时
% 支,地支亥卯未合成木局。古歌说「甲乙生人寅卯辰,又名仁
% 寿两堪评,亥卯未全嫌白帝,若逢坎位必身荣。” 虽说一生荣
% 宠,但也劳心劳神,时处逆境,死于七十九岁丙运。袁树珊评
% 其死于七十九岁的原因为「七十九岁丙运,岁 值辛丑,竟翦于
% 京师贤良寺者,此乃辛乙交战,丑未相冲之故,与丙运无涉
% 也」但陈小翠女士则另有看法「丙是伤官,木最忌出火,力寿
% 终。辛又正官(当为偏官之误〉,此是伤官见官,大祸也。”
% 〔康有为〕
% (年〉 戊午
% (月) 乙卯
% (H) 壬子
% •301 •
% (时) 庚子
% 康有为(公元1858 1927 年),原名祖诒,字广厦,号长
% 素,广东南海人。平生先为改良派,后又为保皇派。一生著述
% 很多,才思喷涌,为人主观执着,这和他八字中水木清华的格
% 局,或许有关。对于这一命造,近代命理学家陈小翠批为「身
% 弱有印,绝顶聪明。飞天禄马。岁丁卯,红炉火,辛亥所最忌,
% 终。”又批为「柔金畏木
% 〔端方〕
% (年) 辛酉
% (月) 壬辰
% (日) 己亥
% (时) 壬申
% 端方(公元 1861-1911 年),清末满洲正白旗人。慈禧太
% 后宠臣,两江总督,又移督直隶。1911 年起用为川权铁路大
% 臣,后在资州(今资中)被起义新军所杀。陈小翠批曰「缺印
% 身弱,大水崩堤?此颇可作从财格,丁火(运)生印则从财不
% 成,故凶。”又说「亥运终,辛亥年亥月,才官太旺,身弱不胜其
% 任,故凶。况有官无印,即失其主权,能不为下所欺乎?”
% 〔孙中山〕
% (年) 乙丑
% (月) 丁亥
% (0) 丁酉
% (时) 壬寅
% 孙中山(公元1866—1925年),名文,字逸仙,广东香山人。
% 孙中山先曾做过医生,后来毅然弃医从政,为推翻清王朝的封
% 建统治,作出了卓越的贡献,从而成为我国近代史上闻名遛迩
% •302 •
的革命先驱,世称“国父”。结合命造,丁火生于亥月,虽然失
令,但有二印一比肩生扶,所以综合分析,属于日主偏弱的命
造。弱者宜扶,权衡利弊,用神当取月干比肩,年干偏印,加之
日时天干丁壬作合化水,地支寅中伤官制服官星,不致为害,
有水、木、火相继而生之妙,所以必定大有作为。再看大运,喜
水、喜木、喜火,忌金、忌土。民国十四年乙丑为中山先生忌年,
主要是因为大运在巳,巳酉丑三合金局,巳运又冲提纲亥水,
故而不禄。
〔胡适〕
(年) 辛卯
(月) 庚子
(日) 丁丑
(时) 丁未
胡适(公元1891-1962年),字适之,安徽绩溪人。早年留
学美国,归国后任北京大学校长。我国新文化运动的著名骁
将。在学术上,他那“大胆假设,小心求证”的研究方法,至今
在学术界仍有较大影响。解放前赴美国,最后病死台湾。
〔释虚云〕
(年) 庚子
(月) 甲申
丁巳
(时) 壬寅
虚云(公元1840-1959年),俗姓萧,举德清;四十岁后在
福州鼓山涌泉寺剃度出家。后来遍游名山大刹,广宏佛法,足
迹所到,远及西藏、缅甸、锡兰、泰国、槟榔屿等地。此后,他
非但重兴鸡足山迎祥寺,并且创立鼓山佛学院,撰《楞严经玄
•303*
要》、《法华经略疏》、《遗教经注释》、《圆觉经玄义》、《心经释》
等,在国内外佛学界享有崇高的威望。1953 年,被推举为中国
佛教协会名誉会长。1959 年,病逝于江西云居山真如寺。

\section{旧时星命家的道德要求}

旧时代虽然从上到下,风行星相占卜之术,社会上以此作
为职业的,不乏其人,可是对于这些专职星命家来说,却也历
来相沿,有着一定的职业道德的要求。这种有关星命家的职
业道德,袁树珊先生把它归纳为《星家十要》。
当年,袁树珊父亲在课读之暇,曾把医卜星相等学传授给
袁树珊说「读书人士途通达,虽然可以列身庙堂,为苍生造
福,士途不达,也可凭着一技之长立身社会。汉朝的贾谊曾经
说过「古之圣人,不居朝廷,必在医卜之中。' 原因是卜可决
疑,医可治病,同为人生日常所需。”
后来,袁树珊未能通过读书跻身士途,以医卜而混迹世
间。平时,每当他读到古代名医陈实功的《医家十要》,张路玉
的《医家十戒》,总是禁不住叹服古人存心之厚,立论之高,于
是便就依样葫芦,仿照其例,著为《星家十要》一篇。篇中宗
旨,大抵不外汉朝大占卜家司马季主与臣言忠,对子说孝之
意。
《星家十要》的要领为:
〔学问〕
长安赵展如中丞在《子平真诠》的序言中说「星命虽为小
• 304 •
道,而所系大焉。近世术士,为糊口计,莫能深究其理,故学术
多不精,学术不精则信者寡,信者寡则非分之营求愈炽,而安
命者愈希,君子忧之。”
从赵展如前言的这段话里,可知学问之道,贵在深究其
理。要深究其理,就非得多读书不可。对于星命家如何读书,
以及读书的作用,袁树珊归结为「(星命家)不仅宜多读星命
书,凡经、史、子、集,有关于星命学者,亦宜选读,既增学问,又
益身心。用之行道,则吉凶瞭然,批谈不俗;用之律己,则行藏
合理,人格自高。有心斯道者,首当如此。”
于此可见,一个学者型的星命家,和混迹江湖,混口饭吃
的江湖派术士,两者对于学问上的探讨和人品高低,竟是有着
多么的不同。
〔常变〕
赵展如中丞曾经说过「禄命之说,未必尽验,然验者常十
之七八。其或因山川风土而小异,由门第世德而悬殊。又一
行之善恶,一时之殃祥,忽焉转移于不知,此则常变之不同,造
化之不测也。要其常理,自不能废,而常人多不能逃。”
从赵展如的话里可以看出,星命家为人推命,常有不验的
地方,原因还是在于“常”和“变”的不同。由于“常”、“变”不同
的客观存在,所以平时星命家如果总是用常法去套,怎么能够
不差之毫厘,失之千里呢?
为此,袁树珊指出「为星家者,欲求事功圆满,万无一差,
必须参以人情物理,询其山川风土,门第世德,以及生时之风
雨晦明,而尤须鉴别其心术之善恶,处世之殃祥,然后定其富
贵贫贱,寿夭穷通,乃可合法。”
其实,从客观上说,袁树珊的出发点再好,再结合命理学
•305
之外的种种人情物理,可是要求其事的圆满成功,万无一失,
也还是无法做到的。不过作为一种推命的补充,袁树珊的看
法还是十分高明的。
〔言语:)
儒家祖师爷孔子指出,做人要「敏于事而慎于言。”西方
哲人苏格拉底也说:“天赋人以两耳两目一口,使人多闻多见
而少言语。”这是说,先哲垂教,重视言语上的谨慎,从而减少
人们在言语所可能导致的过失。
然而,世界上的任何事情,总是一分为二的。比如滔滔雄
辩,也是社会上不可缺少的一个重要方面,所以社会上教育家
的登台讲授,演说家的当众演说,如果缺少了滔滔雄辩的一
面,那末他还能获得授课或演讲上的成功吗?
不管慎言也好,雄辩也好,不过对于星命家来说,袁树珊
认为,最主要的,还是在于「言语之道,宜忠实,忌阿谀,宜雅
驯,忌卑劣I宜从容,忌躁急。至于繁简得当,巨微检点,尤为
要诀J
当然,这又牵涉到星命家平时的品格修养了。
〔敦品〕
儒家先师孔子要求人们「非礼勿视,非礼勿听,非礼勿
言,非礼勿动J
以上视、听、言、动,随处都可表现出一个人的心术邪正,
品行贤愚来,所以为人应世,如果不在这上面下点检点功夫,
处处用礼来加以规范约束,那末即使你平时的穿着再华丽,家
里的陈设再精致,也还是免不了要被那些明达君子所看轻。
为此之故,袁树珊提醒「故吾人欲知敦品,当以视、听、
言、动为本,衣服、陈设为末。苟能如是,则信用远孚,声名振
• 306 •
大,有不期然者。此固尽人当知之理,为星家者尤宜注意。”
不但命星家要以视、听、言、动为本,社会上其他各行各
业,又何尝不要以视、听、言、动为本!如若人们处处能以礼来
约束检点每个人自己的一言一行,那就精神文明,蔚然成风
了。
〔廉洁〕
廉洁是历史上古往今来,多么为人们所向往、歌颂的美
德!当年孔子就曾说过「见利思义。”《曲礼》也说「临财毋苟
得。”所谓“义”和“苟得”的分界在于:“顺取之财为义,逆取之
财为苟得。”
对于顺取、逆取,以及廉洁之道,袁树珊由星命家而推及
世人,精辟地概括为「尽我所长,忠告善道,而得报酬者,此为
顺取;苟且塞责,伪言欺人,而得报酬者,此为逆取。顺取者数
虽巨,亦不失其廉洁;逆取者数虽少,亦不免为贪污(和现在所
说的贪污不同〉。吾人处世行道,当以廉洁自勉,而以贪污为
戒J
星命家如果一味贪污逆取,那就道德败坏,有亏于世道人
心了。
〔劝勉〕
《史记 • 日者列传》载司马季主对宋忠、贾谊的一段话说:
“贤之行也,直道以正谏,三谏不听则退。其誉人也不望其报,
恶人也不顾其怨,以便国家利众为务。故官非其任不处也,见
人不正,虽贵不敬也;见人有污,虽尊不下也;得不为喜,去不
为恨;非其罪也,虽累辱而不愧也。”又说:“且夫卜筮者,扫除
设坐,正其冠带,然后乃言事,此有礼也。言而鬼神或以饷,忠
臣以事其上,孝子以养其亲,慈父以畜其子,此有德者也。”
•307
作为一个星命家来说,袁树珊要求对于各类前来预测,身
份各不相同的求问之士,做好思想上的劝勉工作。他说:
故为政客言,当勉以忠君爱民,显祖流芳,如杨
椒山诗云“男儿欲绘凌烟阁,第一功名不爱钱”之类。
为刑官言,当 勉以虚心听讼,勿 逞意气,如《书》云“罪
疑惟轻,功疑惟重,与其杀不辜,宁失不经”,欧阳修
《淀冈忏表》云“求其生而不得,则死者与我皆无恨”
之类。为 武员言,当 勉以身先士卒,捍卫国家,如曾
子云“战阵无勇,非孝也”,马援云'效命疆场,男儿幸
事”之类。为有老亲者言,当勉以色养无违,如孟郊
诗云“谁言寸草心,报得三春晖”,古诗云“万恶淫为
首,百行孝为先”之 类。为有幼子者云,当劝其教养
兼施,如古人云“子孙虽贤,不宜 溺爱,子孙虽愚,亦
贵读书”之 类。至于为富贵者宜勉其学宽,为 晦明者
宜劝其学厚,为士者宜劝其敦品劝学,为农者宜劝其
尽力田畴,为 工者宜劝其专心技艺,为 商者宜劝其诚
信无欺。此皆星命家应尽之天职,不可不知。
〔警励〕
宁陵吕叔简说「奔走营运则生活,安逸惰慢则死亡。”一
个人生活在世上,要生存,要活命,就得奔走营运;要死寂,要
灭亡,就听任安逸惜慢。由此,奔走营运为生存、活命的兴奋
剂,安逸惰慢为死寂、灭亡的催化剂。这是不言而喻的,而吕
叔简这话的要旨,也正是在于这里。
面对这种情况,所以袁树珊认为「盖生活为万事之根本,
•308-
人无生活则不能仰事父母,俯畜妻子,而亦不能保命也。”为
此,袁树珊并为星命家提出要求「故凡为失业之人推命,务劝
其弃大就小,自营生活,尤须以先哲格言,求人不如求己,能屈
始可能伸之意,反复开导之,万不可使其因循坐误,年复一年,
致蹈闲居丧家之覆辙。古人云「当局者迷,旁观者清。' 为星
家者,能不尽力以警励之乎?”
〔治生〕
《孟子 • 梁惠王上》说:“无恒产而有恒心者,唯士为能。若
民,则无恒产,因无恒心。苟无恒心,放俯邪侈,无不为已。”书
中所说的“恒心”,就是人所常有的善良本心。《史记 • 管晏列
传》也引管仲的话说:“仓以实而知礼节,衣食足而知荣辱。”如
此说来,一个人的经济情况,往往对于他的行为,有着一定的
决定作用。
关于这方面,袁树珊为星命家提出了这样的要求 故凡
为人推命,当呢其于得意时拼节用度,力戒奢侈,以有余之资,
多置恒产,免致失意时一无凭籍而贻悔无穷。至于为纨挎子
弟推命,又当劝其保守旧业,毋求急功,以勉失败。此为星家
必要之议论,不可不知J
〔济贫〕
古人说过,“一言兴邦”,“一言丧邦”,可见言语作用之大。
懂得了这个道理,所以凡是星命家为贫困难堪之人推命,即使
料定他一生真的没有好运,也不可直争谈相,以免断了他的希
望之路。这时应当婉言开导:“大富由 小康由勤。”并对他
指出,你如果能够勤勉职业,好好地干,并且在开支上省减消
费,将来又在某个阶段得到运的帮助,当也不难发达0
对于以上这种做法,并非滑头欺骗,而是对失意者的一种
• 309 •
鞭策和激励,所以袁树珊谆谆告诫「此非虚伪阿谀,盖不如是
不足以保其生命也。至于润笔,务宜璧谢。为星家者,不能以
金钱济贫,已属憾事,又岂能吝此区区智识者。”
我国古代星命之士,为贫困之人推命往往璧还润资,大概
也可算作是个传统了。
(:节义〕
先哲有云「富不易妻。”古代名士宋弘也说「贫贱之交不
可忘,糟糠之妻不下堂。”社会上有些人,往往稍一发迹,便就
摒弃原配夫人,纳妾狎姬,无所不至,当然这是一种应该受到
谴责的不义行为。所以,星命家如果碰到这些人来预测命运,
袁树珊的态度是:“凡为此等人推命,务以婉词劝之,使其琴瑟
调和,俾免家庭恶感。此星家应有之言论,亦大圣与人为善之
微意也。”
此外,对于当时社会上因为没有子嗣而纳妾,以及寡妇改
嫁,袁树珊也有着自己的看法「若为育子而欲纳妾者,又当劝
其慎择。至于孀妇改嫁,当察其贫富、,及有无子息以为断。若
家贫而无子者,既无赡养,又无希望,不得已而再醮,姑置不
论。若有子息,虽家贫亦当劝其含辛茹苦,抚孤守节。若家道
饶余,即无子息,亦当劝其早立嗣,固守节操,且可以由之节妇
而得青史留芳,彤管扬休者,为之模范,使之坚定不移,而成美
德。此为星家应尽之天职,亦维持风化之一端也。”
这里,袁树珊的时代局限是明显的,即使愿望再好,也只
能是为封建说教摇旗呐喊。不过话虽如此,人们若能对此加
以分析批判,则还是能够在去粗取精的基础上,发现其思想上
所确实存在着一些闪光点的。
至于近来港台一带,星相之术盛行,人们往往趋之若鹫,
•310・
使星相家的职业大为红火。不过,他们那里对于正宗开馆的
星相家,往往也常提出一定的职业道德要求。比如在台湾凤
山市中兴街 69 号开凤山大学士命相馆的郑景峰先生,于1984
年夏在为《卜筮正宗》作新式标点时,就在书前写下了五条《星
相预言家守则》:
要有高尚的人格
不欺骗客人,不恶意攻击其他星相界同行。
要有高深的学问
没有学历的限制,但必须博通五术,有真正的才学。
要有宗教家的慈悲
不可敛财骗色,必须本着慈悲心劝人行善布施。收费不
宜太高,贫困的酌予免费。
要有心理医生的水准
聆听客人的问题,指点可行之道,帮助解决困难。
要有地上神仙的风骨
有缘才可渡,不必勉强无缘客,用不着为五斗米而折腰。
当然,社会上以骗钱敛财为目的的江湖术士比比皆是,根
本不理睬《星相预言家守则》,但毕竟这是问题的另一方面,我
们并不因此就抹煞了正统星相家,确是有着职业道德以作为
约束个人言行,与人为善的一面。
•311 •
