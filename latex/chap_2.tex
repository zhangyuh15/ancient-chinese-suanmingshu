\chapter{算命术的基础理论}
\section{天干地支}
李虚中、徐子平创立算命术以后,算命的都少不了要和天
干地支打交道,否则便就寸步难行。
“天干地支”,简称“干支”,又称“干枝”。前人有云:“夫
干,犹木之干,强而为阳;支,犹木之枝,弱而为阴。”可见称为
干支的原始用意。
天干的数目有十位,它们的依次顺序是,甲、乙、丙、丁、
戊、己、庚、辛、壬、癸。
地支的数目有十二位,它们的依次顺序是:子、丑 、寅、gu、
辰、巳、午、未 、申、酉、戌 、亥。
因为天干地支原是取意于树木,所以,对于它们的原始意
义,有这样有趣的说法:
1. 天干
〔甲〕 象草木破土而萌,阳在内而被阴包裹。
〔乙〕 草木初生,枝叶柔软屈曲。
(:丙】 丙,炳也,如赫赫太阳,炎炎火光,万物皆炳然著见
• 35 •
而明。
CTD 草木成长壮实,好比人的成丁。
〔戊〕 茂也,象征大地草木茂盛。
(己〕 起也,纪也,万物抑屈而起,有形可纪。
〔庚〕 更也,秋收而待来春。
〔辛〕 金味辛,物成而后有味。又有认为,辛者新也,万
物肃然更改,秀实新成。
〔壬〕 妊也,阳气潜伏地中,万物怀妊。
〔癸〕 揆也,万物闭藏,怀妊地下,揆然萌芽。
2. 地支
C子J 孳也,草木种子,吸土中水分而出,为一阳萌生的
开始。 '
〔丑〕 草木在土中出芽,屈曲着将要冒出地面。
〔寅〕演也,津也,寒土中屈曲的草木,迎着春阳从地面
伸展。
茂也,日照东方,万物滋茂。
〔辰】 震也,万物震 长,阳气生发已经过半。
〔巳〕 起也,万物盛长而起,阴气消尽,纯阳无阴。
〔午〕 万物丰满长大,阳气充盛,阴气开始萌生。
(:未〕味也,果实成熟而有滋味C
〔申〕身也,物体都已长成。
(:酉〕维也,万物到这时都缩缩收敛。
【戌 灭也,草木凋零,生气灭绝。
〔亥〕劾也,阴气劾杀万物,到此已达极点。
据说,对•于这种有趣的天干地支,发明者是四五千年前上
古轩辕时期的大挠氏。起先,夭干仅是用来记日,因为每个月
• 36 •
的天数都是以十进位的;地支用来记月,因为一年十二个月,
正好用十二地支来相配。可是随之不久,人们感到单用天干
记日,每个月里仍然会有三天同一天干,所以便用一个天干和
一个地支分别依次搭配起来的办法来记日期,如《尚书 •顾
命》就有“惟四月哉生魄,王不怪。甲子,王乃洪庙水,相被冕
服,凭玉几"的记载,这用现在的话来说,就是四月初,王的身
体很不舒服。甲子这一天,王才沐发洗脸,太仆为王穿上礼
服,王依在玉几上坐着。后来,干支记日的办法又被渐渐引进
到了记年、记月和记时。这样,干支记年、记月 、记日、记时的
一整套体系就在实践过程中,渐次地形成了。
十天干和十二地支的最小公倍数是六十,所以它们依次
从头结合到底一个循环,通称“六十甲子”。“六十甲子”的次
第是:
甲子
己巳
甲戌
己卯
甲申
己丑
甲午 ©
己亥 ®
甲辰 @
己酉 ©
甲寅 ®
己未
乙丑 丙寅
庚午 辛未
乙亥 丙子
庚辰 辛巳
乙酉 丙戌
庚寅 辛卯
乙未 @ 丙申
庚子 够 辛丑
乙巳 命 丙午
庚戌 @ 辛亥
乙卯 匈 丙辰
庚申 @ 辛酉
丁卯 戊辰
壬午 癸未
丁酉 颈 戊戌
壬寅® 癸卯
领戊申
领丁巳
壬戌 @ 癸亥

以上"六十甲子" 每个单位都可按照先后顺序分别代表
• 37 •
不同的年、月 、日、时。比如以日为例,清代道光二十二年壬寅
(1842)农历四月十一日是己亥,那末四月十二日、十三日就可
顺次推知为庚子、辛丑……,四月十日、九日就可逆次推知为
戊戌、丁酉……。这样六十甲子循环往复,周而复始,以至无穷。
早在春秋战国时期,历法混乱,夏历、殷历和周历同时并
存。三种历法之间的主要区别是每年开头的月建不同。秦始
皇统一中国后,曾改用夏历建亥之月的十月为一年的开头。此
后直到公元前 104 年汉武帝改用太阳历,才正式确定以夏历
建寅之月的正月,作为一年的开头。打这以后的二千年间,除
了王莽、魏明帝时一度改用殷历,唐代武则天和唐肃宗一度改
用周历,一般都以夏历的寅月作为一年的起始。现将夏历建
寅之月作为岁首的月份和地支对照名称列表如下,

【38页表】

记月之外,古人还用十二地支记时,这样一个昼夜下来,
就是十二个时辰。这用现在的时间概念来说,每个时辰恰好
等于两个小时。所谓“小时”,就是“小时辰”,也就是“半个时
辰”的意思。时辰和小时的对照情况,仍用清晰的表格形式反
映如下:

【38页表】

表里的俗称,是指十二时辰在古代的一种通俗叫法。这
种俗称,主要是借助一些自然特征和生物特征来表示的。“鸡
鸣”、“人定”,借助于半夜鸡叫和人入夜睡觉的特征。“食时”、
“晡时”,借助吃饭时刻表示时间。古人一日两餐,早饭在日出
以后,隅中以前,所以称这段时间为"食时、晚饭在日跌(太阳
偏西)以后,日入以前,所以称这段时间为“晡时 除此之外,
其余留下的八个时间俗称,多半以太阳位置为主要特征来表
示的。这里要注意的是对于子时的划分,因为上半时在夜半
前,所以属上一天,下半时在夜半后,所以属下一天。
在十二时辰化为现代时间上,如果碰上夏时制,我们可以
作相应的调整,如原来的寅时是凌晨 3 点到 5点,夏时制则可
顺推为 4 点到 6 点。其他时辰也按照这个规律类推。
值得一提的是,用十二地支所记月、时是固定不变的。如
子月必定是十一月,子时必定是夜半 23 到 1点,然而与之相
配的天干却不是一成不变的,它们循环往复的顺序是如前所
说的.六十甲子,
在年份上,满六十甲子称为一个花甲。人们平时常说“年
逾花甲”,就是超过六十岁的意思。记年时,满一个甲子后再
从头算起,也和记日一样,周而复始,如环无端。如咸丰十年
(I860)是庚申年,那末咸丰十一年顺推就是辛酉年,咸丰九年
倒推就是己未年。隔六5年后又从庚申开始,再依次记叙下
去。
干支记年的办法和现在记年的办法比较起来,虽然笨拙
得多,但它在我国历史上,却差不多一直沿用到清朝灭亡以前
的整个历史时期。

\section{阴阳五行}
我国古代,阴阳五行是个哲学概念。用这个概念,可以概
括天地自然和人类社会的一切。
《易 •系辞上》说:“易有太极,是生两仪,两仪生四象,四
象生八卦。”这两仪就有阴阳的含义在内。早先,阴阳只是作为
太阳日光向背的意义而出现的,向日的叫阳,背日的叫阴。不
久,又引申解释为气候的寒与暖。后来随着人们认识的不断提
高,就把阴阳作为一种哲学概念,用来广泛解释自然界和人类
社会两种互相对立消长,矛盾而又统一着的动态平衡势力。比
如日月、昼夜、明暗、动静、内外、寒热、雌雄、男女、刚柔、迟速等
等,都可分成阴阳两个方面,然而这两个方面又是协调统一,相
反相成的。因为有着这个原因,所以《易传》有“一阴一阳之谓
道”的说法。这个道,指的就是天地自然变化发展的基本规律。
也正因为阴阳广泛包涵着事物对立统一的两个方面,因
此反过来说,世界上的任何事物也都可分阴阳两个方面,这说
明阴阳这种现象是无所不在的。就举一本书为例吧。书本的
封面是阳,背面是阴,书本的表面是阳,里面是阴。如果打开
书本,暴露在光下的内页是阳,翻到背面的封面又变成了阴。
这又进一步说明,阴阳也并不是一成不变的,它可以随着外界
条件的转化而转化,所以《老子》说「万物负阴而抱阳J
当然,如果再深入一层,天地万物的阴面可以包涵着阳,
夭地万物的阳面同样也可包涵着阴。《素问 • 金匾真言论》有
• 40 •
言「平旦至日中,天之阳,阳中之阳也;日中至黄昏,天之阳,
阳中之阴也;合夜至鸡鸣,天之阴,阴中之阴也;鸡鸣至平旦,
天之阴,阴中之阳也广
这种阴阳的概念早先原本是朴素的,唯物的。到了战国
末期,由于以邹衍为代表的阴阳家“乃深观阴阳消息,而作怪
迂之变”,也就从这以后,本来质朴无华的“阴阳”,多少地给抹
上了一重神秘的油彩。
同样如此,五行的早期也是朴素而又唯物的。《尚书 • 洪
范》曾这样记载说「五行,一曰水,二曰火,三曰木,四曰金,五
曰土。水曰润下,火曰炎上,木曰曲直,金曰从革(顺从人的要
求变革形状),土爰稼稿(指庄稼 洞下作咸,炎上作苦,曲直
作酸,从革作辛,稼稽作甘。”古人认为,天地万物都是由金、
木、水、火、土等五种基本物质组成的。由于这五种基本物质
的运动变化,从而构成了丰富多彩的物质世界。
战国时期,“五行”学说很是风行一时,并且还进一步总结
摸索出了一套“五行相生相胜”的原理。所谓“相生”,就是一种
物质对另一物质有着生发促进的作用,1如木能生火便是I所谓
“相胜”,也就是“相克”的意思,就是一种物质对另一种物质有 - 着克制约束的作用,如水能克火便是。正因为广泛存在在自
俄界的五行有着这种相生相胜的相互作用,所以天地万物才
得到了动态的平衡。否则只生不克,或者只克不生,想要这天
地万物维持下去,简直是不可想象的事。
对于这种五行生克的规律,古人的说法是:
1. 相生:木生火,火生土,土生金,金生水,水生木。
2. 相克:木克土,土克水,水克火,火克金,金克木。
右面这张附图,就是根据这木火土金水的五行生克关系

· 41 ·

【42页图】

绘制出来的。口诀是「顺次相
生,隔一相克J
为什么顺次相 生呢?《命
理探原》的解释是「木生火者,
木性温暖,火伏其中,钻灼而
生,故木生火;火生土者,火热
故能焚木,木焚而成灰,灰即土
也,故火生土;土生金者,金居
石依山,津润而生,聚土成山,
土必生石,故土生金;金生水者,少阴之气温润流泽,销金亦为
水,故金生水;水生木者,因水润而能出,故水生木也J
为什么又隔一相克呢?《白虎通义》的认识是 “五行所以
相害(相克)者,天地之性,众胜寡,故水胜(克)火也;精胜坚,
故火胜金;刚胜柔,故金胜木;专胜散,故木胜土;实胜虚,故土
胜水也
后来,随着唯心主义思想家的出现,尤其是命理学家的出
现,五行也和阴阳一贤,敲披上了一件眩人眼目的神秘外衣,
变得难以捉摸了。

\section{天干地支和阴阳五行的配合}

命理学家认为,天地万物的发展变化既然和阴阳五行的
变化土克不着密不可分的联系,那末“人身一小天地”,通过对
一个人出生年二 日时干支所涵阴阳五行不同变化的推测,不
• 42 '
就可以推知他一生的吉凶祸福了吗?
对此,王充《论衡 • 初禀》篇说「人生性命当富贵者,初禀
自然之气,养育长大,富贵之命效矣 命谓初所禀得而生也。
人生受性则受命矣,性 命俱禀,同时并得,非先禀性,后乃受命
也。”接着他还举例说「文王在母身之中已受命也。”认为一个
人的富贵贫贱,早在父母交合之时就已决定了的,不管将来长
大后操行如何,都没有办法改变。然而,王充的这种说法,给
后世的命理学家推命带来了莫大的困难,虽然元代孔齐《至正
直记》曾有过这样的记载「前辈多言推人五行定休咎,今以受
胎日时为准,但以所生时甲子合,得十月数某甲子是也。如甲
子则推己丑(原注,甲与己合,子与丑合),乙丑则庚子之类(原
注:乙与庚合,子与丑合)也。又云,唐宫如此。未详。”可是这是
在十月怀胎正常分娩情况下所作的一种推算受胎时间法,如
若逢上早产、晚产,那就只好束手无策了。
《古今名人命鉴》是民国命理学家东海乐吾三十年代的著
作,我们再来看看他的说法。在自序中,他是这样分析的「佛
言四大(地水风火),儒言五行,人之一身由四大和合而成,亦
即五行秉赋而成也。光热为火,润泽为水,流动为风,质实为
地。而儒家五行之分类,除水火相同外,金属为金,纤维质为
木,不属于金木之质为土,故土又名杂气,此与今之科学家人
体物质之分析,固有不谋而合者也。”接着他笔锋一转,拦入正
题,“人之秉受不同,其原因固安在乎?曰由于感受太阳之光
线、星球之吸力随时有不同也。春之气和煦,秋之气肃杀,夏
热而冬寒,此显而易见者J所以“凡人脱离母体之时”,“得气
之厚,神守气足则寿I 得气之强,体大用宏则贵。反是则不永
其年,或所为辄阻,贫贱夭折,必居其一」显然,这里作者认
• 43 •
为,只有五代徐子平根据出生年月日时所立的“四柱”推算法,
才是符合科学而合理可行的。
那末,“凡人脱离母体之时”的年、月、日、时的天干地支又
是怎样和阴阳五行配合挂钩的呢?

先说阴阳。干支和阴阳的配合比较简单,不比干支和五
行配合那样来得复杂多变。具体划分是,根据阳数奇(单)而
阴数偶(双)的原则,十个天干和十二地支里,凡是逢单的都属
阳,逢双的都属阴。这可用下面的图表加以表示,

【44页表】

再说五行。在十天干中,五行的分配是甲、乙属木,丙、丁属
火,戊、己属土,庚、辛属金,壬、癸属水。在十二地支中,五行的
分配是寅、卯、辰属木,巳、午、未属火,辰、戌、丑 、未属土,申、
酉、戌属金,亥、子、丑属水。现列表如下,看起来可方便清楚些。

【44页表】

这里,天干的五行要比地支的强些,再加上干支的阴阳不
同,因此同样是木,可又是不全相同的。比如天干的甲乙木和
地支的寅卯木不同,而同是天干的甲乙木,因为甲是阳木,属
于栋梁之木,乙是阴木,属于花果之木,所以也有着一定的区
别。对于这种五行在天干上的阴阳大小不同,古人大致有这
样一种说法:
甲木 栋梁之木。
乙木 花果之木。
丙火——太阳之火。
丁火
——灯烛之火。
戊土——城墙之土。
己土
——田园之土。
庚金——斧钺之金。
辛金 首饰之金。
壬水
——江河之水。
癸水 雨露之水。

然而对于这种说法,清代命理学象任铁樵却力辟其谬说:
“夫十干之气,:以先天言之,故一原有出;以后天言之,亦一气
相包。甲乙一木也,丙丁一火也,戊己一土也,庚辛一金也,壬
癸一水也。即分别所用,不过阳刚阴柔,阳健阴顺而已,朝怪
命家作为歌赋,比拟失伦,竟以甲木为梁栋,乙木为花果,丙作
太阳,丁作灯烛,戊作城墙,己作田园,庚作顽铁,辛作珠玉,壬
作江河,癸为雨露,相沿已久,牢不可破。用之论命,诚大谬
也。如谓甲为无根死木,乙为有根活木,同是木而分生死,岂
阳木独禀死气,阴木独禀生气乎。又谓活木畏水泛,死木不畏
水泛,岂活木遇水且漂,而枯槎遇水反定乎。论断诸干,如此
» 45 •
之类,不一而足,当尽辟之,以绝将来之谬J
从地支来说,寅、卯 、辰虽说同样是木,但寅是初生之木,
卯是极盛之木,辰是渐衰之木。同样,从火来说,巳是初生之
火,午是极盛之火,未是渐衰之火;从金来说,申是初生之金,
酉是极盛之金,戌是渐衰之金;从水来说,亥是初生之水,子是
极盛之水,丑是渐衰之水。至于辰、戌、丑、未四季,非但有属
土之称,并且另外还有着四库的说法。其中“丑为金库,生亥
子而克寅卯I辰为水库,生寅卯而克巳午;未为木库,生巳午而
受金克1戌为火库,克申金而受水制”(《三命通会》卷五)。 正
因为这样,辰、戌、丑、未又称杂气。关于'寄旺于四季”,是指
土寄狂于季春、季夏、季秋、季冬一年四季的最后一个月说的,
也就是说,土寄旺于春天的三月,夏天的六月,秋天的九月.冬
天的十二月。
比较复杂的是,地支的五行不象天干那样,甲木就是甲
木 丙火就是丙火,而是在一定程度上除了本气,还包含着一
个或几个天干的五行成分在内。比如寅支,里面除了含有本
飞天干甲木,还兼有着丙火和戊土的成分在内。所谓本气,就
是每一地支中所藏足以代表自己性质的一个天干。在十二地
支的本气中,寅的本气是甲木,卯的本气是乙木,辰的本气是
戊土,巳的本气是丙火,午的本气是丁火,未的本气是己土,申
的本气是庚金,酉的本气是辛金,戌的本气是戊土,亥的本气
是壬水,子的本气是癸水,丑的本气是己土。对于地支所藏本
气和其他天干,另有古歌一首道:
子 宫癸水在其中,丑癸 辛金己 土同。
寅 宫甲木秉丙戊,卯官 乙木独相逢.
辰 藏乙戍三分癸,巳中庚金丙戊丛。
• 46 •
午 宫丁火并己土,未宫 乙己丁共宗。
申位庚金壬水戊,酉官辛字独丰隆。
戌 宫辛金及丁虎,亥藏壬甲是真踪。
为了清晰起见,现将十二地支所含天干五行列成表格,

【47页表】

除了天干地支和五行的这种正规配角,还有一种把六十
甲子和五音十二律结合起来,其中一律含五音,总数共为六十
的“纳音五行”。对此,古歌有云:
甲子乙丑海中金,丙寅 丁卯沪中火,
成辰己巴大林木,庚午辛未路旁土,
壬申癸酉剑锋金,甲戌 乙亥山头火,
丙子丁丑涧下水,戊寅 己卯城头土,
庚辰辛巳白腊金,壬午 癸未杨柳木,
甲申乙酉泉中水,丙戌 丁亥屋 上主,
戊子己丑霹雳火,庚寅辛卯松柏木,
壬辰癸巳长流水,甲午 乙未沙中金,
丙申丁酉山下火,戊戌 己亥平地木,
庚子辛丑壁上土,壬寅 癸卯金箔金,
甲辰乙巳覆灯火,丙午丁未天 河水,
戍申己酉大驿土,庚戌 辛亥钗钏金,
壬子癸丑桑柘木,甲寅乙卯大 溪水,
• 47 •
丙辰丁巳沙中土,戍午 己未天上火,
庚申辛酉石榴木,壬戌 癸亥大海水。

对于这种“纳音五行”,前人曾解释其搭配原因为:
甲子乙丑海中金
子属水,又为湖,又为水旺之地,加之金死于子,墓于丑,
水旺而金死、墓,所以说是“海中金”。
丙寅丁卯炉中火
寅为三阳,卯为四阴。火既得位,又得寅卯之木以生之,
此时天地开炉,万物始生,所以叫做“炉中火”。
戊辰己巳大林木
辰为原野,巳为六阳。木至六阳则枝叶繁茂,以繁茂之大
林木而生于原野之间,所以称为“大林木工
庚午辛未路旁土
以未中之木,生午未之火,火旺则土于斯而受刑。土之所
生,未能自物,犹路旁土也。
壬申癸酉剑锋金
以申、酉,金之正位,兼临官申,帝旺酉,金既生旺,则诚刚
矣,刚则无逾于剑锋,所以名为“剑锋金”。
甲戌乙亥山头火
以戌、亥为天门,火照天门,其光至高,所以称作“山头
火”。
丙子丁丑涧下水
因为水旺于子,衰于丑,由于由旺而衰,难以成江河,所以
叫做'涧下水”。
戊寅己卯城头土
天干戊、己属土,寅为艮山。土积而为山,所以称为“城头
• 48 •
±Bo
庚辰辛巳白蜡金
金的养地在辰,长生于巳,因其形质初成,未能坚利,所以
取名为“白蜡金”。
壬午癸未杨柳木
木死于午,墓于未,甲木既已死、墓,虽有天干壬、癸之水
相生,终究只能算是柔木,所以说是“杨柳木二
甲申乙酉泉中水
金的临官是申,帝旺是酉,这时金既生旺,水便得力,可是
当此方生之际,水量一时还未洪大,所以只称为是“泉中水”。
丙戌丁亥屋上土
丙、丁属火,戌、亥为天门,火既炎上,那就土从上生,所以
姑称之为“屋上土”。
戊子己丑霹雳火
子属水,丑属土,水居正位,而纳音乃是火水中之火,非龙
神则无,所以叫做“霹雳火”。
庚寅辛卯松柏木
木的临官在寅,帝旺在卯,这时木气生旺,非柔弱之比,所
以名为“松柏木”。
壬辰癸巳长流水
辰为水库,巳为金的长生之地,金能生水,又逢水库,如此
则泉源不绝,所以人称“长流水工
甲午乙未沙中金 ・
午为火旺之地,火旺则金败;未为火衰之地,火衰则金冠
带。败而冠带,金力不足,所以只能称为是“沙中金”。
丙申丁酉山下火
• 49 •
申为地户,酉为日入之门,太阳之火到此而光辉敛藏,所
以叫作“山下火”。
戊戌己亥平地木
戊为原野,亥为生木之地,由于大木生于原野,非一根一
林之比,所以称为“平地木工
庚子辛丑壁上土
丑虽土家正位,而子则水旺之地,土见水多则为泥,所以
只得称为“壁上土二
壬寅癸卯金箔金
寅卯为木旺之地,木旺则金羸,加之金绝于寅,胎于卯,所
以这时金气无力,名之为‘金箔金二
甲辰乙巳覆灯火
辰为食时,巳为隅中,日之将午,艳阳之势光于天下,所以
比之为“覆灯火工
丙午丁未天河水
丙、丁属火,午为火旺之地,然而水从火出,阴阳互根,非
银汉不能及此,所以称作“天河水”。
戊申己酉大驿土
申在八卦为坤,坤为地;酉在八卦为兑,兑为泽。戊、己之
土加到坤、兑之上,非浮薄之土师以同日而语,所以叫做“大驿
土二
庚戌辛亥钗钏金
金遇戌而衰,至亥而病,金既衰、病,当然柔弱,所以名为
.钗钏金”。
壬子癸丑桑柘木
子属水,丑为金库,水方生木,斧金伐之,就好比桑柘方
• 50 •
生,便以戕伐,所以被说成为‘桑柘木二
甲寅乙卯大溪水
寅为冻方,卯为东方正位,川涧池沼,由西而顺流向东,所
以人称“大溪水”。
丙辰丁巳沙中土
丙、丁之火冠带于辰,临官于巳,这时火来生土,但未大
任,所以说成为是*沙中土工
戊午己未天上火
午为火旺之地,未中之火又复生之,火性本属炎上,这时
又逢木库生助,所以喻为“天上火”。
庚申辛酉石榴木
申为七月,酉为八月,这时木气已绝,只有石榴之木结果,
所以名为“石榴木”。
壬戌癸亥大海水
水冠带在戌,临官在亥,这时水力雄厚,非他水可比,所以
称作“大海水二
以上解释虽说粗粗可通,可是不少地方都很勉强,难以深
诘。然而古人既已这样说了,加之后人承袭已久,故而也就约 一
定俗成,墨守成规。

关于正五行和纳音之间的关系,徐子平专用正五行,后来
因为用正五行算常常与实际有所出入,于是便用纳音五行作
为补充。因为这个原因,所以两者之间的关系可以把它看作
是五行为经,纳音为纬。正如《命理探原》所说的那幅 “大概
看日元之强弱,定用神之得失,皆以正五行为主。若欲补偏补
弊,酌盈济虚,又当参看年、月 、日 、时之纳音J
不过,对于纳音五行,旁及一些其他说法,张神峰却独持
• 51 •
批判意见。他说:娄景以炉中、海中、大林、路旁等配纳五行
为歌,使人成诵,后世谓为实然,若《三车一览》、《望斗真经》、
《兰台妙选》等书,俱不论生克正理,漫以江山、水石 、风雨立
说,又以人之生年十二支生肖所属,论人吉凶,尤为谬妄。”
张神峰这种对于纳音五行的批判意见,至今已得到越来
越多有识之士的赞同。


\section{五行和四时五方}
五行观念,是中国算命术中最为关键核心的观念,但它们
在使用时,却并不是孤立的,在很大程度上要和一年的春、夏、
秋、冬四时,以及方位的东、南、西 、北 、中 五方紧密结合起来,
通盘考虑,才能过细入微。
这原因主要是因为,五行在一年四季和方位的定向当中,
各自有着它们所旺的季节和所主的方向,现在我们且看下表:

【52页表】

表里土旺于四季的“四季”,和春夏秋冬的四季解释不同。这
• 52 •
里的“季”,原是“末”的意思。所谓‘四季”,指的就是四个季度
的最后一个月。比如春天三个月,可依次分别称作孟春、仲
春、季春;夏天三个月,可分别称作孟夏、仲夏、季夏;秋天三个
月,可分别称作孟秋、仲秋、季秋, 冬天三个月,可分别称作孟
冬、仲冬、季冬。对于“土旺于四季”,前人又有说成是“土寄旺
于四季”,也就是寄旺于立春、立夏、立秋、立冬“四立”前各 18
天的。如此则春木 72 天,夏火 72 天,四季土 72 天,秋金 72
天,冬水 72 天,平分秋色,合计一年 360 天。
关于五行和四时的联系,前人另有一种十二月支中所藏
天干五行,各分日用事的“司令”说法。所谓“司令”,就是人元
(地支所藏天干五行)当令之气,“乃此月曰之用事神也”。命
理学家认为,月支所藏五行司令用事,便可取用为格,如若所
藏五行不司令的,则不足为重。其司令用事议法为,
〔子〕 从大雪起壬水用事.10 天,接下来癸水用事 20 天
到交小寒止。
〔丑〕 从小寒起癸水用事 9 天,接下来辛金用事 3 天,己
土用事 18 天到交立春止。
(寅〕 从立春起戊土用事 7 天,丙火用事 7 天,甲木用事
16 天到惊蛰止。
〔卯〕‘从惊蛰起甲木用事 10 天,接下来乙木用事 20 天
到交清明止。
〔辰〕 从清明起乙木用事 9 天,癸水用事 3 天,戊土用事
18 夭到交立夏止。
〔巳〕 从立夏起戊土用事 5 天,庚金用事 9 天,丙火用事
16 天到交芒种止。
[午〕 从芒种起丙火用事 10 天,己土用事 9 天,丁火用
• 53 •
事 11 天到交小暑止。
〔未〕 从小暑起丁火用事 9 天,乙木用事 3 天,己土用事
18 天到交立秋止。 '
〔申〕 从立秋起己土用事 7 天,戊土用事 3 天,壬水用事
3 天,庚金用事 17 天到交白露止。
[:酉〕从白露起庚金用事 10 天,辛金用事 20 天到交寒
露止。
〔戌〕 从寒露起辛金用事 9 天,丁火用事 3 天,戊土用事
18 天到交立冬止。
(:亥〕 从立冬起戊土用事 7 天,甲木用事 5 天,壬水用事
18 天到交大雪止。
以上十二月支所藏天干分日司令用事,用歌诀概括起来,
就是:
大雪壬水十日看,廿天癸水逢小寒(子八
小寒九癸兼三辛,己旺十八又立春(丑
立 春成七还丙七,甲 木十六交惊蛰(寅几
惊蛰十日甲木行,余 皆乙木是清明(卯)。
清明乙九三癸寓,虎 土十八到立夏(辰
立夏五戍庚金九,丙火十六及芒种(巳)*
芒种十丙九己取,丁 火十日迎小暑(午)。
小暑九丁乙三日,己 旺十八又立秋(未3
立秋七己兼三成,三 壬交庚十七日(申)。
白露庚金管一旬,辛金二十接寒露(酉3
寒露辛九丁三逢,戊 旺十八又立冬(戌
立冬成七甲五日,壬水十八交大雪(亥斯
前人这种十二月支所藏天干分日司令用事的说法,常为
• 54 •
星命学家看月令时所采用。对于这种相沿已久的说法,清代
陈素庵相国则不以为然,他说「旧书十二月支中所藏诸干,俱
分日用事,相沿既久,遵若金科玉律,但理实不然。推本论之,
寅卯只是甲乙木,巳午只是丙丁火,申酉只是庚辛金,亥子只
是壬癸水,辰戌丑未只是戊己土。若亥有甲,寅有丙,巳有庚,
申有壬,盖木火金水生地之故;未有乙,戌有丁,丑有辛,辰有
癸,盖木火金水墓地之故;辰又有乙,未又有丁,戌又有辛,丑
又有癸,盖木火金水余气之故;寅巳又有戊,午又有己,盖土随
火母生旺之故。总之,但有其气,非盟分诸支之位,而各得若
干日也。唯有其气,故论命者必兼取之,惟不能分其位,故论命
者必以本支为主,而后及其所藏也。”接着又说J再考历法,木
火金水,分旺四时,各七十二日,土旺四季,各十八日。立春日
始,甲木用事三十六日,惊蛰后六日,乙木用事三十六日-,清明
后十二日,戊土用事十八日,余仿此。是则卯月前六日,当用
甲不用乙,辰月前十二日,当用乙不用戊癸,然昔人论命,甲木
生卯月前六日,取卯为刃,不以为本气,生辰月前十二日,先论
季土,次取透干之乙癸,未有竟取乙者,盖既已分建,卯自当从
乙,辰自当从戊,且命法不同历法也。"
“司令”之外,还有一种“进气”“退气”的学说。这种‘进
气”“退气”学说,一般专就月令的“四库”而言。其中辰月金进
气,木退气,未月水进气,火退气;戌月木进气,金退气;丑月火
进气,水退气。为此古歌道:

辰月金进气,癸令(司令)方是真。
未月水进气,乙木是真神。
戌月木进气,最忌值辛金。
丑月火进气,辛癸是忌神。
• 55 •
不可年嵯世沙N进气”等繁琐说法,命理学家又是怎样
在算命前根奖二个人出生时间的五行,和四季结合起来推算
吉凶的呢?这里自有他们的口诀。现把袁树珊《命理探原》引
《穷通宝鉴》所说的那一套,条陈如下:
1. 论四时之木
春月之木,犹有余寒。得火温之,始无盘屈之患,得水
润之,乃有舒畅之美。然水多则木湿,水缺则木枯,必须水火
既济方佳。至于土多则损力堪虞,土薄则丰财可许。如逢金
重,见火无伤;假使木强,得金乃发。
夏月之木,根干叶燥。由曲而直,由屈而,伸。 喜水盛
以润之,忌火炎以焚之。宜薄土不宜厚土,厚则为灾,恶多金
不恶少金,多则受制。若夫重重见木,徒自成林;叠叠逢华,终
无结果。
秋月之木,形渐凋零。初秋则火气犹在,喜水土以资
生,中秋则果实已成,爱刚金以斫削;霜降后不宜水盛,水盛则
木漂,寒露前又宜火炎,火炎则木实。木多则多材之美,土厚
无自立之能。
冬月之木,盘屈在地。欲土多以培养,恐水盛则亡形。
金纵多,克伐无害;火重见,温暖有功。归根复命之时,木病安
能辅助。惟忌死绝,只宜生旺。
2. 论四时之火宜忌
春月之火,母旺子相,势力并行。喜木生扶,不宜过
旺,旺则火炎, 欲水既济,不宜太多,多则火灭。土多则晦,火
旺则亢。见金可以施功,纵叠见富余可望。
夏月之火,势力当权。逢水制,则免自焚之咎I 见木
助,必遭夭折之忧。遇金必发,得土皆良。然金土虽为美利,
• 56 •
无水则金燥土焦。若再火盛,太过必致倾危。
秋月之火,性息体休。得木生,则有复明之庆I 遇水
克,难逃熄灭之灾。土重掩光,金多夺势,火见火以光辉,虽叠
见亦有利。
冬月之火,体绝形亡。喜木生而有救,遇水克以为殃。
欲土制为荣,爱火比为利。见金则难任为财,无金则不遭磨
折。
3. 论四时之土宜忌
春月之土,其势最孤,喜火生扶,忌木克削。喜比助
力,忌水扬波。得金制木为强,金重又盗土气。
夏月之土,其性最燥。得盛水滋润成功,见旺火亢燥
为害。木助火炎,生克不取。金生水泛,财禄有余。见比肩蹇
滞不通,如太过又宜木袭。
秋月之土,子旺母衰。金多则盗泄其气,木盛则制伏
纯良。火重不厌,水泛非祥。得比肩则能助力,至霜降不比无
妨。
冬月之土,外寒内温。水旺财丰,金多身贵。火盛有
荣,木多无咎。再逢土助尤佳,惟喜身强益寿。
4. 论四时之金宜忌
春月之金:余寒未尽,贵乎火气为荣,体弱性柔,欲得
土生乃妙。水盛贝i金寒,有用等于无用。木盛则金折,至刚转
为不刚。金来比助,扶持最喜。比而无火,失类非良。
夏月之金,尤为柔弱。形质未备,更忌身衰。水盛呈
祥,火多不妙。遇金则扶持精壮,见木则助鬼伤身。土厚埋没
无光,土薄资生有益。
秋月之金,当权得令。火来锻炼,遂成钟鼎之材;土复
• 57 •
资生,反有顽浊之气。见水则精神越秀,逢木则琢削施威。金
助愈刚,过刚则折。
冬月之金,形寒性冷。木多则难施斧凿之功,水盛则
不免沉潜之患。土能制水,金体不寒。火来生土,子母成功。
喜比肩类聚相扶,欲官印温养为妙。
5. 论四时之水宜忌
春月之水,性滥滔淫。若逢土制,则无横流之害;再逢
水助,必有崩堤之忧。喜金生扶,不宜金盛;欲火既济,不宜火
炎。见木施功,无土散漫。
夏月之水,外实内虚。时当涸际,欲得比肩。喜金生
助体,忌火旺太炎。木盛则耗泄其气,土盛则克制其源。
秋月之水,母旺子相。得金助则清澄,逢土旺则混浊。
火多而财盛,太过不宜;木重而身荣,中和为贵。重重见水,增
其泛滥之忧,叠叠逢土,始得清平之象。
冬月之水,正应司权。遇火除寒,见土归宿。金多反
致无义,木盛是为有情。水太微则喜比为助,水太盛则喜土为
堤。

再说五行与五方。按照前表所列,我们可以把五行五方
和天干地支捏在一起,凑成几句便于记忆的话。那就是,
东方甲乙寅卯木,
南方丙丁巳午火,
西方庚辛申酉金,
北方壬癸亥子水,
中央戍 己辰成丑未土.
为什么五行和五方要作这样的联系呢?原因是木的禀性温和
向阳,而东方正是太阳初升的地方,所以木便和东方结了缘
• 58 •
中央戊己
辰戌丑未
土
凰
说到五行和四时五方的关系,我们还可结合八卦方位,画
成 59 页这样一个图表,以便观览。
命理学家把五方观念引进命里,主要是因为通过对一个
人生辰八字阴阳五行的推算,可以看出他行运的方向,以及该
在什么方位生活活动,才最为有利。比如有的人利于行东方
木运,不利于西方金运,如果行运一碰上西方金运,就会倒运。
份,灭的禀性炎热盛长,而南方气候炎热,有利于万物生长,所
以火便和南方结了缘份;金的禀性清凉肃杀,而西方正是太阳
落山,草木不生的地方,所以金便和西方结了缘份;.水的禀性
澄澈寒冷,而北方水冰地寒,所以水便和北方结了以份;土的
禀性厚实适中,有利于万物生长,中央地处东南西北的中间,
所以土便和中央结了缘份。

【59页图】

说到五行和四时五方的关系,我们还可结合八卦方位,画
成 59 页这样一个图表,以便观览。
命理学家把五方观念引进命里,主要是因为通过对一个
人生辰八字阴阳五行的推算,可以看出他行运的方向,以及该
在什么方位生活活动,才最为有利。比如有的人利于行东方
木运,不利于西方金运,如果行运一碰上西方金运,就会倒运。

• 59 •

又如在外出上,有利于行东方木运的,最好往东方跑,如果硬
要去西方,往往不利。至于其他利于南方,不利于北方等等,
可以按照这个原则类推。

\section{五行的旺相休囚死和寄生十二宫}

五行的“旺相休囚死”也是和四时密切相关,并被命理学
家谈得较多的一个问题。这里面总的精神,就是在春、夏、秋、
冬四个季节里,每个季节都有一个五行处于“旺”,一个五行处
于“相”,一个五行处于“休”,一个五行处于“囚”,一个五行处
于“死”的状态。
那末,什么叫做旺、相、休、囚、死呢?解释是:
(1旺〕 处于旺盛状态。
〔相〕 处于次旺状态。
〔休〕 休然无事,亦即退休。
〔囚〕衰落被囚。
〔死〕 被克制而生气全无。
现在我们把五行在四时中的旺、相、休、囚、死,简括如
下: .
[春] 木旺 火相 水休 金囚 土死
〔夏〕 火旺 土相 木休 水囚 金死
〔秋〕金旺 水相 土休 火囚 木死
(1冬〕 水旺 木相 金休 土囚 火死
1四季〕 土旺 金相 火休 木瓜1 水死
• 60 •
从上简括,我们可以明显看出这样的规律,就是当令的
旺,我生的相,生我的休,克我的囚,我克的死。比如用木举
例,春天是木当令的季节,所以木旺;火是木生出来的,所以火
相;水是生木的母亲,现在木已长成旺盛之势,母亲便可退居
一旁,所以水休;春木旺盛,金已无力克伐,所以靠边站而金
囚;土是木所克的,现在木既当令,气势强旺,所以土死。在具
体应用中,一个人如果春天出生,八字中以木为主的,就是当
令得时,八字中以金为主的,就是被囚而不得时了。其他依次
类推。为了便于观照,我们现在再反过来,以五行为主线,分
别把它们处在四季旺相休囚死状态括要如下:
〔木〕 春旺 冬相 夏休 四季囚 秋死
〔火〕 夏旺 春相 四季休 秋囚 冬死
〔土〕 四季旺: 夏相 秋休 冬囚 春死
(:金〕 秋旺 春囚 夏死 四季相 冬休
〔水〕 冬旺 四季死 春休 夏囚 秋相
我们在懂得五行的旺、相、休、囚、死后,还不要忘了五行
寄生十二宫的原理,因为这更是搞命理深入下去必不可少的
一种理论指导。
五行寄生十二宫的原理,也就是每一个具体五行在十二
个月中从生长到死亡过程的原理。按照《三命通会》的说法,
十二宫的名称和解释是:
〔绝〕 又叫“受气”,或“胞”,“以万物在地中,未有其象,
如母腹空,未有物也”。
C胎〕 就是“受胎”,“天地气交,前氯造物,其物在地中萌
芽,始有其气,如人受父母之气也”。
〔养〕 就是“成形”,“万物在地中成形,如人在母腹成形
• 61 •
也”。
〔长生〕“万物发生向荣,如人始生而向长也"
〔沐浴〕 又叫“败”,“以万物始生,形体柔脆,易为所损,
如人生后三日,以沐浴之,几至困绝也”。
〔冠带〕“万物渐荣秀,如人具衣冠也)
〔临官〕“如人之临官也”。
〔帝旺〕“万物成熟,如人之兴旺也”。
〔衰〕“万物形衰,如人之气衰也”。
〔病〕“万物病,如人之病也”。
〔死〕“万物死,如人之死也”。
〔墓〕 又叫“库”,“以万物成功而藏之库,如人之终而归
墓也”。
为了明白起见,现将五行寄生十二宫“长生”、“沐浴”、“冠
带”、“临官纪“帝旺”等情况列表如下:

【62页表】

表里,五阳干和五阴干中的甲木、乙木等,指的是出生那
一天天干的五行,而亥、子、丑、寅、卯等十二地支,则又分别指
的是出生的月份。通过这表,我们可以得知,出生一天的天干
如果是甲木,出生的月份是十月亥月,那么这甲木就处于万
物发生向荣的“长生” 状态。如果出生一天的天干换成乙木,
而月份则仍是十月亥月不变,那末这乙木就处于万物临终的
“死”的状态。这就告诉我们,出生一天的天干同样是木,但属
于阳木的甲木是生,属于阴木的乙木却是死,彼此互换。原理
是:“阳之所生,即阴之所死。”反过来,甲木死于午,则乙木必
生于午。这个规律用另外一句话来说,就是“阳干顺行,阴干
逆行,这样彼此交叉下来,就是“阳临官则阴帝旺”,如阳木甲
木临官是寅,那末对于乙木来说,寅就变成乙木的帝旺了。
可是,对于“阳之所生,即阴之所死”的说法,任铁樵却深
表异议道:“古法只有四长生,从无子、午、卯、酉为阴长生之
说。水生木,申为天关,亥为天门,天一生水,即生生不息. 故
木皆生在亥。木死午为火旺之地,木至午发泄已尽,故木皆死
在午。言木而余可类推矣。夫五阳育于生方.盛干本方.唉 、
泄方,尽于劫方,于理为顺。五阴生于泄方,死于主方,不叩*
背。即曲为之说,而子、午之地,终无产金产木之道,寅、亥之
地,终无灭火灭木之道。古人取格,丁遇酉,以财论;乙遇午,
己遇酉,辛遇子,癸遇卯,以泄食神无论,俱不以生论;乙遇亥,
癸遇申,以印论,俱不以死论;即己遇寅藏之丙火,辛遇巳歌之
戊土,亦以印论,不以死论。由此观之,阴阳同生同死可知也。
若执定阴阳顺逆,而以阳生阴死,阴生阳死论命,则大谬矣J
有关五行寄生十二宫的吉凶休咎,《渊海子平》有诗概括
为,
• 63 •
长生诗诀
长生管取命长荣,时日重临主性灵,
更得吉时相会遇,少年及第入王庭。
长生若也得相逢,坐下须招祖业隆。
父母妻儿无克陷,安然享福保初终。
沐浴诗诀
沐浴凶神切忌之,多成 多败少人知。
男人值此应孤独,女命逢之定别离。
泉浴那堪吉位逢,更兼引从在其中。
读书 必定登科甲,莫比诸神例作凶。
桃花 沐浴不堪闻,叔伯如姨合共婚,
日月时胎如犯此,定知无义乱人伦。
咸池无禄号桃花,酒色多因败坏家,
更被凶神来克破,瘠赢病死莫咨嗟。
女命若还逢沐浴,破败两三家不足。
父母离乡寿不长,头生长女须防哭。
冠带
命逢冠带少人知,初生贫寒中主宜,
更得贵人加本位,功名成就又何疑。
• 64 •
人命若还逢冠带,兄弟妻孝无陷害。
因何缓租绍箕裘?只为胎中有冠盖。
临官诗诀
临官 帝旺最为奇,业绍箕裘显祖宗。
若不上元登上第,直须黄甲脱麻衣。
帝旺
临官 帝旺两相逢,业绍 箕裘显祖宗。
失位纵然居世上,也须名姓达天聪。
衰、病、死
纳音衰病死重逢,成败之中见吉凶。
若得吉神来救助,变灾为福始亨通。
衰病两逢兼值死,世人至老无妻子,
不惟衣食不丰隆,灾病绵绵终损己。
墓库
墓库元来是葬神,一为 正印细推论。
相生相顺无相克,富贵之中次第分。
胞胎
绝中逢生少人知,
反本还原宜细辨,
却去当生命里推。
忽然速否无猜疑。
• 65 •
胞神一位难为绝,克限妻孥家道穷。
不惟朝暮走茫茫,羊食狼贪无以别。
胎、养
胎养须宜细审详,半凶半吉两相当,
贵 神相会应为福,恶杀重逢见祸殃。
用出生一天干支对照出生月份,找出寄生十二宫的各种
状态,从而象征一个人的命运好坏,虽然未免简单,但算命家
们却用得很是普遍。可话也得说回来,这种办法也不是绝对
的,还要参考其他的众多因素才能决定,所以《三命通会》在
《论五行旺相休囚死并寄生十二宫》结束语中说道:“凡推造
化,见生旺者未必便作吉论,见休囚死绝未必便作凶言。如生
旺太过,宜乎制伏;死绝不及,宜乎生扶。妙在识其通变。古
以胎、生、旺、库为 '四贵',死、绝、病、败为 '四忌 余为 '四
平',亦大概而言之。”可见还是比较活泛的。

\section{天干地支的刑冲害化合}

在天干地支中,彼此之间的刑、冲 、害、化 、合,是看命的重
要依据之一,所以不得不谈。
先说刑。刑就是彼此刑妨,互不相和的意思。按照命书
的说法,十二地支共有三刑,就是:
子卯,一刑也;
• 66
寅巳申,二刑也;
丑未戌,三刑也。
这就是说,一个人八字的地支中如果有子、卯两个地支碰在一
起,或者寅、巳、申三个地支碰在一起,或者八字中虽然没有这
种现象,可是在流年或大运中碰上的,都可认为是刑。其中子
卯相刑,有“无礼”的说法,所以“女命见之,尤为不良”。
然而,对于刑,也要根据一个人八字的具体情况进行分
析,不可一见有刑就认为是不吉之兆,所以“鬼谷遗文”说:
君子不刑定不发,若居士途多腾达。
小人到此必为灾,不然也被官鞭挞。
可见对于“三刑”,在封建社会是君子得之则吉,小人得之则凶
的。然而也有认为「刑之义无所取”。
接着说冲。冲有天干相冲和地支相涉的不同。天干相冲
的有甲庚、乙辛、壬丙、癸丁四对关系,因为东甲西庚,东乙西
辛,北壬南丙,北癸南丁,方向两两相对,性质截然相反,所以
就冲了起来。对于天干中的丙庚、丁辛,固然彼此间有着相克
的关系,可是丙南庚西,丁南辛西,方向对不起来.所以只克不
冲。此外,甲庚、乙辛、壬丙、癸丁相冲的理由还有,甲庚都属
于阳,乙辛都属于阴,壬丙都属于阳,癸丁都属于阴,阳阳与阴
阴同性相斥,配不起来,也是一个原因,不比甲己、乙庚、丙辛、
丁壬、戊癸,即使相克,然而克而不斥,所以还是合化起来,成
了夫妻。•至于戊己居中,没有方向上的相对,所以也就没有
冲。
在十二地支相冲中,因为每隔六位数就要彼此冲激起来,
所以叫做“六冲 比如子午相冲,子代表水,午代表火,并且
方向相对,故而就冲了起来。六冲的具体情况是,
• 67 •
子午相冲;
丑未相冲;
寅申相冲;
卯酉相冲;
辰戌相冲;
巳亥相冲。
这组概念,从方位来说,都是对的,就五行来说,都是克的, 就
阴阳而言,都是阳克阳,阴克阴,阴阳不能配合,所以就冲了起
来。
在命理学中,六冲是一个很重要的概念。就一般情况言,
六冲给人留下的印象似乎并不太好,可是在命书里,也常作具
体分析。比如辰戌丑未彼此两两相冲,但在十二地支中,辰戌
丑未都被解作是库,也就是仓库的意思。“四库所藏,为十干
财官印绶等物,尤喜冲激”(《三命通会》卷二),因为仓库平时
总是锁着,里面虽有库藏,可就是只好望断秋水,到不了手。现
在经过一冲,库里的财官印绶都被冲了出来,这对于命的主人
来说,不就成了好事?此外如命里 “寅申巳亥全,子午卯酉
全”,也是一种大的格局,不作破败定论。按照《三命通会》的
说法,六冲中怕就怕冲得不全,或者同类相冲。所谓同类相
冲,就是八字干支的天干相同,地支相冲,比如甲子见甲午,己
卯见己酉之类,其中甲和甲同,但子和午却冲了起来,己和己
相同,但卯和酉却冲了起来。如果是犯着这类相冲的,怕就不
太妙了,即使一时禄高名重,将来也难免终有一失。
关于地支六冲,当今台湾命理学家李铁笔还认为:
子午冲,一身不安。
卯酉冲,不重亲戚多忧愁。
68 •
寅申冲,多情而败。
巳亥冲,喜好助人不利己。
年月支冲,背祖离乡。
年日支冲,与亲不和睦。
年时支冲,与子不和。
年冲日、月、时,易罹疾。
日月支冲,与父母兄弟不和。
日时支冲,克妻损子。
但同时李铁笔又认为「以上亦仅论命之参考,绝不可轻
易以此武断之。”
在旧社会,男女缔结婚姻,人们一般都力求避开六冲。这
也是命理学家带给民俗上的一种迷信影响,不足为据。话虽
这么说,可做起来却并不那么容易,可见封建迷信一旦和民俗
结合起来,其势力就是那么的根深蒂固,难以铲除。
再说害。害又名穿,就是彼此损害的意思。命书记载,害
也有六种情况,就是:
子未相害;
丑午相害;
相害; •
卯辰相害;
申亥相害;
酉戌相害。
不过六冲和六害比较起来,六冲是每隔六个地支就要两两相
冲,这样十二地支就有六对相冲。而六害的数目虽然也是六
对,但《三命通会》又说「六,六亲。害,损也。犯之,主六亲上
有损克,故谓六害。”可是,清代命理学家任铁樵却认为,“且刑
• 69 •
既不足为凭,而害之义尤为穿凿,总以论其生克为是J
至于化,这是就十个天干而说的。命书上说,十个天干两
两相化,共有五种情况:
甲己化土;
乙庚化金;
丙辛化水;
丁壬化木;
戊癸化火。
因为化的条件是合,只有合起来才能化,所以化又称 “合” 或
“合化”。《三命通会》卷二有《论十干合》一篇,就是说的这种
情况。书中认为,所谓“合”,就是“和谐”的意思,为什么合化
以后就和谐了呢?书中的说法是,东方甲乙木最怕西方庚辛
金来克,甲是阳木为兄,乙是阴木为妹,于是甲木就想方设法
把妹妹乙木嫁给阳金庚做老婆,这不就阴阳和合了吗?古来
女子嫁鸡随鸡,嫁犬随犬,所以乙木嫁庚金后,就从一而终地
跟了庚金了。
末了说合。在十二地支中,又有六合和三合的不同。其中
六合是I
子丑合土;
寅亥合木;
卯戌合火;
辰酉合金;
巳申合水:
午茶为太阳太阴。
《三命通会》在《论支元六合篇》里说「夫合者,和也。乃阴阳相
和,其气自合。子、寅、辰、午、申、戌六者为阳,丑 、卯 、巳、未、
• 70 •
酉、亥六者为阴,是以一阴一阳和而谓之合。”至于为什么一定
要子丑合土,寅亥合木,这除了这些地支中含有相应五行外,
还有一个“气数中要占阳气为尊”的原理。比如子为一阳,丑
为二阴,一和二合起来是阳数的单数三;寅为三阳,亥为六阴,
, 三和六合起来是阳数的单数九等等。其实这种说法,也是很
牵强的。此外结合看命,六合中还有合禄、合贵、合马,以及男
子忌合绝,女子忌合贵的说法,也是很有趣的。
三合和六合的字面含义不同。六合是十二地支两两相合,
总起来的数目是六,三合则不是这样,主要是说十二地支中三
个三个合起来的意思。三合的内容是:
申子辰合水;
亥卯未合木;
寅午戌合火;
巳酉丑合金。
三合在五行中没有土的原因是,水、木 、火、金四行都要依赖土
才能形成格局,这就是万物都归藏于土的原理。至于辰、戌、
丑、未凑在一起,那就自然合成土局了。
据说一个人生辰八字和三合配合得好的,还可以出现一
种“三会禄格”的格局:假如有幸得了这种格局,那末“折月中
之仙桂”,就肯定没有疑问了。
天干地支的刑、冲、害、化、合看来内容较多,但旧时算命
先生为了终身受用,也是不得不掌握的。

\section{十二生肖和地支}

生肖,这一人类社会的奇特现象,不 仅中国有,就是外国
也有,不过只是具体的动物有所不同罢了。费尔巴哈在《费尔
巴哈哲学著作选集》下卷指出「人之所以为人要依靠动物,而
人的生命和存在所依靠的东西,对于人来说就是神。” 由此可
见,生肖起源于古代人们对动物的崇拜,大概不会有多大疑问。
所谓“生肖”,就是生下来的那年类属于什么动物的意思。
因为从字面上解释,肖就是象,比如子年生的肖鼠,就是说,逢
子年出生的,不管是甲子、丙子、戊子,还是庚子、壬子,都可能
生性多疑好动,胆小警觉,类似于鼠的性格。按照古代术数家
的说法,地支有十二,生肖也有十二,彼此一一相配,哪一年出
生的人肖什么,都有一定的规定。现将十二地支配合十二种
生肖的规定对照如下表(见第 73 页)。
按照这个对照表,我们便可知道哪一地支年份出生的人
肖什么了。至于子年生的为什么肖鼠,丑年生的为什么肖牛,
寅年生的为什么肖虎,卯年生的为什么肖兔,它们的排列顺序
又为什么要这样,术数家们虽有解释,可却牵强附会,生硬得
很。明代李诩《戒庵老人漫笔》卷七引王文恪公的话认为,十
二生肖的排列顺序和天上二十八宿星象的位序相合。他的解
释是二十八宿分布周天,以值十二时辰。每个时辰二宿,子午
卯酉三宿,而各有所象。把二十八宿的禽名简化,按天空次序
排列,由北而东,而南,而西,这样周转下来,正好是十二生肖
• 72・

【73页表】

绕天一周。
危(燕)、虚(鼠)、女(蝙)都值子时,取鼠代表子。
牛(牛)、斗(邂)都值丑,取牛代表丑。
箕(豹)、尾(虎)都值寅,取虎代表寅。
心(狐)、房(兔)、氐(貉)都值卯,取兔代表卯。
亢(龙)、角(蛟)都值辰,取龙代表辰。
轸(蚓)、翼(蛇)都值巳,取蛇代表巳。
张(鹿)、星(马)、柳(獐)都值午,取马代表午。
鬼(羊)、井(狎)都值未,取羊代表未。-
参(猿)、鹫(猴)都值申,取猴代表申。
毕(乌)、昴(鸡)、胃(雉)都值酉,取鸡代表酉。
娄(狗)、奎(狼)都值戌,取狗代表戌。
壁(输)、室(猪)都值亥,取猪代表亥。
这种把十二生肖和天上二十八宿联系起来,从而与十二时辰
地支发生关系的解释,早在南宋朱熹《十二辰诗》中已经有了
隐约的苗子。诗中,朱熹从半夜听老鼠咬席子写起,写一天亮
牛就去耕地,一直写到“客来犬吠催煮茶,不用东家卖猪肉”。
从而把十二个动物一天的活动,按时间顺序叙述了一遍。这
里,朱嶷也认为十二生肖是以活动时问来排列顺序的。
我国古代,“肖”又叫*属”。《周书 •宇文护传》说:"生汝
兄弟,大者属鼠,次者属兔,汝身属蛇。”这里所说的“属”,就是
“肖”的意思。因此古代“十二生肖”,又常称为“十二相属”,或
• 73 •
者称为“十二属相”。
从文献看,东汉王充《论衡 • 物势》,蔡邕《月令问答》,晋代
葛洪《抱朴子 • 登涉》等篇都有零星记载,根据这些记载,清朝
的赵翼在《陵余丛考》卷三十四中断定,十二相属正式的说法
起于东汉,因为在这以前并没有什么人系统提起过这玩意儿。
然而却也有人认为,十二生肖早在西周或春秋战国时候就已
有了,比如《诗经 •小雅 •吉日》的“吉日庚午,既差我马”,把
午和马对应联系起来;又如《左传 •僖公五年》有“龙尾伏辰”
这样的话,把辰和龙对应起来。然而由于这种对应只是偶然零
星的记载,形不成完全的系统,所以只好暂且悬在那里再说。
自从算命术盛行以后,和十二地支有联系的生肖,自然逃
不了被涂上一重迷信色彩的命运。在算命先生的影响下,人
们一遇休戚祸福,总会情不自禁把自己和周围人的属相牵扯
联系起来,甚至在婚配里也要注意避开男女双方生肖的冲撞,
什么“鸡狗断头婚”,“龙虎不相容”等等,就是对婚姻而说的。
再如在社会上,属虎的人常为人们所畏忌,尤其是属虎的女
人,简直没人敢要她,否则发起雌威,谁受得了?
由于十一生肖是和十二地支联系在一起的,所以十二地
支的相合和冲害学说,就自然和生肖的相合和冲害捆到了一
起。我们且看下列三组对照:
1. 十二地支相合和生肖相合
地支子丑相合,生肖鼠牛相合;
地支寅亥相合,生肖虎猪相合;
地支卯戌相合,生肖兔狗相合;
地支辰酉相合,生肖龙鸡相合;
地支巳申相合,生肖蛇猴相合;
• 74 •
地支午未相合,生肖马羊相合;
地支申子辰相合,生肖猴鼠龙相合,
地支寅午戌相合,生肖虎马狗相合,
地支亥卯未相合,生肖猪兔羊相合,
地支巳酉丑相合,生肖蛇鸡牛相合。
2, 十二地支相冲和生肖相冲
地支子午相冲,生肖鼠马相冲;
地支丑未相冲,生肖牛羊相冲;
地支寅申相冲,生肖虎猴相冲;
地支卯酉相冲,生肖兔鸡相冲;
地支辰戌相冲,生肖龙狗相冲;
地支巳亥相冲,生肖蛇猪相冲。
3. 十二地支相害和生肖相害
地支子未相害,生肖鼠羊相害;
地支丑午相害,生肖牛马相害;
地支寅巳相害,生肖虎蛇相害,
地支卯辰相害,生肖兔龙相•害;
地支申亥相害,生肖猴猪相害;
地支酉戌相害,生肖鸡狗相害。
在这种观念支配下,属龙的人喜和属猴、属鼠的人结合,
却怕和属狗的婚配;属蛇的人喜和属鸡、属牛的人结合,却
怕和属猪的婚配;属鸡的人,喜和属蛇、属牛的结合,却怕和属
兔的婚配,等等。至于鼠羊相害,牛马相害,虎蛇相害等等,情
况虽没相冲严重,可是为了趋吉避凶,结婚时对于大多数人来
说,也是不大肯冒这个险的。当然这种无稽之谈的迷信说法,
到了今天,市场已经显得愈来愈小了。
• 75 •
