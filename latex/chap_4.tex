\chapter{算命术的批判}
\section{墨子的“非命”观}
所谓“非命”,就是否定、反对世界上有所谓天命的一种观
点。作为学说的一种,“非命”观在先秦诸子的墨家学派中,也
和“兼爱”、“非攻”一样,同样占着十分重耍的地位。可以这样
说,在前时一片弥漫着天命观的混浊空气中,典家学派打出的
这一放帜鲜明的观点,无疑为当时的学坛和意识形志领域,吹
进了一阵醒人耳目的清风。
《墨子》一书,《非 命》共有上、中、下三篇,文中集中体现了
墨家代表人物墨翟对于“天命”观的批判,是我国古代反对天
命论的辉煌篇章。
在《非命》上篇中,墨子说道,古代治理国家的王公大人,
都希望国家富足,人民众多,政局安定,然而他们所得到的,不
是富足而是贫困,不是众多而是稀少,不是安定而是祸乱。这
就是说,实际上他们没有得到原先所希望得到的,而是得到了
他们原先所不希望得到的。这是什么原因呢?回答是“执(主
张)有命(命运)者”混在民间的太多了。这些“执有命者”认
312 
为「命里注定富足就富足,命里注定贫困就贫困,命里注定人
多就人多,命里注定人少就人少,命里注定安定就安定,命里
注定祸乱就祸乱,命里注定长寿就长寿,命里注定短命就短
命。即使你使出多大的力气,又有什么用处呢?”他们既把这
一套向上兜售给王公大人,又向下影响了百姓干活的积极性。
所以,“执有命者”是不仁的。对于他们这些惑乱视听的言论,
不可不辨个彻底的水落石出。
那末,怎样才能辨个彻底的水落石出呢?墨子的说法是,
立论必定先要有个标准。立论如果没有标准,就好像在运转
着的制陶转轮上去辨别方向,是怎么也辨不清楚的。正因为
这样,所以墨子提出了立论一定要有“三表”(三项标准)的原
则。什么叫“三表”? 一是指推究本源,二是指弄清过程,三是
指检验实践。怎样才能推究本源呢?这就要上推古代圣王的
事迹了。怎样才能弄清过程呢?这就要下考百姓耳闻目睹的
实情。怎样才能检验实践呢?这就要在刑政实施中检验是不
是符合国家和人民的利益。
如今天下士君子中,认为有命运的,何不往上观察一下圣
王的事迹呢?古代夏桀搞乱了天下,商汤把它接过来治理好
了;商纣搞乱了天下,周武王把它接过来治理好了。这期间社
会没有改变,老百姓没有变换,由桀、纣统治则天下大乱,由商
汤、周武王统治却天下大治,这难道可以说是归之于命运吗?
如今天下士君子中,认为有命运的,何不往上翻看一下先
王的典籍?先王的典籍,原是国家制定出来,公布施行到百姓
中去的一种有关宪制。龙王的宪制,何曾说过“幸福不可求得
而灾祸不可避免,善良没有好处而凶残没有害处”的话?先王
用来审判案件制裁犯罪的,是国家的刑律。先王的刑律,又何
 313 
曾说过“幸福不可求得而灾祸不可避免,善良没有好处而凶残
没有害处”的话?先王用来整治军队,指挥军队进退的,是先
王的军令。先王的军令,又何曾有过“幸福不可求得而灾祸不
可避免,善良没有好处而凶残没有害处”的话?所以墨子说
道:我们还没有完全统计过天底下的好书,即使统计起来也统
计不完,可是从大的方面来看,基本就数宪制、刑律、军令这三
个大类了。现在只要一看那些“执天命者”的言论,都是些古
代先王典籍里找不到的,这不明摆着可以丢弃了吗?二看那
些“执天命者”的言论,是违背天下道义的。而又正是那些违
背天下道义的言论,使得百姓困苦不堪而不能自拔。把百姓
弄得困苦不堪而不能自拔当作自己快乐的,就是残害天下的
人。
再看,人们为什么要坚守正义的人治理国家呢?回答是
义人在上,天下必治,上帝山川鬼神有了正统的继承人,万民
就会得到莫大的好处。怎么才能证实这一点呢?古代商汤封
在亳邑(今河南省商丘县),这地方长短大小总计起来,不过百
里见方的土地,可是商汤却能和百姓“兼相爱,交相利,侈(多
余)则分”,率领百姓尊敬上天,事奉鬼神,这样上天鬼神就使
商汤的天下富裕起来,结果诸侯归附,百姓亲近,贤士投奔,没
有终结他的一生就称王天下,做了诸侯的头头。又如古代周
文王封在岐周(今陕西省岐山县),这地方长短大小总计起来,
不过百里见方的土地,可是周文王却和百姓“兼相爱,交相利,
侈则分”,所以住在近处的百姓乐意受他的统治,住在远处的
百姓也听说他的德政而竞相归附。当时只要听到周文王名字
的,不仅人们都会立即起身投奔到他那里,就是连体弱病残
的,也都会守候在自己的住处热切盼望说J要是文王的土地
314 
扩展到我们这里,那末我们岂不也成了文王的百姓?”也就因
为这个原因,上天鬼神就使文王的天下富裕起来,结果诸侯归
附,百姓亲近,贤士投奔,没有终结他的一生就称王天下,做了
诸侯的头头。刚才我们不是说过:“义人在上,天下必治,上帝
山川鬼神有了正统的继承人,万民就会得到莫大的好处」结
论就是根据这些事实推论出来的。
所以,古代圣王制定法律,硕布政令,设立奖惩条例,原是
用来鼓励好人,制约坏人的。刑政赏罚明白,百姓们在家就孝
顺父母,出门就尊敬师长,进进出出都有一定的规矩礼节,男
男女女都不杂处。国家如果派这样的人治理官府就不会偷盗,
守护城池就不会背叛,国君有难就誓死保卫,国君流亡就跟随
护送。而这些美德,正是国君赞赏,百姓称誉的。可是对于这
些,“执天命者”却说「君王要赞赏,是这些人命里本来就该得
到的,并不是因为做了好事才被赏的。”在这种思想支配下,有
些人可以在家不孝父母,出门不敬师长,进进出出都不讲规矩
礼节,男男女女都杂在一起。国家如果派这样的人去治理官
府就会偷盗,守护城池就会背叛,国君有难就不肯死节,国君
流亡就不肯护送。而这些劣迹,却正是国君惩戒,百姓责备
的。可是对于这种劣迹,“执天命者”又会说「国君要惩罚,是
这些人命里本该招致的,并不是因为劣迹斑斑才被惩罚的。”
在这种思想支配下,为君的就可以不守正义,为臣的就可以不
忠国君,为父的就可以不爱护子女,为子女的就可以不孝顺父
母,为兄的就可以不关心弟弟,为弟的就可以不尊敬兄长。所
以那些“执天命者”,简直就是一切谬论和劣迹的制造者。
那么,怎么进一步说明“执天命者”是一切谬论和劣迹的
制造者呢?且看上古时候那些不开化的百姓,吃起来贪心不
315
足,干起活来却偷懒得很,所以吃的和穿的都告匮缺,这就难
免担心自己受饿挨冻。可是对于这种受饿挨冻的简单原理,他
们非但不从“我这个人太懒惰不中用,干活不得力”去考虑,反
而坚持认为:“我的命本来就是个受饿挨冻的命。”再看上古时
候那些暴虐的君王,既不克制他们耳目声色的欲望和心里头
的邪念,又不孝顺他们的父母,这就难免最终导致国破家亡。
可是对于这种国破家亡的简单原理,他们非但不从“我这个人
太懒惰不中用,不善于处理国政家事”去考虑,反而坚持认为:
“我的命本来就是个要国破家亡的命。”《书经  仲虺之告》中
说「我听说夏朝人假托天命,发布命令于天下,于是上帝便就
怒而讨伐他们的罪行,因此夏朝就失掉了他们的军队。”这是
商汤在否定夏桀天命观时所说的话。又如《书经  太誓》中
说「商纣在平时不肯事奉上帝鬼神,丢开他们的祖先神祇不
去祭祀,竟说「我有好命,不必尽力做事。'这样上帝也就放弃
了商纣而不去保佑他了。”这些,又是周武王在否定商纣王天
命观时所说的话。
眼下,由于这些“执天命者”的言论,弄得国君不治理国
家,百姓不好好干活。国君不治理国家就政局混乱,百姓不好
好干活就财用不足,结果弄得上对上帝鬼神没有米饭甜酒可
供祭祀,下对天下贤人达士无法收容安抚,外对诸侯宾客不能
应接招待,内对贫民百姓无以充饥御寒,更不要说是将息调养
老弱病残了。所以天命观“上不利于天,中不利于鬼,下不利
于人”。这样“执天命者”不筒直就是一切谬论和劣迹的制造
者?
所以最后墨子在篇末总结道「今天下之士君子,忠实(真
心实意)欲天下之富而恶其贫,欲天下之治而恶其乱,执有命
 316 
者之言不可不非,此天下之大害也。”
文中,墨翟通过推究本源,弄清过程,检验实践的“三表”
原则,采用层层列举事实,步步讲清道理的方法,有力地驳斥
了“执有命者之言”对于天下的严重危害性,可谓入木三分。这
里面虽然也有由于时代对作者所造成的局限,使得墨翟在驳
斥反对万事命定的同时,又搬出了天帝鬼神的一套,可是无论
如何,这在当时一片烟障雾隔,信命如狂的社会风气中,其文
章思想的光辉瑰奇,却是无可置疑的。

\section{古人都相信算命吗?}
自从唐李虚中发明用年、月、日三柱和徐子平奠定用年、
月、日、时四柱推算命理以来,一时算命术大行天下,学者风
从。就以那位为李虚中作墓志的大文豪韩愈来说,就是个十
分信命的人。此后宋元明清,学者君子,信命的代不乏人,更
不要说是一般民间的平民百姓了。
明代张瀚曾在《松窗梦语》卷六中说,有一年,作者有个名
叫孙季泉的朋友,邀乡人同事一起饮酒。席间,孙季泉一个个
地询何各人出生的年月日时,心里暗暗推算,可就是不出声。
后来酒酣席散,孙季泉暗里拉作者到一旁说「我和你为同年
友,现在只有我们两人。我看你中年运限不利,然而不知到底
怎样,现在再仔细为你推算一下」算后,孙季泉很有把握地
说「中年虽然运行西方,只是宦途淹滞不利,对身家性命却没
有多大影响。行过西方金运,进入南方火运,那就豁然通达
317
了。”当年孙季泉高中一甲,作者中二甲。后来孙季泉官至宗
伯的高位,过了十几年后,作者也爬上了冢宰的高位。至此,
作者不禁深深感叹「夫以数十年之迟速显晦,决于八字之间,
公之精于术数如此!”
《玉堂丛语》是明朝学者焦航的一本笔记杂著。书中卷七,
作者说了这样一件事,提学萧鸣凤精于子平之术,正德丁丑年
间廷试,有人拿了好多考生八字前来求教说:“请你算算这次
廷试,谁是状元?”萧鸣凤把各人八字一一看过后说:“这位舒
梓溪可以高中廷试第一,状元爷的桂冠他摘定了 结果廷试
下来,果然应验了他的话。
类似以上记载的,在明清人的著述中,真是难以一一枚
举,可以作为古人信命的铁证。可是,是不是古人都信命呢?
答案自然是否定的。
一个人生活中贫富寿夭的遭遇,原要受着历史、社会、政
治、文化、境遇等等多种因素的制约,而命理学家撇开这些不
谈,大谈其命,自然是荒谬而站不住脚的。对此,在不相信命
的古人中,除了春秋战国时期《墨子》一书打出《非命》三篇,对
儒家的天命观念作无情批驳的那位墨家学派创始人 墨翟外,
唐太宗时著名哲学家吕才也是其中一个。吕才在他所著的《算
命篇》中说「汉宋忠、贾谊讥司马季主(占卜术士)曰:'卜筮者
高人禄命(说好别人的命),以悦人心,矫言祸福,以规(图谋)
人财。'王充曰「见骨体,知命禄,见命禄,知骨体。' 此则言禄
命尚矣。推索本原,固其不然。积善之家,必有余庆,岂建禄
(命中有禄)而后吉乎?积恶之家,必有余殃,岂劫杀(命中有
劫财、七杀)而后灾乎?”接着他又举例说:“文王忧勤损寿,非
初值空亡《一种凶煞);长平坑降卒,非俱犯三刑(被秦坑死的
318 
四十万赵国降卒,并不都是命里犯了三刑);南阳(汉光武帝)
多近亲,非俱当六合(命中地支)。”与此同时,他还详论鲁庄公
的命说,如果按照他出生的年、月、日去算,应该是个穷贱的
命,可实际上,他却当上了一国之君。有意味的是,偏偏这个
说命不准的人,却是个历史上对阴阳、舆地有着极深研究的专
家。在《旧唐书》本传中,至今还保存着他的《叙宅经》、《叙禄
命》、《叙葬书》等多篇。看来,大概正因为他曾深入这个营垒,
所以才能反戈一击,致强敌于死命。
赵宋之时,费衮著《梁溪漫志》十卷。书的第九卷中,有
《谈命》一则说:“近世士大夫多喜谈命,往往自能推步,有精绝
者。予尝见人言「日者阅人命,盖未始见年、月、日、时同者,
纵有一二,必唱言于人以为异。尝略计之,若时无同者,则一
时(时辰)生一人(一种),一日当生十二人I 以岁计之,四千三
百二十人;以一甲子计之,止有二十五万九千二百人而已。今
只以一大郡计,其户口之数尚不减数十万,况举天下之大,自
王公大人以至小民,何啻亿兆?虽明于数者,有不能历算,则
生时同者,必不为少矣。其间王公大人始生之时,则必有庶民
同时而生者,又何贵贱贫富之不同也?‘” 末文作者虽然自谦
"予不晓命术,姑记之,以俟深于五行者折衷焉”,可是却又认
为“此说似有理 可见这里,作者的天平是倾向于不信命的。
《鸡肋编》是南北宋之际的著述,作者庄绰在书中卷上说
道,世上以五行星历论命者多矣,这里抄录先贵而后凶的命几
则:“张邦昌,元丰四年辛酉七月十六日亥时。王破,元丰二年
己未十一月初二日卯时。燕瑛,熙宁十年丁巳五月二十六日
寅时。聂山,元丰元年戊午八月初十日卯时。赵野,元丰七年
甲子正月十九日丑时。朱励,熙宁八年乙卯十月二十六日申
319 
时。王泉,元丰元年戊午正月初六日子时。蔡攸,熙宁十年丁
巳三月三十日寅时。邓绍密,熙宁六年癸丑九月二十三日戌
时。鲁贯,皇祐六年三月初五卯时。”对于这些人的命造,作者
又说,当他们处在全盛期时,算命的都没能够说出他们未来的
灾祸,由此可见,阴阳家的话是不可太听信的,只有端正身心,
好好做人,才是立身处世的唯一办法。
清代大诗人王士稹,对于算命这种玩意,他在《池北偶谈》
卷二十一中引陆象的话批判说「五行书以人始生年月日时所
值辰,推贵贱夭寿祸福甚详,乃独略于智愚贤不肖,曰纯粹清
明,则归之富贵福寿,曰驳杂浊晦,则归之贱贫夭祸。《易》有
否泰,君子小人之道,迭相消长,各有盛衰。纯驳清浊明晦之
辨,不在盛衰,而在君子小人。今顾略于智愚贤不肖,而必归
之富贵贫贱寿夭祸福,何耶?”这种从君子小人,智愚贤不肖角
度入手,对算命术只推人贵贱贫富夭寿祸福所做的批判,却也
别具一格。
清代名士袁枚,虽然也曾有过“人各有命”的说法,可是却
也时抱怀疑态度,也是不尽信命的一位。在《随园随笔》中,他
说,当初大挠作甲子,原来不过为了记数而已,就好比数一二
三四一样,并没有什么多大意义,更谈不上什么五行生克配合
了。听到袁枚的这种说法,当时曾有人反对道「人在社会上
本来也没姓名,可是一旦取名以后,人家一叫他就应了。天干
地支这玩意既然古人早就给它派上了阴阳五行,不也和人应
答一样道理吗?”对于这种暗中偷换概念的狡辩,袁枚也用同
样的狡辩术驳斥道「人是天地万物之灵,所以一叫就应,如果
派给草木禽兽什么姓名,就叫不应了,又何况天干地支这种本
来就是子虚乌有的东西呢?”
320 
清代道光年间的文人笔记中,吴炽昌的《客窗闲话》,以其
流畅的文笔,多彩的内容而为读者所熟悉。书中*续集”卷七
有《禄命》一则道:近来有个姓赵的算命先生,精于子平之术,
自己推算下来应该得个四品的官,可是因为读书不多,难求功
名。后来赵姓来到京师,看到做官的都下级承奉上级,以谄誉
获利,心里感到很不是味儿,于是出都回到扬州推牌算命。平
时他住在楼上,前来算命的先要用钱挂号登记次序,然后再用
飨筐把人家的八字从楼下吊到楼上,除了大富贵人,一般人很
难和他见面,所以名噪一时。一次,本郡太守派仆人去赵姓那
里算命,赵姓看到他的八字和自己一模一样,心里很是诧异,
就用纸条放下楼去询问来人说「如果生在南方,和我的命差
不多,如果生在北方,就有四品的官职。”后来仆人回答”我家
老爷是北方旗籍。”果然被他算中。
可是在实际中,八字完全一样而命运不一样的到处都有。
那时浙江巡抚的儿子和镇江一个卖豆腐人家的儿子出生在同
年同月同日同时,后来巡抚的儿子因为荫袭得官,当老子逝世
之后,这做儿子的也当上了浙江巡抚的官,可是那卖豆腐人家
的儿子,却仍接替他的先人,做着卖豆腐的勾当。又如《消夏
录》载纪晓岚学士的侄子,和家里奴仆的儿子刘云鹏一起降生
人间,其侄十六岁而夭,刘云鹏却依然健在。当时出生,只隔
着一扇窗子,两个孩子同时产出,连分秒都一样,可就是一尊
一卑,一夭一寿,这又怎么解释呢?可见唐朝太常博士吕才驳
论命理,千古不移。
末了,吴炽昌总结「天下之大,每日万生万死。帝皇夭寿
之日,岂无同者?昔明太祖密谕各布政,确搜与同八字之人
(和明太祖朱元璋八字一样的人 乃述三人:一僧、一丐、一
321 
市价。帝以问刘青田,亦无以对。故曰命之理微,圣人罕见
之。”这里吴炽昌的结论虽有保守成分,可是对于命理所持的
怀疑态度,则是显而易见的。
对于不信命或对命理持怀疑态度的学者,除了以上几位,
当然还有很多的人。可见在信命和不信命之间,从古以来就
有着针锋相对的斗争。

\section{算命术中海市蜃楼的象征律和风雨飘摇的演绎法}
中国算命术尽管博大精深,万化千变,搞得神乎其神,好
象果真泄了什么天地造化之秘似的,可是剖开来看,它也有一
个最基本、最原则的运算法,自始至终地贯穿在整个算命术的
中间,那就是象征律和在象征律基础上推进演化出来的演绎
法。
尽管阴阳五行在我国古代哲学体系中,有着它朴素的、唯
物的一面,但是它也不可避免地存在着种种不足,尤其是五
行:用木、火、土、金、水五种物质包罗自然界的万物,难免粗糙
生硬;五行相生用金生水,只是就金属在高温下的液化状态来
说的,金属的液化状态又怎能和水划上等号?再如土为万物
之母,能生木、生金、生火、生水,却把五种原先本属各自独立
并列的元素,说成了母子关系。其实,举其一点,土就是土,水
就是水,土里所含的水,本是水的一种存在形式,又怎么能认
为是土生出来的呢?
 322 
当然,我们不能用现代的科学水平来要求古人,否则社会
就没有前进了。不过,这至少可以提供给我们这样的思考,就
是建立在这种纯看生辰八字五行理论基础上的算命术,完全
弃后天人为和社会因素于不顾,它的科学性又到底有多强?它
的揭示人生命历程的预测又到底有多可靠?
现在暂时先不管这些,言归象征律本身。
〔人身一小天地〕 把人象征天地宇宙或自然界,是古人
的一种普遍认识。天地自然运周不休,人也运周不休,天地自
然有阴阳五行,人也有阴阳五行。人既有阴阳五行,那末,推
究每个人出生年月日时干支所秉受的阴阳五行之气的不同,
从而推测他们一生的人生历程,就被引进到算命术中来了。
1人与四时合序〕 这是由人身一小天地生发出来的一种
象征律。阴阳家把十天干和十二地支分为阴阳五行,按照日
与天会的原理而记年,月与日会的原理而记月。这样一年有
四季十二个月,算命的就把一个人降生时碰上的天干地支,分
为年、月、日、时四柱,从而推定他一生的吉凶。此外,结合出
生季节看五行的旺相休囚死,用的也是一种人与四时合序的
象征方法。
〔五行寄生十二宫 这是一种完全模拟自然界生物在一
年十二个月中生长衰绝的象征律。在五行寄生十二宫的理论
中,分别有长生、沐浴、冠带、临官、帝旺、衰、病、死、墓、绝、胎、
养等状态。这些状态,循环无端,周而复始,而秉有五行之气
的人,在生理上同样也有着这种相似。其中首先是绝,绝又叫
受气或胞,好比万物处在地里,还没有成形,又象母亲腹空,没
有怀孕;二是受胎,这时天地气交,氤量造物,物在地下萌芽,
好比人受父母之气I三是养而成形,万物在地里成形,就象孩
 323 
子在母腹成形一样;四是长生,万物发生向荣,象人初成形而
生长;五是沐浴,沐浴又叫作败,因为万物始生,形体柔脆,容
易损伤,似婴儿刚出母腹三天,给他洗浴,容易困绝;六是冠
带,这时万物渐渐秀荣,有如人开始穿起了衣冠;七是临官,万
物既已渐趋秀实,就象人临官似的;八是帝旺,夭地万物至此
成熟,人亦至此精力健旺;九是衰,万物由成熟开始转向形衰,
好比人由盛壮转向衰老J十是病,万物有病,如同人有病一样;
十一是死,万物死亡,和人的死亡没有什么两样;十二是墓,墓
又叫库,万物成功藏进仓库,就象人死进入坟墓。归墓以后,
万物又受气胞胎而生,就这样周而复始,直到无穷。
[:五行秉性和相貌性情〕 人既秉受天地五行之气而生,
那末按照命理学家的说法,不同五行秉性的人,也就自然有着
不同的相貌性情了。命主以木为主的人瘦长清朴,因为木形
修长,木质清朴;命主以火为主的面赤聪明,因为火色红赤,火
性闪烁J命主以土为主的人面黄忠淳,因为土属黄色,土性敦
厚;命主以金为主的人面白刚毅,因为金属色白,性质坚硬;命
主以水为主的人面黑机灵,因为水色沉黑,水性流动。
〔象征社会伦理纲常的用神] 命理学家论命看重用神,
而用神名称的由来,多半象征着封建社会的伦理纲常。《三命
通会》探索古人立印、食、官、财名义时说「生我者有父母之
义,故立名印绶。印,荫也;绶,授也。譬父母有恩德,荫庇子
孙,子孙得受其福,朝廷设官分职,畀以印绶,使之掌管。官而
无印,何所凭据?人无父母,何所怙恃?其理通一无二,故曰
印绶。我生者有子孙之义,故立名食神。食者如虫吃物,盖伤
之也。虫得食物则饱,人得食物则益。被食则损,造化以子成
而致养,即人子致养父母之道也,故曰食神。克我者,我受制
 324 
于人之义,故立名官、煞。官者棺也,煞者害也。朝廷以官与
人,此身属于公家,任其驱使,赴汤蹈火,不敢有违,至于盖棺
而已,是官害之也。凡人梦棺则得官,亦是此义,故曰官、煞。
我克者是人受制于我之义,故立名妻财。如人娶妻,而妻有妆
奁田土,赍以事我,终身无违,我得自然享用,不致困乏,况人
成家立产,须得妻室内助,故曰妻财。是四者,术家立名之大
义。”毫无疑问,这些算命家立名的大义,是深深打上了封建社
会伦理纲常烙印的。
以上所举五种,只是就算命术象征律中一些主要方面而
说的。其实,算命术中的象征律还远远不止这些。说实在的,
几乎在所有的命理观念中,都是或多或少地浸透着这种象征
律的。而这种象征律,在某些局部还有点类似于机械类比推
理的味道。虽然这种类比推理的办法比较粗糙。
在这种天地阴阳五行象征基础上建立起来的算命术,在
具体的推算过程中,采用最广的是一种演绎的推理方法。在
这种演绎推理中,命理学家先把一个人出生年月曰时的干支
五行和他的大运推排出来,然后又把这作为推理的前提,一步
一步地演绎推算下去,如命中五行自身属金,并且金多土多水
少木少火少,就可演算出这个人秉有金的属性,生性刚正不
阿,由于金水相生,汩汩流通,又 应聪明过人,技术超群,在运
行中,不喜比劫的金和印绶的土,因为这在八字中已经够多的
了,倒是食神伤官的水、正财偏财的木和官星的火,可以为我
所用,所以一行到水运、木运、火运,必当发迹无疑。
在实际过程中,算命先生对一个人八字的演算要远比上
面所说的复杂得多。比如一个命的日干是金,在推演时不仅
要考虑到整个八字五行对日主金的影响关系,并且还要综合
 325 
考虑到出生的月份和寄生十二宫对自身的利弊,八字彼此之
间的刑冲化合,以及干支彼此交通往来所形成的种种吉神和
凶煞等等。所以这种独特的演绎法,虽然有些类似于我们今
天逻辑学上的演绎推理法,然而又不能完全等同起来。因为
演绎推理要求,推理的前提必须是正确的,否则就推不出正确
的结论来。可是对于命理学家来说,他们所信奉或假设的前
提有多大的可靠性,那就得打上几个大大的问号了。
通过说理分析,我们已不难想见,算命术象征律所象征
的,不过是通过“人身一小天地”、“天人感应”等观念而作出的
对自然现象的一些表面象征。然后,又通过出生年、月、日 、时
所得五行和这些表面象征挂起钩来进行千变万化,无穷无尽
的演阵推算。这种推算,既不顾个人对自己前程的努力如何,
又不顾整个社会发展对人类所造成的种种重大影响,单从海
市蜃楼的徒有堂皇表面的象征出发,那末建立在这基础上的
演绎法,岂非是不攻自破?算准是偶然的,算不准是必然的,
现在我们到了为算命术下这种结论的时候了。

\section{学术乎?迷信乎?}

中国算命术比起世界上任何算命术来,因为有着一个表
面看去似乎十分完整严密的学术体系,所以远比其他国家的
种种算命术复杂得多,难学得多。也就因为这个原因,所以一
千多年来,非但一直盛行民间不衰,并且还博得了一些学者大
儒的普遍青睐。撇开算命术发明以前已深信天命的孔子、列
 326 
子等名满天下的巨子不说,单就算命术发明以后,南宋的大儒
朱熹、明代的闻人刘基、清代的学者俞曲园等,都是信命而且
自己又会算命的一些代表人物。
由于这些学者巨儒的介入,使算命术这种玩意,更为社会
上一些学问不高的平民百姓所深信不疑。因为这些学者大儒
的崇高声望,确实影响了一大批人。
学者大儒的介入,主要是因为算命术所依据的,是我国天
人感应和阴阳五行的哲学理论。在这种貌似科学的理论支配
下,东汉大学者王充尽管可以不信鬼神,但却坚信命运。他认
为:“人禀气而生,含气而长,得贵则贵,得贱则贱J“或贵或
贱,或贫或富,富或累金,贫或乞食,贵至封侯,贱至奴仆,非天
禀施有左右也,人物受性有厚薄也」并且断言「富贵贫贱皆
在初禀之时,不在长大之后随操行而至也J为什么他要坚持
提出这种“人受命在父母施气之时,以(已)得吉凶矣”的说法
呢?他说得很清楚,就是"天施气而众星布精,天所施气,众星
之气在其中矣”。原来自然界中充满了一种天气和众星的精
气,人在结胎之初受了这种气的或厚或薄的影响,就会影响到
他今后的一生。这种宇宙之间的气,自然也包涵了金、木、水、
火、土布施出来的五行之气。
差不多和王充同时,《白虎通  五行篇》中还把自古以来
的阴阳五行思想,和社会人事作了种种紧密的联系。书中煞
有介事地说「父母生子养长子,何法?法水生木,木长大也「
“男不离父母,何法?法火不离木也。女离父母,何法?法水
流去金也。”“不娶同姓,何法?法五行离类乃相生也。”“子丧
父母,何法?法木不见水则憔悴也。”“父死子继,何法?法木
终火旺也。”“臣谏君,何法? 法金正木也。子谏父,何法?法
327-
火揉直木也。”真还被说得头头是道。
后来,唐朝的李虚中和五代的徐子平接过东汉学者论命
论五行的学说,在论命中大加发挥,并从而形成了一套完整的
学术体系。
从上所述,由于算命术采用了阴阳五行哲学和天文星象
中的一些现象作为立命的理论基础,加上通人达士学者大儒
的加入和肯定,就使得神秘的中国算命术在无意中披上了一
层学术的堂皇外衣。说实在的,如果撤开算命目的,单就这种
算命术本身着眼,它确实有着一整套完整的体系,若要深入探
究下去,也够你一辈子研究的,可是算命术的发明,毕竟是有
看它的目的的,这目的就是探究每个人一生未来的吉凶荣枯、
寿夭贫富等等,属于一种多少年来人们一直向往着的预测术。
预测术在我国有史以来,一直是一门人们前赴后继,悉心
探索着的学问,它的内容除了四柱算命外,还广泛地包括着占
卜、星相、拆字、起课详梦、扶乩等术,这里面尽管掺杂着种种
江湖骗子和浓重的迷信色彩,可从另一方面来说,毕竟也注进
了一些通儒学者历时绵久的研究探索,因为这种预测术对于
整个人类来说,确实是太神秘,太具吸引力了,尽管到头来,这
种探索努力只能是失败的。
那末说到这里,算命术是不是就不迷信了呢?答案是,就
算命术本身的一整套完整体系来说,里面多少蕴涵着一种学
术思想,但就预测术的目的而言,由于算命术作为演算基本的
大前提,绝口不谈个人因素和社会因素,却空谈什么秉自先天
的五行生克之类的哲理,其本身的合理程度由此可见。加上
结合天文星象,又混进了好多宇宙中并不存在的凶神恶煞等
等,所以非但推断不准,并且还在它长期的存在过程中,不可
328 
避免地给江湖骗子以可乘之机,从而使算命术更加笼上了一
层浓重迷信的神秘外衣。 ’
在目前,社会上迷信算命术的仍然不乏其人,这说明它的
存在,有着极其复杂的社会因素和历史根源。要彻底根除人
们对算命术的迷信,单纯靠国家禁令和行政的堵,是无济于事
的。看来,最好的办法还是静下心来,对它进行无情的解剖,
把它的原原本本暴露在光天化日之下,让人们自己来作一番
科学而又深入的评判,这样,不仅这种预测术究竟灵验不灵验
可以尽人皆知,并且还使那些江湖骗子失去了混饭吃的资本,
岂不美哉?
李虚中发明算命术之初,原是根据一个人出生年、月、日
所碰上的干支进行推算的,由于推算的结果,相同的人实在太
多,于是五代的徐子平才又加上时间的干支,从而奠定了年、
月、日、时四柱八字推命的基础。可是这样下来,据说也只有
五十一万几千种命,因此社会上八字相同的人还是不少。并
且在一些相同的八字中,有的还命运截然不同,判若泾渭。按
照宋人有关资料记载,蔡京的八字是丁亥、壬寅、壬辰、辛亥,
按理在八字的格中,这是属于壬骑龙背的格局,该是大富大贵
的,可是偏偏在京城里郑粉儿子的八字,也和蔡京一模一样,
但在人生的道路上,这小子却潦倒不堪,没能混好。这就给我
们的命理学家出了个不小的难题。.尽管前面我们说过,算命
先生在碰上问题棘手时,自会举出种种遁辞,自圆其说,然而
这种局面的出现,毕竟是很严峻的。
对于这种难以解释的现象,就是命书本身,也不得不作彻
底的承认。《三命通会》是明末以来最为权威的一本命书,内
中卷六曾载《十干十二年生大贵人例》一篇,说是只要在六甲
 329 
年丁卯月乙未日戊寅时,六乙年己卯月甲戌日乙亥时,六丙年
庚寅月丁巳日丙午时,六丁年丙午月壬辰日丁未时,六戊年壬
戌月己丑日戊寅时,六己年辛未月己未日丙寅时,六庚年甲申
月庚申日辛巳时,六辛年丙申月庚午日辛巳时,六壬年辛亥月
壬辰日丁未时,六癸年丙辰月丙辰日戊子时,这六十个时辰出
生的人,必定是建功立业大贵的人,不然至少也得是个出尘的
神仙。以上这六十个时辰出生的人分配到六十花甲中去,每
年只有一日一时,才有大贵人应世。可是对于这种说法,《三
命通会》作者育吾山人无可奈何地感叹道:“大贵人莫过帝王。
考历代创业之君,及明朝诸帝,无一合者。余尝谓天下之大,
兆民之众,如此年、月 、日 、时生者,岂无其人,然未必皆大贵
人。要之天生大贵人,必有冥数气运以主之,年、月、日、时多
不足凭。”
好一句“必有冥数气运以主之,年、月 、日、时多不足凭”,
其中“冥数气运以主之”是虚晃一枪的遁辞,只有“年、月、日、
时多不足凭”一句,才是书主人多年来为人看命的甘苦之言。
事实上,作者在一生命理研究的生涯中,看到的缙绅人家和凡
夫俗子同命的,多得数也数不过来。就是在缙绅和缙绅之中,
八字相同而命运不同的也大有人在。因此作者接着说道:“如
黄懋官侍郎,与申价副使同命,黄死于兵祸,申死牖下。申先
黄死,官之大小,又不论也d 朱衡与李庭龙同命,朱发科壬辰,
李发科癸丑(两人没同一年登科 朱官至尚书,李止大参,寿
又不永。其子孙之多寡贤否,又不论也。万泉与饶才同命,万
举进士,官至卿贰,饶止举人,官至太守。然饶多子而万则少,
又万以谪戍死,而饶则否,其寿夭得丧又难论也。三河黄且斋
兄弟同产,而功名先后,亦自不同。”为之,作者不由感叹:“况
330 
夭下之大,九州之广,兆民之众,其八字同者何限,又乌以例论
耶?”
说到“遁词”,花样也真还不少。易宗夔《新世说》载一则
星命故事说,康熙年间,史胄司由裸阳携家入都,泊舟水驿而
妻子临盆生子,取名贻直。当时由于风大无法行舟,于是胄司
便登岸纵步,见一冶工家也正好刚生孩子。一问下来,那冶工
家的孩子和自己孩子的八字,竟然完全相同。二十年后,胄司
的儿子史贻直官至清禁,自己则告老还乡。还乡途中,胄司途
经过去泊舟的水驿,为了看看冶工家的孩子现在情况如何,于
是便上岸进行寻访。谁知访问下来,冶工家的门庭依然如故,
只见门里一位面色白皙的少年正在操作打铁,就是当年和他
孩子同年同月同日同时生的铁匠儿子。归家之后,胄司因精
于子平之术,心想两个孩子八字完全一模一样,而遭遇怎么竟
会如此不同呢?想到后来忽然悟曰「这两个孩子四柱中火气
太盛,少水制约。好在我那孩子,因为生在舟中,所以得水之
气,可补不足,而铁匠家的那个孩子,则因以火济火而失去了
调剂之妙,所以就难以发迹了。” 这真可谓是造“遁辞”硬作解
释的顶峰了。
罗大经的《鹤林玉露》,堪称宋人笔记中的佼佼者。书中
记有“大算数”一则道,一天有人拜访黄直卿,说是善算星数,
能够预知吉凶祸福。对此,黄直卿回答道:“我也有个大算数,
《书》说:'惠迪吉,从逆凶。作善,降之百祥,作不善,降之百
殃。’《大学》也说:'言悖而出者,亦悖而入。货悖而入者,亦悖
而出 这个数,从古到今没有差错,难道不比你的算数强吗?”
这里,黄直卿引《尚书》和《大学》的话,大意是说一个人做善事
就吉,做恶事就凶,做善事,上天就会赐福给你,做坏事,上天
就会降灾给你。说话背理伤人的,也会被人家所伤。用不正
当手段弄进货物的,也会被人家用不正当的手段弄去。
这里,黄直卿把《尚书》、《大学》的这段话当作为人处世的
大算数,从而风趣生动地批判了客人的星数,可谓笔力扛鼎。
的确,社会上立身处世,最要紧的还是 “大算数”,因为这
是自己给自己算命的最佳方案。种瓜得瓜,种豆得豆,佛家的
因果报应论在这里和中国的传统道德,在某种程度上该说是
一线吻合的。平时佛家反对算命,这也主要是他们信奉“众善
奉行,诸恶莫作”的信条,好事归我做,至于上天如何安排处
置,不是我应该过问的事。
事实上,偏信命运安排,忽视“大算数”而栽跟斗的也大有
人在。据说明清之际有个染坊儿子,八字算下来是个大富大
贵,高官厚禄的命。家里人听说孩子生了这么个高贵的命,都
大喜过望,从小开始就什么都听他的。后来孩子长大酗酒游
荡,不务正业,结果酒醉落水而死,死时才十九岁。这难道不
是偏信算命,从小失于教育所招致的祸患?
再如从前文推子息的歌诀来看,也是很荒谬的。歌中谈
到最多的是五子,而封建社会里达官贵人广蓄姬妾,生子在五
个以上的比比皆是。又如目前社会上实行计划生育,提倡一
对夫妻只生一个孩子,而歌里却三个四个五个迭出,又作如何
解释?这不使命理学家犯愁了么?
因此结论是:对于中国文化中的算命术,我们先要了解
它,剖析它,因为了解、剖析为全过程,就是批判的全过程。堵
不如导,这是一条早已被证实了的历史规律。
 332 

\section{事 在人为}
我们中国有句老古话,叫做“人定胜天”。这是说,人的意
志和力量,可以战胜自然,战胜老天爷对于人间的一切安排,
事在人为嘛。刘祁《归潜志》十二说:‘人定亦能胜天。”《逸周
书  文传》把这称为“人强胜天”,《史记  伍子胥列传》则称之
为“人众胜天”。
对于老天爷的安排和人定胜天,我国自古以来,在儒、释、
道三教中,就各有各的看法。儒教又称孔教,他们是笃信天命
的;道家祖师爷老子讲“道法自然”,在一定程度上排斥了天命
对于人的主宰作用;至于佛教,由于四大皆空,甘于清苦寂寞
的人生观,决定他们不愿和天命观打交道。
说到这里,有关诸葛亮祈禳的故事,或许另有启发。当年
羽扇纶巾,为刘氏江山呕心听血的蜀相诸葛亮,晚年驻兵五丈
原时,由于重病缠身,自知不久人世,可是为了阻遏魏将司马
懿进攻蜀中,还是演出了一场为自己祈禳改运,希图借延寿命
的悲剧。故事见《三国演义》第 三回。
话说当夜孔明扶病出帐,仰视天文,看到'三台星中,客星
倍明,主星幽暗,相辅刘曜,其光昏昏”,因而判断自己命在旦
夕。
蜀将姜维知道孔明自有秘术,可以延寿,于是劝道「天象
虽然如此,丞相何不用祈禳之法加以挽回?”孔明点头。
这夜正值八月中秋,银河耿耿,玉露零零,旌旗不动,刁斗
 333 
无声。孔明在军帐设坛,点起了本命灯,同时祭以香花果物。
他白天照样计议军机,晚上则披发仗剑,步罡踏斗,镇压
将星,使不殒落。
正当祈禳到第六夜,还有一夜就可暂改天命的紧要关头,
忽有司马懿派夏侯霸突击蜀营,大将魏延飞步入禀:“魏兵至
矣。”由于脚步慌乱,竟将坛中的本命灯扑灭。
孔明见此,禁不住脸色发青,弃剑叹曰:“生死有命,不可
得而禳也!”
过不多久,孔明就一命呜呼,抱恨终天了,享年才五十四
岁。
暂不问孔明祈禳是否真有其事,但故事至少说明两个问
题,第一是用祈禳之法是不能够改变命运的,诸葛亮尚且如
此,又何况他人?二是祈禳之法不能改变命运,并不等于说用
别的方法就不能改变命运了。比如修心补相,修心改命的改
命之法,就一直为古来的有识之士所肯定。由此,古人有修心
劝世古歌一首道:
心好命又好,富贵直到老。
命好心不好,福变为 祸兆。
心好命不好,祸转为 福报。
心命皆不好,遭殃且贫夭。
心可挽乎命,最要存仁道。
命实造于心,吉凶惟人召。
信命不修心,阴阳恐虚矫。
修心一听命,天地自相保。

【335页图】

古歌虽然肯定天命,然而它的积极意义,在于肯定的同
时,又大声疾呼地唱出“最要存仁道,命实造于心,吉凶惟人
召”这样闪耀着思想光芒的名句。
“命由心造”,人定胜天,人的因素第一。
新中国成立以来,特别是改革开放以后,人民生活水平普
遍提高,什么人间奇迹都可创造出来。不光是大陆,就是现在
还风行着算命术的港澳和台湾,连一些命理学家也自觉不自
觉地在自己的著述中,流露出一些改变命运的思想或方法。比
如颜昭博《子平八字大突破 补运之道》就说J宇宙万物,一 物克一物,一物生一物。丙火虽强,辛可怯之;癸水虽弱,戊土
燥厚,无癸不生。是以逢衰败,不必颓丧,绝中自有生处,或往
东南西北,或求之兄弟父母,命中自有安排,但依路寻找,绝无
差错,所谓天无绝人之路,一枝草一点露,但相仍须由心生。”
接着还举了四个例子,其中一例:己亥年、丁卯月、辛亥日、己
丑时,命中以偏印为用神,偏印指长上,或父母或亲长,岁运逢
衰败,可求助长上,并往中部发展。”
再如从另一角度看,命里缺什么喜什么,就可在平时的生
活起居和经营谋生等方面,作如下的种种人为补偿,
〔喜木J 平时衣着服怖,家中布置装潢,可尽量采用青绿
系列。白天工作或晚上龌眠,头面的方位可向家面,有益健
廉,并宜住在方向朝东的屋子里。对于经营谋利,出外求取进
展,当首先考虑以东方为佳。
〔喜火:! 平时衣着服饰,家里布置陈设,可尽量采用红色
系列。白天工作或晚上陲眠,头面方位应当朝南,可以增益健
康和助进运势,并宜住在方向朝南的屋子里。对于经营谋利,
出外求取进展,当首先考虑以南方为佳。
336
(:喜土〕 平时衣着服饰,家里陈设布置,应尽量采用黄色
系列。对于经营谋生,应首先考虑在本地求取发展。因为土
的方位在于中央,所以对于平时生活工作的所喜方位,理应结
合其他木、火、水、金四行喜忌,再行决定。
〔喜金〕 平时衣着服饰,家里陈设布置,应尽量采用白色
系列。白天工作或晚上睡眠,头面的方位应当朝西,借以增进
健康或助益运势。对于经营谋生,应首先考虑以西方为隹。
[:喜水] 平时衣着肥饰,家里陈设布置,应尽一采用黑色
系列。白天工作或晚上睡眠,头面的方便以朝北为父..借以培
进健康或助益运势。对于经营谋生,当 首先考虑以 北 方 为
又如《现代命学》认为,改运之说纯为无稽之谈,当心被良
赛不齐的术士改运骗财。要是你真的耍非改不可,作者建议
了一个不错的办法,“就是运带破财欠安, 则拿一笔钱当救济
金或捐给慈善机构。若带血光运则到捐血中心去捐血,可逢
凶化解」 '
当然,单从这方面来理解人定胜天自?视苍白无力,很消
- 极,很勉强,甚至是很荒谬的。因为这种'人定"的目的,还不
是为了应顺天命,补其不足?和“胜天”的字眼压根儿沾不上
什么边。
然而从积极方面看,台湾李铁笔著《八字命学范例》,内中
倒也颇有些可采的思想。虽然他一方面大谈其命理学,可毕
竟还是饶有力度地深深叹道:人为万物之灵,会肯如此宿命
吗?愿意如此消极吗?能够接受如此的宿命论吗?当然不可
能,更不愿意束手就缚地等待命运来任意宰割,否则这世界怎
么可能如此的多彩多姿?人类怎能漫游外太空,怎能享受到
 337 
今日高度的文明生活,物质享受?就因为不愿宿命,不想坐以
待毙,唯有向命运挑战,开创命运,而此股力量就是推动社会
进步的原动力,亦是源源不断的知识积累的泉源,当然亦是人
定胜天,命运操纵在自己手中的结晶。
与此同时,李铁笔在论述与生俱来的先天宿命过程中,还
断然肯定种种后天因素对于命运的影响。这些 后天因素为,
人为、环境、教育、风俗、习惯、社交、他力、天灾、人祸、时势、修
养、道德等等。这些后天因素扮演的角色和影响,往往超过先
天命理因素。两颗原子弹在日本长崎、广岛一投,则一视同
仁,不管八字是命佳运顺大富大贵者,或命劣运背贫贱卑下
者,均得同年同月同日同时报销。双胞胎的人生命运,多未必
完全相同,这些都是众所皆知,不难求证的事实。
再如,《中国古代算命术》问世以来,先后收到来自全国各
地数以千计的读者来信,其中有一部分读者或因工作不顺利,
或因生活中屡遭坎坷,或因儿女婚姻大事,一时思想不通,情
绪低落,甚至悲观厌世。在找不到正确答案和切实可行的解
决办法而感到痛苦无望时,他们往往把一切咎之于命运的安
排,因此失去了战胜困难,改变处境的勇气和力量。其实,他
们本身的问题有时并不见得多么严重,严重的是缺乏自信心
和精神支柱。在这种情形下,他们最需要的是外界给予安慰、
鼓励和开导,动之以情,晓之以理,将他们本身的主观能动性
调动起来,使他们建立信心,增强勇气,变消极悲观为积极努
力。这样,尽管他们仍然生活在原来的环境里,那些恼人的问
题依然存在,可是由于人的思想方法变了,不再视一切困难为
畏途,只知消极地抱怨生活不公,命运不好,而是以崭新的面
貌出现在生活中,以积极向上的态度,重新看得人生,重新品
 338 
味人生的意义和价值,只有这时,他们才会发现自己是有能力
的,应该好好面对生活。
辽宁有位女同志来信诉说:“从少年时起,工作上、家庭里
从未安顺过,如愿过,每天生活在一种压抑、苦恼、无奈、无聊
的环境中……不如一死了之。” 因此认为自己命不好而求算。
后经委托,由北京赵锡玲居士劝慰开导,精神状态大为改观。
不久,她不仅给赵居士打去了热情洋溢的感谢电,还写去了发
自肺腑的感谢信。她说:“收到来信感到有一种意外收获的欣
喜心情,……。您信中的话语如同面对面谈心一样,十分亲
切,强烈地感染着我的心。”接着并表示:“我要振作精神以积
极的态度面对生活。”
广东一位读者来信说,当地算命的说他儿子的未婚妻克
父又克夫。因此对儿子的这门婚事有些担心害怕,想算算他
儿子能不能和这位女友结婚。试想,如果不能说服老人转变
思想,就会断送一对年轻恋人的幸福,同时使父子之间的矛盾
激化,造成家庭不睦。后也经过赵耐心说服,晓之以理,结
果他儿子怀着激动的心情给赵寄去了一封感谢信。信中他说:
“婚姻这件大事正像您所说的那样,'以爱情为基础,以性格志
趣相投为第一'。我真不知怎样感谢您给我的指点,只好用书
信略表一片心意。以后,我遇到疑难之事时,希望再次能得期
您的帮助和指点。”信尾署名:“一个被您解开心中迷雾的人 X
X X J
广州一读者因做生意向别人借了几万元作本钱,没想到
被坏人将钱骗去,妻子又为此事与之反目。一时内外交困,逼
得他走投无路,开始怀疑起自己的命运来。在这种时候,人最
需要的是安慰和鼓励,使他认清困难是暂时的,不幸总会过
 339 
去,未来的日子还长着呢。通过赵居士语重心长的开导,这位
读者在给赵的信中说「我太感谢你了,你在信中虽然寥寥数
语,但句句中肯,言简意底,使我倍受鼓舞,我一定要经受住疾
风暴雨的打击……以死都不怕的精神来面对眼前的困难,那
么这困难就无所畏惧了。我要用拚命的精神来工作,保证在
今年内还清债务,然后去拜访你。我们虽然素昧平生,但我却
像对老朋友似的唠叨个没完,因为这都是我的肺腑之言。如你
能来我处作客,我当万分高兴,随时欢迎。”有了这样积极的生
活态度,眼前的困难还愁战胜不了么?
江西一读者在给赵的复信中说「谢谢您复信的启示,真
没想到一个素不相识的读者来信,会受到重视和礼遇,确实太
难能可贵了。做为一个农村的读者,能得到这样热情的对待,
真不知怎样才能表示对你们的谢意。”
浙江一位读者也在信中感谢赵居士道「收到回信非常高
兴,因为我的处境和社会地位足以使我自悲,信发出后总认为
希望渺茫,没想到您们对我的信这样重视。希望能常常给我
启发帮助,我非常感谢您。”
还有许多读者来信,这里不一一例举。平时由于本人身
体不好,工作又忙,对于众多读者来信,一般都无法回复,或偶
而请于建华、赵锡玲代为作复。仅从以上这几位致赵锡玲的
读者信中,可以看到这样一个率实,就是:人是需要密助的,话
是开心的饰匙。一个人遇到不原心的*,往往喜欢钻生角尖,
越想趟倒餐,越器I越无路可走。当局者迷,旁就者清,这时候如
果有人稍加点拔,使他从'山穷水发挖无路"的心境中窥到“柳
暗花明又一村”的光明前景,认识到困难总会有办法克服的,
坎坷人人都可能遇上,不幸的并非你一人,关健是如何看待和
 340*
处理。是弱者,就一味信命、认命,消极等待,得过且过;是强
者,就正视现实,勇于向命运挑战,认真分析矛盾所在,总结经
验教训,找出切实可行的解决办法。这就是,人为"因素,就是
“他力”,再加上一颗爱心和一份真诚,后者可以算是“道德”吧,
有了这几种因素就可以改变一个人的精神面貌,帮助一些人
从命运的羁绊中挣脱出来,站在更高的角度看待人生,看待自
己,转变悲观厌世的情绪,振作精神,重新鼓起生活的风帆。思
想疙瘩一经解开,人就会焕发出巨大的精神力量,这力量体现
在生活和工作中,必然给生活和工作带来新的起色,而且人的
精神一变好,头脑相应也就显得聪明起来,一些实际问题也就
可以迎刃而解。那时,命运自然也会有所改变,这种积极的改
命态度应该得到人们的赞许和社会的肯定,当然也就毋庸置
疑了。通过启发诱导使人们的思想由消极变积极,由悲观厌
世变乐观进取,这正是“命由己造”人定胜天的具体体现。这
种巨大的“改命”力量,恐怕是老天爷也始料不及的吧。
近来阅读刘冬泉编著的《黄道吉日析》,内有《命和运》一
段,说得浅近而又深入,很受启发「所以,「块理不是一成不变
的,事在人为。不是说穷则变,变则通吗?理自在人心,看你
以什么样的心态和方法去定论罢了。理可知一而推十,知十
而推百,举一可反三,如此类推。事系在于人天生性情,或将
错就错,一意孤行,或没勇气面对现实的挫折,动辄自我安慰,
给自己一个 '命也'的借口,这 '命也 会令你永无再立之日。
如果孙中山在起义遇挫之初,即鸣鼓收兵,说不定你我仍留辫
子,仍缠足,仍在列强的欺侮下度天日。命最积极的注释是,
疏导自己心中的困顿、茅塞、纷扰,技巧性的化解凶厄,更达
观、更勇敢地面对自己和环境,鼓舞起而改革只蒙头不藏身的
 341 
鸵鸟精神,如此,上天不会疏忽你,难为你的」
接着,刘冬泉又在同书《命运与缘份》一节中说:“其实,天
并不掌管你的命,我们自己的事情还要靠我们.自己去做。”“人
要以积极的态度从能知的限度里,承担自己的命,保护发展好
的方面,努力转变不利的方面,不听天由命,不怨天尤人。因
为你可知道,在你慷慨地把命托付给缘的时候,同时就放弃了
再创奇迹的机缘!”“人的生机何在?希望何在?动力何在?全
在有你一颗自尊、自爱、自强、再磨练、再教育自己的心。”
综上所述,作为一种学术文化,我们在对其进行研究的同
时,应该站在科学、理性的高度,对其进行综合的分析评判,而
这种分析评判,又必须是理性的,客观的,公正的,任何简单、
粗糙,甚至谩骂,不允许人们去接触,畏之如洪水猛兽的做法,
都将因无济于事,收不到预期的效果而堕入形而上学、唯心主
义的泥潭,这是不言而喻的。